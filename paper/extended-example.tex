\section{Extended Example}

Consider building a lens which converts between between citations given
in EndNote format, and in BibTex format.  For simplicity, we are only
considering the author and title fields.  The synthesis problem for these
two regular expressions is provided below.

\begin{figure}
\begin{lstlisting}
Name    = [A-Z][a-z]*
Initial = [A-Z] "."
Word    = ([A-Z] | [a-z] | "-" | "," | ":")+
Title   = Word (" " Word)*

FN      = Name", " Name (" " (Name | Initial))*
Type    = "@inproceedings" | "@article"
BibTex  = 
  Type "{" 
    Word ","
    ("" | ("title={" Title "},"))
    ("" | ("author={" FN (" and " FN)* "},")) 
  "}"

EName   = Name (" " (Name | Initial))* (" " Name)
Class   = "Conference Paper" | "Journal Article"
EndNote = "%0 " Class "\n"
	 ("%A " EName "\n")*
	 ("" | "%T TITLE")
	  "%1 " Word "\n"

BibEndConv : BibTex <=> EndNote = ?
\end{lstlisting}
\caption{Synthesis Problem: Simplified Bibtex to EndNote}
\end{figure}

\begin{lstlisting}
Name = [A-Z][a-z]*

BNames = Name ", " Name (" " Name)*
Bib = ""
    | "author={" BNames (" and " BNames)* "},"

ENames = Name (" " (Name))* " " Name
End = ("%A " ENames "\n")*

BibEnd : Bib <=> End = ?
{ "author={Conway, Horton John and Kleene, Stephen}"
<-> "%A John Horton Conway\nStephen Kleene" }
\end{lstlisting}


We are trying to synthesize \BibEndConv{}.

There is no lens that goes between \BibTex{} and \EndNote{}, so one of them must
be expanded.
Indeed, both of them must be expanded, as \BibTex{} will never be able to be
the source of any lens in this problem, nor will \EndNote{} be able to be the
target.
This can be generalized for the sub-expressions of \BibTex{} and \EndNote{},  
If, any user defined data type does not exist in the possible types of the
other side, it must be expanded.

FIGURE:
Trees of user defined regular expressions, with red x's over those that must be
removed.

After completing these necessary expansions, the problem becomes synthesizing a
lens between the regular expression
\begin{lstlisting}
BibTex'=("@inproceedings" | "@article" | "@book") "{" Word "," ("" | ("title={" Title "},")) ("" | ("author={" (Name", " Name (" " (Name | Initial))*) (" and " (Name", " Name (" " (Name | Initial))*))* "},"))
\end{lstlisting}
and
\begin{lstlisting}
EndNote'="%0 " ("Conference Paper" | "Journal Article" | "Book") "\n" ("%A " (Name (" " (Name | Initial))* (" " Name)) "\n")* ("" | ("%T " Title "\n")) "%1 " Word "\n"
\end{lstlisting}

A basic synthesis approach from here will quickly fail.  Both \texttt{BibTex'}
and \texttt{EndNote'} are of the form $\Regex_1\Concat\Regex_2$, but there is
not a lens between \texttt{"\%0"} and
\texttt{("@inproceedings" | "@article" | "@book")}.

These issues can be handled by normalizing the regular expressions slightly.
By putting them in a form that normalizes away the equivalences of
associativity, distributivity, and \EmptyString{} identity.

The normalized form of these equivalences is too long to be contained in this
paper.  Instead, we will work through \afm{or should we just do the easier
example shown above}.

With those equivalences are handled, perhaps the lens can be synthesized.
A lens can be synthesized, if there is some pairing of sequences such that
there are sequence lenses between the two elements of the pair for all pairings.
A sequence lens can be synthesized if there is some pairing of atoms, such that
there is an atom lens between the two atoms.  An atom lens can be synthesized
if the regular expressions the same user defined data type, or if there is a dnf
lens between the two dnfs under the stars.

FIGURE:

"" | Author Author*

AuthorEndNote*

Looking at the number of sequences, we can easily see that these dnf regular
expressions do not have a lens between them, without some rewrite rules applied.
However, there are a large number of possible rewrite rules that can be applied.
Instead of taking and testing all of them, each rewrite rule is applied, and
then placed in a priority queue.
The priority for each problem is a linear combination of the number of rewrites
applied, and the distance between the two dnf regular expressions by a
pseudometric.

This pseudometric is defined as the sum of two metrics.
The first metric is a standard distance metric over the size of the two dnf
regular expressions.
The second metric is defined based on the distribution of user defined regular
expressions.
We can create an infinite dimensional free module over the integers from the
regular expressions, generated by the set $\SetOf{(U,i)\SuchThat U\text{is a
user defined data type}\BooleanAnd i\in\Nats}$.
We can provide a function $f$ that maps all regular expressions into the element
which counts how many user defined data types there are at a given depth.
\afm{i should probably write down this function with a mapsto style, instead of
just through text}
The second metric is the taxicab metric on the elements of the vector space
mapped to by the regular expressions.

FIGURE:

\begin{lstlisting}
"" | (name,name) (name,name)* -> \{ ((Author,0),2) , ((Author,1),4) \}
(name,name)* -> \{ ((name,1),2) \}

d("" | (name,name) (name,name)*,(name,name)*) =
taxicab(\{ ((Author,0),2) , ((Author,1),4) \},other)+(8-2)
\end{lstlisting}

Empirically, good transformations, like (name,name)* -> (name,name)+
do very well, giving a distance of zero, where bad transformations, like
name -> [A-Z][a-z]* do not do as well.

Using the priority queue, the problem that gets popped would be
"" | (name,name) (name,name)* <=> "" | (name,name) (name,name)*.
Unlike the previous example, this cannot be so easily determined to not be
equivalent.

A naive approach to this would be to attempt all possible pairings of sequences,
and all pairings of atoms within those sequences.  However, the complexity of
trying all permutations is (\# permutations)!.

Instead, an ordering can be provided to atoms, clauses, and dnf regular
expressions, where two elements are equal in the ordering if there exists a lens
between them.  Upon applying that ordering to the elements of the dnf regular
expression, determining whether there is a lens between two dnf regular
expressions amounts to merely seeing if each element of the ordering is
equal with respect to the ordering, as the one on the other side.

FIGURE
\begin{lstlisting}
e | (n,n)(n,n)* <=> e | (n,n)(n,n)*
reorder
e | (n,n)(n,n)* <=> e | (n,n)(n,n)*
matched together
e <=> e , (n,n) <=> (n,n) , etc.
\end{lstlisting}

This still isn't complete yet, for we don't know if the generated lens will
satisfy the examples.  The same problem arises with multiple permutations
being possible.  If all the sequences can be mapped to each other, then there is
a valid lens for each matching of the sequences!  And not all of these
valid lenses satisfy the examples.
To resolve this issue, we embed the parse trees of the examples into the
regular expressions, and treat regular expressions with different aspects of the
parse trees embedded as differently in the ordering.

FIGURE
e | name name(name, name)* , lalalaexample

\begin{lstlisting}
 /\
e  |
   /\
 name *
      .
     /\
   name name

lalaexample -> or2 (lala . Iterated(ex.am,p.le)

combine to

(the top thing but with the parse tree embedded alongside the expressions)
\end{lstlisting}




