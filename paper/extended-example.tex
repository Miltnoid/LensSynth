\section{Lens Synthesis by Example}

\dpw{It may be good to hack lstlisting so it can be in a backslash small font.
There is perhaps another environment we could use --- a code environment.}

In order to explain the major components of our synthesis algorithm,
we will consider a running example involving conversion between citations given
in highly simplified BibTex and EndNote formats.  Our BibTex format
consists entirely of specifying a list of authors on a single line as follows.
\begin{lstlisting}
author={Conway, John and Kleene, Stephen Cole}
\end{lstlisting}
Our EndNote format contains the same information, but uses different separators,
rearranges the order of first and last names, and uses a separate line for each
author. 
\begin{lstlisting}
%A John Conway
%A Stephen Cole Kleene
\end{lstlisting}

\begin{figure}
\begin{lstlisting}
Name = [A-Z][a-z]*
Names = (" " Name)*

BNames = Name ", " Name Names
Bib = ""
    | "author={" BNames (" and " BNames)* "},"

ENames = Name Names " " Name
End = ("%A " ENames "\n")*

BibEnd : Bib <=> End = {
"author={Conway, John and Kleene, Stephen Cole}"
<-> "%A John Conway\n%A Stephen Cole Kleene" }
\end{lstlisting}
\caption{Synthesis Problem Definition: Idealized Bibtex to EndNote}
\label{fig:bibend-spec}
\end{figure}

Figure~\ref{fig:bibend-spec} presents
the user's specification of such a problem.  In general, such specifications are
given by defining the source and target data formats~\footnote{Lenses
operate both left-to-right and right-to-left, so both formats are sometimes
sources and sometimes targets. Nevertheless, we will often call the
format on the left, the ``source,'' and the format on the right, 
the ``target.''} as well as some examples.  In this case, the lens
we wish to synthesize is \CF{BibEnd}.  It converts between 
the formats \CF{Bib} and \CF{End}.
Such formats are defined using a standard syntax for regular expressions.
Intermediate regular expressions are named in order to facilitate
construction of complex formats.

\subsection{Regular Bijective Lenses}

Once we have a synthesis specification, we need to find a lens that
will convert back and forth between formats and that satisfies the
given examples.  In this paper, we begin our work with a simple and 
relatively standard language of bijective string lens combinators.
In order to guarantee the functions generated by these combinators
form proper bijections between source and target, they are given types using regular expressions.  
We typicallly write 
$\Lens \OfType \Regex \Leftrightarrow \RegexAlt$ to indicate that
a lens $\Lens$ converts back and forth between data belonging
to the languages of regular expressions
$\Regex$ and $\RegexAlt$.

The simplest lens in the combinator language is the identity lens 
at type $R$,
$\IdentityLensOf{R}$, which copies data matching regular expression $R$
from source to target, and vice versa.  The
constant lens $\ConstLensOf{\String_1}{\String_2}$, when operated left-to-right, 
replaces string $\String_1$ with $\String_2$, and when operated right-to-left,
replaces string $\String_2$ with $\String_1$.

Each of the other lenses manipulate structured data.  For instance,
$\ConcatLensOf{\Lens_1}{\Lens_2}$



\paragraph*{Step 1:  Inlining Definitions.}


\dpw{----------------}

We are trying to synthesize \BibEnd{}.

There is no lens that goes between \BibTex{} and \EndNote{}, so one of them must
be expanded.
Indeed, both of them must be expanded, as \BibTex{} will never be able to be
the source of any lens in this problem, nor will \EndNote{} be able to be the
target.
This can be generalized for the sub-expressions of \BibTex{} and \EndNote{},  
If, any user defined data type does not exist in the possible types of the
other side, it must be expanded, shown in Figure~\ref{fig:expanded-defs}.

\begin{figure}
  \Tree[.\textcolor{dkred}{\sout{\texttt{Bib}}}
    [.\textcolor{dkred}{\sout{\texttt{BNames}}}
      \texttt{Name}
      [.\texttt{Names}
        \texttt{Name} ]]]
  \Tree[.\textcolor{dkred}{\sout{\texttt{End}}}
    [.\textcolor{dkred}{\sout{\texttt{ENames}}}
      \texttt{Name}
      [.\texttt{Names}
        \texttt{Name} ]]]
\caption{Marking of Definitions which must be expanded}
\label{fig:expanded-defs}
\end{figure}

                                
Trees of user defined regular expressions,
with those that must be replaced by their definitions crossed out.

After completing these necessary expansions, the problem becomes synthesizing a
lens between the regular expression
\begin{lstlisting}
Bib': ""   | ("author={" Name ", " Name Names (" and " (Name ", " Name Names))* "},")

End': ("%A " (Name Names " " Name) "\n")*
\end{lstlisting}
and

A basic type directed synthesis approach from here will quickly fail.
Both \texttt{Bib'}
and \texttt{End'} are of the form $\Regex_1\Concat\Regex_2$, but there is
not a lens between \texttt{"\%0"} and
\texttt{("@inproceedings" | "@article" | "@book")}.

These issues can be handled by normalizing the regular expressions slightly.
By putting them in a form that normalizes away the equivalences of
associativity, distributivity, and \EmptyString{}.

\begin{lstlisting}
DNFBib':
  ""
  | "author=" Name ", " Name "" Names ""
    (" and " Name ", " Name "" Names "")* ""

DNFEnd':
  "" ("%A " Name "" Names " " Name "\n")* ""
\end{lstlisting}

With those equivalences handled, one might think it is now possible to
synthesize the lens.
Unfortunately, we are still unable to preform any synthesis because the types
on the left and right are quite different, and the lens connectives require
similar types of the left and right.
For example, as \texttt{DNFBib'} has an outer |, we know an or lens must be
synthesized.
However, an or lens requires an outer | on both sides, and \texttt{DNFEnd'} does
not have any outer |.

While, with normalization, certain equivalences have been handled, there are
still many equivalences that haven't been handled.
For example, the star in \texttt{End'} can be unrolled, turning \texttt{End'},
and \texttt{DNFEnd'} into

\begin{lstlisting}
End'': "" | ("%A " (Name Names " " Name) "\n"
("%A " (Name Names " " Name) "\n")*)

DNFEnd'':
  ""
  | "%A " Name "" Names " " Name "\n"
    ("%A " Name "" Names " " Name "\n")* ""
\end{lstlisting}

\texttt{DNFBib'} and \texttt{DNFEnd'} now look very similar.
They both have an outer |, which has on one side a string, and on the other
side, two instances of \texttt{Name}, and an instance of \texttt{Names} outside
of any stars,
with two instances of \texttt{Name}, and an instance of \texttt{Name} underneath
a star.
From this, a lens can easily be made.

However, there is some element of choice in how the lens can be made.
The outer \texttt{Name}s can each be sent to each other.
and the inner \texttt{Name}s can each be sent to each other, so which
should get sent to which.
This can be determined from the examples.
If the first \texttt{Name} in \texttt{DNFBib'} were sent to the first
\texttt{Name} in \texttt{DNFEnd'}, then \texttt{Conway, John} would
get mapped to \texttt{\%a Conway John}, which is inconsistent with the
examples.
The examples can be used, when this situation arises, to specify which parts
of a regular expression can be mapped to which, when the types alone leave
this ambiguous.

\afm{I am not sure going into the metric and priority queue are worth going
  into,
  and least not here?}
%%% Local Variables:
%%% TeX-master: "main"
%%% End: