\section{Discussion}

We found that in practice, requiring a user to put in more information for the
synthesis of a string transformation creates a more reliable algorithm.
However, this comes at a few costs.  The user needs to know how to write out
regular expressions.

The largest difficulty in this is the restriction to bijective lenses.
Oftentimes, we find the need to hide information away in some alternate form.
For example, there is oftentimes whitespace in the definition of a regular
expression.  This whitespace must be copied somehow over to the other side.
Oftentimes this requires the use of having a portion of the string dedicated to
holding the whitespace information of the other side.

Another annoyance is the fact that if the correct lens does not exist,
rarely does the tool terminate.  Much of the time it will just spin trying to
perform increasingly complicated transformations to find the correct lens.
A better solution would be to find something close to the correct lens, and have
a user interaction model that supports iterative synthesis.  The user could give
more examples which fix where the program is wrong.

\subsection{Future Work}
Some of the issues seen in this have already been observed.  For example, in the
quotient lens paper, the issue of having to hack a place in the source string to
move whitespace information from the view to the source was noted.  An extension
we would like to work on would be able to synthesize quotient lenses.

Another approach to having disparate information on each side of the lens is
through the use of symmetric lenses.  This could be another approach to the
whitespace problem. 

%%% Local Variables:
%%% TeX-master: "main"
%%% End: