\section{Introduction}


\begin{enumerate}
\item Search through all possible lenses by: searching for good regular
expressions, then searching within those regular expressions for type directed
lenses
\item Putting the regular expressions in a normal form reduces the number of
regular expressions we must search through
\item Using user defined data types allows the synthesis algorithm to keep the
internals of regular expressions abstracted, unless the internals of those
regular expressions are changed, and the abstraction must be broken.
\item Using a pseudometric on regular expressions, based on the
distribution of user defined data types and size of the regular expression,
allows for an efficient search through equivalent regular expressions
\item We use a data structure combining examples with regular expressions to
synthesize only candidate programs which obey the examples
\item We develop an efficient strategy for finding if two
regular expressions have a lens between them requiring no rewrites which amounts
to merely ordering the components.
\end{enumerate}


The growing dependence on big data makes tools for reliably parsing
printing, cleaning, and transforming data increasing important.
One such class of tools are \emph{lenses}~\cite{?}, bi-directional
programs that help users solve the classic ``view update problem.''
More specifically, a lens provides a user with two functions, a
\emph{get} function and a \emph{put} function.  The \emph{get}
function translates native data into a new format, or \emph{view}, for
the user.  If the user updates the new view, the corresponding
\emph{put} function can translate the edited data back into
the native format.

\dpw{The following paragraph isn't so great. It might be helpful if
we cited more places lenses had been used in practice and then
augeas was just one of many examples.}
One concrete example of where lenses provide value is in managing
linux configuration files.  Such files come in a wide variety of
different ``ad hoc'' data formats, and even experienced system
administrators have a hard time keeping track of all the format
variations.  Moreover, a missing comma, forgotten keyword or 
misplaced vertical bar can cause the systems that depend on such
configurations to crash or misbehave.  In order to manage such configuration
files more reliably,  RedHat developed Augeas~\cite{}, a configuration
editing system based around lenses.  At the moment, Augeas contains more
than 200 stock lenses to provide views of common configuration files.

Lenses are popular because they typically come with strong guarantees
about the get-put (or put-get) round-trip behavior.  Such guarantees
help users guard against data corruption while reading and editing
high-value data sources.  These guarantees are often achieved through
the use of domain-specific languages with strong type systems that
guarantee invertability of lens programs~\cite{?,?,?}.  Unfortunately, such
lens languages are naturally quite non-standard, and, moreover, the
process of writing a lens is challenging as a user must think
about data conversion in two directions at once --- an
unusual task for the everyday programmer.  As a consequence, 
for the casual user of lenses, learning the language, its type system and its
unorthodox programming style is time-consuming and, in the end,
a disincentive to adopting the technology, despite its advantages.

In order to overcome the steep learning curve involved in programming
with lenses, we have developed a new system for synthesis of lenses
from their types and a collection of representative examples.  More
specifically, we focus on algorithms for the synthesis of bijective
string lenses.  The types for such lenses are regular expressions,
which describe the format of the source and target strings.  Given
such types, and a very small number of representative examples (sometimes
no examples at all), we synthesize a well-typed lens that implements
a bijection between data formats.

Recently, researchers have developed a number of new algorithms for
type-directed synthesis~\cite{?,?,?}.  Many of those
algorithms~\cite{osera+:pldi15,frankle+:popl16,armando+:pldi16} are
directed in part by the syntax of types.  For instance, to find an
expression with type $A \rightarrow B$, such algorithms will assume
that the expression in question is a function with the form $\lambda
x{:}A. e$, and they will continue the search for an expression $e$
with type $B$ under the hypothesis that the parameter $x$ has type
$A$.  In the case of the simply-typed lambda calculus, such an
approach reduces the complexity of the search for terms inhabiting the
type without losing completeness.  Unfortunately, such simple
syntax-directed algorithms are not directly applicable when searching
the space of terms defined by richer type systems, in which the types
exhibit complex isomorphisms, as is the case for string lenses and
their regular expression types. \dpw{I was trying to generalize a little
in that last sentence, but it may be awkward.}

\paragraph*{The Central Challenge.}
To illustrate the central challenge facing the design of our synthesis system
in greater detail,
consider the concatenation lens, $\Lens_1 \Concat \Lens_2$,
one of the simplest string lens combinators.  
When viewed from left to right, this lens defines a function \emph{get}, which,
given a string 
$\String$, this lens divides $\String$ into two substrings $\String_1$ 
and $\String_2$, applies $\Lens_1$ to the first, yielding $t_1$,
and $\Lens_2$ to the second, yielding $t_2$.  When viewed from
right to left, it defines a second function \emph{put}, which, maps
data back from strings $t$ to strings $s$ in an analogous way.

The type system for lenses ensures such \emph{get} and \emph{put} functions
satisfy strong invertibility properties.  In our case, the two functions
form a bijection.  In general, to 
describe the type of a lens, we write $\Lens \OfType \Regex \Leftrightarrow \RegexAlt$, which may be read ``lens $\Lens$ converts back and forth between
data in the language of regular expression $\Regex$ and data in the language
of regular expression $\RegexAlt$.  More specifically, to give a type
to the concatenation lens, we use the following typing rule.
\dpw{I think we should delete $\LanguageOf{}$ from all the unambiguity
constraints in the typing rules, but I didn't do that.  We should be
careful with consistency, and I didn't know how that would mess with 
all the proofs and other definitions.}
%To understand how
%this lens makes the choice about how to divide up the input string $\String$,
%it is easiest to look at the typing rule for the lens, which follows.
\[
\inferrule
{
\Lens_1 \OfType \Regex_1 \Leftrightarrow \RegexAlt_1\\
\Lens_2 \OfType \Regex_2 \Leftrightarrow \RegexAlt_2\\\\
\UnambigConcatOf{\LanguageOf{\Regex_1}}{\LanguageOf{\Regex_2}}\\
\UnambigConcatOf{\LanguageOf{\RegexAlt_1}}{\LanguageOf{\RegexAlt_2}}
}
{
\ConcatLensOf{\Lens_1}{\Lens_2} \OfType \Regex_1\Regex_2 \Leftrightarrow \RegexAlt_1\RegexAlt_2
}
\]
Above, the notation \UnambigConcatOf{\LanguageOf{\Regex_1}}{\LanguageOf{\Regex_2}} asserts that $\Regex_1$ and $\Regex_2$ are
unambiguously concatenable, meaning there is
exactly one way to divide up any
string $\String$ in the language of $\Regex_1\Regex_2$ into $\String_1$
and $\String_2$ such that $\String_1$ is in the language of $\Regex_1$
and $\String_2$ is in the language of $\Regex_2$.  These unambiguous
concatenation constraints guarantee the \emph{get} and \emph{put}
functions are indeed functions and that they form a bijection if
the underlying lenses are bijective.

One other useful lens to know at this point is the identity lens for a 
type $\Regex$, written $\IdentityLensOf{\Regex}$.  This lens
has the type $\Regex \Leftrightarrow \Regex$ and it simply copies
its input to its output (in either direction).

Now, with these typing rules in hand, consider trying to search for
a lens with the type $(A B) C \Leftrightarrow A' (B' C)$ where
we know of lenses with types 
$A \Leftrightarrow A'$, $B \Leftrightarrow B'$, and 
$C \Leftrightarrow C'$.  A straightforward,
syntax-directed search suggests applying the rule for concatenation,
and looking for two simpler lenses with the types $A B \Leftrightarrow A'$
and $C \Leftrightarrow B' C'$ respectively.  Unfortunately, at this point,
we are simply stuck---we have no lenses with those types.  Of course,
the natural
solution to this problem
is to introduce a type conversion rule to the system:
\[
\inferrule
{
\Lens \OfType \Regex_1 \Leftrightarrow \Regex_2\\
\Regex_1 \equiv \Regex_1'\\
\Regex_2 \equiv \Regex_2'
}
{
\Lens \OfType \Regex_1' \Leftrightarrow \Regex_2'
}
\]
With such a type conversion rule, we could first 
convert the type $(A B) C \Leftrightarrow A' (B' C)$
to the better aligned $(A B) C \Leftrightarrow (A' B') C$.
When we then proceed with structure decomposition using the
concatenation rule, we generate sub-problems involving
finding lenses for types
$A \Leftrightarrow A'$, $B \Leftrightarrow B'$, and 
$C \Leftrightarrow C'$, which is just what we had in hand.

Because equivalence of regular expressions is decideable, when
given a lens $\Lens$ with type $\Regex_1 \Leftrightarrow \Regex_2$,
it is easy to \emph{check} whether that lens also has type 
$\Regex_1' \Leftrightarrow \Regex_2'$, and lens programming languages
such as Boomerang \cite{foster:thesis} do so.  However, 
when we turn to synthesis and are given the lens type 
$\Regex_1' \Leftrightarrow \Regex_2'$ and the option to apply this
rule, the search space becomes unmanageably large --- we must guess
some arbitrary new lenses $\Regex_1$ and $\Regex_2$ that are
equivalent to  $\Regex_1'$ and $\Regex_2'$ respectively
and there are infinitely many such regular expressions to choose
from.  In essence,
we must solve the problem of search \emph{in two dimensions}:  In one
dimension, we must search for a regular expression \emph{type}
with an appropriate ``syntactic shape'' and in a second dimension,
we must search for a lens \emph{expression} that has the given type and
is consistent with the user's examples.

\dpw{Note:  I recall there is another aspect to the difficulty of
doing the search that I'm not mentioning here.  It has to do with
``horizontal'' composition -- I think we came up with an example of
lens that you could not write without horizontal composition, which
also makes the search space bigger.  Ideally, we should expand the
introduction to illustrate that additional problem.  The fact that you
need horizontal composition was surprising.  Extending the intro 
with that material will
further peak the reader's interest and show them we are doing something
still harder.}

\paragraph*{Our Solution.}  Our solution to this problem is to define
a new language of lenses that
cuts down the synthesis search space \emph{in both dimensions}.  \dpw{Is that
true?  In \emph{both} dimensions?  Because we eliminate the need to search
for compositions? But we haven't said that yet.}
This language is designed so that its regular expression types are limited 
to those in a pseudo-canonical form.  This pseudo-canonical form
demands that regular expressions be defined as a flattened 
union of concatenated atoms, where each atom is a simple string or
the Kleene star of an expression that is recursively in normal form.
When converted into this pseudo-canonical form, two regular expressions that 
differed only by associativity of concatenation will be syntactically equal.
For example, rather than searching for a lens with type 
 $(A B) C \Leftrightarrow A' (B' C)$, our algorithm will always search
for one with type  $(A B) C \Leftrightarrow (A' B') C$
  
More generally, many, but not all, structural incongruities that
prevent the existence of lenses disappear.
Not all structural incogruities disappear because our regular expressions
are only \emph{pseudo-canonical} (no true canonical form for regular
expressions exists~\cite{?}).  For instance, our pseudo-normal form
does not automatically unify lenses such as
$A B* \Leftrightarrow A (\epsilon + BB*)$ that result from an
unrolling of Kleene Star, for instance.
Consequently, we structure our synthesis algorithm in two parts:  The
first is a syntax-directed search for a lens based on the structure of the given
regular expression types and, if that fails, the second is a search,
using Conway's axioms~\cite{conway} for 
equivalent regular expression types that might allow a new search
in the first phase of the search to succeed.  

One other complication that appears in our setting but not elsewhere in
the literature on type-directed synthesis, is that the types 
describing data formats may be very large --- they are regular
grammars describing complex data formats.  On the one hand, this is
very useful as such types provide a wealth of information that constrains
the search process effectively.  On the other hand, there is a lot of
data here that must be managed carefully by the synthesis system.
One important observation is that users naturally construct
such types compositionally, using named abbreviations for subcomponents,
and the structure inherent in these named abbreviations can be 
exploited by the synthesis algorithm to speed up the search dramatically.

For example, if the name $A$ appears once on both the left and the right,
the system will hypothesize that the identity lens can be used to convert
between occurrences of $A$.  On the other hand, if the name $A$ appears
on the left, but not on the right, the system will hypothesize that the
name $A$ should be replaced by its definition.  Likewise, differing
numbers of occurrences of names can serve as a guide for unrolling
Kleene star or applying other regular expression equivalence axioms.
Hence, in general, to guide the transformation of regular expression
types, we define a metric based on the occurrence of names.

\paragraph*{Contributions.}  Overall, we have designed and implemented
a new system for the synthesis of bijective string lenses.
In addition, this paper makes the following contributions:

\begin{itemize}
\item It defines a new language of lenses designed specifically for synthesis
with types based on pseudo-canonical regular expressions, with a term 
language to match.
\item We formalize the semantics of the language and prove it sound with
respect to an existing, standard language for bijective string lenses.
\item We designed a basic, new, type-directed, and two-dimensional 
algorithm for synthesizing lenses in our new domain-specific language.
\item We developed a collection of optimizations that speed up the basic
synthesis algorithm.
\item We have implemented the language and evaluated it on a range of
over 40 benchmarks including micro-benchmarks, a few examples from FlashFill,
and a number of examples from the Augeas system for transforming linux
file formats.  Our synthesis algorithm succeeds on 36 out of 42 of our
benchmarks.  When it succeeds, synthesis time is typically less than 1 second.
\end{itemize}


%% This rule gives the lens $\ConcatLens{\Lens_1}{\Lens_2}$ the type
%% $\Regex_1\Regex_2 \Leftrightarrow \RegexAlt_1\RegexAlt_2$, meaning it
%% converts between strings in the language of $\Regex_1\Regex_2$ and
%% strings in the language of $\RegexAlt_1\RegexAlt_2$.  In order for the
%% lens to be well-typed,  $\Regex_1$ and $\Regex_2$ must be
%% unambiguously concatenable (written \UnambigConcatOf{\LanguageOf{\Regex_1}}{\LanguageOf{\Regex_2}}), meaning there must be exactly 1 way to divide up any
%% string $\String$ in the language of $\Regex_1\Regex_2$ into $\String_1$
%% and $\String_2$ such that





%%% Local Variables:
%%% TeX-master: "main"
%%% End: