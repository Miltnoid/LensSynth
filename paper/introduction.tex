\section{Introduction}


\begin{enumerate}
\item Search through all possible lenses by: searching for good regular
expressions, then searching within those regular expressions for type directed
lenses
\item Putting the regular expressions in a normal form reduces the number of
regular expressions we must search through
\item Using user defined data types allows the synthesis algorithm to keep the
internals of regular expressions abstracted, unless the internals of those
regular expressions are changed, and the abstraction must be broken.
\item Using a pseudometric on regular expressions, based on the
distribution of user defined data types and size of the regular expression,
allows for an efficient search through equivalent regular expressions
\item We use a data structure combining examples with regular expressions to
synthesize only candidate programs which obey the examples
\item We develop an efficient strategy for finding if two
regular expressions have a lens between them requiring no rewrites which amounts
to merely ordering the components.
\end{enumerate}


In the era of data, big and small, tools for reliably parsing,
printing, cleaning, and transforming data are increasing important.
One class of such tools are \emph{lenses}~\cite{?}, bi-directional
programs that help users solve the classic ``view update problem.''
More specifically, a lens provides a user with two functions, a
\emph{get} function and a \emph{put} function.  The \emph{get}
function translates native data into a new format, or \emph{view}, for
the user.  If the user updates the new view, the corresponding
\emph{put} function can translate the edited data back into
the native format.

One concrete example of where lenses provide value is in managing
linux configuration files.  Such files come in a wide variety of
different ``ad hoc'' data formats, and even experienced system
administrators have a hard time keeping track of all the format
variations.  Moreover, a missing comma, forgotten keyword or 
misplaced vertical bar can cause the systems that depend on such
configurations to crash or misbehave.  In order to manage such configuration
files more reliably,  RedHat developed Augeas~\cite{}, a configuration
editing system based around lenses.  At the moment, Augeas contains more
than 200 stock lenses to provide views of common configuration files.

Lenses are popular because they typically come with strong guarantees
about the get-put (or put-get) round-trip behavior.  Such guarantees
help users guard against data corruption while reading and editing
high-value data sources.



