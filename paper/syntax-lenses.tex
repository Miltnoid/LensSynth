
\section{Syntax Driven Lenses}
Unfortunately, with retyping rule, it is a difficult task.
There are an infinite number of possible retypings.
They are still relevant, depending on what the examples are.
Retyping is needed to express all lenses.

For example, consider $r_1=a*$, $r_2=b*$, with examples $a \leftrightarrow bb, aa \leftrightarrow b, aaa \leftrightarrow bbb$.
Without retyping, the only possible lens is $iterate(const(a,b))$, which doesn't express it correctly.
However, using retyping, we can first retype to $\epsilon + a + aa + aaaa*$ and $\epsilon + b + bb + bbbb*$.
This is able to be solved.

Because of the difficulty for lenses including retyping, lets first only worry about syntactic lenses, lenses that don't include retyping.
\begin{figure}
\centering
\begin{mathpar}
\inferrule[Constant Lens]
{
\String_1 \in \Sigma^*\\
\String_2 \in \Sigma^*\\
}
{
const(\String_1,\String_2) : \String_1 \Leftrightarrow \String_2
}

\inferrule[Identity Lens]
{
}
{
\IdentityLens : \Regex \Leftrightarrow \Regex 
}

\inferrule[Iterate Lens]
{
\Lens : \Regex_1 \Leftrightarrow \Regex_2\\
\Regex_1^{!*}\\
\Regex_2^{!*}\\
}
{
\IterateLens{\Lens} : \Star{\Regex_1} \Leftrightarrow \Star{\Regex_2}
}

\inferrule[Concat Lens]
{
\Lens_1 : \Regex_{1,1} \Leftrightarrow \Regex_{1,2}\\
\Lens_2 : \Regex_{2,1} \Leftrightarrow \Regex_{2,2}\\
\Regex_{1,1} .! \Regex_{2,1}\\
\Regex_{1,2} .! \Regex_{2,2}\\
}
{
\ConcatLens{\String_1}{\String_2} : \Regex_{1,1}\Regex_{2,1} \Leftrightarrow \Regex_{1,2}\Regex_{2,2}
}

\inferrule[Swap Lens]
{
\Lens_1 : \Regex_{1,1} \Leftrightarrow \Regex_{1,2} \\
\Lens_2 : \Regex_{2,1} \Leftrightarrow \Regex_{2,2} \\
\Regex_{1,1} .! \Regex_{2,1}\\
\Regex_{2,2} .! \Regex_{1,2}\\
}
{
\ConcatLens{\String_1}{\String_2} : \Regex_{1,1}\Regex_{2,1} \Leftrightarrow \Regex_{2,2}\Regex_{1,2}
}

\inferrule[Or Lens]
{
\Lens_1 : \Regex_{1,1} \Leftrightarrow \Regex_{1,2}\\
\Lens_2 : \Regex_{2,1} \Leftrightarrow \Regex_{2,2}\\
\Regex_{1,1} \cap \Regex_{2,1} = \emptyset\\
\Regex_{1,2} \cap \Regex_{2,2} = \emptyset\\
}
{
\OrLens{\Lens_1}{\Lens_2} : \Regex_{1,1} | \Regex_{2,1} \Leftrightarrow \Regex_{1,2} | \Regex_{2,2}
}

\inferrule[Compose Lens]
{
\Lens_1 : \Regex_1 \Leftrightarrow \Regex_2 \\
\Lens_2 : \Regex_2 \Leftrightarrow \Regex_3 \\
}
{
\ComposeLens{\Lens_2}{\Lens_1} : \Regex_1 \Leftrightarrow \Regex_3
}
\end{mathpar}
\caption{Syntactic Lens Typing}
\label{fig:syntactic-lens-typing}
\end{figure}


For these syntactic lenses, one can prove that composition and identity can
be applied only on Userdefs without losing expressibility.

We now have the means to enumerate all possible syntactic lenses between two regular
expressions in a terminating way.  We merely enumerate all the possible
terms of a certain type, which there are a finite number of.  Then we can
run all the lenses on the inputs until we find a lens that satisfies the inputs.

However, this is inefficient, as the number of lenses are possibly exponential in the size of the regular expression.  We can prune some of the possible
lenses very quickly with a key insight, when we reduce the synthesis of
the lens into subproblems, we can also make sub-problems for the input-output examples.

For example, consider the problem of synthesizing a lens $l$ between $\Star{a}\Star{b}$ and $\Star{z}\Star{y}$ that satisfies the example $a \leftrightarrow y$.
We can apply either a swap or a concat rule to this.
If we apply a concat rule, we get the subproblems of synthesizing lenses between
$\Star{a}$ and $\Star{z}$, and synthesizing lenses betwen $\Star{b}$ and $\Star{y}$.
By the underlying function of concat, we know that for this concat lens to work on the provided example,
we must have the lens between $\Star{a}$ and $\Star{z}$ send $a$ to $\epsilon$,
and must have the lens between $\Star{b}$ and $\Star{y}$ send $\epsilon$ to $y$.
However, when trying to apply an iterate rule for $\Star{a}$ and $\Star{z}$, we immediately see there is no satisfying lens,
as parsing $a$ into $\Star{a}$, we see $a$ mathes the inner regular expression $a$ once.
However, on parsing $\epsilon$ into $\Star{z}$, we see $\epsilon$ matches the inner regular expression $z$ zero times.
Through the underlying function of iterate, we know this cannot be done,
so there is no lens that does this.
So we immediately know that there is no concat lens between
$\Star{a}\Star{b}$ and $\Star{z}\Star{y}$ that sends $a$ to $y$.
We can then continue with trying the swap lens, which will eventually provide a lens.


\begin{figure}
\centering
\begin{tabular}{l@{\ }l@{\ }c@{\ }l}

% REGEX
(Parse Tree)& \ParseTree{} & \GEq{} & $s \in \Sigma^*$ \\
& & & \GBar{} $u \in \Delta,s\in \Sigma^*$\\
& & & \GBar{} $\StarParse{[\ParseTree_1;\ldots;\ParseTree_n]}$ \\
& & & \GBar{} $\ParseTree_1 \ParseTree_2$ \\
& & & \GBar{} $\LeftChoiceParse{\ParseTree}$ \\
& & & \GBar{} $\RightChoiceParse{\ParseTree}$ \\
\end{tabular}
\caption{Parse Tree Definition}
\label{fig:parsetree-definition}
\end{figure}

\begin{figure}
\centering

\begin{mathpar}
\inferrule[Constant Parse Tree]
{
\String \in \Sigma^*
}
{
\Delta \vdash \String : \String
}

\inferrule[Userdef Parse Tree]
{
\Delta' \vdash \ParseTree : (\Regex, \String)
}
{
\Delta' \cup \{(\Regex,\UserDefined)\} \vdash \String : (\UserDefined, \String)
}

\inferrule[Concat Parse Trees]
{
\Delta \vdash \ParseTree_1 : (\Regex_1,\String_1) \\
\Delta \vdash \ParseTree_2 : (\Regex_2,\String_2) \\
}
{
\Delta \vdash \ParseTree_1 \ParseTree_2 : (\Regex_1 \Regex_2,\String_1 \String_2)
}

\inferrule[Left Or Example]
{
\Delta \vdash \ParseTree : (\Regex,\String)\\
}
{
\Delta \vdash \LeftChoiceParse{\ParseTree} : (\Regex | \Regex',\String)\\
}

\inferrule[Right Or Example]
{
\Delta \vdash \ParseTree : (\Regex,\String)\\
}
{
\Delta \vdash \RightChoiceParse{\ParseTree} : (\Regex' | \Regex,\String)\\
}

\inferrule[Star Empty]
{
}
{
\Delta \vdash \StarParse{[]} : (\Star{\Regex},\epsilon)
}

\inferrule[Star Add Parse]
{
\Delta \vdash \ParseTree : (\Regex,\String_1)\\
\Delta \vdash \StarParse{\ParseTrees} : (\Star{\Regex},\String_2)\\
}
{
\Delta \vdash \StarParse{\ParseTree::\ParseTrees} : (\Star{\Regex},\String_1\String_2)
}
\end{mathpar}
\caption{Parse Tree Typing}
\label{fig:parsetree-typing}
\end{figure}


We can formalize this first by definining a data structure for how a regular
expression parses a string, as in Figure~\ref{fig:parsetree-definition},
and typing for that data structure as in Figure~\ref{fig:parsetree-typing}.

\begin{figure}
\centering
\begin{mathpar}
\inferrule[Constant Lens]
{
\String_1 \in \Sigma^*\\
\String_2 \in \Sigma^*\\
}
{
const(\String_1,\String_2) : \String_1 \Leftrightarrow \String_2, \String_1 \leftrightarrow \String_2
}

\inferrule[Identity Lens]
{
\Delta' \vdash \ParseTree : (\Regex, \String)
}
{
\Delta' \cup \{(\Regex,\UserDefined)\} \vdash \IdentityLens : \UserDefined \Leftrightarrow \UserDefined, \String \leftrightarrow \String
}

\inferrule[Concat Lens]
{
\Delta \vdash \Lens_1 : \Regex_1 \Leftrightarrow \RegexAlt_1, \ParseTree_1 \leftrightarrow \ParseTreeAlt_1\\
\Delta \vdash \Lens_2 : \Regex_2 \Leftrightarrow \RegexAlt_2, \ParseTree_2 \leftrightarrow \ParseTreeAlt_2\\
\Regex_1 .! \Regex_2\\
\RegexAlt_1 .! \RegexAlt_2\\
}
{
\ConcatLens{\Lens_1}{\Lens_2} : \Regex_1\Regex_2 \Leftrightarrow \RegexAlt_1\RegexAlt_2, \ParseTree_1\ParseTree_2 \leftrightarrow \ParseTreeAlt_1\ParseTreeAlt_2
}

\inferrule[Swap Lens]
{
\Delta \vdash \Lens_1 : \Regex_1 \Leftrightarrow \RegexAlt_1, \ParseTree_1 \leftrightarrow \ParseTreeAlt_1\\
\Delta \vdash \Lens_2 : \Regex_2 \Leftrightarrow \RegexAlt_2, \ParseTree_2 \leftrightarrow \ParseTreeAlt_2\\
\Regex_1 .! \Regex_2\\
\RegexAlt_2 .! \RegexAlt_1\\
}
{
\SwapLens{\Lens_1}{\Lens_2} : \Regex_1\Regex_2 \Leftrightarrow \RegexAlt_2\RegexAlt_1, \ParseTree_1\ParseTree_2 \leftrightarrow \ParseTreeAlt_2\ParseTreeAlt_1
}

\inferrule[Or Lens Left]
{
\Delta \vdash \Lens_1 : \Regex_1 \Leftrightarrow \RegexAlt_1, \ParseTree \leftrightarrow \ParseTreeAlt\\
\Lens_2 : \Regex_2 \Leftrightarrow \RegexAlt_2\\
\Regex_1 \cap \RegexAlt_1 = \emptyset\\
\Regex_2 \cap \RegexAlt_2 = \emptyset\\
}
{
\Delta \vdash \OrLens{\Lens_1}{\Lens_2} : \Regex_{1,1} | \Regex_{2,1} \Leftrightarrow \Regex_{1,2} | \Regex_{2,2}, \LeftChoiceParse{\ParseTree} \leftrightarrow \LeftChoiceParse{\ParseTreeAlt}
}

\inferrule[Or Lens Right]
{
\Lens_1 : \Regex_1 \Leftrightarrow \RegexAlt_1\\
\Delta \vdash \Lens_2 : \Regex_2 \Leftrightarrow \RegexAlt_2, \ParseTree \leftrightarrow \ParseTreeAlt\\
\Regex_1 \cap \RegexAlt_1 = \emptyset\\
\Regex_2 \cap \RegexAlt_2 = \emptyset\\
}
{
\Delta \vdash \OrLens{\Lens_1}{\Lens_2} : \Regex_{1,1} | \Regex_{2,1} \Leftrightarrow \Regex_{1,2} | \Regex_{2,2}, \RightChoiceParse{\ParseTree} \leftrightarrow \RightChoiceParse{\ParseTreeAlt}
}

\inferrule[Iterate Lens Empty]
{
\Lens : \Regex \Leftrightarrow \RegexAlt\\
\Regex^{!*}\\
\RegexAlt^{!*}\\
}
{
\Delta \vdash \IterateLens{\Lens} : \Star{\Regex_1} \Leftrightarrow \Star{\Regex_2}, \StarParse{[]} \leftrightarrow \StarParse{[]}
}

\inferrule[Iterate Lens Add Example]
{
\Delta \vdash \IterateLens{\Lens} : \Star{\Regex} \Leftrightarrow \Star{\RegexAlt}, \StarParse{\ParseTrees} \leftrightarrow \StarParse{\ParseTreeAlts}\\
\Delta \vdash \Lens : \Regex \Leftrightarrow \RegexAlt, \ParseTree \leftrightarrow \ParseTreeAlt
}
{
\Delta \vdash \IterateLens{\Lens} : \Star{\Regex} \Leftrightarrow \Star{\RegexAlt}, \StarParse{\ParseTree::\ParseTrees} \leftrightarrow \StarParse{\ParseTreeAlt::\ParseTreeAlts}
}
\end{mathpar}
\caption{Syntactic Lens and Example Typing}
\label{fig:lens-parsetree-typing}
\end{figure}


With the data structure for parsing, we can now define a typing for a lens
which satisfies the parse tree of an example in Figure~\ref{fig:lens-parsetree-typing}.  Now, to generate a lens for an input output example,
between two regular expressions,
needs to generate the parse tree for the example with those expressions,
and then find a lens between those two regular expressions and between the parse trees.

We say a lens $l : \Regex \Leftrightarrow \RegexAlt$ satisfies a set of parse tree input output examples $\ParseTree_1 \leftrightarrow \ParseTreeAlt_1, \ldots, \ParseTree_n \leftrightarrow \ParseTreeAlt_n$
if we can show $l : \Regex \Leftrightarrow \RegexAlt, \ParseTree_i \leftrightarrow \ParseTreeAlt_i$ for all $i$ between $1$ and $n$.

\iffalse
We now can prove some inversion properties.
\begin{itemize}
\item $concat(\Lens_1,\Lens_2) : \Regex \Leftrightarrow \RegexAlt$ or
$swap(\Lens_1,\Lens_2) : \Regex \Leftrightarrow \RegexAlt$
if, and only if $\Regex = \Regex_1\Regex_2$, and $\RegexAlt = \RegexAlt_1\RegexAlt_2$ for some $\Regex_1$,$\Regex_2$
\end{itemize}
This provides us with a nice, syntax driven strategy for synthesis.
We now have an inversion property, where i

When does there exist a lens between two regular expressions.
More, formally, given two regular expressions $r$ and $s$, when does there exist a lens $l : r \Leftrightarrow s$.
We can say that $\SynSim{r}{s}$ if there is such an $l$.
We can prove $\SynSim{r}{s}$ if, and only if:
\begin{itemize}
\item If $r = r_1 r_2$ then either $s = s_1 s_2$ or $s = s_2 s_1$, and $\SynSim{r_1}{s_1}$ and $\SynSim{r_2}{s_2}$
\item If $r = r_1 | r_2$ then $s = s_1 | s_2$ and $\SynSim{r_1}{s_1}$ and $\SynSim{r_2}{s_2}$
\item If $r = \Star{r'}$ then $s = \Star{s'}$ and $\SynSim{r'}{s'}$.
\item If $r = x\in\Star{\Sigma}$, then $s = y\in\Star{\Sigma}$.
\item If $r = u\in\Delta$, then $s = u\in\Delta$
\end{itemize}

With this, it is easy to recursively check whether there exists a syntactic lens between two regular expressions.

How can we see if there exists a lens between two regular expressions, given a set of examples.
Intuitively, a lens matches a set of examples if it brings the parse trees of the left example to the parse trees of the right example.
So, for example, for an example $e\leftrightarrow f$ to be possible under $r_1|r_2 \Leftrightarrow s_1|s_2$, we need if $e$ matches $r_1$, $f$ must match $s_1$.
To formalize that, we can define parse trees as figure \ref{fig:parsetree-definition}.
Given two regular expressions $\Regex$, $\RegexAlt$, and a list of examples $\{e_1\leftrightarrow f_1,\ldots,e_n\leftrightarrow f_n\}$.
We define a relation between parse trees and regular expressions, $\SynSim{(\Regex_1,\ParseTrees)}{(\Regex_2,\ParseTreeAlts)}$ as follows:
\begin{itemize}
\item If $\Regex = \Regex_1 \Regex_2$,
$\RegexAlt = \RegexAlt_1 \RegexAlt_2$,
$\ParseTrees = [\ParseTree_{1,1}\ParseTree_{1,2}, \ldots, \ParseTree_{n,1}\ParseTree_{n,2}]$,
and $\ParseTreeAlts = [\ParseTreeAlt_{1,1}\ParseTreeAlt_{1,2}, \ldots, \ParseTreeAlt_{n,1}\ParseTreeAlt_{n,2}]$
and either\\
($\SynSim{(\Regex_1,[\ParseTree_{1,1}, \ldots,\ParseTree_{n,1}])}{(\RegexAlt_1,[\ParseTreeAlt_{1,1}, \ldots, \ParseTreeAlt_{n,1}])}$\\
and
$\SynSim{(\Regex_2,[\ParseTree_{1,2}, \ldots,\ParseTree_{n,2}])}{(\RegexAlt_2,[\ParseTreeAlt_{1,2}, \ldots, \ParseTreeAlt_{n,2}])}$)\\
or ($\SynSim{(\Regex_1,[\ParseTree_{1,1}, \ldots,\ParseTree_{n,1}])}{(\RegexAlt_2,[\ParseTreeAlt_{1,2}, \ldots, \ParseTreeAlt_{n,2}])}$\\
and
$\SynSim{(\Regex_2,[\ParseTree_{1,2}, \ldots,\ParseTree_{n,2}])}{(\RegexAlt_1,[\ParseTreeAlt_{1,1}, \ldots, \ParseTreeAlt_{n,1}]})$)
\item If $\Regex = \Regex_1 | \Regex_2$,
$\RegexAlt = \RegexAlt_1 \RegexAlt_2$,
$\ParseTrees = [\ParseTree_1;\ldots;\ParseTree_n]$, $\ParseTreeAlts = [\ParseTreeAlt_1;\ldots;\ParseTreeAlt_n]$,
($\ParseTree_i = \LeftChoiceParse{\ParseTree_i'}$ or $\ParseTree_i = \RightChoiceParse{\ParseTree_i'}$),
and if $\ParseTree_i = \LeftChoiceParse{\ParseTree_i'}$ then $\ParseTreeAlt_i = \LeftChoiceParse{\ParseTreeAlt_i'}$,
and if $\ParseTree_i = \RightChoiceParse{\ParseTree_i'}$ then $\ParseTreeAlt_i = \RightChoiceParse{\ParseTreeAlt_i'}$,
and (define $\ParseTrees_l$ = [$\ParseTree_i'$ for each $\ParseTree_i$ = $\LeftChoiceParse{\ParseTree_i'}$ in \ParseTrees],
define $\ParseTrees_r$ = [$\ParseTree_i'$ for each $\ParseTree_i$ = $\RightChoiceParse{\ParseTree_i'}$ in \ParseTrees],
define $\ParseTreeAlts_l$ = [$\ParseTreeAlt_i'$ for each $\ParseTreeAlt_i$ = $\LeftChoiceParse{\ParseTreeAlt_i'}$ in \ParseTreeAlts],
define $\ParseTreeAlts_r$ = [$\ParseTreeAlt_i'$ for each $\ParseTreeAlt_i = \RightChoiceParse{\ParseTreeAlt_i'}$ in \ParseTreeAlts]
)\SynSim{(\Regex_1,)}{a})
\end{itemize}
\ref{fig:syntactic-lens-typing}
\fi
