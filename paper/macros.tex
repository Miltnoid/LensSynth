% Colors
\definecolor{dkblue}{rgb}{0,0.1,0.5}
\definecolor{dkgreen}{rgb}{0,0.6,0}
\definecolor{dkred}{rgb}{0.6,0,0}
\definecolor{dkpurple}{rgb}{0.7,0,0.4}
\definecolor{olive}{rgb}{0.4, 0.4, 0.0}
\definecolor{teal}{rgb}{0.0,0.5,0.5}
\definecolor{orange}{rgb}{0.9,0.6,0.2}
\definecolor{lightyellow}{RGB}{255, 240, 180}
\definecolor{lightgreen}{RGB}{170, 255, 220}
\definecolor{darkbrown}{RGB}{121,37,0}

% renewtheorem https://tex.stackexchange.com/questions/103013/is-there-a-renewtheorem-equivalent-of-renewcommand-using-amsthm-and-not-ntheo
\makeatletter
\def\renewtheorem#1{%
  \expandafter\let\csname#1\endcsname\relax
  \expandafter\let\csname c@#1\endcsname\relax
  \gdef\renewtheorem@envname{#1}
  \renewtheorem@secpar
}
\def\renewtheorem@secpar{\@ifnextchar[{\renewtheorem@numberedlike}{\renewtheorem@nonumberedlike}}
\def\renewtheorem@numberedlike[#1]#2{\newtheorem{\renewtheorem@envname}[#1]{#2}}
\def\renewtheorem@nonumberedlike#1{  
\def\renewtheorem@caption{#1}
\edef\renewtheorem@nowithin{\noexpand\newtheorem{\renewtheorem@envname}{\renewtheorem@caption}}
\renewtheorem@thirdpar
}
\def\renewtheorem@thirdpar{\@ifnextchar[{\renewtheorem@within}{\renewtheorem@nowithin}}
\def\renewtheorem@within[#1]{\renewtheorem@nowithin[#1]}
\makeatother

\pgfplotsset{
% override style for non-boxed plots
    % which is the case for both sub-plots
    every non boxed x axis/.style={} 
}

\newcommand{\FINISH}[3]{\ifdraft\textcolor{#1}{[#2: #3]}\fi}
\newcommand{\bcp}[1]{\FINISH{dkred}{B}{#1}}
\newcommand{\BCP}[1]{\FINISH{dkred}{B}{\bf #1}}
\newcommand{\afm}[1]{\FINISH{dkgreen}{A}{#1}}
\newcommand{\dpw}[1]{\FINISH{dkblue}{D}{#1}}
\newcommand{\saz}[1]{\FINISH{orange}{S}{#1}}
\newcommand{\ksf}[1]{\FINISH{teal}{K}{#1}}

\newcommand{\IE}{\emph{i.e.}}
\newcommand{\EG}{\emph{e.g.}}
\newcommand{\ETC}{\emph{etc.}}

\let\olddefinition\definition
\renewcommand{\definition}{\olddefinition\normalfont}

\theoremstyle{definition}
\renewtheorem{theorem}{Theorem}
\renewtheorem{lemma}{Lemma}
\renewtheorem{corollary}{Corollary}
\renewtheorem{definition}{Definition}
\theoremstyle{plain}
\theoremstyle{remark}
\newtheorem{subcase}{Subcase}
\theoremstyle{remark}
\newtheorem{case}{Case}
\makeatletter
\@addtoreset{subcase}{case}
\@addtoreset{case}{lemma}
\@addtoreset{case}{theorem}
\@addtoreset{case}{corollary}
\@addtoreset{case}{definition}
\makeatother

\algnewcommand\algorithmicswitch{\textbf{switch}}
\algnewcommand\algorithmicmatch{\textbf{match}}
\algnewcommand\algorithmiccase{\textbf{case}}
\algnewcommand\algorithmicwith{\textbf{with}}
\algnewcommand\algorithmicforeach{\textbf{foreach}}
\algnewcommand\Assert[1]{\State \algorithmicassert(#1)}%
% New "environments"
\algdef{SE}[SWITCH]{Switch}{EndSwitch}[1]{\algorithmicmatch\ #1\ \algorithmicwith}{\algorithmicend\ \algorithmicswitch}%
\algdef{SE}[CASE]{Case}{EndCase}[1]{$|~$ #1 $\rightarrow$}{\algorithmicend\ \algorithmiccase}%
\algdef{SE}[FOREACH]{ForEach}{EndForEach}[2]{\algorithmicforeach\ #1 $\in$ #2}{\algorithmicend\ \algorithmicforeach}%
\algdef{SE}[CaseTwo]{CaseTwo}{EndCaseTwo}[2]{$|~$ #1 $\rightarrow$ #2}{\algorithmicend\ \algorithmiccase}%
\algtext*{EndSwitch}%
\algtext*{EndCase}%
\algtext*{EndCaseTwo}%
\algtext*{EndSecondCase}%
\algtext*{EndForEach}%



\newcommand{\CF}[1]{{\small \texttt{#1}}}         % Code Font
\newcommand{\SmallCF}[1]{{\footnotesize \texttt{#1}}}
\newcommand{\PCF}[1]{\textproc{#1}}
\newcommand{\VarCF}[1]{{\color{darkbrown} \CF{#1}}}
\newcommand{\StringCF}[1]{\CF{\textcolor{blue}{#1}}}
\newcommand{\Regex}{\ensuremath{\mathit{S}}}         % Regular Expression
\newcommand{\RegexType}{\ensuremath{\textit{Regex}}}
\newcommand{\EquivRegexType}{\ensuremath{\textit{Regex}/\sim}}
\newcommand{\BooleanAnd}{\ensuremath{\wedge}}
\newcommand{\BooleanOr}{\ensuremath{\vee}}
\newcommand{\BooleanImplies}{\ensuremath{\Rightarrow}}
\newcommand{\Rewrite}{\ensuremath{\rightarrow}}
\newcommand{\RewriteAtom}{\ensuremath{\Rewrite_\Atom}}
\newcommand{\RewriteDNF}{\ensuremath{\Rewrite_\DNFRegex}}
\newcommand{\ConcatDNF}{\ensuremath{\odot}}
\newcommand{\ConcatDNFOf}[2]{\ensuremath{#1\ConcatDNF#2}}
\newcommand{\BigConcatDNF}{\ensuremath{\bigodot}}
\newcommand{\ConcatSequence}{\ensuremath{\odot_{\Sequence}}}
\newcommand{\ConcatSequenceOf}[2]{\ensuremath{#1\ConcatSequence#2}}
\newcommand{\ConcatPermutation}{\ensuremath{\odot}}
\newcommand{\ConcatPermutationOf}[2]{\ensuremath{#1\ConcatPermutation#2}}
\newcommand{\SwapPermutation}{\ensuremath{\circledS}}
\newcommand{\SwapPermutationOf}[2]{\ensuremath{#1\SwapPermutation#2}}
\newcommand{\DistributePermutation}{\ensuremath{\otimes}}
\newcommand{\DistributePermutationOf}[2]{\ensuremath{#1\DistributePermutation#2}}
\newcommand{\DistributeSwapPermutation}{\ensuremath{\otimes^{\mathit{s}}}}
\newcommand{\DistributeSwapPermutationOf}[2]{\ensuremath{#1\DistributeSwapPermutation#2}}
\newcommand{\ConcatSequenceLens}{\ensuremath{\odot_{\SequenceLens}}}
\newcommand{\ConcatSequenceLensOf}[2]{\ensuremath{#1\ConcatSequenceLens#2}}
\newcommand{\ConcatDNFLens}{\ensuremath{\odot}}
\newcommand{\ConcatDNFLensOf}[2]{\ensuremath{#1\ConcatDNFLens#2}}
\newcommand{\SwapSequenceLens}{\ensuremath{\circledS_{\SequenceLens}}}
\newcommand{\SwapSequenceLensOf}[2]{\ensuremath{#1\SwapSequenceLens#2}}
\newcommand{\SwapDNFLens}{\ensuremath{\circledS}}
\newcommand{\SwapDNFLensOf}[2]{\ensuremath{#1\SwapDNFLens#2}}
\newcommand{\RepeatDNFOfTimes}[1]{\ensuremath{^{#1}}}
\newcommand{\RepeatDNFOf}[2]{\ensuremath{{#2}\RepeatDNFOfTimes{#1}}}
\newcommand{\RepeatDNFLensOfTimes}[1]{\ensuremath{^{#1}}}
\newcommand{\RepeatDNFLensOf}[2]{\ensuremath{{#2}\RepeatDNFLensOfTimes{#1}}}
\newcommand{\OrDNF}{\ensuremath{\oplus}}
\newcommand{\OrDNFOf}[2]{\ensuremath{#1\OrDNF#2}}
\newcommand{\OrDNFLens}{\ensuremath{\oplus}}
\newcommand{\OrDNFLensOf}[2]{\ensuremath{#1\OrDNFLens#2}}
\newcommand{\PutRight}{\ensuremath{\mathit{put}}}
\newcommand{\PutLeft}{\ensuremath{\mathit{get}}}
\newcommand{\PutRightOf}[1]{\ensuremath{#1.\PutRight}}
\newcommand{\PutLeftOf}[1]{\ensuremath{#1.\PutLeft}}
\newcommand{\RegexAlt}{\ensuremath{\mathit{T}}}         % Regular Expression
\newcommand{\Or}{\ensuremath{|}}
\newcommand{\RegexOr}[2]{\ensuremath{#1\Or#2}}
\newcommand{\SubN}{\textsubscript{n}}
\newcommand{\RegexConcat}[2]{\ensuremath{#1\cdot#2}}
\newcommand{\EmptyString}{\ensuremath{\epsilon}}
\newcommand{\StringConcat}[2]{\ensuremath{#1\cdot#2}}
\newcommand{\HasSemantics}{\ensuremath{\triangleright}}
\newcommand{\DerivesLens}{\ensuremath{\vdash}}
\newcommand{\DerivesDNFLens}{\ensuremath{\vdash_{\DNFLens}}}
\newcommand{\DerivesSequenceLens}{\ensuremath{\vdash_{\SequenceLens}}}
\newcommand{\DerivesAtomLens}{\ensuremath{\vdash_{\AtomLens}}}
\newcommand{\DerivesStringRegex}{\ensuremath{\vdash}}
\newcommand{\DerivesAtomRewrite}{\ensuremath{\vdash}}
\newcommand{\DerivesDNFRewrite}{\ensuremath{\vdash}}
\newcommand{\Concat}{\ensuremath{\cdot}}
\newcommand{\Union}{\ensuremath{\cup}}
\newcommand{\Intersect}{\ensuremath{\cap}}
\newcommand{\BigUnion}{\ensuremath{\bigcup}}
\newcommand{\BigIntersect}{\ensuremath{\bigcap}}
\newcommand{\denot}[1]{\ensuremath{[ \! [#1] \! ]}}
\newcommand{\SemanticsOf}[1]{\ensuremath{[ \! [#1] \! ]}}
\newcommand{\SetOf}[1]{\ensuremath{\{#1\}}}
\newcommand{\RegexVariable}{\ensuremath{\mathit{U}}}   % User Defined
\newcommand{\RegexVariableAlt}{\ensuremath{\mathit{V}}}
\newcommand{\LensVariable}{\ensuremath{\mathit{L}}}
\newcommand{\ExampledRegex}{\ensuremath{\mathit{er}}} % Exampled Regex
\newcommand{\UnambigItOf}[1]{\ensuremath{#1^{*!}}}
\newcommand{\UnambigConcat}{\ensuremath{\Concat^!}}
\newcommand{\SequenceUnambigConcatOf}[1]{\ensuremath{\UnambigConcat(#1)}}
\newcommand{\UnambigConcatOf}[2]{\ensuremath{#1 \UnambigConcat #2}}
\newcommand{\UnambigOrOf}[2]{\ensuremath{\LanguageOf{#1} \cap \LanguageOf{#2} = \emptyset}}
\newcommand{\Atom}{\ensuremath{\mathit{A}}}          % Atoms
\newcommand{\AtomAlt}{\ensuremath{\mathit{B}}}
\newcommand{\AtomType}{\ensuremath{\mathit{Atom}}}
\newcommand{\Apply}{\ensuremath{\,}}
\newcommand{\Sequence}{\ensuremath{\mathit{SQ}}}
\newcommand{\SequenceType}{\ensuremath{\mathit{Sequence}}}
\newcommand{\LetIn}[2]{\ensuremath{\text{let } #1 = #2\text{ in }}}
\newcommand{\LetWhereIn}[3]{\ensuremath{\text{let } #1 = #2 \text{ where } #3 \text{ in }}}
\newcommand{\Where}{\ensuremath{\text{ where }}}
\newcommand{\ClauseAlt}{\ensuremath{\mathit{bl}}}       % Clauses
\newcommand{\SequenceAlt}{\ensuremath{\mathit{TQ}}}
\newcommand{\DNFRegex}{\ensuremath{\mathit{DS}}}         % Regular Expression
\newcommand{\DNFRegexAlt}{\ensuremath{\mathit{DT}}}    %Alt Regex
\newcommand{\DNFRegexType}{\ensuremath{\mathit{DNF}}}
\newcommand{\LensContext}{\ensuremath{\Gamma}}
\newcommand{\RegexContext}{\ensuremath{\Delta}}  % Context
\newcommand{\FullContext}{\ensuremath{\Delta, \Gamma}}
\newcommand{\String}{\ensuremath{\mathit{s}}}        % String
\newcommand{\StringAlt}{\ensuremath{\mathit{t}}}        % StringAlt
\newcommand{\StringAltAlt}{\ensuremath{\mathit{t}}}        % StringAltAlt
\newcommand{\ExampleNumberList}{\ensuremath{\mathit{enl}}} %Example Number List
\newcommand{\ExampleNumberListList}{\ensuremath{\mathit{enll}}}
\newcommand{\ExampleStringList}{\ensuremath{\mathit{esl}}}
\newcommand{\StringList}{\ensuremath{\mathit{sl}}}
\newcommand{\Natural}{\ensuremath{\mathit{n}}}
\newcommand{\Interleaving}[1]{\ensuremath{\mathit{interleaving}(#1)}}
\newcommand{\Interleave}{\ensuremath{\mathit{interleave}}}
\newcommand{\BinaryInterleave}[2]{\ensuremath{\mathit{interleave}(#1,#2)}}
\newcommand{\NAryInterleave}[2]{\ensuremath{\mathit{interleave}(#1,\ldots,#2)}}
\newcommand{\Combine}{\ensuremath{\mathit{combine}}}
\newcommand{\List}{\ensuremath{\mathit{l}}}
\newcommand{\ValidCombine}[2]{\ensuremath{\mathit{validcombine}(#1,#2)}}
\newcommand{\ValidRegexContext}[2]{\ensuremath{\mathit{validregexcontext}(#1,#2)}}
\newcommand{\Parent}[1]{\ensuremath{\mathit{parent}(#1)}}
\newcommand{\Parented}[1]{\ensuremath{mathit{parented}(#1)}}
\newcommand{\CombineString}[1]{\ensuremath{\mathit{combine}_{\ExampleStringList}(#1)}}
\newcommand{\CombineList}[1]{\ensuremath{\mathit{combine}_{\ExampleNumberListList}(#1)}}
\newcommand{\Length}[1]{\ensuremath{\mathit{len}(#1)}}
\newcommand{\Language}{\ensuremath{L}}
\newcommand{\LanguageOf}[1]{\ensuremath{\mathcal{L}(#1)}}
\newcommand{\LanguageUnderContextOf}[2]{\ensuremath{\Language{}_{#1}(#2)}}
\newcommand{\ParseTree}{\ensuremath{\mathit{p}}}
\newcommand{\ParseTreeAlt}{\ensuremath{\mathit{q}}}
\newcommand{\ParseTrees}{\ensuremath{\mathit{ps}}}
\newcommand{\ParseTreeAlts}{\ensuremath{\mathit{qs}}}
\newcommand{\StarParse}[1]{\ensuremath{\mathit{starparse}(#1)}}
\newcommand{\LeftChoiceParse}[1]{\ensuremath{\mathit{l}.(#1)}}
\newcommand{\RightChoiceParse}[1]{\ensuremath{\mathit{r}.(#1)}}
\newcommand{\RangeExcInc}[2]{\ensuremath{(#1,#2]}}
\newcommand{\RangeIncInc}[2]{\ensuremath{[#1,#2]}}

\newcommand{\Lens}{\ensuremath{\mathit{l}}}
\newcommand{\AtomLens}{\ensuremath{\mathit{al}}}
\newcommand{\IterateAtomType}{\textit{Iterate}}
\newcommand{\ConcatedAtomsLens}{\ensuremath{\mathit{cal}}}
\newcommand{\OredClausesLens}{\ensuremath{\mathit{ocl}}}
\newcommand{\ClauseLens}{\ensuremath{\mathit{cll}}}
\newcommand{\SequenceLens}{\ensuremath{\mathit{sql}}}
\newcommand{\SequenceLensType}{\ensuremath{\mathit{SequenceLens}}}
\newcommand{\DNFLens}{\ensuremath{\mathit{dl}}}
\newcommand{\DNFLensType}{\ensuremath{\mathit{DNFLens}}}
\newcommand{\AtomLensType}{\ensuremath{\mathit{AtomLens}}}
\newcommand{\SynSim}[2]{\ensuremath{#1 \sim_{\mathit{sym}} #2}}
\newcommand{\ExdSynSim}[3]{\ensuremath{#2 \sim_{\mathit{sym},#1} #3}}

\newcommand{\PermutationSetOf}[1]{\ensuremath{S_{#1}}}
\newcommand{\Permutation}{\ensuremath{\sigma}}

\newcommand{\Star}{\ensuremath{^*}}
\newcommand{\StarOf}[1]{\ensuremath{{#1}\Star}}
\newcommand{\ConstLens}{\ensuremath{\mathit{const}}}
\newcommand{\ConstLensOf}[2]{\ensuremath{\ConstLens(#1,#2)}}
\newcommand{\ConcatLens}{\ensuremath{\mathit{concat}}}
\newcommand{\ConcatLensOf}[2]{\ensuremath{\ConcatLens(#1,#2)}}
\newcommand{\ConcatLensShortOf}[2]{\ensuremath{\mathit{c}(#1,#2)}}
\newcommand{\SwapLens}{\ensuremath{\mathit{swap}}}
\newcommand{\SwapLensOf}[2]{\ensuremath{\SwapLens(#1,#2)}}
\newcommand{\SwapLensShortOf}[2]{\ensuremath{\mathit{s}(#1,#2)}}
\newcommand{\OrLens}{\ensuremath{\mathit{or}}}
\newcommand{\OrLensOf}[2]{\ensuremath{\OrLens(#1,#2)}}
\newcommand{\IdentityLens}{\ensuremath{\mathit{id}}}
\newcommand{\IdentityLensOf}[1]{\ensuremath{\IdentityLens_{#1}}}
\newcommand{\IdentityLensShortT}{\ensuremath{\mathit{id}}}
\newcommand{\IdentityLensShortOf}[1]{\ensuremath{\IdentityLensShortT_{#1}}}
\newcommand{\IterateLens}{\ensuremath{\mathit{iterate}}}
\newcommand{\IterateLensOf}[1]{\ensuremath{\mathit{\IterateLens(#1)}}}
\newcommand{\Identity}{\ensuremath{\mathit{id}}}
\newcommand{\Compose}{\ensuremath{\circ}}
\newcommand{\ComposeLensOf}[2]{\ensuremath{#1\mathrel{;}#2}}

% GRAMMAR OPERATORS
\newcommand{\GBar}{\ensuremath{~|~}}
\newcommand{\GIndent}{\hspace{.5in}}
\newcommand{\GEq}{\ensuremath{::=~}}
\newcommand{\GEmp}{\ensuremath{\cdot}}
\newcommand{\Perm}{\ensuremath{\mathit{Perm}}}
\newcommand{\Nats}{\ensuremath{\mathbb{N}}}

\newcommand{\InverseOf}[1]{\ensuremath{#1^{-1}}}
\newcommand{\FloorOf}[1]{\ensuremath{\lfloor#1\rfloor}}
\newcommand{\CeilOf}[1]{\ensuremath{\lceil#1\rceil}}
\newcommand{\OfType}{\ensuremath{:}}
\newcommand{\OfRewritelessType}{\ensuremath{\,\tilde{\OfType}\,}}
\newcommand{\MapsBetweenTypeOf}[2]{\ensuremath{#1 \Leftrightarrow #2}}
\newcommand{\ArrowTypeOf}[2]{\ensuremath{#1 \rightarrow #2}}
\newcommand{\SizeOf}[1]{\ensuremath{|#1|}}

\newcommand{\ToDNFRegex}{\ensuremath{\Downarrow}}
\newcommand{\ToDNFRegexOf}[1]{\ensuremath{\ToDNFRegex\mkern-4mu #1}}
\newcommand{\ToRegex}{\ensuremath{\Uparrow}}
\newcommand{\ToRegexOf}[1]{\ensuremath{\ToRegex\mkern-4mu #1}}

\newcommand{\SuchThat}{\ensuremath{~|~}}

\newcommand{\LeftQuotientOf}[2]{\ensuremath{#1\backslash#2}}
\newcommand{\RightQuotientOf}[2]{\ensuremath{#1\slash#2}}

\newcommand{\SuffixOf}[1]{\ensuremath{S_{#1}}}
\newcommand{\PrefixOf}[1]{\ensuremath{P_{#1}}}

\newcommand{\ComplementOf}[1]{\ensuremath{\bar{#1}}}

\newcommand{\Alphabet}{\ensuremath{\Sigma}}
\newcommand{\Character}{\ensuremath{c}}
\newcommand{\CharacterAlt}{\ensuremath{d}}

\newcommand{\SequenceLeft}{\ensuremath{[}}
\newcommand{\SequenceRight}{\ensuremath{]}}
\newcommand{\SequenceOf}[1]{\ensuremath{\SequenceLeft#1\SequenceRight}}
\newcommand{\SequenceSep}{\ensuremath{;}}
\newcommand{\DNFLeft}{\ensuremath{\langle}}
\newcommand{\DNFRight}{\ensuremath{\rangle}}
\newcommand{\DNFOf}[1]{\ensuremath{\DNFLeft#1\DNFRight}}
\newcommand{\DNFSep}{\ensuremath{;}}
\newcommand{\SequenceLensLeft}{\ensuremath{[}}
\newcommand{\SequenceLensRight}{\ensuremath{]}}
\newcommand{\SequenceLensOf}[1]{\ensuremath{\SequenceLensLeft#1\SequenceLensRight}}
\newcommand{\SequenceLensSep}{\ensuremath{;}}
\newcommand{\DNFLensLeft}{\ensuremath{\langle}}
\newcommand{\DNFLensRight}{\ensuremath{\rangle}}
\newcommand{\DNFLensOf}[1]{\ensuremath{\DNFLensLeft#1\DNFLensRight}}
\newcommand{\DNFLensSep}{\ensuremath{;}}


\newcommand{\ConstantLensRule}{\textsc{Constant Lens}}
\newcommand{\IdentityLensRule}{\textsc{Identity Lens}}
\newcommand{\IterateLensRule}{\textsc{Iterate Lens}}
\newcommand{\ConcatLensRule}{\textsc{Concat Lens}}
\newcommand{\SwapLensRule}{\textsc{Swap Lens}}
\newcommand{\OrLensRule}{\textsc{Or Lens}}
\newcommand{\ComposeLensRule}{\textsc{Compose Lens}}
\newcommand{\RewriteRegexLensRule}{\textsc{Rewrite Regex Lens}}

\newcommand{\AtomUnrollstarLeftRule}{\textsc{Atom Unrollstar\SubLeft}}
\newcommand{\AtomUnrollstarRightRule}{\textsc{Atom Unrollstar\SubRight}}
\newcommand{\ParallelAtomStructuralRewriteRule}{\textsc{Parallel Atom Structural Rewrite}}
\newcommand{\ParallelSwapAtomStructuralRewriteRule}{\textsc{Parallel Swap Atom Structural Rewrite}}
\newcommand{\AtomStructuralRewriteRule}{\textsc{Atom Structural Rewrite}}
\newcommand{\DNFStructuralRewriteRule}{\textsc{DNF Structural Rewrite}}
\newcommand{\ParallelDNFStructuralRewriteRule}{\textsc{Parallel DNF Structural Rewrite}}
\newcommand{\ParallelSwapDNFStructuralRewriteRule}{\textsc{Parallel Swap DNF Structural Rewrite}}
\newcommand{\IdentityRewriteRule}{\textsc{Identity Rewrite}}
\newcommand{\DNFReorderRule}{\textsc{DNF Reorder}}

\newcommand{\SequenceLensRule}{\textsc{Sequence Lens}}
\newcommand{\AtomLensRule}{\textsc{Atom Lens}}
\newcommand{\DNFLensRule}{\textsc{DNF Lens}}
\newcommand{\RewriteDNFRegexLensRule}{\textsc{Rewrite DNF Regex Lens}}

\newcommand{\SubLeft}{\textsubscript{L}}
\newcommand{\SubRight}{\textsubscript{R}}

\newcommand{\Set}{\ensuremath{\mathit{S}}}

\newcommand{\OrIdentityRule}{\textit{+ Ident}}
\newcommand{\EmptyProjectionRightRule}{\textit{0 Proj\SubRight{}}}
\newcommand{\EmptyProjectionLeftRule}{\textit{0 Proj\SubLeft{}}}
\newcommand{\ConcatAssocRule}{\textit{\Concat{} Assoc}}
\newcommand{\OrAssociativityRule}{\textit{\Or{} Assoc}}
\newcommand{\OrCommutativityRule}{\textit{\Or{} Comm}}
\newcommand{\DistributivityLeftRule}{\textit{Dist\SubRight{}}}
\newcommand{\DistributivityRightRule}{\textit{Dist\SubLeft{}}}
\newcommand{\ConcatIdentityLeftRule}{\textit{\Concat{} Ident\SubLeft{}}}
\newcommand{\ConcatIdentityRightRule}{\textit{\Concat{} Ident\SubRight{}}}
\newcommand{\SumstarRule}{\textit{Sumstar}}
\newcommand{\ProductstarRule}{\textit{Prodstar}}
\newcommand{\UnrollstarLeftRule}{\textit{Unrollstar\SubLeft{}}}
\newcommand{\UnrollstarRightRule}{\textit{Unrollstar\SubRight{}}}
\newcommand{\StarstarRule}{\textit{Starstar}}
\newcommand{\DicyclicityRule}{\textit{Dicyc}}
\newcommand{\Derivation}{\ensuremath{\mathcal{D}}}

\newcolumntype{L}{>{$}l<{$}}
\newcolumntype{R}{>{$}r<{$}}

\renewcommand{\subsubsection}[1]{\paragraph{{#1}}}

\newcommand{\Examples}{\ensuremath{\mathit{exs}}}

\newcommand{\ParallelReduction}{\ensuremath{\rightarrow}}
\newcommand{\ParallelRewrite}{\ensuremath{\,\mathrlap{\to}\,{\scriptstyle\parallel}\,\,\,}}
\newcommand{\ParallelRewriteAtom}{\ensuremath{\ParallelRewrite_{\Atom}}}
\newcommand{\ParallelRewriteSwap}{\ensuremath{\ParallelRewrite^{\mathit{swap}}}}
\newcommand{\ParallelRewriteSwapAtom}{\ensuremath{\ParallelRewrite^{\mathit{swap}}_{\Atom}}}

\newcommand{\Property}{\ensuremath{\mathit{p}}}
\newcommand{\Propagator}{\ensuremath{\mathit{q}}}

\newcommand{\Relation}{\ensuremath{\mathit{R}}}
\newcommand{\RelationSet}{\ensuremath{\mathit{RS}}}

\newcommand{\DiamondProperty}{\ensuremath{\mathit{confluent}}}
\newcommand{\DiamondPropertyWithPropertyOf}[1]{\ensuremath{\DiamondProperty_{#1}}}
\newcommand{\IsConfluentWithPropertyOf}[2]
    {\ensuremath{\DiamondPropertyWithPropertyOf{#2}(#1)}}
\newcommand{\BisimilarProperty}{\ensuremath{\mathit{bisimilar}}}
\newcommand{\BisimilarPropertyWithPropertyOf}[1]{\ensuremath{\BisimilarProperty_{#1}}}
\newcommand{\IsBisimilarWithPropertyOf}[2]
    {\ensuremath{\BisimilarPropertyWithPropertyOf{#2}(#1)}}

\newcommand{\Reduces}{\ensuremath{\rightarrow}}

\newcommand{\SortaEquiv}{\ensuremath{\equiv_{sorta}}}

\newcommand{\AtomEquiv}{\ensuremath{\equiv_{\Atom}}}

\newcommand{\Sep}{\ensuremath{\$}}

\newcommand{\Cross}{\ensuremath{\times}}

\newcommand{\Distance}{\ensuremath{\mathit{d}}}

\newcommand{\AbsOf}[1]{\ensuremath{|#1|}}
\newcommand{\Size}{\ensuremath{\mathit{size}}}
\newcommand{\Module}{\ensuremath{\mathit{M}}}
\newcommand{\VectorSpace}{\ensuremath{\mathit{V}}}

\newcommand{\GetDist}{\ensuremath{\mathit{dist}}}
\newcommand{\LOneNorm}{\ensuremath{\ell_1}}

\newcommand{\Sorting}{\ensuremath{\mathit{sorting}}}
\newcommand{\SortingOf}[2]{\ensuremath{\Sorting(#1,#2)}}

\newcommand{\Sort}{\ensuremath{\mathit{sort}}}
\newcommand{\SortOf}[2]{\ensuremath{\Sort(#1,#2)}}

\newcommand{\ListType}{\ensuremath{\mathit{List}}}
\newcommand{\ListTypeOf}[1]{\ensuremath{#1\,\ListType}}
\newcommand{\ListLeft}{\ensuremath{[}}
\newcommand{\ListRight}{\ensuremath{]}}
\newcommand{\ListOf}[1]{\ensuremath{\ListLeft #1 \ListRight}}

\newcommand{\DNFLeq}{\ensuremath{\leq_{DNF}}}
\newcommand{\SequenceLeq}{\ensuremath{\leq_{Seq}}}
\newcommand{\AtomLeq}{\ensuremath{\leq_{Atom}}}
\newcommand{\DNFEq}{\ensuremath{=_{DNF}}}
\newcommand{\SequenceEq}{\ensuremath{=_{Seq}}}
\newcommand{\AtomEq}{\ensuremath{=_{Atom}}}

\newcommand{\NormalizedDNFOf}[1]{\ensuremath{\DNFOf{#1}_n}}
\newcommand{\NormalizedSequenceOf}[1]{\ensuremath{\SequenceOf{#1}_n}}
\newcommand{\NormalizedStarOf}[1]{\ensuremath{\NormalizedStarOf{#1}_n}}

\newcommand{\AtomNormalizer}{\ensuremath{\mathit{AN}}}
\newcommand{\AtomNormalizerType}{\textit{Atom Normalizer}}
\newcommand{\SequenceNormalizer}{\ensuremath{\mathit{SNN}}}
\newcommand{\SequenceNormalizerType}{\textit{Sequence Normalizer}}
\newcommand{\DNFRegexNormalizer}{\ensuremath{\mathit{DNFN}}}
\newcommand{\DNFRegexNormalizerType}{\textit{DNF Normalizer}}

\newcommand{\Normalize}{\ensuremath{\mathcal{N}}}
\newcommand{\NormalizeOf}[1]{\ensuremath{\Normalize(#1)}}

\newcommand{\DNFLensSynth}{\ensuremath{\mathit{DNFLensSynth}}}
\newcommand{\SequenceLensSynth}{\ensuremath{\mathit{SequenceLensSynth}}}
\newcommand{\AtomLensSynth}{\ensuremath{\mathit{AtomLensSynth}}}
\newcommand{\DNFLensSynthOf}[2]{\ensuremath{\DNFLensSynth(#1,#2)}}
\newcommand{\SequenceLensSynthOf}[2]{\ensuremath{\SequenceLensSynth(#1,#2)}}
\newcommand{\AtomLensSynthOf}[2]{\ensuremath{\AtomLensSynth(#1,#2)}}

\newcommand{\DNFLensHasSemanticsOf}[1]{\ensuremath{\xLeftrightarrow{#1}}}
\newcommand{\SatisfiesDNFLensHasSemanticsOf}[3]{\ensuremath{#2\DNFLensHasSemanticsOf{#1}#3}}
\newcommand{\SatisfiesIdentitySemantics}[2]
  {\ensuremath{\SatisfiesDNFLensHasSemanticsOf{\Identity}{#1}{#2}}}
\newcommand{\EquivalenceOf}[1]{\equiv_{#1}}

\newcommand{\SSREquiv}{\ensuremath{\equiv^s}}

\newcommand{\ReflexivityRule}{\textsc{Reflexivity}}
\newcommand{\BaseRule}{\textsc{Base}}
\newcommand{\SymmetryRule}{\textsc{Symmetry}}
\newcommand{\TransitivityRule}{\textsc{Transitivity}}

\newcommand{\BaseRegexType}{\textit{Base}}
\newcommand{\EmptyRegexType}{\textit{Empty}}
\newcommand{\StarRegexType}{\textit{Star}}
\newcommand{\ConcatRegexType}{\textit{Concat}}
\newcommand{\OrRegexType}{\textit{Or}}

\newcommand{\ConstLensType}{\textit{Const}}
\newcommand{\ConcatLensType}{\textit{Concat}}
\newcommand{\IterateLensType}{\textit{Iterate}}
\newcommand{\SwapLensType}{\textit{Swap}}
\newcommand{\OrLensType}{\textit{Or}}
\newcommand{\ComposeLensType}{\textit{Compose}}
\newcommand{\IdentityLensType}{\textit{Identity}}


\newcommand{\StarAtomType}{\textit{Star}}
\newcommand{\MultiConcatSequenceType}{\textit{MultiConcat}}
\newcommand{\MultiOrDNFRegexType}{\textit{MultiOr}}

\newcommand{\AtomToDNF}{\ensuremath{\mathcal{D}}}
\newcommand{\AtomToDNFOf}[1]{\ensuremath{\AtomToDNF(#1)}}
\newcommand{\AtomToDNFLens}{\ensuremath{\mathcal{D}}}
\newcommand{\AtomToDNFLensOf}[1]{\ensuremath{\AtomToDNFLens(#1)}}

\newcommand{\Queue}{\ensuremath{\mathit{Q}}}
\newcommand{\QueueElement}{\ensuremath{\mathit{qe}}}
\newcommand{\QueueElements}{\ensuremath{\QueueElement\mathit{s}}}
\newcommand{\ExpCount}{\ensuremath{\mathit{e}}}
\newcommand{\True}{\ensuremath{\mathit{true}}}
\newcommand{\False}{\ensuremath{\mathit{false}}}
\newcommand{\Null}{\ensuremath{\mathit{null}}}
\newcommand{\DNFRegexs}{\ensuremath{\DNFRegex\mathit{s}}}
\newcommand{\Types}{\ensuremath{\textit{t}}}

\newcommand{\DictionaryOrderL}{\ensuremath{[}}
\newcommand{\DictionaryOrderR}{\ensuremath{]}}
\newcommand{\DictionaryOrderOf}[1]{\ensuremath{\DictionaryOrderL #1 \DictionaryOrderR}}

\newcommand{\SetOfListOrderL}{\ensuremath{\{}}
\newcommand{\SetOfListOrderR}{\ensuremath{\}}}
\newcommand{\SetOfListOrderOf}[1]{\ensuremath{\SetOfListOrderL #1 \SetOfListOrderR}}

\newcommand{\Int}{\ensuremath{i}}
\newcommand{\UserDef}{\ensuremath{U}}

\newcommand{\Optician}{\PCF{Optician}}
\newcommand{\SynthLens}{\PCF{SynthLens}}
\newcommand{\TypeProp}{\PCF{TypeProp}}
\newcommand{\SynthDNFLens}{\PCF{SynthDNFLens}}
\newcommand{\ToLens}{\ensuremath{\Uparrow}}
\newcommand{\ToLensOf}[1]{\ensuremath{\ToLens{}\mkern-4mu #1}}
\newcommand{\ToDNFRegexText}{\PCF{ToDNFRegex}}
\newcommand{\Beautify}{\PCF{Beautify}}
\newcommand{\RigidSynth}{\PCF{RigidSynth}}
\newcommand{\RigidSynthSequence}{\PCF{RigidSynthSeq}}
\newcommand{\RigidSynthAtom}{\PCF{RigidSynthAtom}}
\newcommand{\GetDNFNormalizer}{\PCF{GetDNFNormalizer}}
\newcommand{\CreatePQueue}{\PCF{CreatePQueue}}
\newcommand{\GetTransitiveSet}{\PCF{GetTransitiveSet}}
\newcommand{\GetCurrentSet}{\PCF{GetCurrentSet}}
\newcommand{\Pop}{\PCF{Pop}}
\newcommand{\ExpandOnce}{\PCF{ExpandOnce}}
\newcommand{\ExpandRequired}{\PCF{ExpandRequired}}
\newcommand{\FixProblemElts}{\PCF{FixProblemElts}}
\newcommand{\Expand}{\PCF{Expand}}
\newcommand{\ForceExpand}{\PCF{ForceExpand}}
\newcommand{\Reveal}{\PCF{Reveal}}
\newcommand{\Map}{\PCF{Map}}
\newcommand{\EnqueueMany}{\PCF{EnqueueMany}}
\newcommand{\ReturnVal}[1]{\ensuremath{\Return\,#1}}

\newcommand{\CurrentSet}{\ensuremath{\mathit{CS}}}
\newcommand{\TransitiveSet}{\ensuremath{\mathit{TS}}}

\newcommand{\StringType}{\ensuremath{\mathit{String}}}

\newcommand{\SUBSECTION}[1]{\iffull\subsection{#1}\else\paragraph*{#1}}

\newcommand{\None}{\ensuremath{\mathit{None}}}
\newcommand{\DNFLensOption}{\ensuremath{\DNFLens\mathit{o}}}
\newcommand{\Some}{\ensuremath{\mathit{Some}}}
\newcommand{\SomeOf}[1]{\ensuremath{\Some\,#1}}

\newcommand{\Success}{\ensuremath{\boldsymbol{\color{dkgreen}\checkmark}}}
\newcommand{\Failure}{\ensuremath{\boldsymbol{\color{dkred}\times}}}

\newcommand{\Append}{\ensuremath{+\!\!\!\!+\ }}
\newcommand{\ModernTitle}{\VarCF{modern\_\discretionary{}{}{}title}}
\newcommand{\LegacyTitle}{\VarCF{legacy\_\discretionary{}{}{}title}}
\newcommand{\TextChar}{\VarCF{text\_char}}


\lstset{basicstyle=\footnotesize\ttfamily,breaklines=true,
  moredelim=**[is][\color{blue}]{@}{@},
  moredelim=**[is][\color{darkbrown}]{!}{!},
  }
  
% \lstset{framextopmargin=50pt,frame=bottomline}

%%% Local Variables:
%%% TeX-master: "main"
%%% End:
