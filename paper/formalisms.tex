\section{Background}

\subsubsection{Languages and Alphabets}

\begin{definition}
We fix a finite set of symbols, called \textit{characters}, $\{\Character_1,\ldots,\Character_n\}$.
We call this fixed set an \textit{alphabet}, denoted \Alphabet{}.
\end{definition}

\begin{definition}
We call a sequence of characters $\Character_1\ldots\Character_n$ a
\textit{string}, denoted \String{}.
We denote the set of all strings $\StarOf{\Alphabet}$, and denote the string
with no characters \EmptyString{}.
\end{definition}

\begin{definition}
We call a subset of $\StarOf{\Alphabet}$ a \textit{language},
denoted \Language{}.
\end{definition}

\begin{definition}
If we have two strings $\Character=\Character_1\ldots\Character_n$,
and $\CharacterAlt=\CharacterAlt_1\ldots\CharacterAlt_m$,
then $\Character\Concat\CharacterAlt=\Character_1\ldots\Character_n\CharacterAlt_1\ldots\CharacterAlt_m$.
\end{definition}

\begin{definition}
We call two languages $\Language_1$, $\Language_2$ unambiguously concatenable,
denoted $\UnambigConcatOf{\Language_1}{\Language_2}$,
if, for all strings $\String_1,\StringAlt_1\in\Language_1$,
and $\String_2,\StringAlt_2\in\Language_2$, then
$\String_1\Concat\String_2=\StringAlt_1\Concat\StringAlt_2$
only if $\String_1=\StringAlt_1$ and $\String_2=\StringAlt_2$.
\end{definition}

\begin{definition}
We call a language $\Language$ unambigusously iterable,
denoted $\UnambigItOf{\Language{}}$,
if, for all $n\in\Nats$, for all strings $\String_1,\ldots,\String_n\in\Language$,
and for all strings $\StringAlt_1,\ldots,\StringAlt_n\in\Language$,
$\String_1\Concat\ldots\Concat\String_n=\StringAlt_1\Concat\ldots\Concat\StringAlt_n$
only if $\String_i=\StringAlt_i$ for all $i\in\RangeIncInc{1}{n}$.
\end{definition}

\subsubsection{Regular Expressions}

% fig:regex-syntax
\begin{figure}
\centering
\begin{tabular}{l@{\hspace*{5mm}}l@{\ }c@{\ }l@{\hspace*{5mm}}>{\itshape\/}l}

(Strings)& \String{},\StringAlt{} & \GEq{} & $\String\in\StarOf{\Sigma}$ \\
(Regexs)& \Regex{},\RegexAlt{} & \GEq{} & s & Base \\
& & & \GBar{} $\emptyset$ & Empty \\
& & & \GBar{} \Regex{}* & Star \\
& & & \GBar{} $\RegexConcat{\Regex_1}{\Regex_2}$ & Concat \\
& & & \GBar{} $\RegexOr{\Regex_1}{\Regex_2}$ & Or \\
\end{tabular}
\caption{Regular Expression Syntax}
\label{fig:regex-syntax}
\end{figure}

% fig:regex-semantics
\begin{figure}
\[
\begin{array}{lcl}
\LanguageOf{\String} &=& \{\String\}\\
\LanguageOf{\emptyset} &=& \{\}\\
\LanguageOf{\RegexConcat{\Regex_1}{\Regex_2}} &=&
\{\StringConcat{\String_1}{\String_2} \SuchThat
\String_1\in\LanguageOf{\Regex_1} \BooleanAnd \String_2\in\LanguageOf{\Regex_2}\}\\
\LanguageOf{\RegexOr{\Regex_1}{\Regex_2}} &=&
\{\String \SuchThat
\String\in\LanguageOf{\Regex_1} \BooleanOr \String\in\LanguageOf{\Regex_2}\}\\
\LanguageOf{\StarOf{\Regex}} &=&
\{\String_1\Concat\ldots\Concat\String_n \SuchThat
n\in\Nats \wedge \String_i\in\LanguageOf{\Regex}\}
\end{array}
\]
\caption{Regex Semantics}
\label{fig:regex-semantics}
\end{figure}


The syntax for regular expressions is given in Figure~\ref{fig:regex-syntax}.
These regular expressions have an underlying semantics to express languages,
formalized in Figure~\ref{fig:regex-semantics}.

% fig:regex-equivalence-rules
\begin{figure}
\newcounter{rowcount}
\setcounter{rowcount}{0}
\centering
\begin{tabular}{@{}r@{\hspace{1em}}c@{\hspace{1em}}l@{}r@{}}
\Regex{} & $\equiv$ & \Regex{} & \EqualityRule{}  \\
\RegexOr{\Regex}{\emptyset} & $\equiv$ & \Regex{} & \OrIdentityRule{} \\
$\RegexConcat{\Regex}{\emptyset}$ & $\equiv$ & $\emptyset$ & \EmptyProjectionRuleRightRule{} \\
$\RegexConcat{\emptyset}{\Regex}$ & $\equiv$ & $\emptyset$ & \EmptyProjectionRuleLeftRule{}\SubLeft{} \\
\RegexConcat{(\RegexConcat{\Regex{}}{\Regex'})}{\Regex''} & $\equiv$ & \RegexConcat{\Regex{}}{(\RegexConcat{\Regex'}{\Regex''})} & \ConcatAssocRule{}  \\
\RegexOr{(\RegexOr{\Regex}{\Regex'})}{\Regex''} & $\equiv$ & \RegexOr{\Regex}{(\RegexOr{\Regex'}{\Regex''})} & \OrAssociativityRule{}  \\
\RegexOr{\Regex{}}{\RegexAlt{}} & $\equiv$ & \RegexOr{\RegexAlt{}}{\Regex{}} & \OrCommutativityRule{}\\
\RegexConcat{\Regex{}}{(\RegexOr{\Regex{}'}{\Regex{}''})} & $\equiv$ & \RegexOr{(\RegexConcat{\Regex{}}{\Regex{}'})}{(\RegexConcat{\Regex{}}{\Regex{}''})} & \DistributivityLeftRule{} \\
\RegexConcat{(\RegexOr{\Regex{}'}{\Regex{}''})}{\Regex{}} & $\equiv$ & \RegexOr{(\RegexConcat{\Regex{}'}{\Regex{}})}{(\RegexConcat{\Regex{}''}{\Regex{}})} & \DistributivityRightRule{} \\
\RegexConcat{\Regex{}}{\EmptyString{}} & $\equiv$ & \Regex{} & \ConcatIdentityRule{} \\
\StarOf{(\RegexOr{\Regex{}}{\RegexAlt{}})} & $\equiv$ & \RegexConcat{\StarOf{(\RegexConcat{\StarOf{\Regex{}}}{\RegexAlt{}})}}{\StarOf{\Regex{}}} & \SumstarRule{}\\
\StarOf{(\RegexConcat{\Regex{}}{\RegexAlt{}})} & $\equiv$ & \RegexOr{\EmptyString{}}{(\RegexConcat{\RegexConcat{\Regex{}}{\StarOf{(\RegexConcat{\RegexAlt{}}{\Regex{}})}}}{\RegexAlt{}})} & \ProductstarRule{} \\
${(\Regex{}^*)}^*$ & $\equiv$ & \StarOf{\Regex{}} & \StarstarRule{} \\
\StarOf{(\RegexOr{\Regex}{\RegexAlt})} & $\equiv$ & $\StarOf{(\RegexConcat{(\RegexOr{\Regex}{\RegexAlt})}{\RegexOr{\RegexAlt}{\RegexConcat{{(\RegexConcat{\Regex}{\StarOf{\RegexAlt}})}^n}{\Regex}}})}\Concat$ & \DicyclicityRule{}\\
& & $(\EmptyString\Or(\RegexOr{\Regex}{\RegexAlt})\Concat$\\
& & $({(\RegexConcat{\Regex}{\StarOf{\RegexAlt}})}^0\Or\ldots\Or{(\RegexConcat{\Regex}{\StarOf{\RegexAlt}})}^n))$
\end{tabular}
\caption{Regular Expression Equivalences}
\label{fig:regex-equivalence-rules}
\end{figure}


We call two regular expressions equivalent, denoted $\Regex_1\equiv\Regex_2$,
if $\LanguageOf{\Regex_1}=\LanguageOf{\Regex_2}$.
There exists an equational theory for determining if two regular expressions are equivalent,
presented by Conway \cite{conway}
and proven complete by Krob \cite{Krob}\bcp{The citation for Conway reads
  funny.},
shown in Figure~\ref{fig:regex-equivalence-rules}.

\subsubsection{Bijective Lenses}

% fig:lens-syntax
\begin{figure}
\centering
\begin{tabular}{l@{\ }l@{\ }c@{\ }l@{\ }>{\itshape\/}r}
% REGEX
(Lenses)& \Lens{} & \GEq{} & $\ConstLens{s_1 \in \StarOf{\Alphabet}}{s_2 \in \StarOf{\Alphabet}}$ & Const \\
& & & \GBar{} $\IdentityLens$ & Identity\\
& & & \GBar{} $\IterateLens{\Lens}$ & Iterate \\
& & & \GBar{} $\ConcatLens{\Lens_1}{\Lens_2}$ & Concat \\
& & & \GBar{} $\SwapLens{\Lens_1}{\Lens_2}$ & Swap\\
& & & \GBar{} $\OrLens{\Lens_1}{\Lens_2}$ & Or\\
& & & \GBar{} $\ComposeLens{\Lens_1}{\Lens_2}$ & Compose\\
\end{tabular}
\caption{Lens Syntax}
\label{fig:lens-syntax}
\end{figure}

% fig:lens-semantics
\begin{figure}
\[
\begin{array}{rcl}
\SemanticsOf{const(\String_1,\String_2)} &=& \SetOf{(\String_1,\String_2)}\\

\SemanticsOf{\IdentityLens} &=& \SetOf{(\String,\String)}\\

\SemanticsOf{\IterateLens{\Lens}} &=& \SetOf{(\String_1\Concat\ldots\Concat\String_n,
\StringAlt_1\Concat\ldots\Concat\StringAlt_n)\SuchThat
(\String_i,\StringAlt_i)\in\SemanticsOf{\Lens}}\\

\SemanticsOf{\ConcatLens{\Lens_1}{\Lens_2}} &=&
\SetOf{(\String_1\Concat\String_2,\StringAlt_1\Concat\StringAlt_2)\SuchThat\\
& & \hspace*{2em}(\String_1,\StringAlt_1)\in\SemanticsOf{\Lens_1}\BooleanAnd
(\String_2,\StringAlt_2)\in\SemanticsOf{\Lens_2}}\\

\SemanticsOf{\SwapLens{\Lens_1}{\Lens_2}} &=&
\SetOf{(\String_1\Concat\String_2,\StringAlt_2\Concat\StringAlt_1)\SuchThat\\
& & \hspace*{2em}(\String_1,\StringAlt_1)\in\SemanticsOf{\Lens_1}\BooleanAnd
(\String_2,\StringAlt_2)\in\SemanticsOf{\Lens_2}}\\

\SemanticsOf{\OrLens{\Lens_1}{\Lens_2}} &=&
\SetOf{(\String,\StringAlt)
\SuchThat(\String,\StringAlt)\in\SemanticsOf{\Lens_1}
\BooleanOr(\String,\StringAlt)\in\SemanticsOf{\Lens_2}}\\

\SemanticsOf{\ComposeLens{\Lens_1}{\Lens_2}} &=&
\SetOf{(\String_1,\String_3)\SuchThat\exists\String_2\\
& & \hspace*{2em}(\String_1,\String_2)\in\SemanticsOf{\Lens_1}\BooleanAnd
(\String_2,\String_3)\in\SemanticsOf{\Lens_2}}
\end{array}
\]
\caption{Lens Semantics}
\label{fig:lens-semantics}
\end{figure}

% fig:lens-semantics
\begin{figure}
\centering
\begin{mathpar}
\inferrule[\ConstantLensRule{}]
{
\String_1 \in \StarOf{\Sigma}\\
\String_2 \in \StarOf{\Sigma}
}
{
\ConstLens{\String_1}{\String_2} \OfType \String_1 \Leftrightarrow \String_2
}

\inferrule[\IdentityLensRule{}]
{
}
{
\IdentityLens \OfType \Regex \Leftrightarrow \Regex
}

\inferrule[\IterateLensRule{}]
{
\Lens \OfType \Regex \Leftrightarrow \RegexAlt\\
\UnambigItOf{\LanguageOf{\Regex}}\\
\UnambigItOf{\LanguageOf{\RegexAlt}}
}
{
\IterateLens{\Lens} \OfType \StarOf{\Regex} \Leftrightarrow \StarOf{\RegexAlt}
}

\inferrule[\ConcatLensRule{}]
{
\Lens_1 \OfType \Regex_1 \Leftrightarrow \RegexAlt_1\\
\Lens_2 \OfType \Regex_2 \Leftrightarrow \RegexAlt_2\\
\UnambigConcatOf{\LanguageOf{\Regex_1}}{\LanguageOf{\Regex_2}}\\
\UnambigConcatOf{\LanguageOf{\RegexAlt_1}}{\LanguageOf{\RegexAlt_2}}
}
{
\ConcatLens{\Lens_1}{\Lens_2} \OfType \Regex_1\Regex_2 \Leftrightarrow \RegexAlt_1\RegexAlt_2
}

\inferrule[\SwapLensRule{}]
{
\Lens_1 \OfType \Regex_1 \Leftrightarrow \RegexAlt_1\\
\Lens_2 \OfType \Regex_2 \Leftrightarrow \RegexAlt_2\\
\UnambigConcatOf{\LanguageOf{\Regex_1}}{\LanguageOf{\Regex_2}}\\
\UnambigConcatOf{\LanguageOf{\RegexAlt_2}}{\LanguageOf{\RegexAlt_1}}
}
{
\SwapLens{\Lens_1}{\Lens_2} \OfType \Regex_1\Regex_2 \Leftrightarrow \RegexAlt_2\RegexAlt_1
}

\inferrule[\OrLensRule{}]
{
\Lens_1 \OfType \Regex_1 \Leftrightarrow \RegexAlt_1\\
\Lens_2 \OfType \Regex_2 \Leftrightarrow \RegexAlt_2\\
\UnambigOrOf{\LanguageOf{\Regex_1}}{\LanguageOf{\Regex_2}}\\ \UnambigOrOf{\LanguageOf{\RegexAlt_1}}{\LanguageOf{\RegexAlt_2}}
}
{
\OrLens{\Lens_1}{\Lens_2} \OfType \Regex_1 | \RegexAlt_1 \Leftrightarrow \Regex_2 | \RegexAlt_2
}

\inferrule[\ComposeLensRule{}]
{
\Lens_1 \OfType \Regex_1 \Leftrightarrow \Regex_2\\
\Lens_2 \OfType \Regex_2 \Leftrightarrow \Regex_3\\
}
{
\ComposeLens{\Lens_2}{\Lens_1} \OfType \Regex_1 \Leftrightarrow \Regex_3
}

\inferrule[\RetypeLensRule{}]
{
\Lens \OfType \Regex_1 \Leftrightarrow \Regex_2\\
\Regex_1 \equiv \Regex_1'\\
\Regex_2 \equiv \Regex_2'
}
{
\Lens \OfType \Regex_1' \Leftrightarrow \Regex_2'
}
\end{mathpar}

\caption{Lens Typing}
\label{fig:lens-semantics}
\end{figure}


We focus on bijections between these languages through the language of
bijective lenses.
We define the syntax for these bijective lenses in Figure~\ref{fig:lens-syntax}.
The semantics and typing for these lenses is defined in Figure~\ref{fig:lens-semantics}.
The semantics is only defined alongside the typing, as the functions are only
well defined when the term is well typed.

We would like to synthesize these lenses from types and examples.
Our approach to synthesis is one of enumeration.
We would like to enumerate the programs that satisfy the desired typing until
we find one that satisfies the examples.
We enumerate these programs by creating subproblems based on typing rules
could potentially create the desired type.

For example, given the type $\RegexConcat{\Regex_1}{\Regex_2}\Leftrightarrow\RegexConcat{\RegexAlt_1}{\RegexAlt_2}$.
We know that it can be created from a Concat Lens rule,
which would create the subproblems of finding a lens of type $\Regex_1\Leftrightarrow\RegexAlt_1$
and a lens of type $\Regex_2\Leftrightarrow\RegexAlt_2$.

\section{DNF Regular Expressions}

% fig:dnf-regex-syntax
\begin{figure}
\begin{tabular}{l@{\ }l@{\ }c@{\ }l@{\ }r}

% DNF_REGEX
(Atoms)& \Atom{},\AtomAlt{} & \GEq{} & \StarOf{\DNFRegex{}} & Iterate DNF\\
(Sequence)& \Sequence{},\SequenceAlt{} & \GEq{} &
$\SequenceOf{\String_0\SequenceSep\Atom_1\SequenceSep\ldots\SequenceSep\Atom_n\SequenceSep\String_n}$ & Sequence\\
(DNF Regex)& \DNFRegex{},\DNFRegexAlt{} & \GEq{} & $\DNFOf{\Sequence_1\DNFSep\ldots\DNFSep\Sequence_n}$ & DNF Or\\
\end{tabular}
\caption{DNF Regex Syntax\bcp{Or how about ``hollow square
  brackets'' for conjuncts (using some kind of square brackets seems good,
  since they are ordered) and ``hollow set braces'' for disjuncts?  We
  should try to get this right, since these widgets are our core technical
  device.}} 
\label{fig:dnf-regex-syntax}
\end{figure}

% fig:dnf-regex-semantics
\begin{figure}
\begin{itemize}
\item $\LanguageOf{\StarOf{\DNFRegex}} =
\{\String_1\Concat\ldots\Concat\String_n \SuchThat n\in\Nats \wedge \String_i\in\LanguageOf{\DNFRegex}\}$
\item $\LanguageOf{\SequenceOf{\String_0\SequenceSep\Atom_1\SequenceSep\ldots\SequenceSep\Atom_n\SequenceSep\String_n}}=
\{\String_0\Concat\StringAlt_1\Concat\ldots\Concat\StringAlt_n\Concat\String_n \SuchThat \StringAlt_i\in\LanguageOf{\Atom_i}\}$
\item $\LanguageOf{\DNFOf{\Sequence_1\DNFSep\ldots\DNFSep\Sequence_n}}=$\\
\hspace*{.5em}$\{\String \SuchThat \String \in \LanguageOf{\Sequence_i} \text{ for some $i\in\RangeIncInc{1}{n}$}\}$
\end{itemize}
\caption{DNF Regex Semantics}
\label{fig:dnf-regex-semantics}
\end{figure}


A DNF regular esxpression is, intuitively
a regular expression which is completely distributed
applied, with no associativity information through a list representation,
and with base regular expressions only at fixed locations.
We formalize the syntax of this language in Figure~\ref{fig:dnf-regex-syntax}.
We will use the shorthand $\DNFOf{\SequenceOf{\Atom_1\SequenceSep\ldots\SequenceSep\Atom_n}}$
for $\DNFOf{\SequenceOf{\EmptyString\SequenceSep\Atom_1\SequenceSep\ldots\SequenceSep\Atom_n\SequenceSep\EmptyString}}$.

The outermost layer is a list of sequences.
This layer intuitively corresponds to the choices involved in regular expression matching, and so represents where the Ors of normal regular expressions exist.
The second layer is a list of alternating strings and stars.
After the choice has been made about what will be expressed,
the base strings and iterated portions remains to be expressed.
We keep it in a normal form, by requiring a (possibly empty) string between
each clause.
The clause corresponds to the concatenated data, and is a concatenation of the
fixed data of base strings, and the iterated data of stars.
Finally is the atom, which is a star.
This corresponds to the iteration that takes place in normal stars.
There is no fixed way to break the choices of this into the clause,
as there is an arbitrarily large number of choices made in the stars.
This intuition is formalized by the semantics, given in
Figure~\ref{fig:dnf-regex-semantics}.

% fig:dnf-regex-functions
\begin{figure}
\ConcatSequence{} \OfType{} \ArrowTypeOf{\SequenceType{}}{\ArrowTypeOf{\SequenceType{}}{\SequenceType{}}}\\
$\ConcatSequenceOf{[\String_0\SequenceSep\Atom_1\SequenceSep\ldots\SequenceSep\Atom_n\SequenceSep\String_n]}{[\StringAlt_0\SequenceSep\AtomAlt_1\SequenceSep\ldots\SequenceSep\AtomAlt_m\SequenceSep\StringAlt_m]}=$\\
\hspace*{2ex}$[\String_0\SequenceSep\Atom_1\SequenceSep\ldots\SequenceSep\Atom_n\SequenceSep\String_n\Concat\StringAlt_0\SequenceSep\AtomAlt_1\SequenceSep\ldots\SequenceSep\AtomAlt_m\SequenceSep\StringAlt_m]$\\
\\
\ConcatDNF{} \OfType{} \ArrowTypeOf{\DNFRegexType{}}{\ArrowTypeOf{\DNFRegexType{}}{\DNFRegexType{}}}\\
$\ConcatDNFOf{\DNFOf{\Sequence_1\DNFSep\ldots\DNFSep\Sequence_n}}{\DNFOf{\SequenceAlt_1\DNFSep\ldots\DNFSep\SequenceAlt_m}}=$
\[
\begin{array}{rcccl}
\DNFLeft & \ConcatSequenceOf{\Sequence_1}{\SequenceAlt_1}\DNFSep & \cdots & \ConcatSequenceOf{\Sequence_1}{\SequenceAlt_m}\DNFSep \\
& & \cdots & \\
& \ConcatSequenceOf{\Sequence_n}{\SequenceAlt_1}\DNFSep & \cdots & \ConcatSequenceOf{\Sequence_n}{\SequenceAlt_m} & \DNFRight
\end{array}
\]
\\
\OrDNF{} \OfType{} \ArrowTypeOf{\DNFRegexType{}}{\DNFRegexType{}}\\
$\OrDNFOf{\DNFOf{\Sequence_1\DNFSep\ldots\DNFSep\Sequence_n}}{\DNFOf{\SequenceAlt_1\DNFSep\ldots\DNFSep\SequenceAlt_m}}$=\\
\hspace*{2ex}$\DNFOf{\Sequence_1\DNFSep\ldots\DNFSep\Sequence_n\DNFSep\SequenceAlt_1\DNFSep\ldots\DNFSep\SequenceAlt_m}$\\
\\
(-)\RepeatDNFOfTimes{n} \OfType{} \ArrowTypeOf{\DNFRegexType{}}{\DNFRegexType{}}\\
$\RepeatDNFOf{0}{\DNFRegex}=\DNFOf{\SequenceOf{\EmptyString}}$\\
$\RepeatDNFOf{n}{\DNFRegex}=\ConcatDNFOf{\DNFRegex}{\RepeatDNFOf{n-1}{\DNFRegex}}$\\
\caption{DNF Regex Functions} 
\label{fig:dnf-regex-functions}
\end{figure}


We can create a dnf regular expression from a regular expression through repeated
application of the distributivity rule, and through removal of association information.  To do this, we require some functions defined on DNF regular expressions,
which we define in Figure~\ref{fig:dnf-regex-functions}.
We can use these functions to obtain a completeness result.
From this we can define a function which converts a regular expression into
an equivalent DNF regular expression, \ToDNFRegex{}.  \bcp{No bullets.}
\begin{definition}
\leavevmode
\begin{itemize}
\item $\ToDNFRegex(\String)=\DNFOf{\SequenceOf{\String}}$
\item $\ToDNFRegex(\StarOf{(\Regex)}) = \DNFOf{\SequenceOf{(\StarOf{\ToDNFRegex(\Regex)})}}$
\item $\ToDNFRegex(\RegexConcat{\Regex_1}{\Regex_2}) =$\\
\hspace*{1em}$\ConcatDNFOf{\ToDNFRegex(\Regex_1)}{\ToDNFRegex(\Regex_2)}$
\item $\ToDNFRegex(\RegexOr{\Regex_1}{\Regex_2}) =$\\
\hspace*{1em}$\OrDNFOf{\ToDNFRegex(\Regex_1)}{\ToDNFRegex(\Regex_2)}$
\end{itemize}
\end{definition}
We show that this transformation is a valid transformation.
\begin{restatable}[Completeness of DNF Regexs]{theorem}{dnfrc}
\label{thm:completeness-dnf-lenses}
For all regular expressions \Regex{},
there exists a DNF regular expression \DNFRegex{},
such that \LanguageOf{\DNFRegex{}}=\LanguageOf{\Regex{}}.
\end{restatable}
First we will prove some lemmas.
\begin{lemma}[Equivalence of \ConcatSequence{} and \Concat{}]
If $\LanguageOf{\Regex}=\LanguageOf{\Sequence}$,
and $\LanguageOf{\RegexAlt}=\LanguageOf{\SequenceAlt}$,
then $\LanguageOf{\RegexConcat{\Regex}{\RegexAlt}}=\LanguageOf{\ConcatSequenceOf{\Sequence}{\SequenceAlt}}$.
\end{lemma}
\begin{proof}
Let $\Sequence=\SequenceOf{\String_0\SequenceSep\Atom_1\SequenceSep\ldots
\SequenceSep\Atom_n\SequenceSep\String_n}$, and
let\\ $\SequenceAlt=[\StringAlt_0\SequenceSep\AtomAlt_1\SequenceSep\ldots
\SequenceSep\AtomAlt_m\SequenceSep\StringAlt_m]$\\
\begin{tabular}{@{}L@{}L@{}}
\LanguageOf{\ConcatSequenceOf{\Sequence}{\SequenceAlt}} & = 
\LanguageOf{\SequenceOf{\String_0\SequenceSep\Atom_1\SequenceSep\ldots
\SequenceSep\Atom_n\SequenceSep\String_n\Concat\StringAlt_0\SequenceSep{}
\AtomAlt_1\SequenceSep\ldots\SequenceSep\AtomAlt_m\SequenceSep\StringAlt_m}} \\
& = 
\{\String_0\Concat\String_1'\Concat\ldots\Concat\String_n'\Concat\String_n
\Concat\StringAlt_0\Concat\StringAlt_1'\Concat\ldots
\Concat\StringAlt_m'\Concat\StringAlt_m \\
& \hspace{5em} \SuchThat{} \String_i'\in\LanguageOf{\Atom_i} \BooleanAnd{}
\StringAlt_i'\in\LanguageOf{\AtomAlt_i}\}\\
& = 
\{\String\Concat\StringAlt{} \SuchThat{} \String\in\LanguageOf{\Sequence}
\BooleanAnd{} \StringAlt\in\LanguageOf{\SequenceAlt}\}\\
& =
\{\String\Concat\StringAlt{} \SuchThat{} \String\in\LanguageOf{\Regex}
\BooleanAnd{} \StringAlt\in\LanguageOf{\RegexAlt}\}\\
& =
\LanguageOf{\RegexConcat{\Regex}{\RegexAlt}}
\end{tabular}
\end{proof}

\begin{lemma}[Equivalence of \ConcatDNF{} and \Concat{}]
\label{lem:cdnfeq}
If $\LanguageOf{\Regex}=\LanguageOf{\DNFRegex}$,
and $\LanguageOf{\RegexAlt}=\LanguageOf{\DNFRegexAlt}$,
then $\LanguageOf{\RegexConcat{\Regex}{\RegexAlt}}=
\LanguageOf{\ConcatDNFOf{\DNFRegex}{\DNFRegexAlt}}$.
\end{lemma}
\begin{proof}
Let $\DNFRegex=\DNFOf{\Sequence_0\DNFSep\ldots\DNFSep\Sequence_n}$, and
let $\DNFRegexAlt=\DNFOf{\SequenceAlt_0\DNFSep\ldots\DNFSep\SequenceAlt_m}$
\begin{tabular}{@{}L@{}L@{}}
\LanguageOf{\ConcatDNFOf{\DNFRegex}{\DNFRegexAlt}} & = 
\LanguageOf{\DNFOf{\ConcatSequenceOf{\Sequence_i}{\SequenceAlt_j}
\text{ for $i\in\RangeIncInc{1}{n}$, $j\in\RangeIncInc{1}{m}$}}} \\
& = 
\{\String\SuchThat \String\in\ConcatSequenceOf{\Sequence_i}{\SequenceAlt_j}\\
& \hspace{5em}
\text{ where $i\in\RangeIncInc{1}{n}$, $j\in\RangeIncInc{1}{m}$}\}\\
& = 
\{\String\Concat\StringAlt{} \SuchThat{} \String\in\LanguageOf{\Sequence_i}
\BooleanAnd{} \StringAlt\in\LanguageOf{\SequenceAlt_j}\}\\
& \hspace{5em}
\text{ where $i\in\RangeIncInc{1}{n}$, $j\in\RangeIncInc{1}{m}$}\}\\
& =
\{\String\Concat\StringAlt{} \SuchThat{} \String\in\LanguageOf{\DNFRegex}
\BooleanAnd{} \StringAlt\in\LanguageOf{\DNFRegexAlt}\}\\
& =
\{\String\Concat\StringAlt{} \SuchThat{} \String\in\LanguageOf{\Regex}
\BooleanAnd{} \StringAlt\in\LanguageOf{\RegexAlt}\}\\
& =
\LanguageOf{\RegexConcat{\Regex}{\RegexAlt}}
\end{tabular}
\end{proof}

\begin{lemma}[Equivalence of \OrDNF{} and \Or{}]
\label{lem:odnfeq}
If $\LanguageOf{\Regex}=\LanguageOf{\DNFRegex}$,
and $\LanguageOf{\RegexAlt}=\LanguageOf{\DNFRegexAlt}$,
then $\LanguageOf{\RegexOr{\Regex}{\RegexAlt}}=
\LanguageOf{\OrDNFOf{\DNFRegex}{\DNFRegexAlt}}$.
\end{lemma}
\begin{proof}
Let $\DNFRegex=\DNFOf{\Sequence_0\DNFSep\ldots\DNFSep\Sequence_n}$, and
let $\DNFRegexAlt=\DNFOf{\SequenceAlt_0\DNFSep\ldots\DNFSep\SequenceAlt_m}$
\begin{tabular}{@{}L@{}L@{}}
\LanguageOf{\OrDNFOf{\DNFRegex}{\DNFRegexAlt}} & = 
\LanguageOf{\DNFOf{\Sequence_0\DNFSep\ldots\DNFSep\Sequence_n\DNFSep
\SequenceAlt_1\DNFSep\ldots\DNFSep\SequenceAlt_m}}\\
& = 
\{\String\SuchThat{} \String\in\Sequence_i\vee\String\in\SequenceAlt_j\\
& \hspace{5em}
\text{ where $i\in\RangeIncInc{1}{n}$, $j\in\RangeIncInc{1}{m}$}\}\\
& = 
\{\String{} \SuchThat{} \String\in\LanguageOf{\DNFRegex}
\BooleanOr{} \String\in\LanguageOf{\DNFRegexAlt}\}\\
& =
\{\String \SuchThat{} \String\in\LanguageOf{\Regex}
\BooleanOr{} \String\in\LanguageOf{\RegexAlt}\}\\
& =
\LanguageOf{\RegexOr{\Regex}{\RegexAlt}}
\end{tabular}\\
\end{proof}

\begin{proof}
By structural induction.

Let $\Regex=\String$.
$\LanguageOf{\ToDNFRegex(\String)}=\LanguageOf{\DNFOf{\SequenceOf{\String}}}=
\{\String\}=\LanguageOf{\String}$

Let $\Regex=\emptyset$.
$\LanguageOf{\ToDNFRegex(\emptyset)}=\LanguageOf{\DNFOf{}} =
\{\} = \LanguageOf{\emptyset}$.

Let $\Regex=\StarOf{\Regex'}$.
By induction assumption, $\LanguageOf{\ToDNFRegex(\Regex')}=
\LanguageOf{\Regex'}$.\\
\begin{tabular}{@{}L@{}L@{}}
\LanguageOf{\ToDNFRegex(\StarOf{\DNFRegex'})} & =
\LanguageOf{\DNFOf{\SequenceOf{\StarOf{\ToDNFRegex(\Regex')}}}}\\
& =
\{\String\SuchThat\String\in
\LanguageOf{\SequenceOf{\StarOf{\ToDNFRegex(\Regex')}}}\}\\
& = 
\{\String\SuchThat{} \String\in\LanguageOf{\StarOf{\ToDNFRegex(\Regex')}}\}\\
& =
\{\String_1\Concat\ldots\Concat\String_n\SuchThat{}
n\in\Nats\\
& \hspace*{3em}\BooleanAnd\String_i\in\LanguageOf{\ToDNFRegex(\Regex')}\}\\
& =
\{\String_1\Concat\ldots\Concat\String_n\SuchThat{}
n\in\Nats\BooleanAnd\String_i\in\LanguageOf{\Regex'}\}\\
& = \LanguageOf{\StarOf{\Regex'}}
\end{tabular}

Let $\Regex=\RegexConcat{\Regex_1}{\Regex_2}$.
By induction assumption,
$\LanguageOf[\ToDNFRegex(\Regex_1)=\LanguageOf{\Regex_1}$, and
$\LanguageOf[\ToDNFRegex(\Regex_2)=\LanguageOf{\Regex_2}$.
$\ToDNFRegex(\RegexConcat{\Regex_1}{\Regex_2})=
\ConcatDNFOf{\ToDNFRegex(\Regex_1)}{\ToDNFRegex(\Regex_2)}$.
By Lemma~\ref{lem:cdnfeq},
$\RegexConcat{\Regex_1}{\Regex_2}=
\ConcatDNFOf{\ToDNFRegex(\Regex_1)}{\ToDNFRegex(\Regex_2)}$.

Let $\Regex=\RegexOr{\Regex_1}{\Regex_2}$.
By induction assumption,
$\LanguageOf[\ToDNFRegex(\Regex_1)=\LanguageOf{\Regex_1}$, and
$\LanguageOf[\ToDNFRegex(\Regex_2)=\LanguageOf{\Regex_2}$.
$\ToDNFRegex(\RegexOr{\Regex_1}{\Regex_2})=
\OrDNFOf{\ToDNFRegex(\Regex_1)}{\ToDNFRegex(\Regex_2)}$.
By Lemma~\ref{lem:odnfeq},
$\RegexOr{\Regex_1}{\Regex_2}=
\OrDNFOf{\ToDNFRegex(\Regex_1)}{\ToDNFRegex(\Regex_2)}$.
\end{proof}

Furthermore, we can use prove that language of DNF regular expressions is sound.

\begin{restatable}[Soundness of DNF Regexs]{theorem}{dnfrs}
\label{thm:soundness-dnf-lenses}
For all dnf regular expressions \DNFRegex{},
there exists a regular expression \Regex{},
such that \LanguageOf{\Regex{}}=\LanguageOf{\DNFRegex{}}.
\end{restatable}
First we will prove some lemmas.

\begin{lemma}
\label{lem:conjunct-builder-rx}
If $\LanguageOf{\ConjunctOf{\String_0\ConjunctSep\Atom_1\ConjunctSep\ldots
\ConjunctSep\Atom_n}}=\LanguageOf{\Regex}$
and $\LanguageOf{\AtomAlt}=\LanguageOf{\RegexAlt}$,
then $\LanguageOf{\ConjunctOf{\StringAlt\ConjunctSep\AtomAlt\ConjunctSep
\String_0\ConjunctSep\Atom_1\ConjunctSep\ldots\ConjunctSep\Atom_n}}=
\LanguageOf{\StringAlt\Concat\RegexAlt\Concat\Regex}$.
\end{lemma}
\begin{proof}\leavevmode\\
\begin{tabular}{@{}L@{}L@{}}
\LanguageOf{\ConjunctOf{\StringAlt\ConjunctSep\AtomAlt\ConjunctSep{}
\String_0\ConjunctSep\Atom_1\ConjunctSep\ldots\ConjunctSep\Atom_n}} & = 
\{\StringAlt\Concat\StringAlt'\Concat\String_0\Concat\String_0'\Concat\ldots\Concat\String_n'\Concat\String_n\\
& \hspace{1em} \SuchThat{} \StringAlt'\in\LanguageOf{\AtomAlt}\BooleanAnd{}
\String_i'\in\LanguageOf{\Atom_i}\}\\
& =
\{\StringAlt\Concat\StringAlt'\Concat\String\\
& \hspace{1em} \SuchThat{}
\StringAlt'\in\LanguageOf{\AtomAlt}\BooleanAnd{}
\String\in\LanguageOf{\ConjunctOf{\String_0\ConjunctSep\Atom_1\ConjunctSep\ldots
\ConjunctSep\Atom_n}}\\
& =
\{\StringAlt\Concat\StringAlt'\Concat\String\\
& \hspace{1em} \SuchThat{}
\StringAlt\in\LanguageOf{\StringAlt}\BooleanAnd{}
\StringAlt'\in\LanguageOf{\RegexAlt}\BooleanAnd{}
\String\in\LanguageOf{\Regex}\}\\
& =
\LanguageOf{\StringAlt\Concat\RegexAlt\Concat\Regex}
\end{tabular}
\end{proof}

\begin{lemma}
\label{lem:conjunct-rx}
Let $\ConjunctOf{\String_0\ConjunctSep\Atom_1\ConjunctSep
\ldots\Atom_n\ConjunctSep\String_n}$ be a conjunct,
and for each $\Atom_i$, there exists a regular expression, $\Regex_i$, such that
$\LanguageOf{\Atom_i}=\Regex_i$.
Then, there exists a regular expression $\Regex$, such that
$\LanguageOf{\ConjunctOf{\String_0\ConjunctSep\Atom_1\ConjunctSep
\ldots\Atom_n\ConjunctSep\String_n}}=\LanguageOf{\Regex}$.
\end{lemma}
\begin{proof}
By induction on $n$.

Let $n=0$.
$\Conjunct=\ConjunctOf{\String_0}$.
Consider the regular expression, $\String_0$.
$\LanguageOf{\String_0}=\{\String_0\}=\LanguageOf{\ConjunctOf{\String_0}}$

Let $n>0$,
$\Conjunct=\ConjunctOf{\String_0\ConjunctSep\Atom_1\ConjunctSep
\ldots\Atom_n\ConjunctSep\String_n}$.
By induction assumption, there exists a regular expression $\Regex'$ such that
$\LanguageOf{\Regex'}=\LanguageOf{\String_1\ConjunctSep\Atom_2\ldots
\ConjunctSep\Atom_n\String_n}$.
By problem assumption, there exists a regular expression $\Regex_1$ such that
$\LanguageOf{\Atom_1}=\LanguageOf{\Regex_1}$.
Consider $\String_0\Concat\Regex_1\Concat\Regex'$.
By Lemma~\ref{lem:conjunct-builder-rx},
$\LanguageOf{\String_0\Concat\Regex_1\Concat\Regex'}=
\LanguageOf{\ConjunctOf{\String_0\ConjunctSep\Atom_1\ConjunctSep
\ldots\Atom_n\ConjunctSep\String_n}}$.
\end{proof}



\begin{lemma}
\label{lem:dnf-builder-rx}
If $\LanguageOf{\DNFOf{\Conjunct_2\DNFSep\ldots\DNFSep\Conjunct_n}}
=\LanguageOf{\Regex}$
and $\LanguageOf{\Conjunct_1}=\LanguageOf{\RegexAlt}$,
then $\LanguageOf{\DNFOf{\Conjunct_1\DNFSep\ldots\DNFSep\Conjunct_n}}=
\LanguageOf{\RegexOr{\RegexAlt}{\Regex}}$.
\end{lemma}
\begin{proof}\leavevmode\\
\begin{tabular}{@{}L@{}L@{}}
\LanguageOf{\DNFOf{\Conjunct_1\ldots\Conjunct_n}} & = 
\{\String\SuchThat{} \String\in\LanguageOf{\Conjunct_i}
\text{ for some $i\in\RangeIncInc{1}{n}$}\}\\
& =
\{\String\SuchThat{}
\String\in\LanguageOf{\Conjunct_1}\BooleanOr{}
\String\in\LanguageOf{\Conjunct_i}\\
& \hspace{5em} 
\text{ for some $i\in\RangeIncInc{2}{n}$}\}\\
& =
\{\String\SuchThat{}
\String\in\LanguageOf{\RegexAlt}\BooleanOr{}
\String\in\LanguageOf{\Regex}\}\\
& = \LanguageOf{\RegexOr{\RegexAlt}{\Regex}}\\
\end{tabular}
\end{proof}

\begin{lemma}
\label{lem:dnf-rx}
Let $\DNFOf{\Conjunct_1\DNFSep\ldots\ConjunctSep\Conjunct_n}$ be a dnf regex,
and for each $\Conjunct_i$, there exists a regular expression, $\Regex_i$,
such that $\LanguageOf{\Conjunct_i}=\Regex_i$.
Then, there exists a regular expression $\Regex$, such that
$\LanguageOf{\DNFOf{\Conjunct_1\DNFSep
\ldots\DNFSep\Conjunct_n}}=\LanguageOf{\Regex}$.
\end{lemma}
\begin{proof}
By induction on $n$.

Let $n=0$.  Consider $\emptyset$
$\LanguageOf{\DNFOf{}}=\{\}=\LanguageOf{\emptyset}$.

Let $n>0$.
$\DNFRegex=\DNFOf{\Conjunct_1\DNFSep\ldots\DNFSep\Conjunct_n}$.
By induction assumption, there exists a regular expression $\Regex'$ such that
$\LanguageOf{\Regex'}=
\LanguageOf{\DNFOf{\Conjunct_2\DNFSep\ldots\DNFSep\Conjunct_n}}$.
By problem assumption, there exists a regular expression $\Regex_1$ such that
$\LanguageOf{\Conjunct_1}=\LanguageOf{\Regex_1}$.
Consider $\RegexOr{\Regex_1}{\Regex'}$.
By Lemma~\ref{lem:dnf-builder-rx},
$\LanguageOf{\RegexOr{\Regex_1}{\Regex'}}=
\LanguageOf{\DNFOf{\Conjunct_1\DNFSep
\ldots\Conjunct_n}}$.
\end{proof}


We can further normalize the DNF regular expressions by providing an order on sequences,
and ordering the sequences in the DNF regular expression according to that order.
With this, we have normalized all regular expression equivalences not involving stars.
In other words, if two regular expressions, $\Regex_1$ and $\Regex_2$,
are equivalent up to the transformations of regular expressions in equivalences
not involving stars,
then $\ToDNFRegex(\Regex_1)=\ToDNFRegex(\Regex_2)$.
A reasonable approach to further normalizing this is to attempt to minimize the
regular expression as much as possible, using the star laws.
Unfortunately, this doesn't work
well for finding lenses.  For example, there is clearly a lens of type
$\MapsBetweenTypeOf{\RegexOr{\EmptyString}{(\RegexConcat{a}{\StarOf{a}})}}{\RegexOr{b}{(\RegexConcat{c}{\StarOf{d}})}}$,
namely $\OrLens{\ConstLens{\EmptyString}{a}}{\ConcatLens{\ConstLens{a}{b}}{\IterateLens{\ConstLens{a}{d}}}}$.
However, if one simplifies as much as possible, then they will have to find a lens
of type $\MapsBetweenTypeOf{\StarOf{a}}{\RegexOr{b}{(\RegexConcat{c}{\StarOf{d}})}}$.
Unlike the type before being minified, where there was a type directed way to find
this lens, there is no longer a type directed way: should the lens be an iterate lens
or an or lens.
However, the regular expression equivalences have meaning.
If we only allow one rewrite, we see that
$\RegexOr{\EmptyString}{(\RegexConcat{a}{\StarOf{a}})}$ can be used for more
complicated lenses than merely $\StarOf{a}$.
A regular expression merely of the form $\StarOf{a}$ only cares about how the iterated case is handled, where a regular expression of the form
$\RegexOr{\EmptyString}{(\RegexConcat{a}{\StarOf{a}})}$ potentially acts differently on the empty
string case than on the nonempty case.

% fig:dnf-regex-rewrites
\begin{figure}
\begin{mathpar}
\inferrule[\AtomSumstarRule{}]
{
S\subseteq\RangeIncInc{1}{n}\\
\DNFRegex_{\Set} = [\Sequence_{\Set_1};\ldots;\Sequence_{\Set_{\SizeOf{\Set}}}]\\
\DNFRegex_{\ComplementOf{\Set}} = [\Sequence_{\ComplementOf{\Set}_1};\ldots;\Sequence_{\ComplementOf{\Set}_{\SizeOf{\ComplementOf{\Set}}}}]
}
{
\StarOf{\DNFOf{\Sequence_1\DNFSep\ldots\DNFSep\Sequence_n}}\RewriteAtom\\\\
\DNFOf{\SequenceOf{\StarOf{(\ConcatDNFOf{\DNFOf{\SequenceOf{\StarOf{\DNFRegex_S}}}}{\DNFRegex_{\ComplementOf{\Set}}})}\SequenceSep
\StarOf{\DNFRegex_{\Set}}}}\\
}

\inferrule[\AtomUnrollstarLeftRule{}]
{
}
{
\StarOf{\DNFRegex}\RewriteAtom
\OrDNFOf{\DNFOf{\SequenceOf{\EmptyString}}}{(\ConcatDNFOf{\DNFRegex}{\DNFOf{\SequenceOf{\StarOf{\DNFRegex}}}})}
}

\inferrule[\AtomUnrollstarRightRule{}]
{
}
{
\StarOf{\DNFRegex}\RewriteAtom
\OrDNFOf{\DNFOf{\SequenceOf{\EmptyString}}}{(\ConcatDNFOf{\DNFOf{\SequenceOf{\StarOf{\DNFRegex}}}}{\DNFRegex})}
}

\inferrule[\AtomPowerstarRule{}]
{
n\in\Nats_{\geq1}
}
{
\StarOf{\DNFRegex}\RewriteAtom
\ConcatDNFOf{\DNFOf{\SequenceOf{\StarOf{(\RepeatDNFOf{n}{\DNFRegex})}}}}
{(\RepeatDNFOf{0}{\DNFRegex}\OrDNF\ldots\OrDNF\RepeatDNFOf{n}{\DNFRegex})}
}

\inferrule[\DicyclicRewriteStarRule{}]
{
S\subseteq\RangeIncInc{1}{n}\\
\DNFRegex_{\Set} = [\Sequence_{\Set_1};\ldots;\Sequence_{\Set_{\SizeOf{\Set}}}]\\
\DNFRegex_{\ComplementOf{\Set}} = [\Sequence_{\ComplementOf{\Set}_1};\ldots;\Sequence_{\ComplementOf{\Set}_{\SizeOf{\ComplementOf{\Set}}}}]
}
{
\StarOf{\DNFOf{\Sequence_1\DNFSep\ldots\DNFSep\Sequence_n}}\RewriteAtom\\\\
\StarOf{(\ConcatDNFOf{(\OrDNFOf{\DNFRegex_{\Set}}{\DNFRegex_{\ComplementOf{\Set}}})}{\OrDNFOf{\DNFRegex_{\ComplementOf{\Set}}}{\ConcatDNFOf{{(\ConcatDNFOf{\DNFRegex_{\Set}}{\StarOf{\DNFRegex_{\ComplementOf{\Set}}}})}^n}{\DNFRegex_{\Set}}}})}\ConcatDNF\\
(\EmptyString\OrDNF(\OrDNFOf{\DNFRegex_{\Set}}{\DNFRegex_{\ComplementOf{\Set}}})\ConcatDNF\\
({(\ConcatDNFOf{\DNFRegex_{\Set}}{\StarOf{\DNFRegex_{\ComplementOf{\Set}}}})}^0\OrDNF\ldots\OrDNF{(\ConcatDNFOf{\DNFRegex_{\Set}}{\StarOf{\DNFRegex_{\ComplementOf{\Set}}}})}^n))
}

\inferrule[\DNFRewriteStarRule{}]
{
\DNFRegex \RewriteDNF \DNFRegex'
}
{
\StarOf{\DNFRegex} \RewriteAtom \DNFOf{\SequenceOf{\StarOf{\DNFRegex'}}}
}

\inferrule[\AtomDNFRewriteRule{}]
{
\Atom_j \RewriteAtom \DNFRegex
}
{
\DNFLeft\Sequence_1\DNFSep\ldots\DNFSep\Sequence_{i-1}\DNFSep\\\\
\SequenceOf{\String_0\SequenceSep\Atom_1\SequenceSep\ldots\SequenceSep\String_{j-1}\SequenceSep\Atom_j\SequenceSep\String_j\SequenceSep\ldots\SequenceSep\Atom_m\SequenceSep\String_m}\DNFSep\\\\
\Sequence_{i+1}\DNFSep\ldots\DNFSep\Sequence_n\DNFRight\RewriteDNF\\\\
\DNFOf{\Sequence_1\DNFSep\ldots\DNFSep\Sequence_{i-1}} \OrDNF\\
\DNFOf{\SequenceOf{\String_0\SequenceSep\Atom_1\SequenceSep\ldots\SequenceSep\String_{j-1}}}\ConcatDNF\DNFRegex\ConcatDNF\SequenceOf{\String_j\SequenceSep\ldots\SequenceSep\Atom_m\SequenceSep\String_m} \OrDNF\\
\DNFOf{\Sequence_{i+1}\DNFSep\ldots\DNFSep\Sequence_n}
}

\inferrule[\DNFRewriteCompositionRule{}]
{
\DNFRegex \RewriteDNF \DNFRegex'\\
\DNFRegex' \RewriteDNF \DNFRegex''
}
{
\DNFRegex \RewriteDNF \DNFRegex''
}

\inferrule[\IdentityDNFRewriteRule{}]
{
}
{
\DNFRegex \RewriteDNF \DNFRegex
}

\end{mathpar}
\caption{DNF Regex Rewrite Rules}
\label{fig:dnf-regex-rewrites}
\end{figure}


Because of this, instead of writing equivalences for DNF regular expressions like
exist for normal regular expressions,
instead we write rewrite rules as shown in
Figure~\ref{fig:dnf-regex-rewrites}.
Intuitively, these rewrite rules correspond to expanding a regular expression
into a regular expression which can be used in a more complicated lens, without requiring retyping.
We define these rules as rewrites that turn atoms into DNF Regular expressions,
and a rule that expresses how these rewrites can become rewrites on
the DNF regular expressions themselves.  \bcp{Here we {\em really} need some
examples.}

Roughly, the \AtomSumstarRule{} rewrite corresponds to \SumstarRule{},
the \AtomUnrollstarLeftRule{} and \AtomUnrollstarRightRule{} rewrites correspond to \ProductstarRule{}.
and the \DicyclicRewriteStarRule{} rewrite corresponds to \DicyclicityRule{}.
There is no rewrite corresponding to Regular Expression equivalence 10, as that
equivalence is an ambiguity introducing equivalence.
An application of \AtomSumstarRule{} corresponds to doing different things after the last time a certain event has occured, and before the last time that event occurs.
An application of \AtomUnrollstarLeftRule{} corresponds to making a different action for
the empty case of an iteration, the first case of an iteration, and all further cases of the iteration,
and similarly with \AtomUnrollstarRightRule{} for the empty case, the last case, and all previous cases, of the iteration.
An application of \DicyclicRewriteStarRule{} corresponds to TODO: understand this
horrible horrible rule.

\section{DNF Lenses}

% fig:dnf-lens-syntax
\begin{figure}
\centering
\begin{tabular}{@{}l@{\ }l@{\ }c@{}l@{\ }>{\itshape\/}r@{}}
% REGEX
(Atom Lenses) &\AtomLens{} & \GEq{} & $Iterate(\DNFLens)$ & Iterate\\
& & & \GBar{} \IdentityLens{} & Identity\\
(Sequence Lenses) &\SequenceLens{} & \GEq{} &
$(\SequenceLensOf{(\String_0,\StringAlt_0)\SequenceLensSep\AtomLens_1\SequenceLensSep\ldots\SequenceLensSep\AtomLens_n\SequenceLensSep(\String_n,\StringAlt_n)}$, &\\
& & & $\sigma \in S_n)$ & Clause\SubN{}\\
(DNF Lenses)& \DNFLens{} & \GEq{} & $(\DNFLensOf{\SequenceLens_1\DNFLensSep\ldots\DNFLensSep\SequenceLens_n}, \sigma \in S_n)$ & DNF\SubN{}\\
\end{tabular}
\caption{DNF Lens Syntax}
\label{fig:dnf-lens-syntax}
\end{figure}

% fig:dnf-lens-semantics
\begin{figure}
Semantics of Atom Lenses:\\
$\SemanticsOf{\IterateLens{\DNFLens}}$=\\
\hspace*{3em}$\SetOf{(\String_1\Concat\ldots\Concat\String_n,
\StringAlt_1\Concat\ldots\Concat\StringAlt_n)\SuchThat
n\in\NatsZero\BooleanAnd(\String_i,\StringAlt_i)\in\SemanticsOf{\DNFLens}}$\\
\\
Semantics of Sequence Lenses:\\
$\SemanticsOf{(\SequenceLensOf{(\String_0,\StringAlt_0)\SequenceLensSep
\AtomLens_1\SequenceLensSep\ldots\SequenceLensSep\AtomLens_n
\SequenceLensSep(\String_n,\StringAlt_n)},\Permutation)}=$\\
\hspace*{3em}$\SetOf{
(\String_0\Concat\String_1'\Concat\ldots\Concat\String_n'\Concat\String_n,
\String_{\Permutation(0)}\Concat\String_{\Permutation(1)}'\Concat\ldots
\Concat\String_{\Permutation(n)}'\Concat\String_{\Permutation(n)})\SuchThat$\\
\hspace*{4em}$(\String_i',\StringAlt_i')\in\SemanticsOf{\Atom_i}}$\\
\\
Semantics of DNF Lenses:\\
$\SemanticsOf{(\DNFLensOf{\SequenceLens_1\DNFLensSep\ldots\DNFLensSep
\SequenceLens_n},\Permutation)}=$\\
\hspace*{3em}$\SetOf{(\String,\StringAlt)\SuchThat
(\String,\StringAlt)\in\SequenceLens_i\text{ for some $i$}}$
\caption{DNF Lens Semantics}
\label{fig:dnf-lens-semantics}
\end{figure}

% fig:dnf-lens-semantics
\begin{figure}
\centering
\begin{mathpar}
\inferrule[\IterateAtomLensRule{}]
{
\DNFLens \OfType \DNFRegex \Leftrightarrow \DNFRegexAlt
}
{
\IterateLens{\DNFLens} \OfType \StarOf{\DNFRegex} \Leftrightarrow \StarOf{\DNFRegexAlt}\\
(\PutRight\Apply\String_1)\Concat\ldots\Concat(\PutRight\Apply\String_n),\\
\lambda \String.\LetWhereIn{\String_1\Concat\ldots\Concat\String_n}{\String}{\String_i\in\LanguageOf{\DNFRegexAlt}}\\
(\PutLeft\Apply\String_1)\Concat\ldots\Concat(\PutLeft\Apply\String_n)
}

\inferrule[\SequenceLensRule{}]
{
\AtomLens_1 \OfType \Atom_1 \Leftrightarrow \AtomAlt_1\\
\ldots\\
\AtomLens_n \OfType \Atom_n \Leftrightarrow \AtomAlt_n\\
\sigma \in \PermutationSetOf{n}\\
\forall i \in \RangeIncInc{1}{n-1} \UnambigConcatOf{\LanguageOf{\String_{i-1}\Atom_i\epsilon}}{\LanguageOf{\String_i\Atom_{i+1}\epsilon}}\\
\forall i \in \RangeIncInc{1}{n-1} \UnambigConcatOf{\LanguageOf{\StringAlt_{i-1}\AtomAlt_{\sigma(i)}\epsilon}}{\LanguageOf{\StringAlt_i\AtomAlt_{\sigma(i+1)}\epsilon}}\\
}
{
(\SequenceLensOf{(\String_0,\StringAlt_0)\SequenceLensSep\Atom_1\SequenceLensSep\ldots\SequenceLensSep\Atom_n\SequenceLensSep(\String_n,\StringAlt_n)},\sigma) \OfType\\
\SequenceOf{\String_0\SequenceSep\Atom_1\SequenceSep\ldots\SequenceSep\Atom_n\SequenceSep\String_n}\Leftrightarrow
\SequenceOf{\StringAlt_0\SequenceSep\AtomAlt_1\SequenceSep\ldots\SequenceSep\AtomAlt_n\SequenceSep\StringAlt_n}
}

\inferrule[\DNFLensRule{}]
{
\SequenceLens_1 \OfType \Sequence_1 \Leftrightarrow \SequenceAlt_1\\
\ldots\\
\SequenceLens_n \OfType \Sequence_n \Leftrightarrow \SequenceAlt_n\\
\sigma \in \PermutationSetOf{n}\\
i \neq j \Rightarrow \Sequence_{i} \cap \Sequence_{j}=\emptyset\\
i \neq j \Rightarrow \SequenceAlt_{i} \cap \SequenceAlt_{j}=\emptyset\\
}
{
(\DNFLensOf{\SequenceLens_1\DNFLensSep\ldots\DNFLensSep\SequenceLens_n},\sigma) \OfType\\
\DNFOf{\Sequence_1\DNFSep\ldots\DNFSep\Sequence_n}
\Leftrightarrow \DNFOf{\SequenceAlt_{\sigma(1)}\DNFSep\ldots\DNFSep\SequenceAlt_{\sigma(n)}}
}

\inferrule[\DNFRewriteLensRule{}]
{
\DNFRegex \RewriteDNF \DNFRegex'\\
\DNFRegexAlt \RewriteDNF \DNFRegexAlt'\\
\DNFLens \OfType \MapsBetweenTypeOf{\DNFRegex}{\DNFRegexAlt}
}
{
\DNFLens \OfType \MapsBetweenTypeOf{\DNFRegex'}{\DNFRegexAlt'}
}

\end{mathpar}
\caption{DNF Lens Typing}
\label{fig:dnf-lens-semantics}
\end{figure}


Armed with these rewrites, we have the capabilities to define a sufficiently
strong language for lenses on these regular expressions.
The syntax is defined in Figure~\ref{fig:dnf-lens-syntax}.
Similarly to the language of lenses, we aim to have the typing of the lenses
correspond closely to the syntax for the lenses themselves.
The typing and semantics of these dnf lenses are defined in Figure~\ref{fig:dnf-lens-semantics}.
Intuitively, a DNF Regex Lens corresponds roughly to an n-ary version of an or lens,
a Sequence Lens corresponds to const lenses for the strings, and a combination of
Concat and Swap lenses for the Atoms.
Furthermore, these lenses are sufficiently strong that Compose Lenses are not
necessary.
Indeed, these lenses are strong enough to express everything expressible in the language of lenses.
\begin{restatable}[Completeness of DNF Lenses Without Rewrites]{theorem}{dnflc}
\label{thm:completeness-dnf-lenses}
If there exists a derivation of $\Lens \OfType \MapsBetweenTypeOf{\Regex}{\RegexAlt}$,
then there exists a derivation of $\DNFLens \OfType \MapsBetweenTypeOf{\DNFRegex}{\DNFRegexAlt}$ such that
$\SemanticsOf{\DNFLens}=\SemanticsOf{\Lens}$,
$\LanguageOf{\DNFRegex}=\LanguageOf{\Regex}$, and
$\LanguageOf{\DNFRegexAlt}=\LanguageOf{\RegexAlt}$,
and the derivation for the typing includes no use of the rewrite rules.
\end{restatable}
We first must prove some Lemmas.

\begin{lemma}[Completeness of Composition]\leavevmode
\label{lem:composition-completeness}
\begin{enumerate}
\item If there are two atom lenses
$\AtomLens_1 \OfType \Atom_1 \Leftrightarrow \Atom_2$ and
$\AtomLens_2 \OfType \Atom_2 \Leftrightarrow \Atom_3$,
then there exists an atom lens
$\AtomLens \OfType \Atom_1 \Leftrightarrow \Atom_3$, such that
$\SemanticsOf{\AtomLens}=\SetOf{(\String_1,\String_3)\SuchThat
\exists\String_2
(\String_1,\String_2)\in\SemanticsOf{\AtomLens_1}\BooleanAnd
(\String_2,\String_3)\in\SemanticsOf{\AtomLens_2}}$

\item If there are two sequence lenses
$\SequenceLens_1 \OfType \Sequence_1 \Leftrightarrow \Sequence_2$ and
$\SequenceLens_2 \OfType \Sequence_2 \Leftrightarrow \Sequence_3$,
then there exists an sequence lens
$\SequenceLens \OfType \Sequence_1 \Leftrightarrow \Sequence_3$, such that
$\SemanticsOf{\SequenceLens}=\SetOf{(\String_1,\String_3)\SuchThat
\exists\String_2
(\String_1,\String_2)\in\SemanticsOf{\SequenceLens_1}\BooleanAnd
(\String_2,\String_3)\in\SemanticsOf{\SequenceLens_2}}$

\item If there are two DNF lenses
$\DNFLens_1 \OfType \DNFRegex_1 \Leftrightarrow \DNFRegex_2$ and
$\DNFLens_2 \OfType \DNFRegex_2 \Leftrightarrow \DNFRegex_3$,
then there exists a DNF lens
$\DNFLens \OfType \DNFRegex_1 \Leftrightarrow \DNFRegex_3$, such that
$\SemanticsOf{\DNFLens}=\SetOf{(\String_1,\String_3)\SuchThat
\exists\String_2
(\String_1,\String_2)\in\SemanticsOf{\DNFLens_1}\BooleanAnd
(\String_2,\String_3)\in\SemanticsOf{\DNFLens_2}}$
\end{enumerate}
\end{lemma}

\begin{proof}
By mutual induction

Let $\StarOf{\DNFRegex_1}$, $\StarOf{\DNFRegex_2}$, $\StarOf{\DNFRegex_3}$
be three atoms, and $\IterateLens{\DNFLens_1} \OfType
\StarOf{\DNFRegex_1} \Leftrightarrow \StarOf{\DNFRegex_2}$ with
$\IterateLens{\DNFLens_2} \OfType
\StarOf{\DNFRegex_2} \Leftrightarrow \StarOf{\DNFRegex_3}$
lenses between them.
By induction assumption, there a typing of a lens

\begin{mathpar}
\inferrule[]
{
\Derivation{}
}
{
\DNFLens \OfType \DNFRegex_1 \Leftrightarrow \DNFRegex_3
}
\end{mathpar}

such that $\SemanticsOf{\DNFLens}=\SetOf{(\String_1,\String_3)\SuchThat
\exists \String_2
(\String_1,\String_2)\in\SemanticsOf{\DNFLens_1}\BooleanAnd
(\String_2,\String_3)\in\SemanticsOf{\DNFLens_2}}$

By inversion lemma (TODO), there exists a $\IterateLens{\DNFLens_1}$ and
$\IterateLens{\DNFLens_2}$ came from $\IterateLensRule$, so I know
$\UnambigItOf{\LanguageOf{\DNFRegex_1}}$,
$\UnambigItOf{\LanguageOf{\DNFRegex_2}}$, and
$\UnambigItOf{\LanguageOf{\DNFRegex_3}}$.

Consider the lens

\begin{mathpar}
\inferrule*
{
\inferrule*
{
\Derivation{}
}
{
\DNFLens \OfType \DNFRegex_1 \Leftrightarrow \DNFRegex_3
}\\
\UnambigItOf{\LanguageOf{\DNFRegex_1}}\\
\UnambigItOf{\LanguageOf{\DNFRegex_2}}
}
{
\IterateLens{\DNFLens} \OfType
\StarOf{\DNFRegex_1} \Leftrightarrow \StarOf{\DNFRegex_3}
}
\end{mathpar}

This lens has the semantics

\begin{tabular}{@{}L@{}L@{}}
\SemanticsOf{\IterateLens{\DNFLens}}
& = \SetOf{(\String_{1,1}\Concat\ldots\Concat\String_{1,n},
\String_{3,1}\Concat\ldots\Concat\String_{3,n})\\
& \hspace{2em}
\SuchThat(\String_{1,i},\String_{3,i})\in\SemanticsOf{\DNFLens}}\\
& =
\SetOf{(\String_{1,1}\Concat\ldots\Concat\String_{1,n},
\String_{3,1}\Concat\ldots\Concat\String_{3,n})\\
& \hspace{2em}
\SuchThat
\exists\String_{2,i} (\String_{1,i},\String_{2,i})\in\SemanticsOf{\DNFLens_1}\\
& \hspace{4em}
\BooleanAnd
(\String_{2,i},\String_{3,i})\in\SemanticsOf{\DNFLens_2}}\\
& =
\SetOf{(\String_{1,1}\Concat\ldots\Concat\String_{1,n},
\String_{3,1}\Concat\ldots\Concat\String_{3,n}) \\
& \hspace{2em}
\SuchThat
\exists\String_{2,1}\Concat\ldots\Concat\String_{2,n}\\
& \hspace{4em}
(\String_{1,1}\Concat\ldots\Concat\String_{1,n},
\String_{2,1}\Concat\ldots\Concat\String_{2,n})
\in\SemanticsOf{\IterateLens{\DNFLens}}\\
& \hspace{4em}
\BooleanAnd
(\String_{2,1}\Concat\ldots\Concat\String_{2,n},
\String_{3,1}\Concat\ldots\Concat\String_{3,n})
\in\SemanticsOf{\IterateLens{\DNFLens}}}\\
& =
\SetOf{(\String_1,\String_3)\SuchThat\exists\String_2
(\String_1,\String_2)\in\SemanticsOf{\IterateLens{\DNFLens_1}}\\
& \hspace{4em}\BooleanAnd
(\String_2,\String_3)\in\SemanticsOf{\IterateLens{\DNFLens_2}}}
\end{tabular}

Let $\SequenceOf{\String_{1,0}\SequenceSep\Atom_{1,1}\SequenceSep
\ldots\SequenceSep\Atom_{1,n}\SequenceSep\String_{1,n}}$,
$\SequenceOf{\String_{2,0}\SequenceSep\Atom_{2,\Permutation_1(1)}\SequenceSep
\ldots\SequenceSep\Atom_{2,\Permutation_1(n)}\SequenceSep\String_{2,n}}$,
and $\SequenceOf{\String_{3,0}\SequenceSep
\Atom_{3,\Permutation_2\Compose\Permutation_1(1)}\SequenceSep
\ldots\SequenceSep\Atom_{3,\Permutation_2\Compose\Permutation_1(n)}
\SequenceSep\String_{3,n}}$ be sequences,
with $(\SequenceLensOf{(\String_{1,0},\String_{2,0})\SequenceLensSep
\AtomLens_{1,1}\SequenceLensSep\ldots\SequenceLensSep\AtomLens_{1,n}
\SequenceLensSep(\String_{1,n},\String_{2,n})},\Permutation_1)$ and
$(\SequenceLensOf{(\String_{2,0},\String_{3,0})\SequenceLensSep
\AtomLens_{2,1}\SequenceLensSep\ldots\SequenceLensSep\AtomLens_{2,n}
\SequenceLensSep(\String_{2,n},\String_{3,n})},\Permutation_2)$ be lenses between them.
By induction assumption, there is a typing of lenses
\begin{mathpar}
\inferrule[]
{
\Derivation_i
}
{
\AtomLens_i \OfType \Atom_{1,i} \Leftrightarrow \Atom_{3,i}
}
\end{mathpar}
such that \SemanticsOf{\AtomLens_i} = \SetOf{(\String_1,\String_3)\SuchThat
\exists \String_2 (\String_1,\String_2)\in\SemanticsOf{\AtomLens_{1,i}}
\BooleanAnd (\String_2,\String_3)\in\SemanticsOf{\Atom_{2,i}}}
Define $\Permutation = \Permutation_2\Compose\Permutation_1$.

By inversion, I know that
$\SequenceUnambigConcatOf{\SequenceOf{\String_{1,0}\SequenceSep\Atom_{1,1}
\SequenceSep\ldots\SequenceSep\Atom_{1,n}\SequenceSep\String_{1,n}}}$
and
$\SequenceUnambigConcatOf{\SequenceOf{\String_{3,0}\SequenceSep
\Atom_{\Permutation(3),1}\SequenceSep\ldots
\SequenceSep\Atom_{\Permutation(3),n}\SequenceSep\String_{3,n}}}$.

Consider the typing of the lens
\begin{mathpar}
{
\inferrule{
\inferrule*
{
\Derivation_0
}
{
\AtomLens_0 \OfType \Atom_{1,0} \Leftrightarrow \Atom_{3,0}
}\\
\ldots\\
\inferrule*
{
\Derivation_n
}
{
\AtomLens_n \OfType \Atom_{1,n} \Leftrightarrow \Atom_{3,n}
}\\
\sigma\in\PermutationSetOf{n}\\
\SequenceUnambigConcatOf{\SequenceOf{\String_{1,0}\SequenceSep\Atom_{1,1}
\SequenceSep\ldots\SequenceSep\Atom_{1,n}\SequenceSep\String_{1,n}}}\\
\SequenceUnambigConcatOf{\SequenceOf{\String_{3,0}\SequenceSep
\Atom_{\Permutation(3),1}\SequenceSep\ldots
\SequenceSep\Atom_{\Permutation(3),n}\SequenceSep\String_{3,n}}}
}
{
(\SequenceLensOf{(\String_{1,0},\String_{3,0})\SequenceLensSep\AtomLens_1
\SequenceLensSep\ldots\SequenceLensSep\AtomLens_n
\SequenceLensSep(\String_{1,n},\String_{3,n})},\Permutation) \OfType\\
\SequenceOf{\String_{1,0}\SequenceSep\Atom_{1,1}\SequenceSep\ldots
\SequenceSep\Atom_{1,n}\SequenceSep\String_{1,n}} \Leftrightarrow
\SequenceOf{\String_{3,0}\SequenceSep\Atom_{3,\Permutation(1)}\SequenceSep\ldots
\SequenceSep\Atom_{3,\Permutation(n)}\SequenceSep\String_{3,n}}
}
}
\end{mathpar}

Furthermore, we can prove the desired property of the semantics.\\
\SemanticsOf{(\SequenceLensOf{(\String_{1,0},\String_{3,0})\SequenceLensSep
\AtomLens_1\SequenceLensSep\ldots\SequenceLensSep\AtomLens_n
\SequenceLensSep(\String_{1,n},\String_{3,n})},\Permutation)}=\\
\SetOf{(\String_{1,0}\Concat\String_1\Concat\ldots
\Concat\String_n\Concat\String_{1,n},
\String_{3,0}\Concat\StringAlt_{\Permutation(1)}\Concat\ldots
\Concat\StringAlt_{\Permutation(n)}\Concat\String_{1,n})\\
\hspace*{1em}\SuchThat(\String_i,\StringAlt_i)\in\SemanticsOf{\AtomLens_i}}=\\
\SetOf{(\String_{1,0}\Concat\String_1\Concat\ldots
\Concat\String_n\Concat\String_{1,n},
\String_{3,0}\Concat\StringAlt_{\Permutation(1)}\Concat\ldots
\Concat\StringAlt_{\Permutation(n)}\Concat\String_{1,n})\\
\hspace*{1em}\SuchThat\exists\String_i'\in\LanguageOf{\Atom_{2,i}}
(\String_i,\String_i')\in\SemanticsOf{\AtomLens_i}\BooleanAnd
(\String_i',\StringAlt_i)\in\SemanticsOf{\AtomLens_i}}\\
\SetOf{(\String_{1,0}\Concat\String_1\Concat\ldots
\Concat\String_n\Concat\String_{1,n},
\String_{3,0}\Concat\StringAlt_{\Permutation(1)}\Concat\ldots
\Concat\StringAlt_{\Permutation(n)}\Concat\String_{1,n})\\
\hspace*{1em}\SuchThat\exists
\String_{2,0}\Concat\String_{\Permutation_1(1)}'\Concat\ldots\Concat
\String_{\Permutation_1(n)}'\Concat\String_{2,n}\in\LanguageOf{\Atom_{2,i}}\\
\hspace*{2em}
(\String_{1,0}\Concat\String_1\Concat\ldots
\Concat\String_n\Concat\String_{1,n},\String_{2,0}\Concat
\String_{\Permutation_1(1)}'\Concat\ldots\Concat
\String_{\Permutation_1(n)}'\Concat\String_{2,n})\\
\hspace*{3em}\in\SemanticsOf{(\SequenceLensOf{(\String_{1,0},\String_{2,0})
\SequenceLensSep\AtomLens_{1,1}\SequenceLensSep\ldots\SequenceLensSep
\AtomLens_{1,n}\SequenceLensSep(\String_{1,n},\String_{2,n})},\Permutation_1)}
\BooleanAnd\\
\hspace*{2em}
(\String_{2,0}\Concat\String_{\Permutation_1(1)}'\Concat\ldots\Concat
\String_{\Permutation_1(n)}'\Concat\String_{2,n},\String_{3,0}\Concat\StringAlt_{\Permutation(1)}\Concat\ldots
\Concat\StringAlt_{\Permutation(n)}\Concat\String_{1,n})\\
\hspace*{3em}\in\SemanticsOf{(\SequenceLensOf{(\String_{2,0},\String_{3,0})\SequenceLensSep
\AtomLens_{2,1}\SequenceLensSep\ldots\SequenceLensSep\AtomLens_{2,n}
\SequenceLensSep(\String_{2,n},\String_{3,n})},\Permutation_2)}}\\
\SetOf{(\String_1,\String_3)\SuchThat\exists\String_2\\
\hspace*{2em}
(\String_1,\String_2)\\
\hspace*{3em}\in\SemanticsOf{(\SequenceLensOf{(\String_{1,0},\String_{2,0})
\SequenceLensSep\AtomLens_{1,1}\SequenceLensSep\ldots\SequenceLensSep
\AtomLens_{1,n}\SequenceLensSep(\String_{1,n},\String_{2,n})},\Permutation_1)}\\
\hspace*{2em}
(\String_2,\String_3)\\
\hspace*{3em}\in\SemanticsOf{(\SequenceLensOf{(\String_{2,0},\String_{3,0})
\SequenceLensSep\AtomLens_{2,1}\SequenceLensSep\ldots\SequenceLensSep
\AtomLens_{2,n}\SequenceLensSep(\String_{2,n},\String_{3,n})},\Permutation_2)}}

Let $\DNFRegex_1=\DNFOf{\Sequence_{1,1}\DNFSep\ldots\DNFSep\Sequence_{1,n}}$,
$\DNFRegex_2=\DNFOf{\Sequence_{2,\Permutation_1(1)}\DNFSep\ldots
\DNFSep\Sequence_{2,\Permutation_1(n)}}$,
and $\DNFRegex_3=\DNFOf{\Sequence_{3,\Permutation_2\Compose\Permutation_1(1)}
\DNFSep\ldots\DNFSep\Sequence_{3,\Permutation_2\Compose\Permutation_1(n)}}$
be three DNF regular expressions.
Let $\DNFLens_1=(\DNFLensOf{\SequenceLens_{1,1}\DNFLensSep\ldots
\DNFLensSep\SequenceLens_{1,n}},\Permutation_1)\OfType
\DNFRegex_1\Leftrightarrow\DNFRegex_2$ and
$\DNFLens_2=(\DNFLensOf{\SequenceLens_{2,1}\DNFLensSep\ldots
\DNFLensSep\SequenceLens_{2,n}},\Permutation_2)\OfType
\DNFRegex_2\Leftrightarrow\DNFRegex_3$ be lenses between them.
By induction assumption, there exists a typing of lenses
\begin{mathpar}
\inferrule*
{
\Derivation_i
}
{
\Sequence_i \OfType \Sequence_{1,i} \Leftrightarrow \Sequence_{3,i}
}
\end{mathpar}


\end{proof}

\begin{lemma}[Completeness of Identity on Strongly Unambiguous DNF Regex,
Clauses, and Atoms]
\leavevmode
\begin{enumerate}
\item If $\DNFRegex$ is a strongly unambiguous DNF Regular expression, then
there exists a DNF lens $\DNFLens \OfType \DNFRegex \Leftrightarrow \DNFRegex$,
such that $\SemanticsOf{\DNFLens}=
\SetOf{(\String,\String)\SuchThat\String\in\LanguageOf{\DNFRegex}}$, where
\DNFLens{} typing includes no rewrite rules.
\item If $\Sequence$ is a strongly unambiguous sequence, then
there exists a sequence lens $\SequenceLens \OfType \Sequence \Leftrightarrow \Sequence$,
such that $\SemanticsOf{\SequenceLens}=
\SetOf{(\String,\String)\SuchThat\String\in\LanguageOf{\DNFRegex}}$, where
\SequenceLens{} typing includes no rewrite rules.
\item If $\Atom$ is a strongly unambiguous atom, then
there exists an atom lens $\AtomLens \OfType \Atom \Leftrightarrow \Atom$,
such that $\SemanticsOf{\AtomLens}=
\SetOf{(\String,\String)\SuchThat\String\in\LanguageOf{\DNFRegex}}$, where
\AtomLens{} typing includes no rewrite rules.
\end{enumerate}
\end{lemma}
\begin{proof}
By mutual induction on the structure of the DNF regular expression,
atom, and clause.

Let \DNFRegex{} be a DNF regular expression.

Let \Sequence{} be a sequence.

Let \Atom{} be an atom.
\end{proof}

Now we can do the proof of completeness.
\dnflc*

\begin{proof}
We will do by structural induction on the typing derivation.

Let the last rule be an application of \ConstantLensRule{}.
\begin{mathpar}
\inferrule[]
{
\String_1 \in \StarOf{\Sigma}\\
\String_2 \in \StarOf{\Sigma}
}
{
\ConstLens{\String_1}{\String_2} \OfType \String_1 \Leftrightarrow \String_2
}
\end{mathpar}
Consider the derivation
\begin{mathpar}
\inferrule[]
{
\inferrule[]
{
\String_1 \in \StarOf{\Sigma}\\
\String_2 \in \StarOf{\Sigma}
}
{
\SequenceLensOf{(\String_1,\String_2)} \OfType
\SequenceOf{\String_1} \Leftrightarrow \SequenceOf{\String_2}
}
}
{
\DNFLensOf{\SequenceLensOf{(\String_1,\String_2)}} \OfType
\DNFOf{\SequenceOf{\String_1}} \Leftrightarrow \DNFOf{\SequenceOf{\String_2}}
}
\end{mathpar}

\begin{tabular}{@{}L@{}L@{}}
\SemanticsOf{\DNFLensOf{\SequenceLensOf{(\String_1,\String_2)}}}
& = \SetOf{(\StringAlt_1,\StringAlt_2)
\SuchThat(\StringAlt_1,\StringAlt_2)\in\SequenceLensOf{(\String_1,\String_2)}}\\
& = \SetOf{(\StringAlt_1,\StringAlt_2)
\SuchThat(\StringAlt_1,\StringAlt_2)\in\SetOf{(\String_1,\String_2)}}\\
& = \SetOf{(\String_1,\String_2)}\\
& = \SemanticsOf{\ConstLens{\String_1}{\String_2}}\\
\end{tabular}

Let the last rule be an application of \IdentityLensRule{}.
We know that for each regular expression there exists a strongly unambiguous
regular expression~\cite{unambigregex}.
Because of that, if we have
$\IdentityLens \OfType \Regex \Leftrightarrow \Regex$,
then we have a strongly unambiguous DNF regular expression,
$\DNFRegex$, such that $\LanguageOf{\Regex}=\LanguageOf{\DNFRegex}$.
Then, by Lemma (TODO: ref), there exists a DNF lens
$\DNFLens \OfType \DNFRegex \Leftrightarrow \DNFRegexAlt$ such that
$\SemanticsOf{\DNFLens}=
\SetOf{(\String,\String)\SuchThat\String\in\LanguageOf{\DNFRegex}}$

Let the last rule be an application of \IterateLensRule{}.
\begin{mathpar}
\inferrule[]
{
\Lens \OfType \Regex \Leftrightarrow \RegexAlt \\
\UnambigItOf{\LanguageOf{\Regex}}\\
\UnambigItOf{\LanguageOf{\RegexAlt}}
}
{
\IterateLens{\Lens} \OfType \StarOf{\Regex} \Leftrightarrow \StarOf{\RegexAlt}
}
\end{mathpar}
By induction assumption, there exists a derivation
\begin{mathpar}
\inferrule[]
{
\Derivation{}
}
{
\DNFLens : \DNFRegex \Leftrightarrow \DNFRegexAlt
}
\end{mathpar}
such that \LanguageOf{\DNFRegex}=\LanguageOf{\Regex},
\LanguageOf{\RegexAlt}=\LanguageOf{\DNFRegexAlt}, and
\SemanticsOf{\DNFLens}=\SemanticsOf{\Lens}.
This means \UnambigItOf{\LanguageOf{\DNFRegex}}
and \UnambigItOf{\LanguageOf{\DNFRegexAlt}}

Consider the derivation
\begin{mathpar}
\inferrule[]
{
\Derivation
}
{
\inferrule[]
{
\DNFLens : \DNFRegex \Leftrightarrow \DNFRegexAlt
\UnambigItOf{\LanguageOf{\DNFRegex}}\\
\UnambigItOf{\LanguageOf{\DNFRegexAlt}}\\
}
{
\inferrule[]
{
\IterateLens{\DNFLens} \OfType
\StarOf{\DNFRegex} \Leftrightarrow \StarOf{\DNFRegexAlt}
}
{
\inferrule
{
\SequenceLensOf{\IterateLens{\DNFLens}} \OfType
\SequenceOf{\StarOf{\DNFRegex}}\Leftrightarrow \SequenceOf{\StarOf{\DNFRegexAlt}}
}
{
\DNFLensOf{\SequenceLensOf{\IterateLens{\DNFLens}}} \OfType
\DNFOf{\SequenceOf{\StarOf{\DNFRegex}}} \Leftrightarrow
\DNFOf{\SequenceOf{\StarOf{\DNFRegexAlt}}}
}
}
}
}
\end{mathpar}

By Lemma~(TODOREF), I know that, as \LanguageOf{\Regex}=\LanguageOf{\DNFRegex},
and as \LanguageOf{\RegexAlt}=\LanguageOf{\DNFRegexAlt}, then
\LanguageOf{\DNFOf{\SequenceOf{\StarOf{\DNFRegex}}}}=
\LanguageOf{\StarOf{\Regex}}
and \LanguageOf{\DNFOf{\SequenceOf{\StarOf{\DNFRegexAlt}}}}=
\LanguageOf{\StarOf{\RegexAlt}}.

\begin{tabular}{@{}L@{}L@{}}
\SemanticsOf{\DNFLensOf{\SequenceLensOf{\IterateLens{\DNFLens}}}}
& = \SetOf{(\String,\StringAlt)
\SuchThat(\String,\StringAlt)\in\SemanticsOf{\SequenceLensOf{\IterateLens{\DNFLens}}}}\\
& = \SetOf{(\String,\StringAlt)
\SuchThat(\String,\StringAlt)\in\SemanticsOf{\IterateLens{\DNFLens}}}\\
& = \SetOf{(\String_1\Concat\ldots\Concat\String_n,
\StringAlt_1\Concat\ldots\Concat\StringAlt_n)
\SuchThat \String_i\in\SemanticsOf{\DNFLens}}\\
& = \SetOf{(\String_1\Concat\ldots\Concat\String_n,
\StringAlt_1\Concat\ldots\Concat\StringAlt_n)
\SuchThat \String_i\in\SemanticsOf{\Lens}}\\
& = \SemanticsOf{\IterateLens{\Lens}}
\end{tabular}

Let the last rule be an application of \ConcatLensRule{}.

Let the last rule be an application of \SwapLensRule{}.

Let the last rule be an application of \OrLensRule{}.
\begin{mathpar}
\inferrule[]
{
\Lens_1 \OfType \Regex_1 \Leftrightarrow \RegexAlt_1\\
\Lens_2 \OfType \Regex_2 \Leftrightarrow \RegexAlt_2\\
\UnambigOrOf{\LanguageOf{\Regex_1}}{\LanguageOf{\Regex_2}}\\
\UnambigOrOf{\LanguageOf{\RegexAlt_1}}{\LanguageOf{\RegexAlt_2}}
}
{
\OrLens{\Lens_1}{\Lens_2} \OfType
\RegexOr{\Regex_1}{\RegexAlt_1} \Leftrightarrow \RegexOr{\Regex_2'}{\RegexAlt_2'}
}
\end{mathpar}
By induction assumption, there exists a derivation
\begin{mathpar}
\inferrule[]
{
\Derivation
}
{
(\DNFLensOf{\SequenceLens_1,\ldots,\SequenceLens_n},\Permutation_1) \OfType\\
\DNFOf{\Sequence_1\DNFSep\ldots\DNFSep\Sequence_n} \Leftrightarrow
\DNFOf{\SequenceAlt_{\Permutation_1(1)}\DNFSep\ldots
\DNFSep\SequenceAlt_{\Permutation_1(n)}}
}
\end{mathpar}
and a derivation
\begin{mathpar}
\inferrule[]
{
\Derivation'
}
{
(\DNFLensOf{\SequenceLens_{n+1},\ldots,\SequenceLens_{n+m}},\Permutation_2) \OfType\\
\DNFOf{\Sequence_{n+1}'\DNFSep\ldots\DNFSep\Sequence_{n+m}} \Leftrightarrow
\DNFOf{\SequenceAlt_{n+\Permutation_2(1)'}\DNFSep\ldots
\DNFSep\SequenceAlt_{n+\Permutation_2(m)'}}
}
\end{mathpar}
such that, if we define:\\
\begin{tabular}{@{}L@{}L@{}}
\DNFRegex_1= &
\DNFOf{\Sequence_1\DNFSep\ldots\DNFSep\Sequence_n}\\
\DNFRegexAlt_1= &
\DNFOf{\SequenceAlt_{\Permutation_1(1)}\DNFSep\ldots
\DNFSep\SequenceAlt_{\Permutation_1(n)}}\\
\DNFRegex_2= &
\DNFOf{\Sequence_{n+1}\DNFSep\ldots\DNFSep\Sequence_{n+m}}\\
\DNFRegexAlt_2= &
\DNFOf{\SequenceAlt_{n+\Permutation_2(1)}\DNFSep\ldots
\DNFSep\SequenceAlt_{n+\Permutation_2(m)}}\\
\end{tabular}
then we have
\LanguageOf{\DNFRegex}=\LanguageOf{\Regex},
\LanguageOf{\DNFRegexAlt}=\LanguageOf{\RegexAlt},
\LanguageOf{\DNFRegex'}=\LanguageOf{\Regex'},
\LanguageOf{\DNFRegexAlt'}=\LanguageOf{\RegexAlt'}.

By inversion (TODO: inversion lemma), we know that the last step of those derivations were applications
of \DNFLensRule{}, so we know that, for all $i\in\RangeIncInc{1}{m}$,
there exists derivations
\begin{mathpar}
\inferrule[]
{
\Derivation_i
}
{
(\SequenceLens_i \OfType
\Sequence_i \Leftrightarrow \SequenceAlt_i)
}
\end{mathpar}
For all $i\neq j$, $i,j\in\RangeIncInc{1}{n}$,
$\Sequence_i \Intersect \Sequence_j = \emptyset$, and
$\SequenceAlt_i \Intersect \SequenceAlt_j = \emptyset$.
For all $i\neq j$, $i,j\in\RangeIncInc{n+1}{n+m}$,
$\Sequence_i \Intersect \Sequence_j = \emptyset$, and
$\SequenceAlt_i \Intersect \SequenceAlt_j = \emptyset$.
Because $\LanguageOf{\Regex}\Intersect\LanguageOf{\RegexAlt}=\emptyset$,
and $\LanguageOf{\Regex'}\Intersect\LanguageOf{\RegexAlt'}=\emptyset$,
and because of the previously stated equivalences between regular expressions,
and DNF regular expressions, for each $i\in\RangeIncInc{1}{n}$,
$j\in\RangeIncInc{n+1}{m}$, $\Sequence_i\Intersect\Sequence_j=\emptyset$
and $\SequenceAlt_i\Intersect\SequenceAlt_j=\emptyset$.
So, we have for all $i\neq j$, $i,j\in\RangeIncInc{1,n+m}$,
$\Sequence_i \Intersect \Sequence_j = \emptyset$ and
$\SequenceAlt_i \Intersect \SequenceAlt_j = \emptyset$.
We can build a new permutation $\Permutation$ defined as
\[
\Permutation(x)=
\begin{cases}
\Permutation_1(x) & \text{ if $x\in\RangeIncInc{1}{n}$}\\
\Permutation_2(x-n)+n & \text{ if $x\in\RangeIncInc{n+1}{n+m}$}
\end{cases}
\]
Consider the derivation
\begin{mathpar}
\inferrule[]
{
\Derivation_1\\
\ldots\\
\Derivation_{n+m}\\
i\neq j\Rightarrow \Sequence_i\Intersect\Sequence_j=\emptyset\\
i\neq j\Rightarrow \Sequence_i\Intersect\SequenceAlt_j=\emptyset\\
}
{
(\DNFLensOf{\SequenceLens_1\SequenceLensSep\ldots
\SequenceLensSep\SequenceLens_{n+m}},\Permutation) \OfType\\
\DNFOf{\Sequence_1\DNFSep\ldots\DNFSep\Sequence_{n+m}} \Leftrightarrow
\DNFOf{\SequenceAlt_{\sigma(1)}\DNFSep\ldots\DNFSep\SequenceAlt_{\sigma(n+m)}}
}
\end{mathpar}

We can prove the resulting DNF regular expressions are equivalent to
the original regular expressions.

\begin{tabular}{@{}L@{}L@{}}
\DNFOf{\Sequence_0\DNFSep\ldots\DNFSep\Sequence_{n+m}} & =
\DNFOf{\Sequence_1\DNFSep\ldots\DNFSep\Sequence_n}\OrDNF\\
& \hspace{1em}
\DNFOf{\Sequence_{n+1}\DNFSep\ldots\DNFSep\Sequence_{n+m}}\\
& =
\OrDNFOf{\DNFRegex_1}{\DNFRegex_2}
\end{tabular}\\
So by Lemma (TODO: find ref),
$\LanguageOf{\DNFOf{\Sequence_0\DNFSep\ldots\DNFSep\Sequence_{n+m}}}=
\LanguageOf{\RegexOr{\Regex_1}{\Regex_2}}$

\begin{tabular}{@{}L@{}L@{}}
\DNFOf{\SequenceAlt_{\Permutation(0)}\DNFSep\ldots\DNFSep
\SequenceAlt_{\Permutation(n+m)}} & =
\DNFOf{\SequenceAlt_{\Permutation(1)}\DNFSep\ldots\DNFSep
\SequenceAlt_{\Permutation(n)}}\OrDNF\\
& \hspace{1em}
\DNFOf{\Sequence_{\Permutation(n+1)}\DNFSep\ldots\DNFSep
\Sequence_{\Permutation(n+m)}}\\
& =
\DNFOf{\SequenceAlt_{\Permutation_1(1)}\DNFSep\ldots\DNFSep
\SequenceAlt_{\Permutation_1(n)}}\OrDNF\\
& \hspace{1em}
\DNFOf{\Sequence_{n+\Permutation_2(1)}\DNFSep\ldots\DNFSep
\Sequence_{n+\Permutation_2(m)}}\\
& =
\OrDNFOf{\DNFRegexAlt_1}{\DNFRegexAlt_2}
\end{tabular}\\
So by Lemma (TODO: find ref),
$\LanguageOf{\DNFOf{\SequenceAlt_{\Permutation(0)}\DNFSep\ldots\DNFSep
\SequenceAlt_{\Permutation(n+m)}}}=
\LanguageOf{\RegexOr{\RegexAlt_1}{\RegexAlt_2}}$

Furthermore, we can prove it has the same semantics.

\begin{tabular}{@{}L@{}L@{}}
\SemanticsOf{\DNFLensOf{\Sequence_1\DNFLensSep\ldots\DNFLensSep\Sequence_{n+m}},
\Permutation} & =
\SetOf{(\String,\StringAlt)\SuchThat
(\String,\StringAlt)\in\SemanticsOf{\Sequence_i}\\
& \hspace{3em}
\text{for some $i\in\RangeIncInc{1}{n+m}$}}\\
& =
\SetOf{(\String,\StringAlt)\SuchThat
(\String,\StringAlt)\in\SemanticsOf{\Sequence_i}\\
& \hspace{3em}
\text{for some $i\in\RangeIncInc{1}{n}$}\\
& \hspace{2em}\BooleanOr(\String,\StringAlt)\in\SemanticsOf{\Sequence_i}\\
& \hspace{3em}\text{for some $i\in\RangeIncInc{n+1}{n+m}$}}\\
& =
\SetOf{(\String,\StringAlt)\SuchThat
(\String,\StringAlt)\in\SemanticsOf{\Lens_1}\\
& \hspace{2em}\BooleanOr(\String,\StringAlt)\in\SemanticsOf{\Lens_2}}\\
& =
\SemanticsOf{\OrLens{\Lens_1}{\Lens_2}}
\end{tabular}

Let the last rule be an application of \ComposeLensRule{}, so we have
$\Lens_1 \OfType \Regex_1 \Leftrightarrow \Regex_2$, and
$\Lens_2 \OfType \Regex_2 \Leftrightarrow \Regex_3$,
and $\ComposeLens{\Lens_1}{\Lens_2} \OfType \Regex_1 \Leftrightarrow \Regex_3$.
By induction assumption, there is a typing of
$\DNFLens_1 \OfType \DNFRegex_1 \Leftrightarrow \DNFRegex_2$,
and $\DNFLens_2 \OfType \DNFRegex_2 \Leftrightarrow \DNFRegex_3$, where
$\SemanticsOf{\DNFLens_1}=\SemanticsOf{\Lens_1}$,
$\SemanticsOf{\DNFLens_2}=\SemanticsOf{\Lens_2}$,
$\LanguageOf{\DNFRegex_1}=\LanguageOf{\Regex_1}$,
$\LanguageOf{\DNFRegex_2}=\LanguageOf{\Regex_2}$, and
$\LanguageOf{\DNFRegex_3}=\LanguageOf{\Regex_3}$.
By Lemma~\ref{lem:composition-completeness},
there is a lens $\DNFLens \OfType \DNFRegex_1 \Leftrightarrow \DNFRegex_3$,
such that $\SemanticsOf{\DNFLens}=\SetOf{(\String_1,\String_3)\SuchThat
\exists \String_2 (\String_1,\String_2)\in\SemanticsOf{\DNFLens_1}\BooleanAnd
(\String_2,\String_3)\in\SemanticsOf{\DNFLens_2}}=
\SemanticsOf{\ComposeLens{\Lens_1}{\Lens_2}}$.

Let the last rule be an application of \RetypeLensRule{}.
\begin{mathpar}
\inferrule[]
{
\Lens \OfType \Regex \Leftrightarrow \RegexAlt \\
\Regex \equiv \Regex'\\
\RegexAlt \equiv \RegexAlt'
}
{
\Lens \OfType \Regex' \Leftrightarrow \RegexAlt'
}
\end{mathpar}
By induction assumption, there exists a derivation

\begin{mathpar}
\inferrule[]
{
\Derivation
}
{
\DNFLens \OfType \DNFRegex \Leftrightarrow \DNFRegexAlt
}
\end{mathpar}

where \LanguageOf{\DNFRegex}=\LanguageOf{\Regex}
and \LanguageOf{\DNFRegexAlt}=\LanguageOf{\RegexAlt},
and \SemanticsOf{\DNFLens}=\SemanticsOf{\Lens}.
Because \LanguageOf{\Regex'}=\LanguageOf{\Regex}
and \LanguageOf{\RegexAlt'}=\LanguageOf{\RegexAlt},
we know \LanguageOf{\Regex'}=\LanguageOf{\DNFRegex}
and \LanguageOf{\RegexAlt'}=\LanguageOf{\DNFRegexAlt'}.
\end{proof}

With some additional lemmas related to an equational theory of
DNF Regular Expressions, we are able to prove an even stronger result.

We can use the equational theory for unambiguous regular expressions, and the conversion
from regular expressions into DNF regular expressions, to create an equational
theory for DNF regular expressions.

\begin{theorem}[Unambiguous Regex Equational Theory]
The Equational Theory for Unambiguous Regular Expressions is the Equational
Theory for Regular Expressions without \StarstarRule{}.
\end{theorem}
\begin{proof}
TODO: Same approach of showing things commute, maybe?  Need more thought here.
\end{proof}

\begin{definition}[Equational Theory for Unambiguous DNF Regular Expressions]
\leavevmode\\
\begin{tabular}{lrcl}
\EqualityRule{} & \Regex{} & $\equiv$ & \Regex{} \\
\OrCommutativityRule{} & \OrDNFOf{\DNFRegex{}}{\DNFRegexAlt{}} & $\equiv$
& \OrDNFOf{\DNFRegexAlt{}}{\DNFRegex{}}\\
\SumstarRule{} & \DNFOf{\SequenceOf{\StarOf{(\OrDNFOf{\DNFRegex{}}{\DNFRegexAlt{}})}}} & $\equiv$ & \DNFOf{\SequenceOf{\StarOf{(\ConcatDNFOf{\DNFOf{\SequenceOf{\StarOf{\DNFRegex{}}}}}{\DNFRegexAlt{}})}}}\\
& & & \ConcatDNF\DNFOf{\SequenceOf{\StarOf{\DNFRegex{}}}}\\
\ProductstarRule{} & \StarOf{(\ConcatDNFOf{\DNFRegex{}}{\DNFRegexAlt{}})} & $\equiv$ & \DNFOf{\SequenceOf{\EmptyString{}}}\OrDNF(\DNFRegex{}\\
& & & \ConcatDNF\DNFOf{\SequenceOf{\StarOf{(\DNFRegexAlt{}\ConcatDNF\DNFRegex{})}}}\ConcatDNF\DNFRegexAlt{})\\
\DicyclicityRule{} & \DNFOf{\SequenceOf{\StarOf{(\OrDNFOf{\DNFRegex}{\DNFRegexAlt})}}} & $\equiv$ & $\DNFLeft\SequenceLeft((\OrDNFOf{\DNFRegex}{\DNFRegexAlt})\ConcatDNF\DNFRegexAlt\OrDNF$\\
& & & ${(\DNFRegex\ConcatDNF\StarOf{\DNFRegexAlt})}\ConcatDNF\DNFRegex)\Star{}\SequenceRight\DNFRight\ConcatDNF$\\
& & & $(\EmptyString\OrDNF(\OrDNFOf{\DNFRegex}{\DNFRegexAlt})\ConcatDNF$\\
& & & $({(\ConcatDNFOf{\DNFRegex}{\StarOf{\DNFRegexAlt}})}^0\OrDNF\ldots$\\
& & & $\OrDNF{(\ConcatDNFOf{\DNFRegex}{\StarOf{\DNFRegexAlt}})}^n))$
\end{tabular}
\end{definition}

\begin{theorem}[Soundness and Completeness of Equational Theory for Unambiguous DNF Regular
Expressions]
\end{theorem}
\begin{proof}
The Equational Theory of Unambiguous Regular Expressions stems merely an
application of the mapping from regular expressions to DNF regular expressions
to the equational theory of regular expressions.
As this mapping is surjective, we know that for a sequence of rules which show
that two regular expressions are equivalent, we can apply the DNF version
of those rules to obtain a sequence of DNF regular expressions, which are
the DNF versions of the regular expressions.  Furthermore, because
we know that the mapping from regular expressions to DNF regular expressions
is sound, we know that this equational theory is sound.
\end{proof}

\begin{lemma}[Equivalence of DNF permutations]
\label{lem:dnfperm}
For all permutations, \Permutation{},
$\LanguageOf{\DNFOf{\Sequence_1\DNFSep\ldots\DNFSep\Sequence_n}}=
\LanguageOf{\DNFOf{\Sequence_{\Permutation(1)}\DNFSep\ldots\DNFSep
\Sequence_{\Permutation(n)}}}$
\end{lemma}
\begin{proof}
Repeated application of \OrCommutativityRule{}.
\end{proof}

\begin{lemma}[Respectfulness of Languages in Rewrites]\leavevmode
\label{lem:language-rewrite-safe}
\begin{enumerate}
\item If $\Atom\RewriteAtom\DNFRegex$, then
$\LanguageOf{\Atom}=\LanguageOf{\DNFRegex}$.
\item If $\DNFRegex\RewriteDNF\DNFRegex'$, then
$\LanguageOf{\DNFRegex}=\LanguageOf{\DNFRegex'}$.
\end{enumerate}
\end{lemma}
\begin{proof}
By mutual induction, use of \ref{lem:dnfperm}, and use of the equational
theory of unambiguous regular expressions.
\end{proof}

\begin{definition}[Non-rewrite DNF Lens equivalence relation]
We say $\Regex\sim\RegexAlt$, if there exists a dnf lens $\DNFLens$, such that
$\DNFLens \OfType \ToDNFRegex(\Regex) \Leftrightarrow \ToDNFRegex(\RegexAlt)$.
\end{definition}

\begin{lemma}[Rewrites respect DNF Lens equivalence relation]
\label{lem:rewrite-respect}
If $\Regex\sim\RegexAlt$, $\Regex\Rewrite\Regex'$, and
$\RegexAlt\Rewrite\RegexAlt'$, then $\Regex'\sim\RegexAlt'$.
Because of this, we can extend \Rewrite{} to operate on \EquivRegexType{}.
\end{lemma}
\begin{proof}
\end{proof}

\begin{lemma}[Rewrites respect the equational theory of regular expressions]
If $\Regex\sim\RegexAlt$, $\Regex\equiv\Regex'$, and
$\RegexAlt\equiv\RegexAlt'$, then $\Regex'\equiv\RegexAlt'$.
Because of this, we can extend $\equiv{}$ to operate on \EquivRegexType{}.
\end{lemma}
\begin{proof}
\end{proof}

\begin{definition}[$\equiv_{\Rewrite}$]
Define $\equiv_{\Rewrite}$ as the equivalence relation generated by \Rewrite{}.
\end{definition}

\begin{lemma}[$\equiv_{\Rewrite} = \equiv$ on \EquivRegexType{}]
\label{lem:rewriteequivequivalence}
Let $\Regex,\Regex'\in\EquivRegexType$.
$\Regex\equiv_{\Rewrite}\Regex'$ if, and only if, $\Regex\equiv\Regex'$.
\end{lemma}
\begin{proof}
\end{proof}

\begin{lemma}[Confluence of rewrites on \EquivRegexType{}]
\label{lem:rewrite-confluence}
Let $\Regex\Rewrite\RegexAlt$, and $\Regex\Rewrite\RegexAlt'$, then there
exists some $\Regex'$ such that $\RegexAlt\StarOf{\Rewrite}\Regex'$,
and $\RegexAlt'\StarOf{\Rewrite}\Regex'$.
\end{lemma}
\begin{proof}
\end{proof}

\begin{lemma}
If $\Regex, \Regex' \in \EquivRegexType$, then if $\Regex\equiv\Regex'$, then
there exists a $\Regex''$ such that $\Regex\StarOf{\Rewrite}\Regex''$ and
$\Regex'\StarOf{\Rewrite}\Regex''$.
\end{lemma}
\begin{proof}
By Lemma~\ref{lem:rewrite-confluence}, Church-Rosser, and
Lemma\ref{lem:rewriteequivequivalence}.
\end{proof}

\begin{lemma}[Semantics Retention through Rewrites]
\label{lem:rewrite-semantic-retention}
If $\Regex,\RegexAlt\in\EquivRegexType$, and
$\Regex\Rewrite\RegexAlt$, then if $\Regex_1,\Regex_2\in\Regex$, and
$\Lens \OfType \Regex_1 \Leftrightarrow \Regex_2$, then there exists
$\RegexAlt_1,\RegexAlt_2\in\RegexAlt$ such that $\Regex_1\Rewrite\RegexAlt_1$,
and $\Regex_2\Rewrite\RegexAlt_2$, and
$\Lens' \OfType \RegexAlt_1 \Leftrightarrow \RegexAlt_2$,
and $\SemanticsOf{\Lens'}=\SemanticsOf{\Lens}$.
\end{lemma}
\begin{proof}
\end{proof}

\begin{lemma}[Semantics Retention Within Equivalence Class]
\label{lem:eqclass-semantic-retention}
If $\Regex^{\sim}\in\EquivRegexType$, and
$\Regex,\RegexAlt\in\Regex^{\sim}$, and there exists a DNF Lens
$\DNFLens \OfType \ToDNFRegex(\Regex) \Leftrightarrow \ToDNFRegex(\RegexAlt)$,
then if $\Regex',\RegexAlt'\in\Regex^{\sim}$,
with $\LanguageOf{\Regex}=\LanguageOf{\Regex'}$,
and $\LanguageOf{\RegexAlt}=\LanguageOf{\RegexAlt'}$,
then there exists a DNF lens
$\DNFLens' \OfType \ToDNFRegex(\Regex') \Leftrightarrow \ToDNFRegex(\RegexAlt')$,
such that $\SemanticsOf{\DNFLens'}=\SemanticsOf{\DNFLens}$.
\end{lemma}
\begin{proof}
\end{proof}

Now we can prove the stronger result.
\begin{theorem}[Strong Completeness of DNF Lenses]
If there exists a derivation of
$\Lens \OfType \MapsBetweenTypeOf{\Regex}{\RegexAlt}$,
then there exists a derivation of
$\DNFLens \OfType \MapsBetweenTypeOf{\Regex}{\RegexAlt}$,
such that 
$\SemanticsOf{\DNFLens}=\SemanticsOf{\Lens}$.
\end{theorem}
\begin{proof}
By Theorem~\ref{thm:dnflc} (Weak Completeness), there exists a $\Regex_1, \RegexAlt_1, \DNFLens$
such that
$\DNFLens_1 \OfType \ToDNFRegex(\Regex_1) \Leftrightarrow \ToDNFRegex(\RegexAlt_1)$.
As such, $\Regex_1$ and $\RegexAlt_1$ are in the same equivalence class,
$\Regex_1^{\sim}$.

Let $\Regex^\sim$ and $\RegexAlt^\sim$ be the equivalence classes of
$\Regex$ and $\RegexAlt$, respectively.
$\Regex^{\sim}\equiv\Regex_1^{\sim}\equiv\RegexAlt^{\sim}$, so there exists
a $\Regex_2^{\sim}$ such that $\Regex^{\sim}\Rewrite\Regex_2^{\sim}$,
$\Regex_1^{\sim}\Rewrite\Regex_2^{\sim}$, and
$\RegexAlt^{\sim}\Rewrite\Regex_2^{\sim}$.

From Lemma~\ref{lem:rewrite-semantic-retention}, there exists a $\DNFLens_2 \OfType
\ToDNFRegex(\Regex_2) \Leftrightarrow \ToDNFRegex(\RegexAlt_2)$ such that
$\SemanticsOf{\DNFLens_2}=\SemanticsOf{\DNFLens_1}$.
Furthermore, there exists a $\Regex_3,\RegexAlt)3\in\Regex_2^{\sim}$
such that $\Regex\StarOf{\Rewrite}\Regex_3$,
and $\RegexAlt\StarOf{\Rewrite}\RegexAlt_3$.
By Lemma~\ref{lem:eqclass-semantic-retention}, there exists a lens
$\DNFLens_3 \OfType
\ToDNFRegex(\Regex_3) \Leftrightarrow \ToDNFRegex(\RegexAlt_3)$,
with $\SemanticsOf{\DNFLens_3}=\SemanticsOf{\Lens}$.

So we have the final typing derivation
\begin{mathpar}
\inferrule*[left=\Derivation]
{
\inferrule*
{
\inferrule*[vdots=1.5em]
{
}
{
}
}
{
\DNFLens_3 \OfType
\ToDNFRegex(\Regex_3) \Leftrightarrow \ToDNFRegex(\RegexAlt_3)
}
}
{
\DNFLens_3 \OfType \Regex_3 \Leftrightarrow \RegexAlt_3
}\\

\inferrule*
{
\Derivation\\
\Regex\StarOf{\Rewrite}\Regex_3\\
\DNFRegex\StarOf{\Rewrite}\DNFRegex_3
}
{
\DNFLens_3 \OfType \Regex \Leftrightarrow \RegexAlt
}
\end{mathpar}
\end{proof}





Furthermore, they only express things expressible in the language of lenses.
\begin{restatable}[Soundness of DNF Lenses]{theorem}{dnfls}
\label{thm:dnfls}
If there exists a derivation of $\DNFLens \OfType \MapsBetweenTypeOf{\DNFRegex}{\DNFRegexAlt} \HasSemantics \PutRight,\PutLeft$,
then there exists a derivation of $\Lens \OfType \MapsBetweenTypeOf{\Regex}{\RegexAlt} \HasSemantics \PutRight,\PutLeft$ such that
$\LanguageOf{\Regex}=\LanguageOf{\DNFRegex}$, and
$\LanguageOf{\RegexAlt}=\LanguageOf{\DNFRegexAlt}$.
\end{restatable}
We will prove a couple of lemmas first.

\begin{lemma}[Creation of Lens from Identity Perm Sequence Lens]
\label{lem:id-clause}
Let $\Sequence=\SequenceOf{\String_0 ; \Atom_1 ; \ldots ; \Atom_n; \String_n}$,
and $\SequenceAlt=\SequenceOf{\StringAlt_0 ; \AtomAlt_1 ; \ldots ; \AtomAlt_n ; \StringAlt_n}$.
If $(\SequenceLensOf{(\String_0,\StringAlt_0);\AtomLens_1 ; \ldots ;
\AtomLens_n;(\String_n,\StringAlt_n)},id) \OfType
\Sequence \Leftrightarrow \SequenceAlt$,
and for each $\AtomLens_i \OfType \Atom_i \Leftrightarrow \AtomAlt_i$,
there exists a $\Lens_i \OfType \ToRegex(\Atom_i) \Leftrightarrow \ToRegex(\AtomAlt_i)$ such that
then there exists a $\Lens \OfType \ToRegex(\Sequence) \Leftrightarrow \ToRegex(\DNFRegexAlt)$ such that
$\SemanticsOf{\Lens} =
\SemanticsOf{(\SequenceLensOf{\AtomLens_1 ; \ldots ; \AtomLens_n},id)}$,
\begin{proof}

By induction on $n$.

Let $n=0$, $(\SequenceLensOf{(\String_0,\StringAlt_0)},id) \OfType
\SequenceOf{\String_0} \Leftrightarrow \SequenceOf{\StringAlt_0}$.
Then consider
\begin{mathpar}
\inferrule[]
{
}
{
\ConstLens{\String_0}{\StringAlt_0}\OfType\String_0\Leftrightarrow\StringAlt_0
}
\end{mathpar}

$\String_0=\ToRegex(\SequenceOf{\String_0})$,
and
$\StringAlt_0=\ToRegex(\SequenceOf{\StringAlt_0})$.
$\SemanticsOf{\ConstLens{\String_0}{\StringAlt_0}}=
\SetOf{\String_0,\StringAlt_0}=
\SemanticsOf{\SequenceOf{(\String_0,\StringAlt_0)},id)}$.

Let $n>0$.
Let $\Sequence'=\SequenceOf{\String_0\SequenceSep\Atom_1\SequenceSep
\ldots\SequenceSep\Atom_{n-1}\SequenceSep\String_{n-1}}$,
and $\SequenceAlt'=\SequenceOf{\StringAlt_0\SequenceSep\AtomAlt_1\SequenceSep
\ldots\SequenceSep\AtomAlt_{n-1}\SequenceSep\StringAlt_{n-1}}$
By induction assumption, there exists a typing derivation
\begin{mathpar}
\inferrule*
{
\inferrule*[vdots=1.5em]
{
}
{
}
}
{
\Lens\OfType\ToRegex(\Sequence')\Leftrightarrow\ToRegex(\SequenceAlt')
}
\end{mathpar}
Satisfying $\SemanticsOf{\Lens}=
\SemanticsOf{(\SequenceLensOf{(\String_0,\StringAlt_0);\AtomLens_1 ;
\ldots ; \AtomLens_n;(\String_n,\StringAlt_{n-1})},id)}$

By problem statement, there exists a typing derivation
\begin{mathpar}
\inferrule*[left=$\Derivation_n$]
{
\inferrule*[vdots=1.5em]
{
}
{
}
}
{
\Lens_{\AtomLens_{n}} \OfType
\ToRegex(\Atom_{n}) \Leftrightarrow \ToRegex(\AtomAlt_{n})
}
\end{mathpar}
satisfying $\SemanticsOf{\Lens_{\Atom_n}}=\SemanticsOf{\Atom_n}$.

Consider the following lens typing
\begin{mathpar}
\inferrule*[left=\Derivation{}]
{
\Derivation_n\\
\inferrule*
{
}
{
\ConstLens{\String_n}{\StringAlt_n}
\OfType
\String_n \Leftrightarrow \StringAlt_n
}
}
{
\ConcatLens{\Lens_{\AtomLens_n}}{\ConstLens{\String_n}{\StringAlt_n}}
\OfType\\
\ToRegex(\Atom_n)\Concat\String_n \Leftrightarrow
\ToRegex(\AtomAlt_n)\Concat\StringAlt_n
}

\inferrule*{
\inferrule*
{
\inferrule*[vdots=1.5em]
{
}
{
}
}
{
\Lens\OfType\ToRegex(\Sequence) \Leftrightarrow \ToRegex(\SequenceAlt)
}\\
\Derivation{}
}
{
\ConcatLens
{\Lens}
{\ConcatLens{\Lens_{\AtomLens_n}}{\ConstLens{\String_n}{\StringAlt_n}}}
\OfType\\
\ToRegex(\Sequence)\Concat\ToRegex(\Atom_n)\Concat\String_n \Leftrightarrow\\
\ToRegex(\SequenceAlt)\Concat\ToRegex(\AtomAlt_n)\Concat\StringAlt_n
}
\end{mathpar}

\SemanticsOf{\ConcatLens
{\Lens}
{\ConcatLens{\Lens_{\AtomLens_n}}{\ConstLens{\String_n}{\StringAlt_n}}}}=\\
\SetOf{(\String'\Concat\String''\String_n,\StringAlt'\Concat\StringAlt''\StringAlt_n)
\SuchThat
(\String',\StringAlt')\in\SemanticsOf{\Lens}\BooleanAnd
(\String'',\StringAlt'')\in\SemanticsOf{\Lens_{\AtomLens_n}}}=\\
\SetOf{
(\String_0\Concat\String_0'\Concat\ldots
\Concat\String_{n-1}'\Concat\String_{n-1}
\Concat \String_n \Concat \String_n'
,
\StringAlt_0\Concat\StringAlt_0'\Concat\ldots
\Concat\StringAlt_{n-1}'\Concat\StringAlt_{n-1}
\Concat \String_n \Concat \String_n')\SuchThat\\
\hspace*{1em}\String_i'\in\Atom_i\BooleanAnd\StringAlt_i'\in\AtomAlt_i}=\\
\SemanticsOf{(\SequenceLensOf{(\String_0,\StringAlt_0);\AtomLens_1 ;
\ldots ; \AtomLens_n;(\String_n,\StringAlt_{n-1})},id)}
\end{proof}
\end{lemma}

\begin{definition}[Adjacent Swapping Permutation]
Let $\sigma_{i} \in S_n$ be the permutation where
$\sigma_{i}(i) = i+1$, $\sigma_{i}(i+1) = i$, $\sigma_{i,j}(k\neq i,i+1) = k$
\end{definition}

\begin{theorem}[Unambiguity of Sep]
Given an alphabet, \Alphabet{}, we can expand the alphabet by adding in a different
element, denoted \Sep{}.
For any sequence of languages $\Language_1\in\StarOf{\Alphabet}$, \ldots,
$\Language_n\in\StarOf{\Alphabet}$, $\Sep\Language_1\Sep\ldots\Sep\Language_i
\UnambigConcat
\Sep\Language_{i+1}\ldots\Sep\Language_n\Sep$.
\end{theorem}
\begin{proof}
We know that all of the string before the $i+1$st instance of $\Sep$ is
part of the first, and the rest is part of the second.
\end{proof}

\begin{lemma}[Expressibility of Adjacent Swapping Permutation Lens]
Let $\sigma_i$ be an adjacent element swapping permutation and
$\SequenceOf{\Sep;\Atom_1;\ldots;\Atom_n;\Sep}$ be a sequence with $\Sep$ as
the strings between atoms.
There exists regular expressions $\Regex$ and $\RegexAlt$, and a lens $\Lens$,
such that $\LanguageOf{\Regex}=\LanguageOf{[\Sep;\Atom_1;\ldots;\Atom_n;\Sep]}$,
$\LanguageOf{\RegexAlt}=\LanguageOf{[\Sep;\Atom_{\sigma(1)};\ldots;\Atom_{\sigma(n)};\Sep]}$, and
$\Lens \OfType \Regex \Leftrightarrow \RegexAlt$,
with semantics
$\SemanticsOf{\Lens}=
\SetOf{(\Sep\Concat\String_1\Concat\ldots\Concat\String_n\Concat\Sep,
\Sep\Concat\String_{\sigma_i(1)}\Concat\ldots\Concat\String_{\sigma_i(n)}\Concat\Sep)
\SuchThat\String_i\in\LanguageOf{\Atom_i}}$
\begin{proof}
By the soundness of regular expressions, define
$\Regex_1, \Regex_2, \Regex_3, \Regex_4$ as
$\Regex_1=\ToRegex([\Sep;\Atom_1;\ldots;\Atom_{i-1};\Sep])$,
$\Regex_2=\ToRegex(\Atom_i)$,
$\Regex_3=\ToRegex(\Atom_{i+1})$, and
$\Regex_4=\ToRegex([\Sep;\Atom_{i+1};\ldots;\Atom_{n};\Sep])$.
Consider the following deduction
\begin{mathpar}

\inferrule*[left=\Derivation{}]
{
\inferrule*
{
}
{
\IdentityLensShort \OfType \Sep \Leftrightarrow \Sep
}
\inferrule*
{
}
{
\IdentityLensShort \OfType \Regex_3 \Leftrightarrow \Regex_3
}
}
{
\SwapLensShort{\IdentityLensShort}{\IdentityLensShort} \OfType 
\String_i\Concat\Regex_3 \Leftrightarrow \Regex_3\Concat\String_i
}

\inferrule*[left=\Derivation{}']
{
\inferrule*
{
}
{
\IdentityLensShort \OfType \Regex_2 \Leftrightarrow \Regex_2
}\\
\Derivation
}
{
\SwapLensShort{\IdentityLensShort}{\SwapLensShort{\IdentityLensShort}{\IdentityLensShort}} \OfType
\Regex_2\Concat\String_i\Concat\Regex_3 \Leftrightarrow \Regex_3\Concat\String_i\Concat\Regex_2
}

\inferrule*[left=\Derivation{}'']
{
\inferrule*
{
}
{
\IdentityLensShort \OfType \Regex_1 \Leftrightarrow \Regex_1
}\\
\Derivation{}'
}
{
\ConcatLensShort{\IdentityLensShort}{\SwapLensShort{\IdentityLensShort}{\SwapLensShort{\IdentityLensShort}{\IdentityLensShort}}} \OfType
\Regex_1\Concat\Regex_2\Concat\String_i\Concat\Regex_3 \Leftrightarrow \Regex_1\Concat\Regex_3\Concat\String_i\Concat\Regex_2
}


\inferrule*
{
\Derivation{}''\\
\inferrule*
{
}
{
\IdentityLens \OfType \Regex_4 \Leftrightarrow \Regex_4
}
}
{
\ConcatLensShort{\ConcatLensShort{\IdentityLensShort}{\SwapLensShort{\IdentityLensShort}{\SwapLensShort{\IdentityLensShort}{\IdentityLensShort}}}}{\IdentityLensShort} \OfType\\
\Regex_1\Concat\Regex_2\Concat\String_i\Concat\Regex_3\Concat\Regex_4 \Leftrightarrow \Regex_1\Concat\Regex_3\Concat\String_i\Concat\Regex_2\Concat\Regex_4
}
\end{mathpar}

By inspection, the lens $
\ConcatLensShort{\ConcatLensShort{\IdentityLensShort}{\SwapLensShort{\IdentityLensShort}{\SwapLensShort{\IdentityLensShort}{\IdentityLensShort}}}}{\IdentityLensShort} \OfType
\Regex_1\Concat\Regex_2\Concat\String_i\Concat\Regex_3\Concat\Regex_4 \Leftrightarrow \Regex_1\Concat\Regex_3\Concat\String_i\Concat\Regex_2\Concat\Regex_4$
satisfies $\LanguageOf{\Regex_1\Concat\Regex_2\Concat\String_i\Concat\Regex_3\Concat\Regex_4}=\LanguageOf{\SequenceOf{\Sep;\Atom_1;\ldots;\Atom_n;\Sep}}$ and
$\LanguageOf{\Regex_1\Concat\Regex_3\Concat\String_i\Concat\Regex_2\Concat\Regex_4}=\LanguageOf{\SequenceOf{\Sep;\Atom_{\sigma_i(1)};\ldots;\Atom_{\sigma_i(n)};\Sep}}$
and has the desired semantics.
\end{proof}
\end{lemma}

\begin{lemma}[Expressibility of Permutation]
\label{lem:perm-exp}
The language of lenses can express $([(\Sep,\Sep) ; \IdentityLens ; \ldots ; \IdentityLens ; (\Sep,\Sep)], \sigma)$
for any permutation $\sigma$.
\begin{proof}
Let $\sigma$ be a permutation.
Consider the clause lens\\ $([(\Sep,\Sep) ; \IdentityLens ; \ldots ; \IdentityLens ; (\Sep,\Sep)], \sigma_{i})$.
From algebra, we know that the group of permutations is generated by all
adjacent swaps $\sigma_i = (i,i+1)$.
So there exists an adjacency swap decomposition of $\sigma = \sigma_{i_1}\ldots\sigma_{i_m}$.
Consider the dnf lens\\ $([(\Sep,\Sep) ; \IdentityLens ; \ldots ; \IdentityLens ; (\Sep,\Sep)], \sigma_{i_j})$ for each $\sigma_{i_j}$.
By the above lemma, there exists a $\Lens_j$ for each of these adjacency swaps.
Consider the lens $\Lens = \ComposeLens{\Lens_{1}}{\ComposeLens{\Lens_{2}}{\ldots \ComposeLens{}{\Lens_{m}}}}$,
with retypes put in where necessary to handle different associations.

\end{proof}
\end{lemma}

\begin{lemma}[Creation of Lens from Identity Perm DNF Lens]
\label{lem:id-dnf}
If $(\DNFLensOf{\SequenceLens_1 ; \ldots ; \SequenceLens_n},id) \OfType
\DNFOf{\Sequence_1 ; \ldots ; \Sequence_n} \Leftrightarrow \DNFOf{\SequenceAlt_1 ; \ldots ; \SequenceAlt_n}$,
and for each $\SequenceLens_i \OfType \Sequence_i \Leftrightarrow \SequenceAlt_i$,
there exists a $\Lens_i \OfType \ToRegex(\Sequence_i) \Leftrightarrow \ToRegex(\SequenceAlt_i)$ such that
$\SemanticsOf{\Lens_i} = \SemanticsOf{\SequenceLens_i}$,
then there exists a $\Lens \OfType \ToRegex(\DNFOf{\Sequence_1 ; \ldots ; \Sequence_n}) \Leftrightarrow \ToRegex(\DNFOf{\SequenceAlt_1 ; \ldots ; \SequenceAlt_n})$ such that $\SemanticsOf{\Lens} = \SemanticsOf{([\SequenceLens_1 ; \ldots ; \SequenceLens_n],id)}$.
\begin{proof}
By induction on n

Let $n=0$.
$\DNFLensOf{} \OfType \DNFOf{} \Leftrightarrow \DNFOf{}$.  Then consider $\Lens$ = $\IdentityLens$.
\\

Let $n>0$.
By induction assumption, I know
there exists a $\Lens \OfType \ToRegex(\DNFOf{\Sequence_1 ; \ldots ; \Sequence_n}) \Leftrightarrow \ToRegex(\DNFOf{\SequenceAlt_1 ; \ldots ; \SequenceAlt_n})$ such that
$\SemanticsOf{\Lens} = \SemanticsOf{(\DNFLensOf{\SequenceLens_1 ; \ldots ; \SequenceLens_{n-1}},id)}$.
By assumption, I know that there exists a $\Lens_{n} \OfType \ToRegex(\Sequence_{n}) \Leftrightarrow \ToRegex(\SequenceAlt_{n})$
where $\SemanticsOf{\Lens_{n}}=\SemanticsOf{\SequenceLens_{n}}$.\\
Consider the lens $\OrLens{\Lens}{\Lens_{n}} \OfType \RegexOr{\ToRegex(\DNFOf{\Sequence_1 ; \ldots ; \Sequence_{n-1}})}{\ToRegex(\Sequence_{n})} \Leftrightarrow \RegexOr{\ToRegex(\DNFOf{\SequenceAlt_1 ; \ldots ; \SequenceAlt_{n-1}})}{\ToRegex(\SequenceAlt_{n})}$.\\
by definition of $\ToRegex$, we know $\RegexOr{\ToRegex(\DNFOf{\Sequence_1 ; \ldots ; \Sequence_{n-1}})}{\ToRegex(\Sequence_{n})}=\ToRegex(\DNFOf{\Sequence_1;\ldots;\Sequence_n})$, and $\RegexOr{\ToRegex(\DNFOf{\SequenceAlt_1 ; \ldots ; \SequenceAlt_{n-1}})}{\ToRegex(\SequenceAlt_{n})}=\ToRegex(\DNFOf{\SequenceAlt_1;\ldots;\SequenceAlt_n})$

$\PutRightOf{\OrLens{\Lens}{\Lens_n}}=\SetOf{(\String,\StringAlt)\SuchThat
(\String,\StringAlt)\in\Lens\BooleanOr(\String,\StringAlt)\in\Lens_n}$\\
$=\SetOf{(\String,\StringAlt)\SuchThat
(\String,\StringAlt)\in\DNFLensOf{\SequenceLens_1;\ldots;\SequenceLens_{n-1}}
\BooleanOr(\String,\StringAlt)\in\DNFLensOf{\SequenceLens_n}}$\\
$=\SetOf{(\String,\StringAlt)\SuchThat
(\String,\StringAlt)\in\SequenceLens_i}$.
\end{proof}
\end{lemma}

\begin{lemma}[Rewrites Respecting Language]
\label{lem:rrl}
If $\DNFRegex\Rewrite\DNFRegexAlt$, then $\LanguageOf{\DNFRegex}=\LanguageOf{\DNFRegexAlt}$
\end{lemma}
\begin{proof}
Each of the rewrite rules are merely an application of one direction of one of
the regular expression star equivalences.
\end{proof}

\begin{lemma}[Soundness of DNF, Sequence, and Atom Lenses]\leavevmode
\label{lem:dnfcal}
\begin{enumerate}
\item Let \DNFRegex{} and \DNFRegexAlt{} be two dnf regular expressions, and $\DNFLens \OfType \DNFRegex \Leftrightarrow \DNFRegexAlt$.  Then there exists a \Lens{} such that $\Lens \OfType \ToRegex(\DNFRegex) \Leftrightarrow \ToRegex(\DNFRegexAlt)$, \SemanticsOf{\Lens}=\SemanticsOf{\DNFLens}

\item Let \Sequence{} and \SequenceAlt{} be two clauses, and $\SequenceLens \OfType \Sequence \Leftrightarrow \SequenceAlt$.  Then there exists a \Lens{} such that $\Lens \OfType \ToRegex(\Sequence) \Leftrightarrow \ToRegex(\SequenceAlt)$, \SemanticsOf{\Lens}=\SemanticsOf{\SequenceLens}.

\item Let \Atom{} and \AtomAlt{} be two atoms, and $\AtomLens \OfType \Atom \Leftrightarrow \AtomAlt$.  Then there exists a \Lens{}, such that $\Lens \OfType \ToRegex(\Atom) \Leftrightarrow \ToRegex(\AtomAlt)$, \SemanticsOf{\Lens}=\SemanticsOf{\AtomLens}.
\end{enumerate}
\begin{proof}
By mutual induction on the structure of the DNF Regex, Sequence, and Atom.\\
\\
Let $(\DNFLensOf{\SequenceLens_1;\ldots;\SequenceLens_n},\sigma) \OfType \DNFOf{\Sequence_1;\ldots;\Sequence_n} \Leftrightarrow \DNFOf{\SequenceAlt_{\sigma(1)};\ldots;\SequenceAlt_{\sigma(n)}}$ be formed from an application of $\DNFLensRule$.
By Induction assumption, for each $\SequenceLens_i \OfType \Sequence_i \Leftrightarrow \SequenceAlt_i$ there exists a $\Lens_i \OfType \ToRegex(\Sequence_i) \Leftrightarrow \ToRegex(\SequenceAlt_i)$.\\
By Lemma~\ref{lem:id-dnf} there exists a $\Lens \OfType \ToRegex(\DNFOf{\Sequence_1;\ldots;\Sequence_{n}}) \Leftrightarrow \ToRegex(\DNFOf{\SequenceAlt_1;\ldots;\SequenceAlt_{n}})$ such that $\SemanticsOf{\Lens}=\SemanticsOf{([\SequenceLens_1;\ldots\SequenceLens_n],id)}$,
$\PutRightOf{([\SequenceLens_1;\ldots\SequenceLens_n],id)}=\PutRightOf{([\SequenceLens_1;\ldots\SequenceLens_n],\sigma)}$, as the order has no impact on the semantics of the lens.
$\LanguageOf{[\SequenceAlt_1;\ldots;\SequenceAlt_n]}=\LanguageOf{[\SequenceAlt_{\sigma(1)};\ldots;\SequenceAlt_{\sigma(n)}]}$, as the order has no impact on the semantics of the DNF regular expression.
As such, using retyping, there exists a lens
$\Lens\OfType\ToRegex(\DNFOf{\Sequence_1;\ldots;\Sequence_n}) \Leftrightarrow \ToRegex(\DNFOf{\Sequence_{\sigma(1)};\ldots;\Sequence_{\sigma(n)}})$
with
$\SemanticsOf{\Lens}=\SemanticsOf{([\SequenceLens_1;\ldots\SequenceLens_n],\sigma)}$.\\
\\
Let $\DNFLens \OfType \DNFRegex \Leftrightarrow \DNFRegexAlt$ be formed from
an application of DNF rewrite rules.  By Lemma~\ref{lem:rrl},
we know that if $\DNFRegex\RewriteDNF\DNFRegex'$, then
$\LanguageOf{\DNFRegex}=\LanguageOf{\DNFRegex'}$,
so we can merely use the typing before the rewrite, and do a retype lens to
$\ToRegex(\DNFRegex) \Leftrightarrow \ToRegex(\DNFRegexAlt)$.\\
\\
Let there be a sequence, $(\SequenceLensOf{(\String_0,\StringAlt_0);\AtomLens_1;\ldots;\AtomLens_n;(\String_n,\StringAlt_n)},\sigma \in S_n) \OfType \SequenceOf{\String_0 ; \Atom_1 ; \ldots ; \Atom_n ; \String_n} \Leftrightarrow \SequenceOf{\StringAlt_0; \AtomAlt_{\sigma(1)} ; \ldots ; \AtomAlt_{\sigma(n)} ; \StringAlt_n}$
From induction assumption, I know for each $\AtomLens_i \OfType \Atom_i \Leftrightarrow \AtomAlt_i$, there exists a $\Lens_i \OfType \Regex_i \Leftrightarrow \RegexAlt_i$ such that $\SemanticsOf{\AtomLens_i}=\SemanticsOf{\Lens_i}$, where $\Regex_i=\ToRegex(\Atom_i)$, and $\RegexAlt_i=\ToRegex(\AtomAlt_i)$,
By Lemma~\ref{lem:id-clause}, there exists a $\Lens \OfType \Regex \Leftrightarrow \RegexAlt$ such that $\SemanticsOf{\Lens}=\SemanticsOf{([(\String_0,\StringAlt_0);\AtomLens_1;\ldots;\AtomLens_n;(\String_n,\StringAlt_n)],id)}$,
$\Regex=\ToRegex(\SequenceOf{\String_0;\Atom_1;\ldots;\Atom_n;\String_n})$, and
$\RegexAlt=\ToRegex(\SequenceOf{\StringAlt_0;\AtomAlt_1;\ldots;\AtomAlt_n;\StringAlt_n})$.
By Lemma~\ref{lem:perm-exp} there exists a $\Lens_{\sigma} \OfType \RegexAlt_{\Sep} \Leftrightarrow \RegexAlt_{\Sep}'$,
such that $\SemanticsOf{\Lens_{\sigma}}=\SetOf{(\Sep\Concat\String_1\Concat\Sep\Concat\ldots\Concat\String_n\Concat\Sep,\Sep\Concat\String_{\Permutation(1)}\Concat\Sep\Concat\ldots\Concat\String_{\Permutation(n)}\Concat\Sep)}$,
$\RegexAlt_{\Sep}=\ToRegex(\SequenceOf{\Sep;\Sequence_1;\ldots;\Sequence_n;\Sep})$, and
$\RegexAlt_{\Sep}'=\ToRegex(\SequenceOf{\Sep;\Sequence_{\sigma(1)};\ldots;\Sequence_{\sigma(n)}\Sep})$.
By Lemma~\ref{lem:id-clause}, there exist lenses $\Lens_{\Sep} \OfType \RegexAlt \Leftrightarrow \RegexAlt_{\Sep}$, and
$\Lens_{\Sep'} \OfType \RegexAlt_{\Sep}' \Leftrightarrow \RegexAlt'$, where
$\RegexAlt=\ToRegex(\SequenceOf{\StringAlt_0; \AtomAlt_{\sigma(1)} ; \ldots ; \AtomAlt_{\sigma(n)} ; \StringAlt_n})$, and the semantics of these add and remove the separators, respectivey.
Consider the lens $\Lens_{\Sep}' \Compose \Lens_{\sigma} \Compose \Lens_{\Sep} \Compose \Lens \OfType \Regex \Leftrightarrow \RegexAlt'$, with semantics as intended.\\
\\
$\IterateLens{\DNFLens} \OfType \StarOf{\DNFRegex} \Leftrightarrow \StarOf{\DNFRegexAlt}$\\
From induction assumption, I know that there exists $\Lens \OfType \Regex \Leftrightarrow \RegexAlt$, such that
$\SemanticsOf{\DNFLens}=\SemanticsOf{\Lens}$,
\Regex=\ToRegex(\DNFRegex), and
$\RegexAlt=\ToRegex(\DNFRegexAlt)$.\\
Consider $\IterateLens{\Lens} \OfType \StarOf{\Regex} \Leftrightarrow \StarOf{\RegexAlt}$.\\
By definition, $\StarOf{\Regex}$ and $\StarOf{\RegexAlt}$ are $\ToRegex(\StarOf{\DNFRegex})$
and $\ToRegex(\StarOf{\Regex})$, respectively.

\begin{tabular}{RcL}
\SemanticsOf{\IterateLens{\Lens}} & = &
\SetOf{(\String_0\ldots\String_n,\StringAlt_0\ldots\StringAlt_n)\SuchThat
(\String_i,\StringAlt_i)\in\SemanticsOf{\Lens}}\\
& = &
\SetOf{(\String_0\ldots\String_n,\StringAlt_0\ldots\StringAlt_n)\SuchThat
(\String_i,\StringAlt_i)\in\SemanticsOf{\DNFLens}}\\
& = &
\SemanticsOf{\IterateLens{\DNFLens}}
\end{tabular}
\end{proof}
\end{lemma}

\dnfls*
\begin{proof}

The soundess of DNF lenses follows immediatley from Lemma~\ref{lem:dnfcal}

\end{proof}

\begin{theorem}[Strong DNF Lens Soundness]
Let $\DNFLens \OfType \DNFRegex \Leftrightarrow \DNFRegexAlt$, be the typing
of a DNF lens $\DNFLens$.
and let $\Regex$ and $\RegexAlt$ be regular expressions, such that
$\LanguageOf{\DNFRegex}=\LanguageOf{\Regex}$,
and $\LanguageOf{\DNFRegexAlt}=\LanguageOf{\RegexAlt}$.
There exists a lens $\Lens : \Regex \Leftrightarrow \RegexAlt$, such that
$\SemanticsOf{\Lens}=\SemanticsOf{\DNFLens}$.
\end{theorem}

\begin{proof}
By Theorem~\ref{thm:dnfls}, there exists regular expressions $\Regex'$,
$\RegexAlt'$, such that there exists a lens $\DNFLens$ such there is the
derivation for a typing.
\begin{mathpar}
\inferrule[]
{
\Derivation{}
}
{
\DNFLens \OfType \Regex' \Leftrightarrow \RegexAlt'
}
\end{mathpar}
\SemanticsOf{\DNFLens}=\SemanticsOf{\Lens},
\LanguageOf{\Regex'}=\LanguageOf{\DNFRegex},
and \LanguageOf{\RegexAlt'}=\LanguageOf{\DNFRegexAlt'}.
Because of this, $\Regex'\equiv\Regex$, and $\RegexAlt'\equiv\RegexAlt$.
Consider the typing
\begin{mathpar}
\inferrule*
{
\inferrule*
{
\Derivation{}
}
{
\DNFLens \OfType \Regex' \Leftrightarrow \RegexAlt'
}\\
\Regex'\equiv\Regex\\
\RegexAlt'\equiv\RegexAlt
}
{
\DNFLens \OfType \Regex \Leftrightarrow \RegexAlt
}
\end{mathpar}

This satisfies the needed requirements!

\end{proof}


However, these lenses are much more suited to synthesis.
The vast majority of the rules have a syntax directed synthesis algorithm.
Furthermore, even the rewrite rules that don't have an immediate syntax directed
synthesis algorithm have an underlying semantic meaning which can be used
to direct the solution.
