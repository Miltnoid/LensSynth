\section{Background}

\bcp{Let's change subsubsection globally to paragraph, to avoid wasting a line.}

\subsubsection{Languages and Alphabets}

\bcp{Too much italics in these definitions.  Italicize just the term being
  defined.}  

\begin{definition}
We fix a finite set of symbols, called characters, $\{\Character_1,\ldots,\Character_n\}$.
We call this fixed set an alphabet, denoted \Alphabet{}.
\end{definition}

\begin{definition}\bcp{the arguments to ``call'' are the wrong way round in
  some of these...''}
We call a string a sequence of characters $\Character_1\ldots\Character_n$.
We call the set of all strings $\StarOf{\Alphabet}$, and denote the empty string as \EmptyString{}.
\end{definition}

\begin{definition}
We call \Language{}, a subset of $\StarOf{\Alphabet}$ a Language\bcp{why is
  it capitalized?}.
\end{definition}

\begin{definition}
If we have two strings $\Character=\Character_1\ldots\Character_n$,
and $\CharacterAlt=\CharacterAlt_1\ldots\CharacterAlt_m$,
then $\Character\Concat\CharacterAlt=\Character_1\ldots\Character_n\CharacterAlt_1\ldots\CharacterAlt_m$.
\end{definition}

\begin{definition}
We call two languages $\Language_1$, $\Language_2$ unambiguously concatenable,
denoted $\UnambigConcatOf{\Language_1}{\Language_2}$,
if, for all strings $\String_1,\StringAlt_1\in\Language_1$,
and $\String_2,\StringAlt_2\in\Language_2$,\bcp{(Nearly) universal rule of
  mathematical writing: always separate adjacent math expressions by more
  than just punctuation.  Things get very hard to parse otherwise.}
$\String_1\Concat\String_2=\StringAlt_1\Concat\StringAlt_2$
only if $\String_1=\StringAlt_1$ and $\String_2=\StringAlt_2$.
\end{definition}

\begin{definition}
We call a language $\Language$ unambigusously iterable,
denoted $\UnambigItOf{\Language{}}$,
if, for all $n\in\Nats$, for all strings $\String_1,\ldots,\String_n\in\Language$,
and for all strings $\StringAlt_1,\ldots,\StringAlt_n\in\Language$,
$\String_1\Concat\ldots\Concat\String_n=\StringAlt_1\Concat\ldots\Concat\StringAlt_n$
only if $\String_i=\StringAlt_i$ for all $i\in\RangeIncInc{1}{n}$.
\end{definition}

\subsubsection{Regular Expressions}
\begin{figure}
\centering
\begin{tabular}{l@{\hspace*{5mm}}l@{\ }c@{\ }l@{\hspace*{5mm}}l}

% REGEX
(String)& \String{},\StringAlt{} & \GEq{} & $\String\in\Star{\Sigma}$ \\
(Regexs)& \Regex{},\RegexAlt{} & \GEq{} & s & \it Base \\
& & & \GBar{} \Regex{}* & \it Star \\
& & & \GBar{} $\RegexConcat{\Regex_1}{\Regex_2}$ & \it Concat \\
& & & \GBar{} $\Regex_1 | \Regex_2$ & \it Or \\
\end{tabular}
\caption{Regex Syntax \bcp{Recommend using capital letters ($R$, $S$,
    ...) for regular expressions.  And left-justifying the RH column and
    putting it in italic (with more space separating it from the rest) so
    that it's clear that it's commentary.}}
\label{fig:regex-syntax}
\end{figure}

\begin{figure}
\[
\begin{array}{lcl}
\LanguageOf{\String} &=& \{\String\}\\
\LanguageOf{\emptyset} &=& \{\}\\
\LanguageOf{\RegexConcat{\Regex_1}{\Regex_2}} &=&
\{\StringConcat{\String_1}{\String_2} \SuchThat
\String_1\in\LanguageOf{\Regex_1} \BooleanAnd \String_2\in\LanguageOf{\Regex_2}\}\\
\LanguageOf{\RegexOr{\Regex_1}{\Regex_2}} &=&
\{\String \SuchThat
\String\in\LanguageOf{\Regex_1} \BooleanOr \String\in\LanguageOf{\Regex_2}\}\\
\LanguageOf{\StarOf{\Regex}} &=&
\{\String_1\Concat\ldots\Concat\String_n \SuchThat
n\in\Nats \wedge \String_i\in\LanguageOf{\Regex}\}
\end{array}
\]
\caption{Regex Semantics}
\label{fig:regex-semantics}
\end{figure}

\bcp{The second sentence here is fine.  The first is not useful: all POPL
  readers know what regular expressions are.  We are not teaching anybody the
  concept here; we are just showing the concrete syntax that we use.  Ditto
  the second sentence of the next paragraph.}
Regular expressions are a tool which uses symbols to express languages.
The syntax for regular expressions is given in Figure~\ref{fig:regex-syntax}.
These regular expressions have an underlying semantics to express languages,
formalized in Figure~\ref{fig:regex-semantics}.

\begin{figure}
\newcounter{rowcount}
\setcounter{rowcount}{0}
\centering
\begin{tabular}{@{}r@{\hspace{1em}}c@{\hspace{1em}}l@{}r@{}}
\Regex{} & $\equiv$ & \Regex{} & \EqualityRule{}  \\
\RegexOr{\Regex}{\emptyset} & $\equiv$ & \Regex{} & \OrIdentityRule{} \\
$\RegexConcat{\Regex}{\emptyset}$ & $\equiv$ & $\emptyset$ & \EmptyProjectionRuleRightRule{} \\
$\RegexConcat{\emptyset}{\Regex}$ & $\equiv$ & $\emptyset$ & \EmptyProjectionRuleLeftRule{}\SubLeft{} \\
\RegexConcat{(\RegexConcat{\Regex{}}{\Regex'})}{\Regex''} & $\equiv$ & \RegexConcat{\Regex{}}{(\RegexConcat{\Regex'}{\Regex''})} & \ConcatAssocRule{}  \\
\RegexOr{(\RegexOr{\Regex}{\Regex'})}{\Regex''} & $\equiv$ & \RegexOr{\Regex}{(\RegexOr{\Regex'}{\Regex''})} & \OrAssociativityRule{}  \\
\RegexOr{\Regex{}}{\RegexAlt{}} & $\equiv$ & \RegexOr{\RegexAlt{}}{\Regex{}} & \OrCommutativityRule{}\\
\RegexConcat{\Regex{}}{(\RegexOr{\Regex{}'}{\Regex{}''})} & $\equiv$ & \RegexOr{(\RegexConcat{\Regex{}}{\Regex{}'})}{(\RegexConcat{\Regex{}}{\Regex{}''})} & \DistributivityLeftRule{} \\
\RegexConcat{(\RegexOr{\Regex{}'}{\Regex{}''})}{\Regex{}} & $\equiv$ & \RegexOr{(\RegexConcat{\Regex{}'}{\Regex{}})}{(\RegexConcat{\Regex{}''}{\Regex{}})} & \DistributivityRightRule{} \\
\RegexConcat{\Regex{}}{\EmptyString{}} & $\equiv$ & \Regex{} & \ConcatIdentityRule{} \\
\StarOf{(\RegexOr{\Regex{}}{\RegexAlt{}})} & $\equiv$ & \RegexConcat{\StarOf{(\RegexConcat{\StarOf{\Regex{}}}{\RegexAlt{}})}}{\StarOf{\Regex{}}} & \SumstarRule{}\\
\StarOf{(\RegexConcat{\Regex{}}{\RegexAlt{}})} & $\equiv$ & \RegexOr{\EmptyString{}}{(\RegexConcat{\RegexConcat{\Regex{}}{\StarOf{(\RegexConcat{\RegexAlt{}}{\Regex{}})}}}{\RegexAlt{}})} & \ProductstarRule{} \\
${(\Regex{}^*)}^*$ & $\equiv$ & \StarOf{\Regex{}} & \StarstarRule{} \\
\StarOf{(\RegexOr{\Regex}{\RegexAlt})} & $\equiv$ & $\StarOf{(\RegexConcat{(\RegexOr{\Regex}{\RegexAlt})}{\RegexOr{\RegexAlt}{\RegexConcat{{(\RegexConcat{\Regex}{\StarOf{\RegexAlt}})}^n}{\Regex}}})}\Concat$ & \DicyclicityRule{}\\
& & $(\EmptyString\Or(\RegexOr{\Regex}{\RegexAlt})\Concat$\\
& & $({(\RegexConcat{\Regex}{\StarOf{\RegexAlt}})}^0\Or\ldots\Or{(\RegexConcat{\Regex}{\StarOf{\RegexAlt}})}^n))$
\end{tabular}
\caption{Regular Expression Equivalences}
\label{fig:regex-equivalence-rules}
\end{figure}

We call two regular expressions equivalent, denoted $\Regex_1\equiv\Regex_2$,
if $\LanguageOf{\Regex_1}=\LanguageOf{\Regex_2}$.
There exists an equational theory for determining if two regular expressions are equivalent,
presented by Conway \cite{conway},\bcp{no comma between two
  sub-sentence-level clauses joined by and / or}
and proven complete by Krob \cite{Krob}\bcp{The citation for Conway reads
  funny.  Also, note that the names are repeated; that can be fixed by using
a different variant of the cite command, but using numeric citations is
probably better anyway here, to save space.},
shown in Figure~\ref{fig:regex-equivalence-rules}.

\subsubsection{Bijective Lenses}


\begin{figure}

let $\Delta$ be the set of user defined data types.
let $\Sigma^*$ be the set of words over the alphabet $\Sigma$\\

\begin{tabular}{l@{\ }l@{\ }c@{\ }l@{\ }r}

% REGEX
(Lenses)& \Lens{} & \GEq{} & $\LensVariable$ & Variable\\
& & & \GBar{} $\ConstLens{s_1 \in \Star{\Sigma}}{s_2 \in \Star{\Sigma}}$ & Const \\
& & & \GBar{} $\IdentityLens$ & Identity\\
& & & \GBar{} $\IterateLens{\Lens}$ & Iterate \\
& & & \GBar{} $\ConcatLens{\Lens_1}{\Lens_2}$ & Concat \\
& & & \GBar{} $\SwapLens{\Lens_1}{\Lens_2}$ & Swap\\
& & & \GBar{} $\OrLens{\Lens_1}{\Lens_2}$ & Or\\
& & & \GBar{} $\ComposeLens{\Lens_1}{\Lens_2}$ & Compose\\
\end{tabular}
\caption{Lens Syntax}
\label{fig:lens-syntax}
\end{figure}

\begin{figure}[b]
\centering
\begin{itemize}
\item $const(\String_1,\String_2)(\String_1,\String_2)$
\item $\IdentityLens(s,s)$
\item $\IterateLens{\Lens}(\epsilon,\epsilon)$
\item $\IterateLens{\Lens}(\String,\StringAlt)$ if $\Lens(\String_1,\StringAlt_1)$ and $\IterateLens{\Lens}(\String_2,\StringAlt_2)$ and $\String=\String_1\String_2$ and $\StringAlt=\StringAlt_1\StringAlt_2$
\item $\ConcatLens{\Lens_1}{\Lens_2}.putr(\String_1,\String_2) = \Lens_1(\String_1).\Lens_2(\String_2)$
\item $\SwapLens{\Lens_1}{\Lens_2}.putr(\String_1,\String_2) = \Lens_2(\String_2).\Lens_1(\String_1)$
\item $\OrLens{\Lens_1}{\Lens_2}.putr(\String) = \Lens_1(\String)$ if $\String \in dom(\Lens_1)$ or $\Lens_2(\String)$ if $\String \in dom(\Lens_2)$
\item $(\ComposeLens{\Lens_2}{\Lens_1}).putr(\String) = \Lens_2.putr(\Lens_1.putr(\String))$
\end{itemize}
\caption{Lens Semantics}
\label{fig:lens-semantics}
\end{figure}

We focus on bijections between these languages.
We can take a language based approach to developing these bijections,
where we construct a programming language where the types are regular expressions,
and the program is only well typed when it expresses a bijection between the languages
of the regular expressions.
This is the language of bijective lenses.
We define the syntax for these bijective lenses in Figure~\ref{fig:lens-syntax}.
The semantics and typing for these lenses is defined in Figure~\ref{fig:lens-semantics}.
The semantics is only defined alongside the typing, as the functions are only
well defined when the term is well typed.

We would like to synthesize these lenses from types and examples.
Our approach to synthesis is one of enumeration.
We would like to enumerate the programs that satisfy the desired typing until
we find one that satisfies the examples.
We enumerate these programs by creating subproblems based on typing rules
could potentially create the desired type.

For example, given the type $\RegexConcat{\Regex_1}{\Regex_2}\Leftrightarrow\RegexConcat{\RegexAlt_1}{\RegexAlt_2}$.
We know that it can be created from a Concat Lens rule,
which would create the subproblems of finding a lens of type $\Regex_1\Leftrightarrow\RegexAlt_1$
and a lens of type $\Regex_2\Leftrightarrow\RegexAlt_2$.

\section{DNF Regular Expressions}
\begin{figure}
\begin{tabular}{l@{\ }l@{\ }c@{\ }l@{\ }r}

% DNF_REGEX
(Atoms)& \Atom{},\AtomAlt{} & \GEq{} & \StarOf{\DNFRegex{}} & Iterate DNF\\
(Sequence)& \Sequence{},\SequenceAlt{} & \GEq{} &
$\SequenceOf{\String_0\SequenceSep\Atom_1\SequenceSep\ldots\SequenceSep\Atom_n\SequenceSep\String_n}$ & Conjoin\\
(DNF Regex)& \DNFRegex{},\DNFRegexAlt{} & \GEq{} & $\DNFOf{\Sequence_1\DNFSep\ldots\DNFSep\Sequence_n}$ & DNF Or\\
\end{tabular}
\caption{DNF Regex Syntax\bcp{Why the dot after conjuncts?  Ah, now I see;
    OK, but maybe it's better to choose two different kinds of brackets
    instead of distinguishing with . and + (I was also confused by the + --
    thought it must mean some kind of repetition).  How about
    $\langle...\rangle$ for disjuncts and $[...]$ for conjuncts?  (This is a
  bit suggestive of type-theoretic notation.)  Or how about ``hollow square
  brackets'' for conjuncts (using some kind of square brackets seems good,
  since they are ordered) and ``hollow set braces'' for disjuncts?  We
  should try to get this right, since these widgets are our core technical
  device.}} 
\label{fig:dnf-regex-syntax}
\end{figure}

\begin{figure}
%\begin{mathpar}
%\inferrule[Atom Userdef]
%{
%\Delta' \vdash \String : \Regex
%}
%{
%\Delta' \cup \{(\Regex,\RegexVariable)\} \vdash \String : \RegexVariable
%}
%
%\inferrule[Atom Empty Star]
%{
%}
%{
%\Delta \vdash \epsilon : \Star{\DNFRegex}
%}
%
%\inferrule[Atom Nonempty Star]
%{
%\Delta \vdash \String_1 : \DNFRegex\\
%\Delta \vdash \String_2 : \Star{\DNFRegex}
%}
%{
%\Delta \vdash \String_1\String_2 : \Star{\DNFRegex}
%}
%
%\inferrule[Clause]
%{
%\lambda i:\RangeIncInc{1}{n}.\Delta \vdash \String_i : \Atom_i
%}
%{
%\Delta \vdash \StringAlt_0\String_1\ldots\String_n\StringAlt_n : (\lambda i:\RangeIncInc{1}{n}.\Atom_i,\lambda i:\RangeIncInc{0}{n}.\StringAlt_i)
%}
%
%\inferrule[DNF Regex]
%{
%\Delta \vdash \String : \Clause_i\\
%}
%{
%\Delta \vdash \String : (\lambda i:\RangeIncInc{1}{n}.\Clause_i)
%}
%\end{mathpar}
\begin{itemize}
\item $\LanguageOf{\RegexContext}{\RegexVariable}=\LanguageOf{\RegexContext\setminus(\RegexVariable,\Regex)}{\Regex}$ if, for some unique \Regex{}, $(\RegexVariable,\Regex)\in\RegexContext$.
\item $\LanguageOf{\RegexContext}{\RegexVariable}=\emptyset$
if there does not exist a unique \Regex{} such that $(\RegexVariable,\Regex)\in\RegexContext$
\item $\LanguageOf{\RegexContext}{\Star{\DNFRegex}} =
\{\String_1\Concat\ldots\Concat\String_n | n\in\Nats \wedge \String_i\in\LanguageOf{\RegexContext}{\DNFRegex}\}$
\item $\LanguageOf{\RegexContext}{[\String_0;\Atom_1;\ldots;\Atom_n;\String_n]}=
\{\String_0\Concat\String_1'\Concat\ldots\Concat\String_n'\Concat\String_n | \String_i'\in\LanguageOf{\RegexContext}{\Atom_i}\}$
\item $\LanguageOf{\RegexContext}{[\Conjunct_1;\ldots;\Conjunct_n]}=
\{\String | \String \in \Conjunct_i \text{ for some $i\in\RangeIncInc{1}{n}$}\}$
\end{itemize}
\caption{DNF Regex Semantics}
\label{fig:dnf-regex-semantics}
\end{figure}

A DNF regular esxpression is, intuitively
a regular expression which is completely distributed
applied, with no associativity information through a list representation,
and with base regular expressions only at fixed locations.
We formalize the syntax of this language in Figure~\ref{fig:dnf-regex-syntax}.
We will use the shorthand $\DNFOf{\ConjunctOf{\Atom_1\ConjunctSep\ldots\ConjunctSep\Atom_n}}$
for $\DNFOf{\ConjunctOf{\EmptyString\ConjunctSep\Atom_1\ConjunctSep\ldots\ConjunctSep\Atom_n\ConjunctSep\EmptyString}}$.

The outermost layer is a list of conjuncts.
This layer intuitively corresponds to the choices involved in regular expression matching, and so represents where the Ors of normal regular expressions exist.
The second layer is a list of alternating strings and stars.
After the choice has been made about what will be expressed,
the base strings and iterated portions remains to be expressed.
We keep it in a normal form, by requiring a (possibly empty) string between
each clause.
The clause corresponds to the concatenated data, and is a concatenation of the
fixed data of base strings, and the iterated data of stars.
Finally is the atom, which is a star.
This corresponds to the iteration that takes place in normal stars.
There is no fixed way to break the choices of this into the clause,
as there is an arbitrarily large number of choices made in the stars.
This intuition is formalized by the semantics, given in
Figure~\ref{fig:dnf-regex-semantics}.  \bcp{Couple examples would help
  here.  Also, the name ``Conjunct'' doesn't seem right: either these are
  ``conjunctions'' of smaller things, or they function as ``disjuncts'' in
  larger expressions.  But even ``conjunction'' connotes an unordered
  grouping, which this is not.  What about ``sequences'' or some such?} 

\begin{figure}
\bf{ConcatConjunct}:\\
\ConcatConjunct \OfType \ArrowTypeOf{\mathit{Conjunct}}{\mathit{Conjunct}}\\
$\ConcatConjunct([\String_0;\Atom_1;\ldots;\Atom_n;\String_n],[\StringAlt_0;\AtomAlt_1;\ldots;\AtomAlt_m;\StringAlt_m])=$\\
\hspace*{2ex}$\ConcatConjunct([\String_0;\Atom_1;\ldots;\Atom_n;\String_n\Concat\StringAlt_0;\AtomAlt_1;\ldots;\AtomAlt_m;\StringAlt_m])$\\
\\
\bf{ConcatDNF}:\\
\ConcatDNF \OfType \ArrowTypeOf{\mathit{DNF}}{\mathit{DNF}}\\
$\ConcatDNF([\Conjunct_0;\ldots;\Conjunct_n],[\ConjunctAlt_0;\ldots;\ConjunctAlt_m])=$
\[
\begin{array}{rcccl}
[ & \ConcatConjunct(\Conjunct_0,\ConjunctAlt_0); & \ldots & \ConcatConjunct(\Conjunct_0,\ConjunctAlt_m); \\
& & \ldots & \\
& \ConcatConjunct(\Conjunct_n,\ConjunctAlt_0); & \ldots & \ConcatConjunct(\Conjunct_n,\ConjunctAlt_m) & ]
\end{array}
\]
\\
\bf{OrDNF}:\\
\OrDNF \OfType \ArrowTypeOf{\mathit{DNF}}{\mathit{DNF}}\\
$\OrDNF([\Conjunct_0;\ldots;\Conjunct_n],[\ConjunctAlt_0;\ldots;\ConjunctAlt_m])$=
\hspace*{2ex}$[\Conjunct_0;\ldots;\Conjunct_n;\ConjunctAlt_0;\ldots;\ConjunctAlt_m]$\\
\\
\bf{RepeatDNF}:\\
\RepeatDNF \OfType \ArrowTypeOf{\Nats}{\ArrowTypeOf{\mathit{DNF}}{\mathit{DNF}}}\\
$\RepeatDNF(0,\DNFRegex)=[[\EmptyString]]$\\
$\RepeatDNF(n,\DNFRegex)=$\\
\hspace*{2ex}$\ConcatDNF(\DNFRegex,\RepeatDNF(n-1,\DNFRegex))$\\
\\
\bf{StarModNDNF}:\\
\StarModNDNF \OfType \ArrowTypeOf{\Nats_{\geq1}}{\ArrowTypeOf{\mathit{DNF}}{\mathit{DNF}}}\\
$\StarModNDNF(1,\DNFRegex)=[[\EmptyString]]$\\
$\StarModNDNF(n,\DNFRegex)=$\\
\hspace*{2ex}$\OrDNF(\StarModNDNF(n-1,\DNFRegex),\RepeatDNF(n-1,\DNFRegex))$

\caption{DNF Regex Functions}
\label{fig:dnf-regex-functions}
\end{figure}

We can create a dnf regular expression from a regular expression through repeated
application of the distributivity rule, and through removal of association information.  To do this, we require some functions defined on DNF regular expressions,
which we define in Figure~\ref{fig:dnf-regex-functions}.
We can use these functions to obtain a completeness result.
From this we can define a function which converts a regular expression into
an equivalent DNF regular expression, \ToDNFRegex{}.  \bcp{No bullets.}
\begin{definition}
\leavevmode
\begin{itemize}
\item $\ToDNFRegex(\String)=\DNFOf{\ConjunctOf{\String}}$
\item $\ToDNFRegex(\StarOf{(\Regex)}) = \DNFOf{\ConjunctOf{(\StarOf{\ToDNFRegex(\Regex)})}}$
\item $\ToDNFRegex(\RegexConcat{\Regex_1}{\Regex_2}) =$\\
\hspace*{1em}$\ConcatDNFOf{\ToDNFRegex(\Regex_1)}{\ToDNFRegex(\Regex_2)}$
\item $\ToDNFRegex(\RegexOr{\Regex_1}{\Regex_2}) =$\\
\hspace*{1em}$\OrDNFOf{\ToDNFRegex(\Regex_1)}{\ToDNFRegex(\Regex_2)}$
\end{itemize}
\end{definition}
We show that this transformation is a valid transformation.
\begin{restatable}[Completeness of DNF Regexs]{theorem}{dnfrc}
\label{thm:completeness-dnf-lenses}
For all regular expressions \Regex{},
there exists a DNF regular expression \DNFRegex{},
such that \LanguageOf{\DNFRegex{}}=\LanguageOf{\Regex{}}.
\end{restatable}
First we will prove some lemmas.
\begin{lemma}[Equivalence of \ConcatSequence{} and \Concat{}]
If $\LanguageOf{\Regex}=\LanguageOf{\Sequence}$,
and $\LanguageOf{\RegexAlt}=\LanguageOf{\SequenceAlt}$,
then $\LanguageOf{\RegexConcat{\Regex}{\RegexAlt}}=\LanguageOf{\ConcatSequenceOf{\Sequence}{\SequenceAlt}}$.
\end{lemma}
\begin{proof}
Let $\Sequence=\SequenceOf{\String_0\SequenceSep\Atom_1\SequenceSep\ldots
\SequenceSep\Atom_n\SequenceSep\String_n}$, and
let\\ $\SequenceAlt=[\StringAlt_0\SequenceSep\AtomAlt_1\SequenceSep\ldots
\SequenceSep\AtomAlt_m\SequenceSep\StringAlt_m]$\\
\begin{tabular}{@{}L@{}L@{}}
\LanguageOf{\ConcatSequenceOf{\Sequence}{\SequenceAlt}} & = 
\LanguageOf{\SequenceOf{\String_0\SequenceSep\Atom_1\SequenceSep\ldots
\SequenceSep\Atom_n\SequenceSep\String_n\Concat\StringAlt_0\SequenceSep{}
\AtomAlt_1\SequenceSep\ldots\SequenceSep\AtomAlt_m\SequenceSep\StringAlt_m}} \\
& = 
\{\String_0\Concat\String_1'\Concat\ldots\Concat\String_n'\Concat\String_n
\Concat\StringAlt_0\Concat\StringAlt_1'\Concat\ldots
\Concat\StringAlt_m'\Concat\StringAlt_m \\
& \hspace{5em} \SuchThat{} \String_i'\in\LanguageOf{\Atom_i} \BooleanAnd{}
\StringAlt_i'\in\LanguageOf{\AtomAlt_i}\}\\
& = 
\{\String\Concat\StringAlt{} \SuchThat{} \String\in\LanguageOf{\Sequence}
\BooleanAnd{} \StringAlt\in\LanguageOf{\SequenceAlt}\}\\
& =
\{\String\Concat\StringAlt{} \SuchThat{} \String\in\LanguageOf{\Regex}
\BooleanAnd{} \StringAlt\in\LanguageOf{\RegexAlt}\}\\
& =
\LanguageOf{\RegexConcat{\Regex}{\RegexAlt}}
\end{tabular}
\end{proof}

\begin{lemma}[Equivalence of \ConcatDNF{} and \Concat{}]
\label{lem:cdnfeq}
If $\LanguageOf{\Regex}=\LanguageOf{\DNFRegex}$,
and $\LanguageOf{\RegexAlt}=\LanguageOf{\DNFRegexAlt}$,
then $\LanguageOf{\RegexConcat{\Regex}{\RegexAlt}}=
\LanguageOf{\ConcatDNFOf{\DNFRegex}{\DNFRegexAlt}}$.
\end{lemma}
\begin{proof}
Let $\DNFRegex=\DNFOf{\Sequence_0\DNFSep\ldots\DNFSep\Sequence_n}$, and
let $\DNFRegexAlt=\DNFOf{\SequenceAlt_0\DNFSep\ldots\DNFSep\SequenceAlt_m}$
\begin{tabular}{@{}L@{}L@{}}
\LanguageOf{\ConcatDNFOf{\DNFRegex}{\DNFRegexAlt}} & = 
\LanguageOf{\DNFOf{\ConcatSequenceOf{\Sequence_i}{\SequenceAlt_j}
\text{ for $i\in\RangeIncInc{1}{n}$, $j\in\RangeIncInc{1}{m}$}}} \\
& = 
\{\String\SuchThat \String\in\ConcatSequenceOf{\Sequence_i}{\SequenceAlt_j}\\
& \hspace{5em}
\text{ where $i\in\RangeIncInc{1}{n}$, $j\in\RangeIncInc{1}{m}$}\}\\
& = 
\{\String\Concat\StringAlt{} \SuchThat{} \String\in\LanguageOf{\Sequence_i}
\BooleanAnd{} \StringAlt\in\LanguageOf{\SequenceAlt_j}\}\\
& \hspace{5em}
\text{ where $i\in\RangeIncInc{1}{n}$, $j\in\RangeIncInc{1}{m}$}\}\\
& =
\{\String\Concat\StringAlt{} \SuchThat{} \String\in\LanguageOf{\DNFRegex}
\BooleanAnd{} \StringAlt\in\LanguageOf{\DNFRegexAlt}\}\\
& =
\{\String\Concat\StringAlt{} \SuchThat{} \String\in\LanguageOf{\Regex}
\BooleanAnd{} \StringAlt\in\LanguageOf{\RegexAlt}\}\\
& =
\LanguageOf{\RegexConcat{\Regex}{\RegexAlt}}
\end{tabular}
\end{proof}

\begin{lemma}[Equivalence of \OrDNF{} and \Or{}]
\label{lem:odnfeq}
If $\LanguageOf{\Regex}=\LanguageOf{\DNFRegex}$,
and $\LanguageOf{\RegexAlt}=\LanguageOf{\DNFRegexAlt}$,
then $\LanguageOf{\RegexOr{\Regex}{\RegexAlt}}=
\LanguageOf{\OrDNFOf{\DNFRegex}{\DNFRegexAlt}}$.
\end{lemma}
\begin{proof}
Let $\DNFRegex=\DNFOf{\Sequence_0\DNFSep\ldots\DNFSep\Sequence_n}$, and
let $\DNFRegexAlt=\DNFOf{\SequenceAlt_0\DNFSep\ldots\DNFSep\SequenceAlt_m}$
\begin{tabular}{@{}L@{}L@{}}
\LanguageOf{\OrDNFOf{\DNFRegex}{\DNFRegexAlt}} & = 
\LanguageOf{\DNFOf{\Sequence_0\DNFSep\ldots\DNFSep\Sequence_n\DNFSep
\SequenceAlt_1\DNFSep\ldots\DNFSep\SequenceAlt_m}}\\
& = 
\{\String\SuchThat{} \String\in\Sequence_i\vee\String\in\SequenceAlt_j\\
& \hspace{5em}
\text{ where $i\in\RangeIncInc{1}{n}$, $j\in\RangeIncInc{1}{m}$}\}\\
& = 
\{\String{} \SuchThat{} \String\in\LanguageOf{\DNFRegex}
\BooleanOr{} \String\in\LanguageOf{\DNFRegexAlt}\}\\
& =
\{\String \SuchThat{} \String\in\LanguageOf{\Regex}
\BooleanOr{} \String\in\LanguageOf{\RegexAlt}\}\\
& =
\LanguageOf{\RegexOr{\Regex}{\RegexAlt}}
\end{tabular}\\
\end{proof}

\begin{proof}
By structural induction.

Let $\Regex=\String$.
$\LanguageOf{\ToDNFRegex(\String)}=\LanguageOf{\DNFOf{\SequenceOf{\String}}}=
\{\String\}=\LanguageOf{\String}$

Let $\Regex=\emptyset$.
$\LanguageOf{\ToDNFRegex(\emptyset)}=\LanguageOf{\DNFOf{}} =
\{\} = \LanguageOf{\emptyset}$.

Let $\Regex=\StarOf{\Regex'}$.
By induction assumption, $\LanguageOf{\ToDNFRegex(\Regex')}=
\LanguageOf{\Regex'}$.\\
\begin{tabular}{@{}L@{}L@{}}
\LanguageOf{\ToDNFRegex(\StarOf{\DNFRegex'})} & =
\LanguageOf{\DNFOf{\SequenceOf{\StarOf{\ToDNFRegex(\Regex')}}}}\\
& =
\{\String\SuchThat\String\in
\LanguageOf{\SequenceOf{\StarOf{\ToDNFRegex(\Regex')}}}\}\\
& = 
\{\String\SuchThat{} \String\in\LanguageOf{\StarOf{\ToDNFRegex(\Regex')}}\}\\
& =
\{\String_1\Concat\ldots\Concat\String_n\SuchThat{}
n\in\Nats\\
& \hspace*{3em}\BooleanAnd\String_i\in\LanguageOf{\ToDNFRegex(\Regex')}\}\\
& =
\{\String_1\Concat\ldots\Concat\String_n\SuchThat{}
n\in\Nats\BooleanAnd\String_i\in\LanguageOf{\Regex'}\}\\
& = \LanguageOf{\StarOf{\Regex'}}
\end{tabular}

Let $\Regex=\RegexConcat{\Regex_1}{\Regex_2}$.
By induction assumption,
$\LanguageOf[\ToDNFRegex(\Regex_1)=\LanguageOf{\Regex_1}$, and
$\LanguageOf[\ToDNFRegex(\Regex_2)=\LanguageOf{\Regex_2}$.
$\ToDNFRegex(\RegexConcat{\Regex_1}{\Regex_2})=
\ConcatDNFOf{\ToDNFRegex(\Regex_1)}{\ToDNFRegex(\Regex_2)}$.
By Lemma~\ref{lem:cdnfeq},
$\RegexConcat{\Regex_1}{\Regex_2}=
\ConcatDNFOf{\ToDNFRegex(\Regex_1)}{\ToDNFRegex(\Regex_2)}$.

Let $\Regex=\RegexOr{\Regex_1}{\Regex_2}$.
By induction assumption,
$\LanguageOf[\ToDNFRegex(\Regex_1)=\LanguageOf{\Regex_1}$, and
$\LanguageOf[\ToDNFRegex(\Regex_2)=\LanguageOf{\Regex_2}$.
$\ToDNFRegex(\RegexOr{\Regex_1}{\Regex_2})=
\OrDNFOf{\ToDNFRegex(\Regex_1)}{\ToDNFRegex(\Regex_2)}$.
By Lemma~\ref{lem:odnfeq},
$\RegexOr{\Regex_1}{\Regex_2}=
\OrDNFOf{\ToDNFRegex(\Regex_1)}{\ToDNFRegex(\Regex_2)}$.
\end{proof}

Furthermore, we can use prove that language of DNF regular expressions is sound.

\begin{restatable}[Soundness of DNF Regexs]{theorem}{dnfrs}
\label{thm:soundness-dnf-lenses}
For all dnf regular expressions \DNFRegex{},
there exists a regular expression \Regex{},
such that \LanguageOf{\Regex{}}=\LanguageOf{\DNFRegex{}}.
\end{restatable}
First we will prove some lemmas.

\begin{lemma}
\label{lem:conjunct-builder-rx}
If $\LanguageOf{\ConjunctOf{\String_0\ConjunctSep\Atom_1\ConjunctSep\ldots
\ConjunctSep\Atom_n}}=\LanguageOf{\Regex}$
and $\LanguageOf{\AtomAlt}=\LanguageOf{\RegexAlt}$,
then $\LanguageOf{\ConjunctOf{\StringAlt\ConjunctSep\AtomAlt\ConjunctSep
\String_0\ConjunctSep\Atom_1\ConjunctSep\ldots\ConjunctSep\Atom_n}}=
\LanguageOf{\StringAlt\Concat\RegexAlt\Concat\Regex}$.
\end{lemma}
\begin{proof}\leavevmode\\
\begin{tabular}{@{}L@{}L@{}}
\LanguageOf{\ConjunctOf{\StringAlt\ConjunctSep\AtomAlt\ConjunctSep{}
\String_0\ConjunctSep\Atom_1\ConjunctSep\ldots\ConjunctSep\Atom_n}} & = 
\{\StringAlt\Concat\StringAlt'\Concat\String_0\Concat\String_0'\Concat\ldots\Concat\String_n'\Concat\String_n\\
& \hspace{1em} \SuchThat{} \StringAlt'\in\LanguageOf{\AtomAlt}\BooleanAnd{}
\String_i'\in\LanguageOf{\Atom_i}\}\\
& =
\{\StringAlt\Concat\StringAlt'\Concat\String\\
& \hspace{1em} \SuchThat{}
\StringAlt'\in\LanguageOf{\AtomAlt}\BooleanAnd{}
\String\in\LanguageOf{\ConjunctOf{\String_0\ConjunctSep\Atom_1\ConjunctSep\ldots
\ConjunctSep\Atom_n}}\\
& =
\{\StringAlt\Concat\StringAlt'\Concat\String\\
& \hspace{1em} \SuchThat{}
\StringAlt\in\LanguageOf{\StringAlt}\BooleanAnd{}
\StringAlt'\in\LanguageOf{\RegexAlt}\BooleanAnd{}
\String\in\LanguageOf{\Regex}\}\\
& =
\LanguageOf{\StringAlt\Concat\RegexAlt\Concat\Regex}
\end{tabular}
\end{proof}

\begin{lemma}
\label{lem:conjunct-rx}
Let $\ConjunctOf{\String_0\ConjunctSep\Atom_1\ConjunctSep
\ldots\Atom_n\ConjunctSep\String_n}$ be a conjunct,
and for each $\Atom_i$, there exists a regular expression, $\Regex_i$, such that
$\LanguageOf{\Atom_i}=\Regex_i$.
Then, there exists a regular expression $\Regex$, such that
$\LanguageOf{\ConjunctOf{\String_0\ConjunctSep\Atom_1\ConjunctSep
\ldots\Atom_n\ConjunctSep\String_n}}=\LanguageOf{\Regex}$.
\end{lemma}
\begin{proof}
By induction on $n$.

Let $n=0$.
$\Conjunct=\ConjunctOf{\String_0}$.
Consider the regular expression, $\String_0$.
$\LanguageOf{\String_0}=\{\String_0\}=\LanguageOf{\ConjunctOf{\String_0}}$

Let $n>0$,
$\Conjunct=\ConjunctOf{\String_0\ConjunctSep\Atom_1\ConjunctSep
\ldots\Atom_n\ConjunctSep\String_n}$.
By induction assumption, there exists a regular expression $\Regex'$ such that
$\LanguageOf{\Regex'}=\LanguageOf{\String_1\ConjunctSep\Atom_2\ldots
\ConjunctSep\Atom_n\String_n}$.
By problem assumption, there exists a regular expression $\Regex_1$ such that
$\LanguageOf{\Atom_1}=\LanguageOf{\Regex_1}$.
Consider $\String_0\Concat\Regex_1\Concat\Regex'$.
By Lemma~\ref{lem:conjunct-builder-rx},
$\LanguageOf{\String_0\Concat\Regex_1\Concat\Regex'}=
\LanguageOf{\ConjunctOf{\String_0\ConjunctSep\Atom_1\ConjunctSep
\ldots\Atom_n\ConjunctSep\String_n}}$.
\end{proof}



\begin{lemma}
\label{lem:dnf-builder-rx}
If $\LanguageOf{\DNFOf{\Conjunct_2\DNFSep\ldots\DNFSep\Conjunct_n}}
=\LanguageOf{\Regex}$
and $\LanguageOf{\Conjunct_1}=\LanguageOf{\RegexAlt}$,
then $\LanguageOf{\DNFOf{\Conjunct_1\DNFSep\ldots\DNFSep\Conjunct_n}}=
\LanguageOf{\RegexOr{\RegexAlt}{\Regex}}$.
\end{lemma}
\begin{proof}\leavevmode\\
\begin{tabular}{@{}L@{}L@{}}
\LanguageOf{\DNFOf{\Conjunct_1\ldots\Conjunct_n}} & = 
\{\String\SuchThat{} \String\in\LanguageOf{\Conjunct_i}
\text{ for some $i\in\RangeIncInc{1}{n}$}\}\\
& =
\{\String\SuchThat{}
\String\in\LanguageOf{\Conjunct_1}\BooleanOr{}
\String\in\LanguageOf{\Conjunct_i}\\
& \hspace{5em} 
\text{ for some $i\in\RangeIncInc{2}{n}$}\}\\
& =
\{\String\SuchThat{}
\String\in\LanguageOf{\RegexAlt}\BooleanOr{}
\String\in\LanguageOf{\Regex}\}\\
& = \LanguageOf{\RegexOr{\RegexAlt}{\Regex}}\\
\end{tabular}
\end{proof}

\begin{lemma}
\label{lem:dnf-rx}
Let $\DNFOf{\Conjunct_1\DNFSep\ldots\ConjunctSep\Conjunct_n}$ be a dnf regex,
and for each $\Conjunct_i$, there exists a regular expression, $\Regex_i$,
such that $\LanguageOf{\Conjunct_i}=\Regex_i$.
Then, there exists a regular expression $\Regex$, such that
$\LanguageOf{\DNFOf{\Conjunct_1\DNFSep
\ldots\DNFSep\Conjunct_n}}=\LanguageOf{\Regex}$.
\end{lemma}
\begin{proof}
By induction on $n$.

Let $n=0$.  Consider $\emptyset$
$\LanguageOf{\DNFOf{}}=\{\}=\LanguageOf{\emptyset}$.

Let $n>0$.
$\DNFRegex=\DNFOf{\Conjunct_1\DNFSep\ldots\DNFSep\Conjunct_n}$.
By induction assumption, there exists a regular expression $\Regex'$ such that
$\LanguageOf{\Regex'}=
\LanguageOf{\DNFOf{\Conjunct_2\DNFSep\ldots\DNFSep\Conjunct_n}}$.
By problem assumption, there exists a regular expression $\Regex_1$ such that
$\LanguageOf{\Conjunct_1}=\LanguageOf{\Regex_1}$.
Consider $\RegexOr{\Regex_1}{\Regex'}$.
By Lemma~\ref{lem:dnf-builder-rx},
$\LanguageOf{\RegexOr{\Regex_1}{\Regex'}}=
\LanguageOf{\DNFOf{\Conjunct_1\DNFSep
\ldots\Conjunct_n}}$.
\end{proof}


We can further normalize the DNF regular expressions by providing an order on conjuncts,
and ordering the conjuncts in the DNF regular expression according to that order.
With this, we have normalized all regular expression equivalences not involving stars.
In other words, if two regular expressions, $\Regex_1$ and $\Regex_2$,
are equivalent up to the transformations of regular expressions in equivalences
not involving stars,
then $\ToDNFRegex(\Regex_1)=\ToDNFRegex(\Regex_2)$.
A reasonable approach to further normalizing this is to attempt to minimize the
regular expression as much as possible, using the star laws.
Unfortunately, this doesn't work
well for finding lenses.  For example, there is clearly a lens of type
$\MapsBetweenTypeOf{\RegexOr{\EmptyString}{(\RegexConcat{a}{\StarOf{a}})}}{\RegexOr{b}{(\RegexConcat{c}{\StarOf{d}})}}$,
namely $\OrLens{\ConstLens{\EmptyString}{a}}{\ConcatLens{\ConstLens{a}{b}}{\IterateLens{\ConstLens{a}{d}}}}$.
However, if one simplifies as much as possible, then they will have to find a lens
of type $\MapsBetweenTypeOf{\StarOf{a}}{\RegexOr{b}{(\RegexConcat{c}{\StarOf{d}})}}$.
Unlike the type before being minified, where there was a type directed way to find
this lens, there is no longer a type directed way: should the lens be an iterate lens
or an or lens.
However, the regular expression equivalences have meaning.
If we only allow one rewrite, we see that
$\RegexOr{\EmptyString}{(\RegexConcat{a}{\StarOf{a}})}$ can be used for more
complicated lenses than merely $\StarOf{a}$.
A regular expression merely of the form $\StarOf{a}$ only cares about how the iterated case is handled, where a regular expression of the form
$\RegexOr{\EmptyString}{(\RegexConcat{a}{\StarOf{a}})}$ potentially acts differently on the empty
string case than on the nonempty case.

\begin{figure}
\begin{mathpar}
\inferrule[Atom Sumstar]
{
A\subsetneq \RangeIncInc{1}{n}
}
{
\Star{([\Conjunct_1;\ldots;\Conjunct_n])}\RewriteAtom\\
\ConcatDNF(\Star{(\ConcatDNF([[\EmptyString;\Star{\DNFRegex_A};\EmptyString]],\DNFRegex_{\ComplementOf{A}}))},
\Star{\DNFRegex_A})\\
\text{where }\DNFRegex_S=[\Conjunct_{S_1};\ldots;\Conjunct_{S_{\SizeOf{S}}}]\\
}

\inferrule[Atom Unrollstar]
{
}
{
\Star{\DNFRegex}\RewriteAtom
\OrDNF([[\EmptyString]],\ConcatDNF(\DNFRegex,[[\EmptyString;\Star{\DNFRegex};\EmptyString]]))
}

\inferrule[Atom Powerstar]
{
n\in\Nats_{\geq1}
}
{
\Star{\DNFRegex}\RewriteAtom
\ConcatDNF([[\EmptyString;\Star{[[\RepeatDNF(n,\DNFRegex)]]};\EmptyString]],\\\\
\StarModNDNF(n,\DNFRegex))
}

\inferrule[DNF Rewrite Star]
{
\DNFRegex \RewriteDNF \DNFRegex'
}
{
\Star{\DNFRegex} \RewriteAtom [[\EmptyString;\Star{\DNFRegex'};\EmptyString]]
}

\inferrule[Atom DNF Rewrite]
{
\Atom_j \RewriteAtom \DNFRegex
}
{
[\Clause_1;\ldots;\Clause_{i-1};\\\\
[\String_0;\Atom_1;\ldots;\String_{j-1};\Atom_j;\String_j;\ldots;\Atom_m;\String_m]\\\\
;\Clause_{i+1};\ldots;\Clause_n]\RewriteDNF\\\\
\OrDNF(\OrDNF([\Clause_1;\ldots;\Clause_{i-1}],
\ConcatDNF(\ConcatDNF(\\\\
[[\String_0;\Atom_1;\ldots;\String_{j-1}]],\DNFRegex),[\String_j;\ldots;\Atom_m;\String_m])),
[\Clause_{i+1};\ldots;\Clause_n])
}

\inferrule[DNF Rewrite Composition]
{
\DNFRegex \RewriteDNF \DNFRegex'\\
\DNFRegex' \RewriteDNF \DNFRegex''
}
{
\DNFRegex \RewriteDNF \DNFRegex''
}

\inferrule[Identity DNF Rewrite]
{
}
{
\DNFRegex \RewriteDNF \DNFRegex
}

\end{mathpar}
\caption{DNF Regex Rewrite Rules}
\label{fig:dnf-regex-rewrites}
\end{figure}

Because of this, instead of writing equivalences for DNF regular expressions like
exist for normal regular expressions,
instead we write rewrite rules as shown in
Figure~\ref{fig:dnf-regex-rewrites}.
Intuitively, these rewrite rules correspond to expanding a regular expression
into a regular expression which can be used in a more complicated lens, without requiring retyping.
We define these rules as rewrites that turn atoms into DNF Regular expressions,
and a rule that expresses how these rewrites can become rewrites on
the DNF regular expressions themselves.  \bcp{Here we {\em really} need some
examples.}

Roughly, the \AtomSumstarRule{} rewrite corresponds to \SumstarRule{},
the \AtomUnrollstarLeftRule{} and \AtomUnrollstarRightRule{} rewrites correspond to \ProductstarRule{}.
and the \DicyclicRewriteStarRule{} rewrite corresponds to \DicyclicityRule{}.
There is no rewrite corresponding to Regular Expression equivalence 10, as that
equivalence is an ambiguity introducing equivalence.
An application of \AtomSumstarRule{} corresponds to doing different things after the last time a certain event has occured, and before the last time that event occurs.
An application of \AtomUnrollstarLeftRule{} corresponds to making a different action for
the empty case of an iteration, the first case of an iteration, and all further cases of the iteration,
and similarly with \AtomUnrollstarRightRule{} for the empty case, the last case, and all previous cases, of the iteration.
An application of \DicyclicRewriteStarRule{} corresponds to TODO: understand this
horrible horrible rule.

\section{DNF Lenses}
\begin{figure}
\centering
\begin{tabular}{l@{\ }l@{\ }c@{\ }l@{\ }r}
% REGEX
(Atom Lenses) &\AtomLens{} & \GEq{} & $Iterate(\DNFLens)$ & Iterate\\
& & & \GBar{} \IdentityLens{} & Identity\\
(Sequence Lenses) &\SequenceLens{} & \GEq{} &
$(\SequenceLensOf{(\String_0,\StringAlt_0)\SequenceLensSep\AtomLens_1\SequenceLensSep\ldots\SequenceLensSep\AtomLens_n\SequenceLensSep(\String_n,\StringAlt_n)}$, &\\
& & & $\sigma \in S_n)$ & Clause\SubN{}\\
(DNF Lenses)& \DNFLens{} & \GEq{} & $(\DNFLensOf{\SequenceLens_1\DNFLensSep\ldots\DNFLensSep\SequenceLens_n}, \sigma \in S_n)$ & DNF\SubN{}\\
\end{tabular}
\caption{DNF Lens Syntax \bcp{italics right column.  And remove the space to
  the left of the LH column.  (Use @\{\} in the array header.)}}
\label{fig:dnf-lens-syntax}
\end{figure}

\begin{figure}
\centering
\begin{mathpar}
\inferrule[\IterateAtomLensRule{}]
{
\DNFLens : \DNFRegex \Leftrightarrow \DNFRegexAlt
\HasSemantics \PutRight,\PutLeft\\
\UnambigItOf{\LanguageOf{\DNFRegex}}\\
\UnambigItOf{\LanguageOf{\DNFRegexAlt}}\\
}
{
\IterateLens{\DNFLens} : \StarOf{\DNFRegex} \Leftrightarrow \StarOf{\DNFRegexAlt} \HasSemantics\\
\lambda \String.\LetWhereIn{\String_1\Concat\ldots\Concat\String_n}{\String}{\String_i\in\LanguageOf{\DNFRegex}}\\
(\PutRight\Apply\String_1)\Concat\ldots\Concat(\PutRight\Apply\String_n),\\\\
\lambda \String.\LetWhereIn{\String_1\Concat\ldots\Concat\String_n}{\String}{\String_i\in\LanguageOf{\DNFRegexAlt}}\\
(\PutLeft\Apply\String_1)\Concat\ldots\Concat(\PutLeft\Apply\String_n)
}

\inferrule[\SequenceLensRule{}]
{
\AtomLens_1 : \Atom_1 \Leftrightarrow \AtomAlt_1 \HasSemantics \PutRight_1,\PutLeft_1\\\\
\ldots\\\\
\AtomLens_n : \Atom_n \Leftrightarrow \AtomAlt_n \HasSemantics \PutRight_n,\PutLeft_n\\\\
\sigma \in \PermutationSetOf{n}\\
\forall i \in \RangeIncInc{1}{n-1} \UnambigConcatOf{\LanguageOf{\String_{i-1}\Atom_i\epsilon}}{\LanguageOf{\String_i\Atom_{i+1}\epsilon}}\\
\forall i \in \RangeIncInc{1}{n-1} \UnambigConcatOf{\LanguageOf{\StringAlt_{i-1}\AtomAlt_{\sigma(i)}\epsilon}}{\LanguageOf{\StringAlt_i\AtomAlt_{\sigma(i+1)}\epsilon}}\\
}
{
(\SequenceLensOf{(\String_0,\StringAlt_0)\SequenceLensSep\Atom_1\SequenceLensSep\ldots\SequenceLensSep\Atom_n\SequenceLensSep(\String_n,\StringAlt_n)},\sigma) :\\
\SequenceOf{\String_0\SequenceSep\Atom_1\SequenceSep\ldots\SequenceSep\Atom_n\SequenceSep\String_n}\Leftrightarrow
\SequenceOf{\StringAlt_0\SequenceSep\AtomAlt_1\SequenceSep\ldots\SequenceSep\AtomAlt_n\SequenceSep\StringAlt_n}\HasSemantics\\
\lambda \String.\LetWhereIn{\String_0\Concat\String_1'\Concat\ldots\Concat\String_n'\Concat\String_n}{\String}{\String_i'\in\LanguageOf{\Atom_i}}\\
\String_0\Concat(\PutRight_{\sigma(1)}\Apply\String_{\sigma(1)}')\Concat\ldots\Concat(\PutRight_n\Apply\String_n')\Concat\String_n,\\\\
\lambda \String.\LetWhereIn{\String_0\Concat\String_1'\Concat\ldots\Concat\String_n'\Concat\String_n}{\String}{\String_i'\in\LanguageOf{\AtomAlt_i}}\\
\String_0\Concat(\PutLeft_{\InverseOf{\sigma}(1)}\Apply\String_{\InverseOf{\sigma}(1)}')\Concat\ldots\Concat(\PutLeft_{\InverseOf{\sigma}(n)}\Apply\String_{\InverseOf{\sigma}(n)}')\Concat\String_n
}

\inferrule[\DNFLensRule{}]
{
\SequenceLens_1 : \Sequence_1 \Leftrightarrow \SequenceAlt_1 \HasSemantics \PutRight_1, \PutLeft_1\\\\
\ldots\\\\
\SequenceLens_n : \Sequence_n \Leftrightarrow \SequenceAlt_n \HasSemantics \PutRight_n, \PutLeft_n\\\\
\sigma \in \PermutationSetOf{n}\\
i \neq j \Rightarrow \Sequence_{i} \cap \Sequence_{j}=\emptyset\\
i \neq j \Rightarrow \SequenceAlt_{i} \cap \SequenceAlt_{j}=\emptyset\\
}
{
(\DNFLensOf{\SequenceLens_1\DNFLensSep\ldots\DNFLensSep\SequenceLens_n},\sigma) : \DNFOf{\Sequence_1\DNFSep\ldots\DNFSep\Sequence_n}
\Leftrightarrow \DNFOf{\SequenceAlt_{\sigma(1)}\DNFSep\ldots\DNFSep\SequenceAlt_{\sigma(n)}} \HasSemantics\\
\lambda \String.\{\PutRight_1(\String) \text{ if $\String\in\LanguageOf{\Sequence_1}$ }, \ldots, \PutRight_n(\String) \text{ if $\String\in\LanguageOf{\Sequence_n}$ }\},\\
\lambda \String.\{\PutLeft_1(\String) \text{ if $\String\in\LanguageOf{\SequenceAlt_1}$ }, \ldots, \PutLeft_n(\String) \text{ if $\String\in\LanguageOf{\SequenceAlt_n}$ }\}\\
}

\inferrule[\DNFRewriteLensRule{}]
{
\DNFRegex \RewriteDNF \DNFRegex'\\
\DNFRegexAlt \RewriteDNF \DNFRegexAlt'\\
\DNFLens \OfType \MapsBetweenTypeOf{\DNFRegex}{\DNFRegexAlt} \HasSemantics \PutRight,\PutLeft
}
{
\DNFLens \OfType \MapsBetweenTypeOf{\DNFRegex'}{\DNFRegexAlt'}
}

\end{mathpar}
\caption{DNF Lens Typing and Semantics}
\label{fig:dnf-lens-semantics}
\end{figure}

Armed with these rewrites, we have the capabilities to define a sufficiently
strong language for lenses on these regular expressions.
The syntax is defined in Figure~\ref{fig:dnf-lens-syntax}.
Similarly to the language of lenses, we aim to have the typing of the lenses
correspond closely to the syntax for the lenses themselves.
The typing and semantics of these dnf lenses are defined in Figure~\ref{fig:dnf-lens-semantics}.
Intuitively, a DNF Regex Lens corresponds roughly to an n-ary version of an or lens,
a Conjunct Lens corresponds to const lenses for the strings, and a combination of
Concat and Swap lenses for the Atoms.
Furthermore, these lenses are sufficiently strong that Compose Lenses are not
necessary.
Indeed, these lenses are strong enough to express everything expressible in the language of lenses.
\begin{restatable}[Completeness of DNF Lenses]{theorem}{dnflc}
\label{thm:completeness-dnf-lenses}
If there exists a derivation of $\Lens \OfType \MapsBetweenTypeOf{\Regex}{\RegexAlt} \HasSemantics \PutRight,\PutLeft$,
then there exists a derivation of $\DNFLens \OfType \MapsBetweenTypeOf{\DNFRegex}{\DNFRegexAlt} \HasSemantics \PutRight,\PutLeft$ such that
$\LanguageOf{\DNFRegex}=\LanguageOf{\Regex}$, and
$\LanguageOf{\DNFRegexAlt}=\LanguageOf{\RegexAlt}$.
\end{restatable}
Furthermore, they only express things expressible in the language of lenses.
\begin{restatable}[Soundness of DNF Lenses]{theorem}{dnfls}
\label{thm:soundness-dnf-lenses}
If there exists a derivation of $\DNFLens \OfType \MapsBetweenTypeOf{\DNFRegex}{\DNFRegexAlt} \HasSemantics \PutRight,\PutLeft$,
then there exists a derivation of $\Lens \OfType \MapsBetweenTypeOf{\Regex}{\RegexAlt} \HasSemantics \PutRight,\PutLeft$ such that
$\LanguageOf{\Regex}=\LanguageOf{\DNFRegex}$, and
$\LanguageOf{\RegexAlt}=\LanguageOf{\DNFRegexAlt}$.
\end{restatable}

However, these lenses are much more suited to synthesis.
The vast majority of the rules have a syntax directed synthesis algorithm.
Furthermore, even the rewrite rules that don't have an immediate syntax directed
synthesis algorithm have an underlying semantic meaning which can be used
to direct the solution.
