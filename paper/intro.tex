
\section{Introduction}
\begin{figure}
\begin{tabular}{l@{\ }l@{\ }c@{\ }l@{\ }r}

% REGEX
(Regexs)& \Regex{} & \GEq{} & \UserDefined{} & variable \\
& & & \GBar{} s $\in \Sigma^*$ & base \\
& & & \GBar{} \Regex{}* & star \\
& & & \GBar{} $\Regex_1 \Regex_2$ & concat \\
& & & \GBar{} $\Regex_1 | \Regex_2$ & or \\
\end{tabular}
\caption{Regex Syntax}
\label{fig:refn-subgrammars}
\end{figure}

\begin{figure}
\[
\begin{array}{lcl}
\LanguageOf{\String} &=& \{\String\}\\
\LanguageOf{\emptyset} &=& \{\}\\
\LanguageOf{\RegexConcat{\Regex_1}{\Regex_2}} &=&
\{\StringConcat{\String_1}{\String_2} \SuchThat
\String_1\in\LanguageOf{\Regex_1} \BooleanAnd \String_2\in\LanguageOf{\Regex_2}\}\\
\LanguageOf{\RegexOr{\Regex_1}{\Regex_2}} &=&
\{\String \SuchThat
\String\in\LanguageOf{\Regex_1} \BooleanOr \String\in\LanguageOf{\Regex_2}\}\\
\LanguageOf{\StarOf{\Regex}} &=&
\{\String_1\Concat\ldots\Concat\String_n \SuchThat
n\in\Nats \wedge \String_i\in\LanguageOf{\Regex}\}
\end{array}
\]
\caption{Regex Semantics}
\label{fig:regex-semantics}
\end{figure}



\begin{figure}

let $\Delta$ be the set of user defined data types.
let $\Sigma^*$ be the set of words over the alphabet $\Sigma$\\

\begin{tabular}{l@{\ }l@{\ }c@{\ }l@{\ }r}

% REGEX
(Lenses)& \Lens{} & \GEq{} & $\LensVariable$ & Variable\\
& & & \GBar{} $\ConstLens{s_1 \in \Star{\Sigma}}{s_2 \in \Star{\Sigma}}$ & Const \\
& & & \GBar{} $\IdentityLens$ & Identity\\
& & & \GBar{} $\IterateLens{\Lens}$ & Iterate \\
& & & \GBar{} $\ConcatLens{\Lens_1}{\Lens_2}$ & Concat \\
& & & \GBar{} $\SwapLens{\Lens_1}{\Lens_2}$ & Swap\\
& & & \GBar{} $\OrLens{\Lens_1}{\Lens_2}$ & Or\\
& & & \GBar{} $\ComposeLens{\Lens_1}{\Lens_2}$ & Compose\\
\end{tabular}
\caption{Lens Syntax}
\label{fig:lens-syntax}
\end{figure}

\begin{figure}
\centering
\begin{mathpar}
\inferrule[Constant Lens]
{
\String_1 \in \Sigma^*\and
\String_2 \in \Sigma^*
}
{
const(\String_1,\String_2) : \String_1 \Leftrightarrow \String_2
}

\inferrule[Identity Lens]
{
}
{
\IdentityLens : \Regex \Leftrightarrow \Regex 
}

\inferrule[Iterate Lens]
{
\Lens : \Regex_1 \Leftrightarrow \Regex_2\and
\Regex_1^{!*}\and
\Regex_2^{!*}
}
{
\IterateLens{\Lens} : \Star{\Regex_1} \Leftrightarrow \Star{\Regex_2}
}

\inferrule[Concat Lens]
{
\Lens_1 : \Regex_{1,1} \Leftrightarrow \Regex_{1,2}\and
\Lens_2 : \Regex_{2,1} \Leftrightarrow \Regex_{2,2}\and
\Regex_{1,1} .! \Regex_{2,1}\and
\Regex_{1,2} .! \Regex_{2,2}
}
{
\ConcatLens{\String_1}{\String_2} : \Regex_{1,1}\Regex_{2,1} \Leftrightarrow \Regex_{1,2}\Regex_{2,2}
}

\inferrule[Swap Lens]
{
\Lens_1 : \Regex_{1,1} \Leftrightarrow \Regex_{1,2} \and
\Lens_2 : \Regex_{2,1} \Leftrightarrow \Regex_{2,2} \and
\Regex_{1,1} .! \Regex_{2,1}\and
\Regex_{2,2} .! \Regex_{1,2}
}
{
\ConcatLens{\String_1}{\String_2} : \Regex_{1,1}\Regex_{2,1} \Leftrightarrow \Regex_{2,2}\Regex_{1,2}
}

\inferrule[Or Lens]
{
\Lens_1 : \Regex_{1,1} \Leftrightarrow \Regex_{1,2}\\
\Lens_2 : \Regex_{2,1} \Leftrightarrow \Regex_{2,2}\\
\Regex_{1,1} \cap \Regex_{2,1} = \emptyset\\
\Regex_{1,2} \cap \Regex_{2,2} = \emptyset\\
}
{
\OrLens{\Lens_1}{\Lens_2} : \Regex_{1,1} | \Regex_{2,1} \Leftrightarrow \Regex_{1,2} | \Regex_{2,2}
}

\inferrule[Compose Lens]
{
\Lens_1 : \Regex_1 \Leftrightarrow \Regex_2 \and
\Lens_2 : \Regex_2 \Leftrightarrow \Regex_3 
}
{
\ComposeLens{\Lens_2}{\Lens_1} : \Regex_1 \Leftrightarrow \Regex_3
}

\inferrule[Retype Lens]
{
\Lens : \Regex_1 \Leftrightarrow \Regex_2 \and
\LanguageOf{\Regex_1} \equiv \LanguageOf{\Regex_1'}\and
\LanguageOf{\Regex_2} \equiv \LanguageOf{\Regex_2'}
}
{
\Lens : \Regex_1' \Leftrightarrow \Regex_2' 
}
\end{mathpar}
\caption{Lens Typing}
\label{fig:lens-typing}
\end{figure}

\begin{figure}[b]
\centering
\begin{itemize}
\item $const(\String_1,\String_2)(\String_1,\String_2)$
\item $\IdentityLens(s,s)$
\item $\IterateLens{\Lens}(\epsilon,\epsilon)$
\item $\IterateLens{\Lens}(\String,\StringAlt)$ if $\Lens(\String_1,\StringAlt_1)$ and $\IterateLens{\Lens}(\String_2,\StringAlt_2)$ and $\String=\String_1\String_2$ and $\StringAlt=\StringAlt_1\StringAlt_2$
\item $\ConcatLens{\Lens_1}{\Lens_2}.putr(\String_1,\String_2) = \Lens_1(\String_1).\Lens_2(\String_2)$
\item $\SwapLens{\Lens_1}{\Lens_2}.putr(\String_1,\String_2) = \Lens_2(\String_2).\Lens_1(\String_1)$
\item $\OrLens{\Lens_1}{\Lens_2}.putr(\String) = \Lens_1(\String)$ if $\String \in dom(\Lens_1)$ or $\Lens_2(\String)$ if $\String \in dom(\Lens_2)$
\item $(\ComposeLens{\Lens_2}{\Lens_1}).putr(\String) = \Lens_2.putr(\Lens_1.putr(\String))$
\end{itemize}
\caption{Lens Semantics}
\label{fig:lens-semantics}
\end{figure}

We have regular expressions as normal regular expressions.
We expand it with having user defined data types as well.
We have lenses, defined in Figure~\ref{fig:lens-syntax}, typed as in Figure~\ref{fig:lens-typing}.
These lenses have a semantic interpretation as a relation, given via Figure~\ref{fig:lens-semantics}.
\begin{theorem}
\label{thm:lens-bij-fcn}
If $\Delta \vdash \Lens : \Regex_1 \Leftrightarrow \Regex_2$,
then for all $x \in \LanguageOf{\Delta}{\Regex_1}$, there exists a unique $y \in \LanguageOf{\Delta}{\Regex_2}$ such that $\denot{\Lens}(x,y)$.
This defines a bijective function called $\Lens.putr : \LanguageOf{\Delta}{\Regex_1} \rightarrow \LanguageOf{\Delta}{\Regex_2}$,
whose inverse is called $\Lens.putl$.
\end{theorem}
We would like to be able to synthesize these lenses automatically, given a
specification as two regular expressions, an a set of values that are
mapped to each other.
