
\section{Introduction}
\ref{fig:lens-alternate-alternate-semantics}
We have regular expressions as normal regular expressions.
We expand it with having user defined data types as well.
We have lenses, defined in Figure~\ref{fig:lens-syntax}, typed as in Figure~\ref{fig:lens-typing}.
These lenses have a semantic interpretation as a relation, given via Figure~\ref{fig:lens-semantics}.
\begin{theorem}
\label{thm:lens-bij-fcn}
If $\Delta \vdash \Lens : \Regex_1 \Leftrightarrow \Regex_2$,
then for all $x \in \LanguageOf{\Delta}{\Regex_1}$, there exists a unique $y \in \LanguageOf{\Delta}{\Regex_2}$ such that $\denot{\Lens}(x,y)$.
This defines a bijective function called $\Lens.putr : \LanguageOf{\Delta}{\Regex_1} \rightarrow \LanguageOf{\Delta}{\Regex_2}$,
whose inverse is called $\Lens.putl$.
\end{theorem}
We would like to be able to synthesize these lenses automatically, given a
specification as two regular expressions, an a set of values that are
mapped to each other.


We can alternatively do this through alternate semantics where semantics are defined on the typing derivations. \ref{fig:lens-alternate-semantics}

