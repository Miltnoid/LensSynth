We first must prove some Lemmas.

\begin{lemma}[Completeness of Composition]\leavevmode
\label{lem:composition-completeness}
\begin{enumerate}
\item If there are two atom lenses
$\AtomLens_1 \OfType \Atom_1 \Leftrightarrow \Atom_2$ and
$\AtomLens_2 \OfType \Atom_2 \Leftrightarrow \Atom_3$,
then there exists an atom lens
$\AtomLens \OfType \Atom_1 \Leftrightarrow \Atom_3$, such that
$\SemanticsOf{\AtomLens}=\SetOf{(\String_1,\String_3)\SuchThat
\exists\String_2
(\String_1,\String_2)\in\SemanticsOf{\AtomLens_1}\BooleanAnd
(\String_2,\String_3)\in\SemanticsOf{\AtomLens_2}}$

\item If there are two sequence lenses
$\SequenceLens_1 \OfType \Sequence_1 \Leftrightarrow \Sequence_2$ and
$\SequenceLens_2 \OfType \Sequence_2 \Leftrightarrow \Sequence_3$,
then there exists an sequence lens
$\SequenceLens \OfType \Sequence_1 \Leftrightarrow \Sequence_3$, such that
$\SemanticsOf{\SequenceLens}=\SetOf{(\String_1,\String_3)\SuchThat
\exists\String_2
(\String_1,\String_2)\in\SemanticsOf{\SequenceLens_1}\BooleanAnd
(\String_2,\String_3)\in\SemanticsOf{\SequenceLens_2}}$

\item If there are two DNF lenses
$\DNFLens_1 \OfType \DNFRegex_1 \Leftrightarrow \DNFRegex_2$ and
$\DNFLens_2 \OfType \DNFRegex_2 \Leftrightarrow \DNFRegex_3$,
then there exists a DNF lens
$\DNFLens \OfType \DNFRegex_1 \Leftrightarrow \DNFRegex_3$, such that
$\SemanticsOf{\DNFLens}=\SetOf{(\String_1,\String_3)\SuchThat
\exists\String_2
(\String_1,\String_2)\in\SemanticsOf{\DNFLens_1}\BooleanAnd
(\String_2,\String_3)\in\SemanticsOf{\DNFLens_2}}$
\end{enumerate}
\end{lemma}

\begin{proof}
By mutual induction

Let $\StarOf{\DNFRegex_1}$, $\StarOf{\DNFRegex_2}$, $\StarOf{\DNFRegex_3}$
be three atoms, and $\IterateLens{\DNFLens_1} \OfType
\StarOf{\DNFRegex_1} \Leftrightarrow \StarOf{\DNFRegex_2}$ with
$\IterateLens{\DNFLens_2} \OfType
\StarOf{\DNFRegex_2} \Leftrightarrow \StarOf{\DNFRegex_3}$
lenses between them.
By induction assumption, there a typing of a lens

\begin{mathpar}
\inferrule[]
{
\Derivation{}
}
{
\DNFLens \OfType \DNFRegex_1 \Leftrightarrow \DNFRegex_3
}
\end{mathpar}

such that $\SemanticsOf{\DNFLens}=\SetOf{(\String_1,\String_3)\SuchThat
\exists \String_2
(\String_1,\String_2)\in\SemanticsOf{\DNFLens_1}\BooleanAnd
(\String_2,\String_3)\in\SemanticsOf{\DNFLens_2}}$

By inversion lemma (TODO), there exists a $\IterateLens{\DNFLens_1}$ and
$\IterateLens{\DNFLens_2}$ came from $\IterateLensRule$, so I know
$\UnambigItOf{\LanguageOf{\DNFRegex_1}}$,
$\UnambigItOf{\LanguageOf{\DNFRegex_2}}$, and
$\UnambigItOf{\LanguageOf{\DNFRegex_3}}$.

Consider the lens

\begin{mathpar}
\inferrule*
{
\inferrule*
{
\Derivation{}
}
{
\DNFLens \OfType \DNFRegex_1 \Leftrightarrow \DNFRegex_3
}\\
\UnambigItOf{\LanguageOf{\DNFRegex_1}}\\
\UnambigItOf{\LanguageOf{\DNFRegex_2}}
}
{
\IterateLens{\DNFLens} \OfType
\StarOf{\DNFRegex_1} \Leftrightarrow \StarOf{\DNFRegex_3}
}
\end{mathpar}

This lens has the semantics

\begin{tabular}{@{}L@{}L@{}}
\SemanticsOf{\IterateLens{\DNFLens}}
& = \SetOf{(\String_{1,1}\Concat\ldots\Concat\String_{1,n},
\String_{3,1}\Concat\ldots\Concat\String_{3,n})\\
& \hspace{2em}
\SuchThat(\String_{1,i},\String_{3,i})\in\SemanticsOf{\DNFLens}}\\
& =
\SetOf{(\String_{1,1}\Concat\ldots\Concat\String_{1,n},
\String_{3,1}\Concat\ldots\Concat\String_{3,n})\\
& \hspace{2em}
\SuchThat
\exists\String_{2,i} (\String_{1,i},\String_{2,i})\in\SemanticsOf{\DNFLens_1}\\
& \hspace{4em}
\BooleanAnd
(\String_{2,i},\String_{3,i})\in\SemanticsOf{\DNFLens_2}}\\
& =
\SetOf{(\String_{1,1}\Concat\ldots\Concat\String_{1,n},
\String_{3,1}\Concat\ldots\Concat\String_{3,n}) \\
& \hspace{2em}
\SuchThat
\exists\String_{2,1}\Concat\ldots\Concat\String_{2,n}\\
& \hspace{4em}
(\String_{1,1}\Concat\ldots\Concat\String_{1,n},
\String_{2,1}\Concat\ldots\Concat\String_{2,n})
\in\SemanticsOf{\IterateLens{\DNFLens}}\\
& \hspace{4em}
\BooleanAnd
(\String_{2,1}\Concat\ldots\Concat\String_{2,n},
\String_{3,1}\Concat\ldots\Concat\String_{3,n})
\in\SemanticsOf{\IterateLens{\DNFLens}}}\\
& =
\SetOf{(\String_1,\String_3)\SuchThat\exists\String_2
(\String_1,\String_2)\in\SemanticsOf{\IterateLens{\DNFLens_1}}\\
& \hspace{4em}\BooleanAnd
(\String_2,\String_3)\in\SemanticsOf{\IterateLens{\DNFLens_2}}}
\end{tabular}

Let $\SequenceOf{\String_{1,0}\SequenceSep\Atom_{1,1}\SequenceSep
\ldots\SequenceSep\Atom_{1,n}\SequenceSep\String_{1,n}}$,
$\SequenceOf{\String_{2,0}\SequenceSep\Atom_{2,\Permutation_1(1)}\SequenceSep
\ldots\SequenceSep\Atom_{2,\Permutation_1(n)}\SequenceSep\String_{2,n}}$,
and $\SequenceOf{\String_{3,0}\SequenceSep
\Atom_{3,\Permutation_2\Compose\Permutation_1(1)}\SequenceSep
\ldots\SequenceSep\Atom_{3,\Permutation_2\Compose\Permutation_1(n)}
\SequenceSep\String_{3,n}}$ be sequences,
with $(\SequenceLensOf{(\String_{1,0},\String_{2,0})\SequenceLensSep
\AtomLens_{1,1}\SequenceLensSep\ldots\SequenceLensSep\AtomLens_{1,n}
\SequenceLensSep(\String_{1,n},\String_{2,n})},\Permutation_1)$ and
$(\SequenceLensOf{(\String_{2,0},\String_{3,0})\SequenceLensSep
\AtomLens_{2,1}\SequenceLensSep\ldots\SequenceLensSep\AtomLens_{2,n}
\SequenceLensSep(\String_{2,n},\String_{3,n})},\Permutation_2)$ be lenses between them.
By induction assumption, there is a typing of lenses
\begin{mathpar}
\inferrule[]
{
\Derivation_i
}
{
\AtomLens_i \OfType \Atom_{1,i} \Leftrightarrow \Atom_{3,i}
}
\end{mathpar}
such that \SemanticsOf{\AtomLens_i} = \SetOf{(\String_1,\String_3)\SuchThat
\exists \String_2 (\String_1,\String_2)\in\SemanticsOf{\AtomLens_{1,i}}
\BooleanAnd (\String_2,\String_3)\in\SemanticsOf{\Atom_{2,i}}}
Define $\Permutation = \Permutation_2\Compose\Permutation_1$.

By inversion, I know that
$\SequenceUnambigConcatOf{\SequenceOf{\String_{1,0}\SequenceSep\Atom_{1,1}
\SequenceSep\ldots\SequenceSep\Atom_{1,n}\SequenceSep\String_{1,n}}}$
and
$\SequenceUnambigConcatOf{\SequenceOf{\String_{3,0}\SequenceSep
\Atom_{\Permutation(3),1}\SequenceSep\ldots
\SequenceSep\Atom_{\Permutation(3),n}\SequenceSep\String_{3,n}}}$.

Consider the typing of the lens
\begin{mathpar}
{
\inferrule{
\inferrule*
{
\Derivation_0
}
{
\AtomLens_0 \OfType \Atom_{1,0} \Leftrightarrow \Atom_{3,0}
}\\
\ldots\\
\inferrule*
{
\Derivation_n
}
{
\AtomLens_n \OfType \Atom_{1,n} \Leftrightarrow \Atom_{3,n}
}\\
\sigma\in\PermutationSetOf{n}\\
\SequenceUnambigConcatOf{\SequenceOf{\String_{1,0}\SequenceSep\Atom_{1,1}
\SequenceSep\ldots\SequenceSep\Atom_{1,n}\SequenceSep\String_{1,n}}}\\
\SequenceUnambigConcatOf{\SequenceOf{\String_{3,0}\SequenceSep
\Atom_{\Permutation(3),1}\SequenceSep\ldots
\SequenceSep\Atom_{\Permutation(3),n}\SequenceSep\String_{3,n}}}
}
{
(\SequenceLensOf{(\String_{1,0},\String_{3,0})\SequenceLensSep\AtomLens_1
\SequenceLensSep\ldots\SequenceLensSep\AtomLens_n
\SequenceLensSep(\String_{1,n},\String_{3,n})},\Permutation) \OfType\\
\SequenceOf{\String_{1,0}\SequenceSep\Atom_{1,1}\SequenceSep\ldots
\SequenceSep\Atom_{1,n}\SequenceSep\String_{1,n}} \Leftrightarrow
\SequenceOf{\String_{3,0}\SequenceSep\Atom_{3,\Permutation(1)}\SequenceSep\ldots
\SequenceSep\Atom_{3,\Permutation(n)}\SequenceSep\String_{3,n}}
}
}
\end{mathpar}

Furthermore, we can prove the desired property of the semantics.\\
\SemanticsOf{(\SequenceLensOf{(\String_{1,0},\String_{3,0})\SequenceLensSep
\AtomLens_1\SequenceLensSep\ldots\SequenceLensSep\AtomLens_n
\SequenceLensSep(\String_{1,n},\String_{3,n})},\Permutation)}=\\
\SetOf{(\String_{1,0}\Concat\String_1\Concat\ldots
\Concat\String_n\Concat\String_{1,n},
\String_{3,0}\Concat\StringAlt_{\Permutation(1)}\Concat\ldots
\Concat\StringAlt_{\Permutation(n)}\Concat\String_{1,n})\\
\hspace*{1em}\SuchThat(\String_i,\StringAlt_i)\in\SemanticsOf{\AtomLens_i}}=\\
\SetOf{(\String_{1,0}\Concat\String_1\Concat\ldots
\Concat\String_n\Concat\String_{1,n},
\String_{3,0}\Concat\StringAlt_{\Permutation(1)}\Concat\ldots
\Concat\StringAlt_{\Permutation(n)}\Concat\String_{1,n})\\
\hspace*{1em}\SuchThat\exists\String_i'\in\LanguageOf{\Atom_{2,i}}
(\String_i,\String_i')\in\SemanticsOf{\AtomLens_i}\BooleanAnd
(\String_i',\StringAlt_i)\in\SemanticsOf{\AtomLens_i}}\\
\SetOf{(\String_{1,0}\Concat\String_1\Concat\ldots
\Concat\String_n\Concat\String_{1,n},
\String_{3,0}\Concat\StringAlt_{\Permutation(1)}\Concat\ldots
\Concat\StringAlt_{\Permutation(n)}\Concat\String_{1,n})\\
\hspace*{1em}\SuchThat\exists
\String_{2,0}\Concat\String_{\Permutation_1(1)}'\Concat\ldots\Concat
\String_{\Permutation_1(n)}'\Concat\String_{2,n}\in\LanguageOf{\Atom_{2,i}}\\
\hspace*{2em}
(\String_{1,0}\Concat\String_1\Concat\ldots
\Concat\String_n\Concat\String_{1,n},\String_{2,0}\Concat
\String_{\Permutation_1(1)}'\Concat\ldots\Concat
\String_{\Permutation_1(n)}'\Concat\String_{2,n})\\
\hspace*{3em}\in\SemanticsOf{(\SequenceLensOf{(\String_{1,0},\String_{2,0})
\SequenceLensSep\AtomLens_{1,1}\SequenceLensSep\ldots\SequenceLensSep
\AtomLens_{1,n}\SequenceLensSep(\String_{1,n},\String_{2,n})},\Permutation_1)}
\BooleanAnd\\
\hspace*{2em}
(\String_{2,0}\Concat\String_{\Permutation_1(1)}'\Concat\ldots\Concat
\String_{\Permutation_1(n)}'\Concat\String_{2,n},\String_{3,0}\Concat\StringAlt_{\Permutation(1)}\Concat\ldots
\Concat\StringAlt_{\Permutation(n)}\Concat\String_{1,n})\\
\hspace*{3em}\in\SemanticsOf{(\SequenceLensOf{(\String_{2,0},\String_{3,0})\SequenceLensSep
\AtomLens_{2,1}\SequenceLensSep\ldots\SequenceLensSep\AtomLens_{2,n}
\SequenceLensSep(\String_{2,n},\String_{3,n})},\Permutation_2)}}\\
\SetOf{(\String_1,\String_3)\SuchThat\exists\String_2\\
\hspace*{2em}
(\String_1,\String_2)\\
\hspace*{3em}\in\SemanticsOf{(\SequenceLensOf{(\String_{1,0},\String_{2,0})
\SequenceLensSep\AtomLens_{1,1}\SequenceLensSep\ldots\SequenceLensSep
\AtomLens_{1,n}\SequenceLensSep(\String_{1,n},\String_{2,n})},\Permutation_1)}\\
\hspace*{2em}
(\String_2,\String_3)\\
\hspace*{3em}\in\SemanticsOf{(\SequenceLensOf{(\String_{2,0},\String_{3,0})
\SequenceLensSep\AtomLens_{2,1}\SequenceLensSep\ldots\SequenceLensSep
\AtomLens_{2,n}\SequenceLensSep(\String_{2,n},\String_{3,n})},\Permutation_2)}}

Let $\DNFRegex_1=\DNFOf{\Sequence_{1,1}\DNFSep\ldots\DNFSep\Sequence_{1,n}}$,
$\DNFRegex_2=\DNFOf{\Sequence_{2,\Permutation_1(1)}\DNFSep\ldots
\DNFSep\Sequence_{2,\Permutation_1(n)}}$,
and $\DNFRegex_3=\DNFOf{\Sequence_{3,\Permutation_2\Compose\Permutation_1(1)}
\DNFSep\ldots\DNFSep\Sequence_{3,\Permutation_2\Compose\Permutation_1(n)}}$
be three DNF regular expressions.
Let $\DNFLens_1=(\DNFLensOf{\SequenceLens_{1,1}\DNFLensSep\ldots
\DNFLensSep\SequenceLens_{1,n}},\Permutation_1)\OfType
\DNFRegex_1\Leftrightarrow\DNFRegex_2$ and
$\DNFLens_2=(\DNFLensOf{\SequenceLens_{2,1}\DNFLensSep\ldots
\DNFLensSep\SequenceLens_{2,n}},\Permutation_2)\OfType
\DNFRegex_2\Leftrightarrow\DNFRegex_3$ be lenses between them.
By induction assumption, there exists a typing of lenses
\begin{mathpar}
\inferrule*
{
\Derivation_i
}
{
\Sequence_i \OfType \Sequence_{1,i} \Leftrightarrow \Sequence_{3,i}
}
\end{mathpar}


\end{proof}

\begin{lemma}[Completeness of Identity on Strongly Unambiguous DNF Regex,
Clauses, and Atoms]
\leavevmode
\begin{enumerate}
\item If $\DNFRegex$ is a strongly unambiguous DNF Regular expression, then
there exists a DNF lens $\DNFLens \OfType \DNFRegex \Leftrightarrow \DNFRegex$,
such that $\SemanticsOf{\DNFLens}=
\SetOf{(\String,\String)\SuchThat\String\in\LanguageOf{\DNFRegex}}$, where
\DNFLens{} typing includes no rewrite rules.
\item If $\Sequence$ is a strongly unambiguous sequence, then
there exists a sequence lens $\SequenceLens \OfType \Sequence \Leftrightarrow \Sequence$,
such that $\SemanticsOf{\SequenceLens}=
\SetOf{(\String,\String)\SuchThat\String\in\LanguageOf{\DNFRegex}}$, where
\SequenceLens{} typing includes no rewrite rules.
\item If $\Atom$ is a strongly unambiguous atom, then
there exists an atom lens $\AtomLens \OfType \Atom \Leftrightarrow \Atom$,
such that $\SemanticsOf{\AtomLens}=
\SetOf{(\String,\String)\SuchThat\String\in\LanguageOf{\DNFRegex}}$, where
\AtomLens{} typing includes no rewrite rules.
\end{enumerate}
\end{lemma}
\begin{proof}
By mutual induction on the structure of the DNF regular expression,
atom, and clause.

Let \DNFRegex{} be a DNF regular expression.

Let \Sequence{} be a sequence.

Let \Atom{} be an atom.
\end{proof}

Now we can do the proof of completeness.
\dnflc*

\begin{proof}
We will do by structural induction on the typing derivation.

Let the last rule be an application of \ConstantLensRule{}.
\begin{mathpar}
\inferrule[]
{
\String_1 \in \StarOf{\Sigma}\\
\String_2 \in \StarOf{\Sigma}
}
{
\ConstLens{\String_1}{\String_2} \OfType \String_1 \Leftrightarrow \String_2
}
\end{mathpar}
Consider the derivation
\begin{mathpar}
\inferrule[]
{
\inferrule[]
{
\String_1 \in \StarOf{\Sigma}\\
\String_2 \in \StarOf{\Sigma}
}
{
\SequenceLensOf{(\String_1,\String_2)} \OfType
\SequenceOf{\String_1} \Leftrightarrow \SequenceOf{\String_2}
}
}
{
\DNFLensOf{\SequenceLensOf{(\String_1,\String_2)}} \OfType
\DNFOf{\SequenceOf{\String_1}} \Leftrightarrow \DNFOf{\SequenceOf{\String_2}}
}
\end{mathpar}

\begin{tabular}{@{}L@{}L@{}}
\SemanticsOf{\DNFLensOf{\SequenceLensOf{(\String_1,\String_2)}}}
& = \SetOf{(\StringAlt_1,\StringAlt_2)
\SuchThat(\StringAlt_1,\StringAlt_2)\in\SequenceLensOf{(\String_1,\String_2)}}\\
& = \SetOf{(\StringAlt_1,\StringAlt_2)
\SuchThat(\StringAlt_1,\StringAlt_2)\in\SetOf{(\String_1,\String_2)}}\\
& = \SetOf{(\String_1,\String_2)}\\
& = \SemanticsOf{\ConstLens{\String_1}{\String_2}}\\
\end{tabular}

Let the last rule be an application of \IdentityLensRule{}.
We know that for each regular expression there exists a strongly unambiguous
regular expression~\cite{unambigregex}.
Because of that, if we have
$\IdentityLens \OfType \Regex \Leftrightarrow \Regex$,
then we have a strongly unambiguous DNF regular expression,
$\DNFRegex$, such that $\LanguageOf{\Regex}=\LanguageOf{\DNFRegex}$.
Then, by Lemma (TODO: ref), there exists a DNF lens
$\DNFLens \OfType \DNFRegex \Leftrightarrow \DNFRegexAlt$ such that
$\SemanticsOf{\DNFLens}=
\SetOf{(\String,\String)\SuchThat\String\in\LanguageOf{\DNFRegex}}$

Let the last rule be an application of \IterateLensRule{}.
\begin{mathpar}
\inferrule[]
{
\Lens \OfType \Regex \Leftrightarrow \RegexAlt \\
\UnambigItOf{\LanguageOf{\Regex}}\\
\UnambigItOf{\LanguageOf{\RegexAlt}}
}
{
\IterateLens{\Lens} \OfType \StarOf{\Regex} \Leftrightarrow \StarOf{\RegexAlt}
}
\end{mathpar}
By induction assumption, there exists a derivation
\begin{mathpar}
\inferrule[]
{
\Derivation{}
}
{
\DNFLens : \DNFRegex \Leftrightarrow \DNFRegexAlt
}
\end{mathpar}
such that \LanguageOf{\DNFRegex}=\LanguageOf{\Regex},
\LanguageOf{\RegexAlt}=\LanguageOf{\DNFRegexAlt}, and
\SemanticsOf{\DNFLens}=\SemanticsOf{\Lens}.
This means \UnambigItOf{\LanguageOf{\DNFRegex}}
and \UnambigItOf{\LanguageOf{\DNFRegexAlt}}

Consider the derivation
\begin{mathpar}
\inferrule[]
{
\Derivation
}
{
\inferrule[]
{
\DNFLens : \DNFRegex \Leftrightarrow \DNFRegexAlt
\UnambigItOf{\LanguageOf{\DNFRegex}}\\
\UnambigItOf{\LanguageOf{\DNFRegexAlt}}\\
}
{
\inferrule[]
{
\IterateLens{\DNFLens} \OfType
\StarOf{\DNFRegex} \Leftrightarrow \StarOf{\DNFRegexAlt}
}
{
\inferrule
{
\SequenceLensOf{\IterateLens{\DNFLens}} \OfType
\SequenceOf{\StarOf{\DNFRegex}}\Leftrightarrow \SequenceOf{\StarOf{\DNFRegexAlt}}
}
{
\DNFLensOf{\SequenceLensOf{\IterateLens{\DNFLens}}} \OfType
\DNFOf{\SequenceOf{\StarOf{\DNFRegex}}} \Leftrightarrow
\DNFOf{\SequenceOf{\StarOf{\DNFRegexAlt}}}
}
}
}
}
\end{mathpar}

By Lemma~(TODOREF), I know that, as \LanguageOf{\Regex}=\LanguageOf{\DNFRegex},
and as \LanguageOf{\RegexAlt}=\LanguageOf{\DNFRegexAlt}, then
\LanguageOf{\DNFOf{\SequenceOf{\StarOf{\DNFRegex}}}}=
\LanguageOf{\StarOf{\Regex}}
and \LanguageOf{\DNFOf{\SequenceOf{\StarOf{\DNFRegexAlt}}}}=
\LanguageOf{\StarOf{\RegexAlt}}.

\begin{tabular}{@{}L@{}L@{}}
\SemanticsOf{\DNFLensOf{\SequenceLensOf{\IterateLens{\DNFLens}}}}
& = \SetOf{(\String,\StringAlt)
\SuchThat(\String,\StringAlt)\in\SemanticsOf{\SequenceLensOf{\IterateLens{\DNFLens}}}}\\
& = \SetOf{(\String,\StringAlt)
\SuchThat(\String,\StringAlt)\in\SemanticsOf{\IterateLens{\DNFLens}}}\\
& = \SetOf{(\String_1\Concat\ldots\Concat\String_n,
\StringAlt_1\Concat\ldots\Concat\StringAlt_n)
\SuchThat \String_i\in\SemanticsOf{\DNFLens}}\\
& = \SetOf{(\String_1\Concat\ldots\Concat\String_n,
\StringAlt_1\Concat\ldots\Concat\StringAlt_n)
\SuchThat \String_i\in\SemanticsOf{\Lens}}\\
& = \SemanticsOf{\IterateLens{\Lens}}
\end{tabular}

Let the last rule be an application of \ConcatLensRule{}.

Let the last rule be an application of \SwapLensRule{}.

Let the last rule be an application of \OrLensRule{}.
\begin{mathpar}
\inferrule[]
{
\Lens_1 \OfType \Regex_1 \Leftrightarrow \RegexAlt_1\\
\Lens_2 \OfType \Regex_2 \Leftrightarrow \RegexAlt_2\\
\UnambigOrOf{\LanguageOf{\Regex_1}}{\LanguageOf{\Regex_2}}\\
\UnambigOrOf{\LanguageOf{\RegexAlt_1}}{\LanguageOf{\RegexAlt_2}}
}
{
\OrLens{\Lens_1}{\Lens_2} \OfType
\RegexOr{\Regex_1}{\RegexAlt_1} \Leftrightarrow \RegexOr{\Regex_2'}{\RegexAlt_2'}
}
\end{mathpar}
By induction assumption, there exists a derivation
\begin{mathpar}
\inferrule[]
{
\Derivation
}
{
(\DNFLensOf{\SequenceLens_1,\ldots,\SequenceLens_n},\Permutation_1) \OfType\\
\DNFOf{\Sequence_1\DNFSep\ldots\DNFSep\Sequence_n} \Leftrightarrow
\DNFOf{\SequenceAlt_{\Permutation_1(1)}\DNFSep\ldots
\DNFSep\SequenceAlt_{\Permutation_1(n)}}
}
\end{mathpar}
and a derivation
\begin{mathpar}
\inferrule[]
{
\Derivation'
}
{
(\DNFLensOf{\SequenceLens_{n+1},\ldots,\SequenceLens_{n+m}},\Permutation_2) \OfType\\
\DNFOf{\Sequence_{n+1}'\DNFSep\ldots\DNFSep\Sequence_{n+m}} \Leftrightarrow
\DNFOf{\SequenceAlt_{n+\Permutation_2(1)'}\DNFSep\ldots
\DNFSep\SequenceAlt_{n+\Permutation_2(m)'}}
}
\end{mathpar}
such that, if we define:\\
\begin{tabular}{@{}L@{}L@{}}
\DNFRegex_1= &
\DNFOf{\Sequence_1\DNFSep\ldots\DNFSep\Sequence_n}\\
\DNFRegexAlt_1= &
\DNFOf{\SequenceAlt_{\Permutation_1(1)}\DNFSep\ldots
\DNFSep\SequenceAlt_{\Permutation_1(n)}}\\
\DNFRegex_2= &
\DNFOf{\Sequence_{n+1}\DNFSep\ldots\DNFSep\Sequence_{n+m}}\\
\DNFRegexAlt_2= &
\DNFOf{\SequenceAlt_{n+\Permutation_2(1)}\DNFSep\ldots
\DNFSep\SequenceAlt_{n+\Permutation_2(m)}}\\
\end{tabular}
then we have
\LanguageOf{\DNFRegex}=\LanguageOf{\Regex},
\LanguageOf{\DNFRegexAlt}=\LanguageOf{\RegexAlt},
\LanguageOf{\DNFRegex'}=\LanguageOf{\Regex'},
\LanguageOf{\DNFRegexAlt'}=\LanguageOf{\RegexAlt'}.

By inversion (TODO: inversion lemma), we know that the last step of those derivations were applications
of \DNFLensRule{}, so we know that, for all $i\in\RangeIncInc{1}{m}$,
there exists derivations
\begin{mathpar}
\inferrule[]
{
\Derivation_i
}
{
(\SequenceLens_i \OfType
\Sequence_i \Leftrightarrow \SequenceAlt_i)
}
\end{mathpar}
For all $i\neq j$, $i,j\in\RangeIncInc{1}{n}$,
$\Sequence_i \Intersect \Sequence_j = \emptyset$, and
$\SequenceAlt_i \Intersect \SequenceAlt_j = \emptyset$.
For all $i\neq j$, $i,j\in\RangeIncInc{n+1}{n+m}$,
$\Sequence_i \Intersect \Sequence_j = \emptyset$, and
$\SequenceAlt_i \Intersect \SequenceAlt_j = \emptyset$.
Because $\LanguageOf{\Regex}\Intersect\LanguageOf{\RegexAlt}=\emptyset$,
and $\LanguageOf{\Regex'}\Intersect\LanguageOf{\RegexAlt'}=\emptyset$,
and because of the previously stated equivalences between regular expressions,
and DNF regular expressions, for each $i\in\RangeIncInc{1}{n}$,
$j\in\RangeIncInc{n+1}{m}$, $\Sequence_i\Intersect\Sequence_j=\emptyset$
and $\SequenceAlt_i\Intersect\SequenceAlt_j=\emptyset$.
So, we have for all $i\neq j$, $i,j\in\RangeIncInc{1,n+m}$,
$\Sequence_i \Intersect \Sequence_j = \emptyset$ and
$\SequenceAlt_i \Intersect \SequenceAlt_j = \emptyset$.
We can build a new permutation $\Permutation$ defined as
\[
\Permutation(x)=
\begin{cases}
\Permutation_1(x) & \text{ if $x\in\RangeIncInc{1}{n}$}\\
\Permutation_2(x-n)+n & \text{ if $x\in\RangeIncInc{n+1}{n+m}$}
\end{cases}
\]
Consider the derivation
\begin{mathpar}
\inferrule[]
{
\Derivation_1\\
\ldots\\
\Derivation_{n+m}\\
i\neq j\Rightarrow \Sequence_i\Intersect\Sequence_j=\emptyset\\
i\neq j\Rightarrow \Sequence_i\Intersect\SequenceAlt_j=\emptyset\\
}
{
(\DNFLensOf{\SequenceLens_1\SequenceLensSep\ldots
\SequenceLensSep\SequenceLens_{n+m}},\Permutation) \OfType\\
\DNFOf{\Sequence_1\DNFSep\ldots\DNFSep\Sequence_{n+m}} \Leftrightarrow
\DNFOf{\SequenceAlt_{\sigma(1)}\DNFSep\ldots\DNFSep\SequenceAlt_{\sigma(n+m)}}
}
\end{mathpar}

We can prove the resulting DNF regular expressions are equivalent to
the original regular expressions.

\begin{tabular}{@{}L@{}L@{}}
\DNFOf{\Sequence_0\DNFSep\ldots\DNFSep\Sequence_{n+m}} & =
\DNFOf{\Sequence_1\DNFSep\ldots\DNFSep\Sequence_n}\OrDNF\\
& \hspace{1em}
\DNFOf{\Sequence_{n+1}\DNFSep\ldots\DNFSep\Sequence_{n+m}}\\
& =
\OrDNFOf{\DNFRegex_1}{\DNFRegex_2}
\end{tabular}\\
So by Lemma (TODO: find ref),
$\LanguageOf{\DNFOf{\Sequence_0\DNFSep\ldots\DNFSep\Sequence_{n+m}}}=
\LanguageOf{\RegexOr{\Regex_1}{\Regex_2}}$

\begin{tabular}{@{}L@{}L@{}}
\DNFOf{\SequenceAlt_{\Permutation(0)}\DNFSep\ldots\DNFSep
\SequenceAlt_{\Permutation(n+m)}} & =
\DNFOf{\SequenceAlt_{\Permutation(1)}\DNFSep\ldots\DNFSep
\SequenceAlt_{\Permutation(n)}}\OrDNF\\
& \hspace{1em}
\DNFOf{\Sequence_{\Permutation(n+1)}\DNFSep\ldots\DNFSep
\Sequence_{\Permutation(n+m)}}\\
& =
\DNFOf{\SequenceAlt_{\Permutation_1(1)}\DNFSep\ldots\DNFSep
\SequenceAlt_{\Permutation_1(n)}}\OrDNF\\
& \hspace{1em}
\DNFOf{\Sequence_{n+\Permutation_2(1)}\DNFSep\ldots\DNFSep
\Sequence_{n+\Permutation_2(m)}}\\
& =
\OrDNFOf{\DNFRegexAlt_1}{\DNFRegexAlt_2}
\end{tabular}\\
So by Lemma (TODO: find ref),
$\LanguageOf{\DNFOf{\SequenceAlt_{\Permutation(0)}\DNFSep\ldots\DNFSep
\SequenceAlt_{\Permutation(n+m)}}}=
\LanguageOf{\RegexOr{\RegexAlt_1}{\RegexAlt_2}}$

Furthermore, we can prove it has the same semantics.

\begin{tabular}{@{}L@{}L@{}}
\SemanticsOf{\DNFLensOf{\Sequence_1\DNFLensSep\ldots\DNFLensSep\Sequence_{n+m}},
\Permutation} & =
\SetOf{(\String,\StringAlt)\SuchThat
(\String,\StringAlt)\in\SemanticsOf{\Sequence_i}\\
& \hspace{3em}
\text{for some $i\in\RangeIncInc{1}{n+m}$}}\\
& =
\SetOf{(\String,\StringAlt)\SuchThat
(\String,\StringAlt)\in\SemanticsOf{\Sequence_i}\\
& \hspace{3em}
\text{for some $i\in\RangeIncInc{1}{n}$}\\
& \hspace{2em}\BooleanOr(\String,\StringAlt)\in\SemanticsOf{\Sequence_i}\\
& \hspace{3em}\text{for some $i\in\RangeIncInc{n+1}{n+m}$}}\\
& =
\SetOf{(\String,\StringAlt)\SuchThat
(\String,\StringAlt)\in\SemanticsOf{\Lens_1}\\
& \hspace{2em}\BooleanOr(\String,\StringAlt)\in\SemanticsOf{\Lens_2}}\\
& =
\SemanticsOf{\OrLens{\Lens_1}{\Lens_2}}
\end{tabular}

Let the last rule be an application of \ComposeLensRule{}, so we have
$\Lens_1 \OfType \Regex_1 \Leftrightarrow \Regex_2$, and
$\Lens_2 \OfType \Regex_2 \Leftrightarrow \Regex_3$,
and $\ComposeLens{\Lens_1}{\Lens_2} \OfType \Regex_1 \Leftrightarrow \Regex_3$.
By induction assumption, there is a typing of
$\DNFLens_1 \OfType \DNFRegex_1 \Leftrightarrow \DNFRegex_2$,
and $\DNFLens_2 \OfType \DNFRegex_2 \Leftrightarrow \DNFRegex_3$, where
$\SemanticsOf{\DNFLens_1}=\SemanticsOf{\Lens_1}$,
$\SemanticsOf{\DNFLens_2}=\SemanticsOf{\Lens_2}$,
$\LanguageOf{\DNFRegex_1}=\LanguageOf{\Regex_1}$,
$\LanguageOf{\DNFRegex_2}=\LanguageOf{\Regex_2}$, and
$\LanguageOf{\DNFRegex_3}=\LanguageOf{\Regex_3}$.
By Lemma~\ref{lem:composition-completeness},
there is a lens $\DNFLens \OfType \DNFRegex_1 \Leftrightarrow \DNFRegex_3$,
such that $\SemanticsOf{\DNFLens}=\SetOf{(\String_1,\String_3)\SuchThat
\exists \String_2 (\String_1,\String_2)\in\SemanticsOf{\DNFLens_1}\BooleanAnd
(\String_2,\String_3)\in\SemanticsOf{\DNFLens_2}}=
\SemanticsOf{\ComposeLens{\Lens_1}{\Lens_2}}$.

Let the last rule be an application of \RetypeLensRule{}.
\begin{mathpar}
\inferrule[]
{
\Lens \OfType \Regex \Leftrightarrow \RegexAlt \\
\Regex \equiv \Regex'\\
\RegexAlt \equiv \RegexAlt'
}
{
\Lens \OfType \Regex' \Leftrightarrow \RegexAlt'
}
\end{mathpar}
By induction assumption, there exists a derivation

\begin{mathpar}
\inferrule[]
{
\Derivation
}
{
\DNFLens \OfType \DNFRegex \Leftrightarrow \DNFRegexAlt
}
\end{mathpar}

where \LanguageOf{\DNFRegex}=\LanguageOf{\Regex}
and \LanguageOf{\DNFRegexAlt}=\LanguageOf{\RegexAlt},
and \SemanticsOf{\DNFLens}=\SemanticsOf{\Lens}.
Because \LanguageOf{\Regex'}=\LanguageOf{\Regex}
and \LanguageOf{\RegexAlt'}=\LanguageOf{\RegexAlt},
we know \LanguageOf{\Regex'}=\LanguageOf{\DNFRegex}
and \LanguageOf{\RegexAlt'}=\LanguageOf{\DNFRegexAlt'}.
\end{proof}

With some additional lemmas related to an equational theory of
DNF Regular Expressions, we are able to prove an even stronger result.

We can use the equational theory for unambiguous regular expressions, and the conversion
from regular expressions into DNF regular expressions, to create an equational
theory for DNF regular expressions.

\begin{theorem}[Unambiguous Regex Equational Theory]
The Equational Theory for Unambiguous Regular Expressions is the Equational
Theory for Regular Expressions without \StarstarRule{}.
\end{theorem}
\begin{proof}
TODO: Same approach of showing things commute.
\end{proof}

\begin{definition}[Equational Theory for Unambiguous DNF Regular Expressions]
\leavevmode\\
\begin{tabular}{lrcl}
\EqualityRule{} & \Regex{} & $\equiv$ & \Regex{} \\
\OrCommutativityRule{} & \OrDNFOf{\DNFRegex{}}{\DNFRegexAlt{}} & $\equiv$
& \OrDNFOf{\DNFRegexAlt{}}{\DNFRegex{}}\\
\SumstarRule{} & \DNFOf{\SequenceOf{\StarOf{(\OrDNFOf{\DNFRegex{}}{\DNFRegexAlt{}})}}} & $\equiv$ & \DNFOf{\SequenceOf{\StarOf{(\ConcatDNFOf{\DNFOf{\SequenceOf{\StarOf{\DNFRegex{}}}}}{\DNFRegexAlt{}})}}}\\
& & & \ConcatDNF\DNFOf{\SequenceOf{\StarOf{\DNFRegex{}}}}\\
\ProductstarRule{} & \StarOf{(\ConcatDNFOf{\DNFRegex{}}{\DNFRegexAlt{}})} & $\equiv$ & \DNFOf{\SequenceOf{\EmptyString{}}}\OrDNF(\DNFRegex{}\\
& & & \ConcatDNF\DNFOf{\SequenceOf{\StarOf{(\DNFRegexAlt{}\ConcatDNF\DNFRegex{})}}}\ConcatDNF\DNFRegexAlt{})\\
\DicyclicityRule{} & \DNFOf{\SequenceOf{\StarOf{(\OrDNFOf{\DNFRegex}{\DNFRegexAlt})}}} & $\equiv$ & $\DNFLeft\SequenceLeft((\OrDNFOf{\DNFRegex}{\DNFRegexAlt})\ConcatDNF\DNFRegexAlt\OrDNF$\\
& & & ${(\DNFRegex\ConcatDNF\StarOf{\DNFRegexAlt})}\ConcatDNF\DNFRegex)\Star{}\SequenceRight\DNFRight\ConcatDNF$\\
& & & $(\EmptyString\OrDNF(\OrDNFOf{\DNFRegex}{\DNFRegexAlt})\ConcatDNF$\\
& & & $({(\ConcatDNFOf{\DNFRegex}{\StarOf{\DNFRegexAlt}})}^0\OrDNF\ldots$\\
& & & $\OrDNF{(\ConcatDNFOf{\DNFRegex}{\StarOf{\DNFRegexAlt}})}^n))$
\end{tabular}
\end{definition}

\begin{theorem}[Soundness and Completeness of Equational Theory for Unambiguous DNF Regular
Expressions]
\end{theorem}
\begin{proof}
The Equational Theory of Unambiguous Regular Expressions stems merely an
application of the mapping from regular expressions to DNF regular expressions
to the equational theory of regular expressions.
As this mapping is surjective, we know that for a sequence of rules which show
that two regular expressions are equivalent, we can apply the DNF version
of those rules to obtain a sequence of DNF regular expressions, which are
the DNF versions of the regular expressions.  Furthermore, because
we know that the mapping from regular expressions to DNF regular expressions
is sound, we know that this equational theory is sound.
\end{proof}

\begin{lemma}[Equivalence of DNF permutations]
\label{lem:dnfperm}
For all permutations, \Permutation{},
$\LanguageOf{\DNFOf{\Sequence_1\DNFSep\ldots\DNFSep\Sequence_n}}=
\LanguageOf{\DNFOf{\Sequence_{\Permutation(1)}\DNFSep\ldots\DNFSep
\Sequence_{\Permutation(n)}}}$
\end{lemma}
\begin{proof}
Repeated application of \OrCommutativityRule{}.
\end{proof}

\begin{lemma}[Respectfulness of Languages in Rewrites]\leavevmode
\label{lem:language-rewrite-safe}
\begin{enumerate}
\item If $\Atom\RewriteAtom\DNFRegex$, then
$\LanguageOf{\Atom}=\LanguageOf{\DNFRegex}$.
\item If $\DNFRegex\RewriteDNF\DNFRegex'$, then
$\LanguageOf{\DNFRegex}=\LanguageOf{\DNFRegex'}$.
\end{enumerate}
\end{lemma}
\begin{proof}
By mutual induction, use of \ref{lem:dnfperm}, and use of the equational
theory of unambiguous regular expressions.
\end{proof}


Now we can prove the stronger result.
\begin{theorem}[Strong Completeness of DNF Lenses]
If there exists a derivation of $\Lens \OfType \MapsBetweenTypeOf{\Regex}{\RegexAlt} \HasSemantics \PutRight,\PutLeft$,
and $\DNFRegex$ and $\DNFRegexAlt$ be any DNF regular expressions
such that 
$\LanguageOf{\DNFRegex}=\LanguageOf{\Regex}$ and
$\LanguageOf{\DNFRegexAlt}=\LanguageOf{\RegexAlt}$,
then there exists a derivation of $\DNFLens \OfType \MapsBetweenTypeOf{\DNFRegex}{\DNFRegexAlt}$ such that
$\SemanticsOf{\DNFLens}=\SemanticsOf{\Lens}$.
\end{theorem}
