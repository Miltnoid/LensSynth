\begin{figure}
\begin{tabular}{l@{\ }l@{\ }c@{\ }l@{\ }r}

% DNF_REGEX
(Atoms)& \Atom{},\AtomAlt{} & \GEq{} & \StarOf{\DNFRegex{}} & Iterate DNF\\
(Sequence)& \Sequence{},\SequenceAlt{} & \GEq{} &
$\SequenceOf{\String_0\SequenceSep\Atom_1\SequenceSep\ldots\SequenceSep\Atom_n\SequenceSep\String_n}$ & Conjoin\\
(DNF Regex)& \DNFRegex{},\DNFRegexAlt{} & \GEq{} & $\DNFOf{\Sequence_1\DNFSep\ldots\DNFSep\Sequence_n}$ & DNF Or\\
\end{tabular}
\caption{DNF Regex Syntax\bcp{Why the dot after conjuncts?  Ah, now I see;
    OK, but maybe it's better to choose two different kinds of brackets
    instead of distinguishing with . and + (I was also confused by the + --
    thought it must mean some kind of repetition).  How about
    $\langle...\rangle$ for disjuncts and $[...]$ for conjuncts?  (This is a
  bit suggestive of type-theoretic notation.)  Or how about ``hollow square
  brackets'' for conjuncts (using some kind of square brackets seems good,
  since they are ordered) and ``hollow set braces'' for disjuncts?  We
  should try to get this right, since these widgets are our core technical
  device.}} 
\label{fig:dnf-regex-syntax}
\end{figure}
