\begin{figure}
\centering
\begin{tabular}{l@{\hspace*{5mm}}l@{\ }c@{\ }l@{\hspace*{5mm}}l}

% REGEX
(String)& \String{},\StringAlt{} & \GEq{} & $\String\in\Star{\Sigma}$ \\
(Regexs)& \Regex{},\RegexAlt{} & \GEq{} & s & \it Base \\
& & & \GBar{} \Regex{}* & \it Star \\
& & & \GBar{} $\RegexConcat{\Regex_1}{\Regex_2}$ & \it Concat \\
& & & \GBar{} $\Regex_1 | \Regex_2$ & \it Or \\
\end{tabular}
\caption{Regex Syntax \bcp{Recommend using capital letters ($R$, $S$,
    ...) for regular expressions.  And left-justifying the RH column and
    putting it in italic (with more space separating it from the rest) so
    that it's clear that it's commentary.}}
\label{fig:regex-syntax}
\end{figure}

\begin{figure}
\[
\begin{array}{lcl}
\LanguageOf{\String} &=& \{\String\}\\
\LanguageOf{\emptyset} &=& \{\}\\
\LanguageOf{\RegexConcat{\Regex_1}{\Regex_2}} &=&
\{\StringConcat{\String_1}{\String_2} \SuchThat
\String_1\in\LanguageOf{\Regex_1} \BooleanAnd \String_2\in\LanguageOf{\Regex_2}\}\\
\LanguageOf{\RegexOr{\Regex_1}{\Regex_2}} &=&
\{\String \SuchThat
\String\in\LanguageOf{\Regex_1} \BooleanOr \String\in\LanguageOf{\Regex_2}\}\\
\LanguageOf{\StarOf{\Regex}} &=&
\{\String_1\Concat\ldots\Concat\String_n \SuchThat
n\in\Nats \wedge \String_i\in\LanguageOf{\Regex}\}
\end{array}
\]
\caption{Regex Semantics}
\label{fig:regex-semantics}
\end{figure}


\bcp{Now would be a good time to add some section headers so we can see
  where everything goes.}

The syntax for nonempty\bcp{why not empty ones too??} regular expressions
is given in Figure~\ref{fig:regex-syntax}.
These regular expressions have an underlying semantics, which expresses languages,
formalized in Figure~\ref{fig:regex-semantics}.



\begin{figure}

let $\Delta$ be the set of user defined data types.
let $\Sigma^*$ be the set of words over the alphabet $\Sigma$\\

\begin{tabular}{l@{\ }l@{\ }c@{\ }l@{\ }r}

% REGEX
(Lenses)& \Lens{} & \GEq{} & $\LensVariable$ & Variable\\
& & & \GBar{} $\ConstLens{s_1 \in \Star{\Sigma}}{s_2 \in \Star{\Sigma}}$ & Const \\
& & & \GBar{} $\IdentityLens$ & Identity\\
& & & \GBar{} $\IterateLens{\Lens}$ & Iterate \\
& & & \GBar{} $\ConcatLens{\Lens_1}{\Lens_2}$ & Concat \\
& & & \GBar{} $\SwapLens{\Lens_1}{\Lens_2}$ & Swap\\
& & & \GBar{} $\OrLens{\Lens_1}{\Lens_2}$ & Or\\
& & & \GBar{} $\ComposeLens{\Lens_1}{\Lens_2}$ & Compose\\
\end{tabular}
\caption{Lens Syntax}
\label{fig:lens-syntax}
\end{figure}

\begin{figure}
%A typing derivation is of the form $\FullContext \HasSemantics \Lens : \Regex \Leftrightarrow \RegexAlt \HasSemantics \PutRight,\PutLeft$.
%This means that under the regular expression context \RegexContext, and under the lens context \LensContext, the lens $\Lens$ has typing $\Regex \Leftrightarrow \RegexAlt$.
%It defines two functions, $\PutRight : \LanguageOf{\Delta}{\Regex}\rightarrow\LanguageOf{\Delta}{\RegexAlt}$, and $\PutLeft : \LanguageOf{\Delta}{\RegexAlt}\rightarrow\LanguageOf{\Delta}{\Regex}$.
%These functions are bijective functions, and are
%inverses of each other.
\centering
\begin{mathpar}
\inferrule[Constant Lens]
{
\String_1 \in \Star{\Sigma}\\
\String_2 \in \Star{\Sigma}\\
}
{
\FullContext \DerivesLens \ConstLens{\String_1}{\String_2} \OfType \String_1 \Leftrightarrow \String_2 \HasSemantics \lambda \String. \String_2 , \lambda \String. \String_1
}

\inferrule[Identity Lens]
{
}
{
\FullContext \DerivesLens \IdentityLens \OfType \Regex \Leftrightarrow \Regex \HasSemantics \lambda \String.\String, \lambda \String . \String
}

\inferrule[Iterate Lens]
{
\FullContext \DerivesLens \Lens \OfType \Regex \Leftrightarrow \RegexAlt \HasSemantics \PutRight , \PutLeft \\
\UnambigIt{\LanguageOf{\Delta}{\Regex}}\\
\UnambigIt{\LanguageOf{\Delta}{\RegexAlt}}\\
}
{
\FullContext \DerivesLens \IterateLens{\Lens} \OfType \Star{\Regex} \Leftrightarrow \Star{\RegexAlt} \HasSemantics\\\\
\lambda \String.\LetWhereIn{\String_1\Concat\ldots\Concat\String_n}{\String}{\String_i\in\LanguageOf{\Delta}{\Regex}} (\PutRight\Apply\String_1)\Concat\ldots\Concat(\PutRight\Apply\String_n),\\\\
\lambda \String.\LetWhereIn{\String_1\Concat\ldots\Concat\String_n}{\String}{\String_i\in\LanguageOf{\Delta}{\RegexAlt}} (\PutLeft\Apply\String_1)\Concat\ldots\Concat(\PutLeft\Apply\String_n)
}

\inferrule[Concat Lens]
{
\FullContext \DerivesLens \Lens_1 \OfType \Regex_1 \Leftrightarrow \RegexAlt_1 \HasSemantics \PutRight_1, \PutLeft_1\\
\FullContext \DerivesLens \Lens_2 \OfType \Regex_2 \Leftrightarrow \RegexAlt_2 \HasSemantics \PutRight_2, \PutLeft_2\\\\
\UnambigConcat{\LanguageOf{\Delta}{\Regex_1}}{\LanguageOf{\Delta}{\Regex_2}}\\
\UnambigConcat{\LanguageOf{\Delta}{\RegexAlt_1}}{\LanguageOf{\Delta}{\RegexAlt_2}}
}
{
\FullContext \DerivesLens \ConcatLens{\Lens_1}{\Lens_2} \OfType \Regex_1\Regex_2 \Leftrightarrow \RegexAlt_1\RegexAlt_2 \HasSemantics\\\\
\lambda \String.\LetWhereIn{\String_1\Concat\String_2}{\String}{\String_i\in\LanguageOf{\Delta}{\Regex_i}} (\PutRight_1\Apply\String_1)\Concat(\PutRight_2\Apply\String_2),\\\\
\lambda \String.\LetWhereIn{\String_1\Concat\String_2}{\String}{\String_i\in\LanguageOf{\Delta}{\RegexAlt_i}} (\PutLeft_1\Apply\String_1)\Concat(\PutLeft_2\Apply\String_2)
}

\inferrule[Swap Lens]
{
\FullContext \DerivesLens \Lens_1 \OfType \Regex_1 \Leftrightarrow \RegexAlt_1 \HasSemantics \PutRight_1,\PutLeft_2\\
\FullContext \DerivesLens \Lens_2 \OfType \Regex_2 \Leftrightarrow \RegexAlt_2 \HasSemantics \PutRight_1,\PutLeft_2\\\\
\UnambigConcat{\LanguageOf{\Delta}{\Regex_1}}{\LanguageOf{\Delta}{\Regex_2}}\\
\UnambigConcat{\LanguageOf{\Delta}{\RegexAlt_2}}{\LanguageOf{\Delta}{\RegexAlt_1}}
}
{
\FullContext \DerivesLens \SwapLens{\Lens_1}{\Lens_2} \OfType \Regex_1\Regex_2 \Leftrightarrow \RegexAlt_2\RegexAlt_1 \HasSemantics\\\\
\lambda \String.\LetWhereIn{\String_1\Concat\String_2}{\String}{\String_i\in\LanguageOf{\Delta}{\Regex_i}} (\PutRight_2\Apply\String_2)\Concat(\PutRight_1\Apply\String_1),\\\\
\lambda \String.\LetWhereIn{\String_1\Concat\String_2}{\String}{\String_i\in\LanguageOf{\Delta}{\RegexAlt_i}} (\PutLeft_2\Apply\String_2)\Concat(\PutLeft_1\Apply\String_1)
}

\inferrule[Or Lens]
{
\FullContext \DerivesLens \Lens_1 \OfType \Regex_1 \Leftrightarrow \RegexAlt_1 \HasSemantics \PutRight_1,\PutLeft_1\\
\FullContext \DerivesLens \Lens_2 \OfType \Regex_2 \Leftrightarrow \RegexAlt_2 \HasSemantics \PutRight_2,\PutLeft_2\\\\
\UnambigOr{\LanguageOf{\Delta}{\Regex_1}}{\LanguageOf{\Delta}{\Regex_2}}\\ \UnambigOr{\LanguageOf{\Delta}{\RegexAlt_1}}{\LanguageOf{\Delta}{\RegexAlt_2}}
}
{
\FullContext \DerivesLens \OrLens{\Lens_1}{\Lens_2} \OfType \Regex_1 | \RegexAlt_1 \Leftrightarrow \Regex_2 | \RegexAlt_2 \HasSemantics\\\\
\lambda \String.\{\PutRight_1(\String) \text{ if $\String\in\LanguageOf{\Delta}{\Regex_1}$ }, \PutRight_2(\String) \text{ if $\String\in\LanguageOf{\Delta}{\Regex_2}$ }\},\\\\
\lambda \String.\{\PutLeft_1(\String) \text{ if $\String\in\LanguageOf{\Delta}{\RegexAlt_1}$ }, \PutLeft_2(\String) \text{ if $\String\in\LanguageOf{\Delta}{\RegexAlt_2}$ }\}
}

\inferrule[Compose Lens]
{
\FullContext \DerivesLens \Lens_1 \OfType \Regex_1 \Leftrightarrow \Regex_2 \HasSemantics \PutRight_1,\PutLeft_1\\
\FullContext \DerivesLens \Lens_2 \OfType \Regex_2 \Leftrightarrow \Regex_3 \HasSemantics \PutRight_2,\PutLeft_2\\
}
{
\FullContext \DerivesLens \ComposeLens{\Lens_2}{\Lens_1} \OfType \Regex_1 \Leftrightarrow \Regex_3 \HasSemantics
\PutRight_2\Compose\PutRight_1,\PutLeft_2\Compose\PutLeft_1
}

\inferrule[Lens Variable]
{
\RegexContext', \LensContext' \DerivesLens \Lens \OfType \Regex \Leftrightarrow \RegexAlt \HasSemantics \PutRight, \PutLeft\\
\forall \Regex' (\RegexVariable,\Regex')\notin\RegexContext'\\
\forall \Regex' (\RegexVariableAlt,\Regex')\notin\RegexContext'\\
\forall \Lens' (\LensVariable,\Lens')\notin\LensContext'\\\\
\RegexContext', \LensContext' \DerivesLens \Lens \OfType \Regex \Leftrightarrow \RegexAlt \HasSemantics \PutRight, \PutLeft\\
}
{
\RegexContext' \cup \{(\RegexVariable,\Regex),(\RegexVariableAlt,\RegexAlt)\},
\LensContext' \cup \{(\LensVariable,\Lens)\}
\DerivesLens \LensVariable \OfType \RegexVariable \Leftrightarrow \RegexVariableAlt
\HasSemantics \PutRight,\PutLeft
}

\inferrule[Retype Lens]
{
\FullContext \DerivesLens \Lens \OfType \Regex_1 \Leftrightarrow \Regex_2 \HasSemantics \PutRight,\PutLeft\\
\LanguageOf{\Delta}{\Regex_1} = \LanguageOf{\Delta}{\Regex_1'}\\
\LanguageOf{\Delta}{\Regex_2} = \LanguageOf{\Delta}{\Regex_2'}
}
{
\FullContext \DerivesLens \Lens \OfType \Regex_1' \Leftrightarrow \Regex_2' \HasSemantics \PutRight,\PutLeft
}
\end{mathpar}

\caption{Lens Semantics and Typing}
\label{fig:lens-alternate-alternate-semantics}
\end{figure}

We focus on the language of bijective lenses, where the
functions between views are bijections, and inverses to each other.
We define the syntax for these bijective lenses in Figure~\ref{fig:lens-syntax}.
The semantics and typing for these lenses is defined in Figure~\ref{fig:lens-alternate-alternate-semantics}.
The semantics is only defined alongside the typing, as the functions are only
well defined when the term is well typed.

A close inspection of the laws behind the lenses shows that the typing of the
lenses corresponds tightly with the actual lenses.
For example, consider the concatenation rule:

\begin{mathpar}
\inferrule*
{
\Lens_1 : \Regex_1 \Leftrightarrow \RegexAlt_1\\
\Lens_2 : \Regex_2 \Leftrightarrow \RegexAlt_2\\\\
\UnambigConcatOf{\LanguageOf{\Regex_1}}{\LanguageOf{\Regex_2}}\\
\UnambigConcatOf{\LanguageOf{\Regex_2}}{\LanguageOf{\RegexAlt_2}}\\
}
{
\ConcatLens{\Lens_1}{\Lens_2} :
\RegexConcat{\Regex_1}{\Regex_2} \Leftrightarrow
\RegexConcat{\RegexAlt_1}{\RegexAlt_2}
}
\end{mathpar}

Concatenation lenses come alongside a regular expression concatenation.
Perhaps, instead of writing out lenses explicitly, we can merely
write its type, and the lenses can be inferred from the types.
Utilizing types for synthesis is not a new idea, and has been explored
in systems like Myth \cite{tds-pldi}.
However, we have found that type directed synthesis is well suited
for domain specific languages like the language of lenses.
The constraints of the type systems of these languages make the search space
tighter, and allow us to synthesize very complicated
programs very quickly.

Our approach to synthesis is one of enumeration.
We would like to enumerate the programs that satisfy the desired typing until
we find one that satisfies the examples.
We enumerate these programs by creating subproblems based on typing rules
could potentially create the desired type.

For example, given the type $\RegexConcat{\Regex_1}{\Regex_2}\Leftrightarrow\RegexConcat{\RegexAlt_1}{\RegexAlt_2}$.
We know that it can be created from a Concat Lens rule,
which would create the subproblems of finding a lens of type $\Regex_1\Leftrightarrow\RegexAlt_1$
and a lens of type $\Regex_2\Leftrightarrow\RegexAlt_2$.
However, this is made more difficult due to the Compose Lens rule,
and the Retype Lens rule.
Both of these rules can be applied to any type, and the subproblems are not immediately apparent.
For example, if the lens of type
$\RegexConcat{\Regex_1}{\Regex_2}\Leftrightarrow\RegexConcat{\RegexAlt_1}{\RegexAlt_2}$
came from a Compose Lens rule, then for some regular expression $\Regex$,
we would have the subproblems of finding a lens with type
$\RegexConcat{\Regex_1}{\Regex_2}\Leftrightarrow\Regex$,
and finding a lens with type
$\Regex\Leftrightarrow\RegexConcat{\RegexAlt_1}{\RegexAlt_2}$.
However, merely from the resulting type,
$\RegexConcat{\Regex_1}{\Regex_2} \Leftrightarrow \RegexConcat{\RegexAlt_1}{\RegexAlt_2}$,
we do not know which regular expression \Regex{} to use.

Similarly, if the lens of type
$\RegexConcat{\Regex_1}{\Regex_2}\Leftrightarrow\RegexConcat{\RegexAlt_1}{\RegexAlt_2}$
came from a Retype Lens rule, then for some regular expressions $\Regex$, and $\RegexAlt$,
where $\LanguageOf{\Regex}=\LanguageOf{\RegexConcat{\Regex_1}{\Regex_2}}$
and $\LanguageOf{\RegexAlt}=\LanguageOf{\RegexConcat{\RegexAlt_1}{\RegexAlt_2}}$,
we have the subproblem of finding a lens with type
$\Regex\Leftrightarrow\RegexAlt$.
However, merely from the resulting type,
$\RegexConcat{\Regex_1}{\Regex_2}\Leftrightarrow\RegexConcat{\RegexAlt_1}{\RegexAlt_2}$,
we do not know which regular expressions \Regex{} and \RegexAlt{} to
use, we only know that they have to meet some semantic restrictions.

However, neither of these rules are admissible and both are important.
Without the Retype Lens rule, it would not be possible to have a lens of the type
$a\Concat (b\Concat (c\Concat d))\Leftrightarrow(a\Concat b)\Concat(c\Concat d)$,
despite the fact that we expect this lens to be inhabited by the lens \IdentityLens{}.
Without the Compose Lens rule, it would not be possible to have a lens $\Lens$ of the type
$\RegexConcat{\Star{a}}{\RegexConcat{\Star{b}}{\RegexConcat{\Star{c}}{\Star{d}}}} \Leftrightarrow
\RegexConcat{\Star{c}}{\RegexConcat{\Star{a}}{\RegexConcat{\Star{d}}{\Star{b}}}}$,
which has $\PutRightOf{\Lens}(a^h\Concat b^i \Concat c^j \Concat d^k)=c^j\Concat a^h\Concat d^k\Concat b^i$, a simple permutation of the letters.

\begin{figure}
\newcounter{rowcount}
\setcounter{rowcount}{0}
\centering
\begin{tabular}{@{}r@{\hspace{1em}}c@{\hspace{1em}}l@{}r@{}}
\Regex{} & $\equiv$ & \Regex{} & \EqualityRule{}  \\
\RegexOr{\Regex}{\emptyset} & $\equiv$ & \Regex{} & \OrIdentityRule{} \\
$\RegexConcat{\Regex}{\emptyset}$ & $\equiv$ & $\emptyset$ & \EmptyProjectionRuleRightRule{} \\
$\RegexConcat{\emptyset}{\Regex}$ & $\equiv$ & $\emptyset$ & \EmptyProjectionRuleLeftRule{}\SubLeft{} \\
\RegexConcat{(\RegexConcat{\Regex{}}{\Regex'})}{\Regex''} & $\equiv$ & \RegexConcat{\Regex{}}{(\RegexConcat{\Regex'}{\Regex''})} & \ConcatAssocRule{}  \\
\RegexOr{(\RegexOr{\Regex}{\Regex'})}{\Regex''} & $\equiv$ & \RegexOr{\Regex}{(\RegexOr{\Regex'}{\Regex''})} & \OrAssociativityRule{}  \\
\RegexOr{\Regex{}}{\RegexAlt{}} & $\equiv$ & \RegexOr{\RegexAlt{}}{\Regex{}} & \OrCommutativityRule{}\\
\RegexConcat{\Regex{}}{(\RegexOr{\Regex{}'}{\Regex{}''})} & $\equiv$ & \RegexOr{(\RegexConcat{\Regex{}}{\Regex{}'})}{(\RegexConcat{\Regex{}}{\Regex{}''})} & \DistributivityLeftRule{} \\
\RegexConcat{(\RegexOr{\Regex{}'}{\Regex{}''})}{\Regex{}} & $\equiv$ & \RegexOr{(\RegexConcat{\Regex{}'}{\Regex{}})}{(\RegexConcat{\Regex{}''}{\Regex{}})} & \DistributivityRightRule{} \\
\RegexConcat{\Regex{}}{\EmptyString{}} & $\equiv$ & \Regex{} & \ConcatIdentityRule{} \\
\StarOf{(\RegexOr{\Regex{}}{\RegexAlt{}})} & $\equiv$ & \RegexConcat{\StarOf{(\RegexConcat{\StarOf{\Regex{}}}{\RegexAlt{}})}}{\StarOf{\Regex{}}} & \SumstarRule{}\\
\StarOf{(\RegexConcat{\Regex{}}{\RegexAlt{}})} & $\equiv$ & \RegexOr{\EmptyString{}}{(\RegexConcat{\RegexConcat{\Regex{}}{\StarOf{(\RegexConcat{\RegexAlt{}}{\Regex{}})}}}{\RegexAlt{}})} & \ProductstarRule{} \\
${(\Regex{}^*)}^*$ & $\equiv$ & \StarOf{\Regex{}} & \StarstarRule{} \\
\StarOf{(\RegexOr{\Regex}{\RegexAlt})} & $\equiv$ & $\StarOf{(\RegexConcat{(\RegexOr{\Regex}{\RegexAlt})}{\RegexOr{\RegexAlt}{\RegexConcat{{(\RegexConcat{\Regex}{\StarOf{\RegexAlt}})}^n}{\Regex}}})}\Concat$ & \DicyclicityRule{}\\
& & $(\EmptyString\Or(\RegexOr{\Regex}{\RegexAlt})\Concat$\\
& & $({(\RegexConcat{\Regex}{\StarOf{\RegexAlt}})}^0\Or\ldots\Or{(\RegexConcat{\Regex}{\StarOf{\RegexAlt}})}^n))$
\end{tabular}
\caption{Regular Expression Equivalences}
\label{fig:regex-equivalence-rules}
\end{figure}

These rules are important, but on each application of the rules, there are an
infinite number of potential subproblems.
Making the composition admissible is not the most complicated problem,
as it's lack of admissibility comes a bit from the rigid structure of the rules,
and only arises in an edge case.
However, retyping is not so easily made admissible.
For the handling of the Retype Lens
rule, we can create a search space for these potential subproblems by
searching through the lenses of equivalent regular expressions.
For this, we use the equational theory presented by Conway \cite{conway},
and proven complete by Krob \cite{Krob},
shown in Figure~\ref{fig:regex-equivalence-rules}.

However, enumerating through all possible regular expressions for lenses is inefficient.
Beyond that, we are not merely searching for the applications of these rules to
a single regular expression, but instead the applications of these to two regular expressions.
This makes this search space incredibly broad.
Instead of searching through this search space, we would instead like to find
a normal form for regular expressions, and find a way to create lenses on
regular expressions of that form.

\begin{figure}
\begin{tabular}{l@{\ }l@{\ }c@{\ }l@{\ }r}

% DNF_REGEX
(Atoms)& \Atom{},\AtomAlt{} & \GEq{} & \StarOf{\DNFRegex{}} & Iterate DNF\\
(Sequence)& \Sequence{},\SequenceAlt{} & \GEq{} &
$\SequenceOf{\String_0\SequenceSep\Atom_1\SequenceSep\ldots\SequenceSep\Atom_n\SequenceSep\String_n}$ & Conjoin\\
(DNF Regex)& \DNFRegex{},\DNFRegexAlt{} & \GEq{} & $\DNFOf{\Sequence_1\DNFSep\ldots\DNFSep\Sequence_n}$ & DNF Or\\
\end{tabular}
\caption{DNF Regex Syntax\bcp{Why the dot after conjuncts?  Ah, now I see;
    OK, but maybe it's better to choose two different kinds of brackets
    instead of distinguishing with . and + (I was also confused by the + --
    thought it must mean some kind of repetition).  How about
    $\langle...\rangle$ for disjuncts and $[...]$ for conjuncts?  (This is a
  bit suggestive of type-theoretic notation.)  Or how about ``hollow square
  brackets'' for conjuncts (using some kind of square brackets seems good,
  since they are ordered) and ``hollow set braces'' for disjuncts?  We
  should try to get this right, since these widgets are our core technical
  device.}} 
\label{fig:dnf-regex-syntax}
\end{figure}

\begin{figure}
%\begin{mathpar}
%\inferrule[Atom Userdef]
%{
%\Delta' \vdash \String : \Regex
%}
%{
%\Delta' \cup \{(\Regex,\RegexVariable)\} \vdash \String : \RegexVariable
%}
%
%\inferrule[Atom Empty Star]
%{
%}
%{
%\Delta \vdash \epsilon : \Star{\DNFRegex}
%}
%
%\inferrule[Atom Nonempty Star]
%{
%\Delta \vdash \String_1 : \DNFRegex\\
%\Delta \vdash \String_2 : \Star{\DNFRegex}
%}
%{
%\Delta \vdash \String_1\String_2 : \Star{\DNFRegex}
%}
%
%\inferrule[Clause]
%{
%\lambda i:\RangeIncInc{1}{n}.\Delta \vdash \String_i : \Atom_i
%}
%{
%\Delta \vdash \StringAlt_0\String_1\ldots\String_n\StringAlt_n : (\lambda i:\RangeIncInc{1}{n}.\Atom_i,\lambda i:\RangeIncInc{0}{n}.\StringAlt_i)
%}
%
%\inferrule[DNF Regex]
%{
%\Delta \vdash \String : \Clause_i\\
%}
%{
%\Delta \vdash \String : (\lambda i:\RangeIncInc{1}{n}.\Clause_i)
%}
%\end{mathpar}
\begin{itemize}
\item $\LanguageOf{\RegexContext}{\RegexVariable}=\LanguageOf{\RegexContext\setminus(\RegexVariable,\Regex)}{\Regex}$ if, for some unique \Regex{}, $(\RegexVariable,\Regex)\in\RegexContext$.
\item $\LanguageOf{\RegexContext}{\RegexVariable}=\emptyset$
if there does not exist a unique \Regex{} such that $(\RegexVariable,\Regex)\in\RegexContext$
\item $\LanguageOf{\RegexContext}{\Star{\DNFRegex}} =
\{\String_1\Concat\ldots\Concat\String_n | n\in\Nats \wedge \String_i\in\LanguageOf{\RegexContext}{\DNFRegex}\}$
\item $\LanguageOf{\RegexContext}{[\String_0;\Atom_1;\ldots;\Atom_n;\String_n]}=
\{\String_0\Concat\String_1'\Concat\ldots\Concat\String_n'\Concat\String_n | \String_i'\in\LanguageOf{\RegexContext}{\Atom_i}\}$
\item $\LanguageOf{\RegexContext}{[\Conjunct_1;\ldots;\Conjunct_n]}=
\{\String | \String \in \Conjunct_i \text{ for some $i\in\RangeIncInc{1}{n}$}\}$
\end{itemize}
\caption{DNF Regex Semantics}
\label{fig:dnf-regex-semantics}
\end{figure}

To this end, we have created DNF regular expressions.  Intuitively, a DNF regular
expression is a regular expression with the distributivity laws (6 and 7) always
applied, with no associativity between operations of the same type,
and with base regular expressions at fixed locations.
We formalize the syntax of this language in Figure~\ref{fig:dnf-regex-syntax}.

The outermost layer is a list of conjuncts.
This layer intuitively corresponds to the choices involved in regular expression matching, and so represents where the Ors of normal regular expressions exist.
The second layer is a list of alternating strings and stars.
After the choice has been made about what will be expressed,
the base strings and iterated portions remains to be expressed.
We keep it in a normal form, by requiring a (possibly empty) string between
each clause.
The clause corresponds to the concatenated data, and is a concatenation of the
fixed data of base strings, and the iterated data of stars.
Finally is the atom, which is a star.
This corresponds to the iteration that takes place in normal stars.
There is no fixed way to break the choices of this into the clause,
as there is an arbitrarily large number of choices made in the stars.
This intuition is formalized by the semantics, given in Figure~\ref{fig:dnf-regex-semantics}.

\begin{figure}
\bf{ConcatConjunct}:\\
\ConcatConjunct \OfType \ArrowTypeOf{\mathit{Conjunct}}{\mathit{Conjunct}}\\
$\ConcatConjunct([\String_0;\Atom_1;\ldots;\Atom_n;\String_n],[\StringAlt_0;\AtomAlt_1;\ldots;\AtomAlt_m;\StringAlt_m])=$\\
\hspace*{2ex}$\ConcatConjunct([\String_0;\Atom_1;\ldots;\Atom_n;\String_n\Concat\StringAlt_0;\AtomAlt_1;\ldots;\AtomAlt_m;\StringAlt_m])$\\
\\
\bf{ConcatDNF}:\\
\ConcatDNF \OfType \ArrowTypeOf{\mathit{DNF}}{\mathit{DNF}}\\
$\ConcatDNF([\Conjunct_0;\ldots;\Conjunct_n],[\ConjunctAlt_0;\ldots;\ConjunctAlt_m])=$
\[
\begin{array}{rcccl}
[ & \ConcatConjunct(\Conjunct_0,\ConjunctAlt_0); & \ldots & \ConcatConjunct(\Conjunct_0,\ConjunctAlt_m); \\
& & \ldots & \\
& \ConcatConjunct(\Conjunct_n,\ConjunctAlt_0); & \ldots & \ConcatConjunct(\Conjunct_n,\ConjunctAlt_m) & ]
\end{array}
\]
\\
\bf{OrDNF}:\\
\OrDNF \OfType \ArrowTypeOf{\mathit{DNF}}{\mathit{DNF}}\\
$\OrDNF([\Conjunct_0;\ldots;\Conjunct_n],[\ConjunctAlt_0;\ldots;\ConjunctAlt_m])$=
\hspace*{2ex}$[\Conjunct_0;\ldots;\Conjunct_n;\ConjunctAlt_0;\ldots;\ConjunctAlt_m]$\\
\\
\bf{RepeatDNF}:\\
\RepeatDNF \OfType \ArrowTypeOf{\Nats}{\ArrowTypeOf{\mathit{DNF}}{\mathit{DNF}}}\\
$\RepeatDNF(0,\DNFRegex)=[[\EmptyString]]$\\
$\RepeatDNF(n,\DNFRegex)=$\\
\hspace*{2ex}$\ConcatDNF(\DNFRegex,\RepeatDNF(n-1,\DNFRegex))$\\
\\
\bf{StarModNDNF}:\\
\StarModNDNF \OfType \ArrowTypeOf{\Nats_{\geq1}}{\ArrowTypeOf{\mathit{DNF}}{\mathit{DNF}}}\\
$\StarModNDNF(1,\DNFRegex)=[[\EmptyString]]$\\
$\StarModNDNF(n,\DNFRegex)=$\\
\hspace*{2ex}$\OrDNF(\StarModNDNF(n-1,\DNFRegex),\RepeatDNF(n-1,\DNFRegex))$

\caption{DNF Regex Functions}
\label{fig:dnf-regex-functions}
\end{figure}

We can create a dnf regular expression from a regular expression through repeated
application of the distributivity rule, and through removal of association information.  To do this, we require some functions defined on DNF regular expressions,
which we define in Figure~\ref{fig:dnf-regex-functions}.
From this we can define a function which converts a regular expression into
an equivalent DNF regular expression, \ToDNFRegex{}.
\begin{definition}
\leavevmode
\begin{itemize}
\item $\ToDNFRegex(\String)=[[\String]]$
\item $\ToDNFRegex(\Star{\Regex}) = [[\Star{\ToDNFRegex(\Regex)}]]$
\item $\ToDNFRegex(\RegexConcat{\Regex_1}{\Regex_2}) =$\\
\hspace*{1em}$\ConcatDNF(\ToDNFRegex(\Regex_1),\ToDNFRegex(\Regex_2))$
\item $\ToDNFRegex(\RegexOr{\Regex_1}{\Regex_2}) =$\\
\hspace*{1em}$\OrDNF(\ToDNFRegex(\Regex_1),\ToDNFRegex(\Regex_2))$
\end{itemize}
\end{definition}
We show that this transformation is a valid transformation.
\begin{restatable}[Completeness of DNF Regexs]{theorem}{dnfrc}
\label{thm:completeness-dnf-lenses}
For all regular expressions \Regex{},
$\LanguageOf{\Regex}=\LanguageOf{\ToDNFRegex(\Regex)}$.
\end{restatable}

Furthermore, we can use this to prove that the language of DNF regular expressions is sound.

\begin{restatable}[Soundness of DNF Regexs]{theorem}{dnfrs}
\label{thm:soundness-dnf-lenses}
The function \ToDNFRegex{} has a right inverse \FromDNFRegex{}.
\end{restatable}

We can further normalize the DNF regular expressions by providing an order on conjuncts,
and ordering the conjuncts in the DNF regular expression according to that order.
With this, we have normalized the regular expression equivalences 1-7.
In other words, if two regular expressions, $\Regex_1$ and $\Regex_2$,
are equivalent up to the transformations of regular expressions in equivalences 1-7,
then $\ToDNFRegex(\Regex_1)=\ToDNFRegex(\Regex_2)$.
A reasonable approach to further normalizing this is to attempt to minimize the
regular expression as much as possible, using the star laws.
Unfortunately, while this approach works well for normalizing, it doesn't work
well for finding lenses.  For example, there is clearly a lens of type
$\MapsBetweenTypeOf{\RegexOr{\EmptyString}{(\RegexConcat{a}{\Star{a}})}}{\RegexOr{b}{(\RegexConcat{c}{\Star{d}})}}$,
namely $\OrLens{\ConstLens{\EmptyString}{a}}{\ConcatLens{\ConstLens{a}{b}}{\IterateLens{\ConstLens{a}{d}}}}$.
However, if one simplifies as much as possible, then they will have to find a lens
of type $\MapsBetweenTypeOf{\Star{a}}{\RegexOr{b}{(\RegexConcat{c}{\Star{d}})}}$.
Unlike the type before being minified, where there was a type directed way to find
this lens, there is no longer a type directed way: should the lens be an iterate lens
or an or lens.
We interpret this as the need to gain semantic information through the expansion of
the regular expressions.
A regular expression merely of the form $\Star{a}$ only cares about how the iterated case is handled, where a regular expression of the form
$\RegexOr{\EmptyString}{(\RegexConcat{a}{\Star{a}})}$ potentially acts differently on the empty
string case than on the nonempty case.

\begin{figure}
\begin{mathpar}
\inferrule[Atom Sumstar]
{
A\subsetneq \RangeIncInc{1}{n}
}
{
\Star{([\Conjunct_1;\ldots;\Conjunct_n])}\RewriteAtom\\
\ConcatDNF(\Star{(\ConcatDNF([[\EmptyString;\Star{\DNFRegex_A};\EmptyString]],\DNFRegex_{\ComplementOf{A}}))},
\Star{\DNFRegex_A})\\
\text{where }\DNFRegex_S=[\Conjunct_{S_1};\ldots;\Conjunct_{S_{\SizeOf{S}}}]\\
}

\inferrule[Atom Unrollstar]
{
}
{
\Star{\DNFRegex}\RewriteAtom
\OrDNF([[\EmptyString]],\ConcatDNF(\DNFRegex,[[\EmptyString;\Star{\DNFRegex};\EmptyString]]))
}

\inferrule[Atom Powerstar]
{
n\in\Nats_{\geq1}
}
{
\Star{\DNFRegex}\RewriteAtom
\ConcatDNF([[\EmptyString;\Star{[[\RepeatDNF(n,\DNFRegex)]]};\EmptyString]],\\\\
\StarModNDNF(n,\DNFRegex))
}

\inferrule[DNF Rewrite Star]
{
\DNFRegex \RewriteDNF \DNFRegex'
}
{
\Star{\DNFRegex} \RewriteAtom [[\EmptyString;\Star{\DNFRegex'};\EmptyString]]
}

\inferrule[Atom DNF Rewrite]
{
\Atom_j \RewriteAtom \DNFRegex
}
{
[\Clause_1;\ldots;\Clause_{i-1};\\\\
[\String_0;\Atom_1;\ldots;\String_{j-1};\Atom_j;\String_j;\ldots;\Atom_m;\String_m]\\\\
;\Clause_{i+1};\ldots;\Clause_n]\RewriteDNF\\\\
\OrDNF(\OrDNF([\Clause_1;\ldots;\Clause_{i-1}],
\ConcatDNF(\ConcatDNF(\\\\
[[\String_0;\Atom_1;\ldots;\String_{j-1}]],\DNFRegex),[\String_j;\ldots;\Atom_m;\String_m])),
[\Clause_{i+1};\ldots;\Clause_n])
}

\inferrule[DNF Rewrite Composition]
{
\DNFRegex \RewriteDNF \DNFRegex'\\
\DNFRegex' \RewriteDNF \DNFRegex''
}
{
\DNFRegex \RewriteDNF \DNFRegex''
}

\inferrule[Identity DNF Rewrite]
{
}
{
\DNFRegex \RewriteDNF \DNFRegex
}

\end{mathpar}
\caption{DNF Regex Rewrite Rules}
\label{fig:dnf-regex-rewrites}
\end{figure}

Because of this, instead of writing equivalences for DNF regular expressions like
exist for normal regular expressions,
instead we write rewrite rules as shown in
Figure~\ref{fig:dnf-regex-rewrites}.\bcp{Don't understand what's going on here.}
Intuitively, these rewrite rules correspond to breaking a regular expression
into a regular expression that encodes additional semantics.
We define these rules as rewrites that turn atoms into DNF Regular expressions,
and a rule that expresses how these rewrites can become rewrites on
the DNF regular expressions themselves.

Roughly, the Atom Sumstar rewrite corresponds to Regular Expression equivalence 8,
the Atom Unrollstar rewrite corresponds to Regular Expression equivalence 9,
and the Atom Powerstar rewrite corresponds to Regular Expression equivalence 11.
There is no rewrite corresponding to Regular Expression equivalence 10, as that
equivalence is an ambiguity introducing equivalence.
An application of Atom Sumstar\bcp{Wrong font} corresponds to doing different things after the last time a certain event has occured, and before the last time that event occurs.
An application of Atom Unrollstar corresponds to making a different action for
the empty case of an iteration, the first case of an iteration, and all further cases of the iteration.
An application of Atom Powerstar corresponds to doing different things if the iteration
is occuring a different number of times, modulo n, for some fixed n.

\begin{figure}
\centering
\begin{tabular}{l@{\ }l@{\ }c@{\ }l@{\ }r}
% REGEX
(Atom Lenses) &\AtomLens{} & \GEq{} & $Iterate(\DNFLens)$ & Iterate\\
& & & \GBar{} \IdentityLens{} & Identity\\
(Sequence Lenses) &\SequenceLens{} & \GEq{} &
$(\SequenceLensOf{(\String_0,\StringAlt_0)\SequenceLensSep\AtomLens_1\SequenceLensSep\ldots\SequenceLensSep\AtomLens_n\SequenceLensSep(\String_n,\StringAlt_n)}$, &\\
& & & $\sigma \in S_n)$ & Clause\SubN{}\\
(DNF Lenses)& \DNFLens{} & \GEq{} & $(\DNFLensOf{\SequenceLens_1\DNFLensSep\ldots\DNFLensSep\SequenceLens_n}, \sigma \in S_n)$ & DNF\SubN{}\\
\end{tabular}
\caption{DNF Lens Syntax \bcp{italics right column.  And remove the space to
  the left of the LH column.  (Use @\{\} in the array header.)}}
\label{fig:dnf-lens-syntax}
\end{figure}

\begin{figure}
\centering
\begin{mathpar}
\inferrule[Iterate Atom Lens]
{
\DNFLens : \DNFRegex \Leftrightarrow \DNFRegexAlt
\HasSemantics \PutRight,\PutLeft\\
\UnambigIt{\LanguageOf{\DNFRegex}}\\
\UnambigIt{\LanguageOf{\DNFRegexAlt}}\\
}
{
\IterateLens{\DNFLens} : \Star{\DNFRegex} \Leftrightarrow \Star{\DNFRegexAlt} \HasSemantics\\
\lambda \String.\LetWhereIn{\String_1\Concat\ldots\Concat\String_n}{\String}{\String_i\in\LanguageOf{\DNFRegex}}\\
(\PutRight\Apply\String_1)\Concat\ldots\Concat(\PutRight\Apply\String_n),\\\\
\lambda \String.\LetWhereIn{\String_1\Concat\ldots\Concat\String_n}{\String}{\String_i\in\LanguageOf{\DNFRegexAlt}}\\
(\PutLeft\Apply\String_1)\Concat\ldots\Concat(\PutLeft\Apply\String_n)
}

\inferrule[Conjunct Lens]
{
\AtomLens_1 : \Atom_1 \Leftrightarrow \AtomAlt_1 \HasSemantics \PutRight_1,\PutLeft_1\\\\
\ldots\\\\
\AtomLens_n : \Atom_n \Leftrightarrow \AtomAlt_n \HasSemantics \PutRight_n,\PutLeft_n\\\\
\sigma \in \PermutationSetOf{n}\\
\forall i \in \RangeIncInc{1}{n-1} \UnambigConcat{\LanguageOf{\String_{i-1}\Atom_i\epsilon}}{\LanguageOf{\String_i\Atom_{i+1}\epsilon}}\\
\forall i \in \RangeIncInc{1}{n-1} \UnambigConcat{\LanguageOf{\StringAlt_{i-1}\AtomAlt_{\sigma(i)}\epsilon}}{\LanguageOf{\StringAlt_i\AtomAlt_{\sigma(i+1)}\epsilon}}\\
}
{
([(\String_0,\StringAlt_0);\Atom_1;\ldots;\Atom_n;(\String_n,\StringAlt_n)],\lambda i:\RangeIncInc{1}{n}.\Atom_i,\sigma) :\\
[\String_0;\Atom_1;\ldots;\Atom_n;\String_n]\Leftrightarrow
[\StringAlt_0;\AtomAlt_1;\ldots;\AtomAlt_n;\StringAlt_n]\HasSemantics\\
\lambda \String.\LetWhereIn{\String_0\Concat\String_1'\Concat\ldots\Concat\String_n'\Concat\String_n}{\String}{\String_i'\in\LanguageOf{\Atom_i}}\\
\String_0\Concat(\PutRight_{\sigma(1)}\Apply\String_{\sigma(1)}')\Concat\ldots\Concat(\PutRight_n\Apply\String_n')\Concat\String_n,\\\\
\lambda \String.\LetWhereIn{\String_0\Concat\String_1'\Concat\ldots\Concat\String_n'\Concat\String_n}{\String}{\String_i'\in\LanguageOf{\AtomAlt_i}}\\
\String_0\Concat(\PutLeft_{\InverseOf{\sigma}(1)}\Apply\String_{\InverseOf{\sigma}(1)}')\Concat\ldots\Concat(\PutLeft_{\InverseOf{\sigma}(n)}\Apply\String_{\InverseOf{\sigma}(n)}')\Concat\String_n
}

\inferrule[DNF Regex Lens]
{
\ConjunctLens_1 : \Conjunct_1 \Leftrightarrow \ConjunctAlt_1 \HasSemantics \PutRight_1, \PutLeft_1\\\\
\ldots\\\\
\ConjunctLens_n : \Conjunct_n \Leftrightarrow \ConjunctAlt_n \HasSemantics \PutRight_n, \PutLeft_n\\\\
\sigma \in \PermutationSetOf{n}\\
i \neq j \Rightarrow \Conjunct_{i} \cap \Conjunct_{j}=\emptyset\\
i \neq j \Rightarrow \ConjunctAlt_{i} \cap \ConjunctAlt_{j}=\emptyset\\
}
{
([\ConjunctLens_1;\ldots;\ConjunctLens_n],\sigma) : [\Conjunct_1;\ldots;\Conjunct_n]
\Leftrightarrow [\ConjunctAlt_{\sigma(1)};\ldots;\ConjunctAlt_{\sigma(n)}] \HasSemantics\\
\lambda \String.\{\PutRight_1(\String) \text{ if $\String\in\LanguageOf{\Conjunct_1}$ }, \ldots, \PutRight_n(\String) \text{ if $\String\in\LanguageOf{\Conjunct_n}$ }\},\\
\lambda \String.\{\PutLeft_1(\String) \text{ if $\String\in\LanguageOf{\ConjunctAlt_1}$ }, \ldots, \PutLeft_n(\String) \text{ if $\String\in\LanguageOf{\ConjunctAlt_n}$ }\}\\
}

\inferrule[DNF Rewrite Lens]
{
\DNFRegex \RewriteDNF \DNFRegex'\\
\DNFRegexAlt \RewriteDNF \DNFRegexAlt'\\
\DNFLens \OfType \MapsBetweenTypeOf{\DNFRegex}{\DNFRegexAlt} \HasSemantics \PutRight,\PutLeft
}
{
\DNFLens \OfType \MapsBetweenTypeOf{\DNFRegex'}{\DNFRegexAlt'}
}

\end{mathpar}
\caption{DNF Lens Typing and Semantics}
\label{fig:dnf-lens-alternate-alternate-semantics}
\end{figure}

Armed with these rewrites, we have the capabilities to define a sufficiently
strong language for lenses on these regular expressions.
The syntax is defined in Figure~\ref{fig:dnf-lens-syntax}.
Similarly to the language of lenses, we aim to have the typing of the lenses
correspond closely to the syntax for the lenses themselves.
The typing and semantics of these dnf lenses are defined in Figure~\ref{fig:dnf-lens-alternate-alternate-semantics}.
Intuitively, a DNF Regex Lens corresponds roughly to an n-ary version of an or lens,
a Conjunct Lens corresponds to const lenses for the strings, and a combination of
Concat and Swap lenses for the Atoms.
Furthermore, these lenses are sufficiently strong that Compose Lenses are not
necessary.
Indeed, these lenses are strong enough to express everything expressible in the language of lenses.
\begin{restatable}[Completeness of DNF Lenses]{theorem}{dnflc}
\label{thm:completeness-dnf-lenses}
If there exists a derivation of $\Lens \OfType \MapsBetweenTypeOf{\Regex}{\RegexAlt} \HasSemantics \PutRight,\PutLeft$,
then there exists a derivation of $\DNFLens \OfType \MapsBetweenTypeOf{\DNFRegex}{\DNFRegexAlt} \HasSemantics \PutRight,\PutLeft$ such that
$\LanguageOf{\DNFRegex}=\LanguageOf{\Regex}$, and
$\LanguageOf{\DNFRegexAlt}=\LanguageOf{\RegexAlt}$.
\end{restatable}
Furthermore, they only express things expressible in the language of lenses.
\begin{restatable}[Soundness of DNF Lenses]{theorem}{dnfls}
\label{thm:soundness-dnf-lenses}
If there exists a derivation of $\DNFLens \OfType \MapsBetweenTypeOf{\DNFRegex}{\DNFRegexAlt} \HasSemantics \PutRight,\PutLeft$,
then there exists a derivation of $\Lens \OfType \MapsBetweenTypeOf{\Regex}{\RegexAlt} \HasSemantics \PutRight,\PutLeft$ such that
$\LanguageOf{\Regex}=\LanguageOf{\DNFRegex}$, and
$\LanguageOf{\RegexAlt}=\LanguageOf{\DNFRegexAlt}$.
\end{restatable}

However, these lenses are much more suited to synthesis.
The vast majority of the rules have a syntax directed synthesis algorithm.
Furthermore, even the rewrite rules that don't have an immediate syntax directed
synthesis algorithm have an underlying semantic meaning which can be used
to direct the solution.
