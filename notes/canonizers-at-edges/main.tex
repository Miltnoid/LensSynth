\documentclass[a4paper,11pt] {article}
\usepackage{amsmath, amssymb, amsthm}
\usepackage[pdftex] {graphicx}
\usepackage{verbatim}
\usepackage{fullpage, setspace}
\usepackage{stmaryrd}
\usepackage{mathtools}
\usepackage{extarrows}
\usepackage{supertabular}
\usepackage{units}
\usepackage{diagrams}
\usepackage{amsmath}
\usepackage[table]{xcolor}
\usepackage{mathtools}
\usepackage{bussproofs}
\usepackage{amsthm}
\usepackage{csvsimple}
\usepackage{thmtools,thm-restate}
\usepackage{changepage}
\usepackage{booktabs}
\usepackage{amssymb}
\usepackage[inline]{enumitem}
\usepackage{multirow,bigdelim}
\usepackage{siunitx}
\usepackage{listings}
\usepackage{sansmath}
\usepackage{url}
\usepackage{flushend}
\usepackage{microtype}
\usepackage[utf8]{inputenc}
\usepackage{mathpartir}
\usepackage{empheq}
\usepackage{array}
\usepackage{pgfplots}
\usepackage{stmaryrd}
\usepackage{courier}
\usepackage{qtree}
\usepackage[normalem]{ulem}
\usepackage{relsize}
\usepackage{tikz}
\usepackage{algorithm}
\usepackage{algorithmicx}
\usepackage[noend]{algpseudocode}
\usepackage{graphicx}
\usepackage{textcomp}

% Colors
\definecolor{dkblue}{rgb}{0,0.1,0.5}
\definecolor{dkgreen}{rgb}{0,0.6,0}
\definecolor{dkred}{rgb}{0.6,0,0}
\definecolor{dkpurple}{rgb}{0.4,0,0.6}
\definecolor{olive}{rgb}{0.4, 0.4, 0.0}
\definecolor{teal}{rgb}{0.0,0.5,0.5}
\definecolor{orange}{rgb}{0.9,0.6,0.2}
\definecolor{lightyellow}{RGB}{255, 255, 179}
\definecolor{lightgreen}{RGB}{170, 255, 220}
\definecolor{teal}{RGB}{141,211,199}
\definecolor{darkbrown}{RGB}{121,37,0}

% remove whitespace before and after multicols
\setlength{\multicolsep}{0pt}

% renewtheorem https://tex.stackexchange.com/questions/103013/is-there-a-renewtheorem-equivalent-of-renewcommand-using-amsthm-and-not-ntheo
\makeatletter
\def\renewtheorem#1{%
  \expandafter\let\csname#1\endcsname\relax
  \expandafter\let\csname c@#1\endcsname\relax
  \gdef\renewtheorem@envname{#1}
  \renewtheorem@secpar
}
\def\renewtheorem@secpar{\@ifnextchar[{\renewtheorem@numberedlike}{\renewtheorem@nonumberedlike}}
\def\renewtheorem@numberedlike[#1]#2{\newtheorem{\renewtheorem@envname}[#1]{#2}}
\def\renewtheorem@nonumberedlike#1{  
\def\renewtheorem@caption{#1}
\edef\renewtheorem@nowithin{\noexpand\newtheorem{\renewtheorem@envname}{\renewtheorem@caption}}
\renewtheorem@thirdpar
}
\def\renewtheorem@thirdpar{\@ifnextchar[{\renewtheorem@within}{\renewtheorem@nowithin}}
\def\renewtheorem@within[#1]{\renewtheorem@nowithin[#1]}
\makeatother

\pgfplotsset{
% override style for non-boxed plots
    % which is the case for both sub-plots
    every non boxed x axis/.style={} 
}

\newenvironment{mathprooftree}
  {\varwidth{.9\textwidth}\centering\leavevmode}
  {\DisplayProof\endvarwidth}

\newcommand{\FINISH}[3]{\ifdraft\textcolor{#1}{[#2: #3]}\fi}
\newcommand{\bcp}[1]{\FINISH{dkred}{B}{#1}}
\newcommand{\BCP}[1]{\FINISH{dkred}{B}{\bf #1}}
\newcommand{\afm}[1]{\FINISH{dkgreen}{A}{#1}}
\newcommand{\dpw}[1]{\FINISH{dkblue}{D}{#1}}
\newcommand{\saz}[1]{\FINISH{orange}{S}{#1}}
\newcommand{\ksf}[1]{\FINISH{teal}{K}{#1}}
\newcommand{\revised}[1]{\FINISH{dkred}{#1}}

\newcommand{\IE}{\emph{i.e.}}
\newcommand{\EG}{\emph{e.g.}}
\newcommand{\ETC}{\emph{etc.}}

\theoremstyle{definition}
\renewtheorem{theorem}{Theorem}
\newtheorem{mylemma}{Lemma}
\renewtheorem{corollary}{Corollary}
\renewtheorem{conjecture}{Conjecture}
\renewtheorem{definition}{Definition}
\newtheorem{property}{Property}
\theoremstyle{plain}
\theoremstyle{remark}
\newtheorem{subcase}{Subcase}
\theoremstyle{remark}
\newtheorem{case}{Case}
\makeatletter
\@addtoreset{subcase}{case}
\@addtoreset{case}{mylemma}
\@addtoreset{case}{theorem}
\@addtoreset{case}{corollary}
\@addtoreset{case}{definition}
\@addtoreset{case}{definition}
\@removefromreset{theorem}{section}
\makeatother

\algnewcommand\algorithmicswitch{\textbf{switch}}
\algnewcommand\algorithmicmatch{\textbf{match}}
\algnewcommand\algorithmiccase{\textbf{case}}
\algnewcommand\algorithmicwith{\textbf{with}}
\algnewcommand\algorithmicforeach{\textbf{foreach}}
\algnewcommand\Assert[1]{\State \algorithmicassert(#1)}%
% New "environments"
\algdef{SE}[SWITCH]{Switch}{EndSwitch}[1]{\algorithmicmatch\ #1\ \algorithmicwith}{\algorithmicend\ \algorithmicswitch}%
\algdef{SE}[CASE]{Case}{EndCase}[1]{$|~$ #1 $\rightarrow$}{\algorithmicend\ \algorithmiccase}%
\algdef{SE}[FOREACH]{ForEach}{EndForEach}[2]{\algorithmicforeach\ #1 $\in$ #2}{\algorithmicend\ \algorithmicforeach}%
\algdef{SE}[CaseTwo]{CaseTwo}{EndCaseTwo}[2]{$|~$ #1 $\rightarrow$ #2}{\algorithmicend\ \algorithmiccase}%
\algtext*{EndSwitch}%
\algtext*{EndCase}%
\algtext*{EndCaseTwo}%
\algtext*{EndSecondCase}%
\algtext*{EndForEach}%



\newcommand{\CF}[1]{\ensuremath{\mathsf{#1}}}         % Code Font
\newcommand{\SmallCF}[1]{{\small \mathsf{#1}}}
\newcommand{\PCF}[1]{\textproc{#1}}
\newcommand{\VarCF}[1]{{\color{darkbrown} \CF{#1}}}
\newcommand{\StringCF}[1]{\CF{\textcolor{blue}{#1}}}
\newcommand{\KW}[1]{\CF{\textcolor{dkpurple}{#1}}}
\newcommand{\Regex}{\ensuremath{\mathit{S}}\xspace}         % Regular Expression
\newcommand{\RegexType}{\ensuremath{\textit{Regex}}}
\newcommand{\EquivRegexType}{\ensuremath{\textit{Regex}/\sim}}
\newcommand{\BooleanAnd}{\ensuremath{~\wedge~}}
\newcommand{\BooleanOr}{\ensuremath{\vee}}
\newcommand{\BooleanImplies}{\ensuremath{\Rightarrow}}
\newcommand{\Rewrite}{\ensuremath{\rightarrow}}
\newcommand{\RewriteAtom}{\ensuremath{\Rewrite_\Atom}}
\newcommand{\RewriteDNF}{\ensuremath{\Rewrite_\DNFRegex}}
\newcommand{\ConcatDNF}{\ensuremath{\odot}}
\newcommand{\ConcatDNFOf}[2]{\ensuremath{#1\ConcatDNF#2}}
\newcommand{\BigConcatDNF}{\ensuremath{\bigodot}}
\newcommand{\ConcatSequence}{\ensuremath{\odot_{\Sequence}}}
\newcommand{\ConcatSequenceOf}[2]{\ensuremath{#1\ConcatSequence#2}}
\newcommand{\ConcatPermutation}{\ensuremath{\odot}}
\newcommand{\ConcatPermutationOf}[2]{\ensuremath{#1\ConcatPermutation#2}}
\newcommand{\SwapPermutation}{\ensuremath{\circledS}}
\newcommand{\SwapPermutationOf}[2]{\ensuremath{#1\SwapPermutation#2}}
\newcommand{\DistributePermutation}{\ensuremath{\otimes}}
\newcommand{\DistributePermutationOf}[2]{\ensuremath{#1\DistributePermutation#2}}
\newcommand{\DistributeSwapPermutation}{\ensuremath{\otimes^{\mathit{s}}}}
\newcommand{\DistributeSwapPermutationOf}[2]{\ensuremath{#1\DistributeSwapPermutation#2}}
\newcommand{\ConcatSequenceLens}{\ensuremath{\odot_{\SequenceLens}}}
\newcommand{\ConcatSequenceLensOf}[2]{\ensuremath{#1\ConcatSequenceLens#2}}
\newcommand{\ConcatDNFLens}{\ensuremath{\odot}}
\newcommand{\ConcatDNFLensOf}[2]{\ensuremath{#1\ConcatDNFLens#2}}
\newcommand{\SwapSequenceLens}{\ensuremath{\circledS_{\SequenceLens}}}
\newcommand{\SwapSequenceLensOf}[2]{\ensuremath{#1\SwapSequenceLens#2}}
\newcommand{\SwapDNFLens}{\ensuremath{\circledS}}
\newcommand{\SwapDNFLensOf}[2]{\ensuremath{#1\SwapDNFLens#2}}
\newcommand{\RepeatDNFOfTimes}[1]{\ensuremath{^{#1}}}
\newcommand{\RepeatDNFOf}[2]{\ensuremath{{#2}\RepeatDNFOfTimes{#1}}}
\newcommand{\RepeatDNFLensOfTimes}[1]{\ensuremath{^{#1}}}
\newcommand{\RepeatDNFLensOf}[2]{\ensuremath{{#2}\RepeatDNFLensOfTimes{#1}}}
\newcommand{\OrDNF}{\ensuremath{\oplus}}
\newcommand{\OrDNFOf}[3]{\ensuremath{#1\OrDNF_{#3}#2}}
\newcommand{\OrDNFLens}{\ensuremath{\oplus}}
\newcommand{\OrDNFLensOf}[2]{\ensuremath{#1\OrDNFLens#2}}
\newcommand{\PutRSym}{\ensuremath{\mathit{putr}}}
\newcommand{\PutLSym}{\ensuremath{\mathit{putl}}}
\newcommand{\PutRSymOf}[1]{\ensuremath{\PutRSym \App #1}}
\newcommand{\PutLSymOf}[1]{\ensuremath{\PutLSym \App #1}}
\newcommand{\RegexAlt}{\ensuremath{\mathit{T}}\xspace}         % Regular Expression
\newcommand{\RegexAltAlt}{\ensuremath{\mathit{U}}\xspace}         % Regular Expression
\newcommand{\Or}{\ensuremath{~|~}}
\newcommand{\RegexOr}[2]{\ensuremath{#1\Or#2}}
\newcommand{\SubN}{\textsubscript{n}}
\newcommand{\RegexConcat}[2]{\ensuremath{#1\cdot#2}}
\newcommand{\EmptyString}{\ensuremath{\epsilon}}
\newcommand{\StringConcat}[2]{\ensuremath{#1\cdot#2}}
\newcommand{\HasSemantics}{\ensuremath{\triangleright}}
\newcommand{\DerivesLens}{\ensuremath{\vdash}}
\newcommand{\DerivesDNFLens}{\ensuremath{\vdash_{\DNFLens}}}
\newcommand{\DerivesSequenceLens}{\ensuremath{\vdash_{\SequenceLens}}}
\newcommand{\DerivesAtomLens}{\ensuremath{\vdash_{\AtomLens}}}
\newcommand{\DerivesStringRegex}{\ensuremath{\vdash}}
\newcommand{\DerivesAtomRewrite}{\ensuremath{\vdash}}
\newcommand{\DerivesDNFRewrite}{\ensuremath{\vdash}}
\newcommand{\Concat}{\ensuremath{\cdot}}
\newcommand{\Union}{\ensuremath{\cup}}
\newcommand{\Intersect}{\ensuremath{\cap}}
\newcommand{\BigUnion}{\ensuremath{\bigcup}}
\newcommand{\BigIntersect}{\ensuremath{\bigcap}}
\newcommand{\denot}[1]{\ensuremath{[ \! [#1] \! ]}}
\newcommand{\SemanticsOf}[1]{\ensuremath{[ \! [#1] \! ]}}
\newcommand{\SetOf}[1]{\ensuremath{\{#1\}}}
\newcommand{\RegexVariable}{\ensuremath{\mathit{U}}}   % User Defined
\newcommand{\RegexVariableAlt}{\ensuremath{\mathit{V}}}
\newcommand{\LensVariable}{\ensuremath{\mathit{L}}}
\newcommand{\ExampledRegex}{\ensuremath{\mathit{er}}} % Exampled Regex
\newcommand{\UnambigItOf}[1]{\ensuremath{#1^{*!}}}
\newcommand{\UnambigConcat}{\ensuremath{\Concat^!}}
\newcommand{\SequenceUnambigConcatOf}[1]{\ensuremath{\UnambigConcat(#1)}}
\newcommand{\UnambigConcatOf}[2]{\ensuremath{#1 \UnambigConcat #2}}
\newcommand{\UnambigOrOf}[2]{\ensuremath{\LanguageOf{#1} \cap \LanguageOf{#2} = \emptyset}}
\newcommand{\Atom}{\ensuremath{\mathit{A}}}          % Atoms
\newcommand{\AtomAlt}{\ensuremath{\mathit{B}}}
\newcommand{\AtomType}{\ensuremath{\mathit{Atom}}}
\newcommand{\App}{\ensuremath{\,}}
\newcommand{\Sequence}{\ensuremath{\mathit{SQ}}}
\newcommand{\SequenceType}{\ensuremath{\mathit{Sequence}}}
\newcommand{\LetIn}[2]{\ensuremath{\text{let } #1 = #2\text{ in }}}
\newcommand{\LetWhereIn}[3]{\ensuremath{\text{let } #1 = #2 \text{ where } #3 \text{ in }}}
\newcommand{\Where}{\ensuremath{\text{ where }}}
\newcommand{\ClauseAlt}{\ensuremath{\mathit{bl}}}       % Clauses
\newcommand{\SequenceAlt}{\ensuremath{\mathit{TQ}}}
\newcommand{\DNFRegex}{\ensuremath{\mathit{DS}}}         % Regular Expression
\newcommand{\DNFRegexAlt}{\ensuremath{\mathit{DT}}}    %Alt Regex
\newcommand{\DNFRegexType}{\ensuremath{\mathit{DNF}}}
\newcommand{\LensContext}{\ensuremath{\Gamma}}
\newcommand{\RegexContext}{\ensuremath{\Delta}}  % Context
\newcommand{\FullContext}{\ensuremath{\Delta, \Gamma}}
\newcommand{\String}{\ensuremath{\mathit{s}}\xspace}        % String
\newcommand{\StringAlt}{\ensuremath{\mathit{t}}}        % StringAlt
\newcommand{\StringAltAlt}{\ensuremath{\mathit{u}}}        % StringAltAlt
\newcommand{\ExampleNumberList}{\ensuremath{\mathit{enl}}} %Example Number List
\newcommand{\ExampleNumberListList}{\ensuremath{\mathit{enll}}}
\newcommand{\ExampleStringList}{\ensuremath{\mathit{esl}}}
\newcommand{\StringList}{\ensuremath{\mathit{sl}}}
\newcommand{\Natural}{\ensuremath{\mathit{n}}}
\newcommand{\Interleaving}[1]{\ensuremath{\mathit{interleaving}(#1)}}
\newcommand{\Interleave}{\ensuremath{\mathit{interleave}}}
\newcommand{\BinaryInterleave}[2]{\ensuremath{\mathit{interleave}(#1,#2)}}
\newcommand{\NAryInterleave}[2]{\ensuremath{\mathit{interleave}(#1,\ldots,#2)}}
\newcommand{\Combine}{\ensuremath{\mathit{combine}}}
\newcommand{\List}{\ensuremath{\mathit{l}}}
\newcommand{\ValidCombine}[2]{\ensuremath{\mathit{validcombine}(#1,#2)}}
\newcommand{\ValidRegexContext}[2]{\ensuremath{\mathit{validregexcontext}(#1,#2)}}
\newcommand{\Parent}[1]{\ensuremath{\mathit{parent}(#1)}}
\newcommand{\Parented}[1]{\ensuremath{mathit{parented}(#1)}}
\newcommand{\CombineString}[1]{\ensuremath{\mathit{combine}_{\ExampleStringList}(#1)}}
\newcommand{\CombineList}[1]{\ensuremath{\mathit{combine}_{\ExampleNumberListList}(#1)}}
\newcommand{\Length}[1]{\ensuremath{\mathit{len}(#1)}}
\newcommand{\Language}{\ensuremath{L}}
\newcommand{\LanguageOf}[1]{\ensuremath{\mathcal{L}(#1)}}
\newcommand{\LanguageUnderContextOf}[2]{\ensuremath{\Language{}_{#1}(#2)}}
\newcommand{\ParseTree}{\ensuremath{\mathit{p}}}
\newcommand{\ParseTreeAlt}{\ensuremath{\mathit{q}}}
\newcommand{\ParseTrees}{\ensuremath{\mathit{ps}}}
\newcommand{\ParseTreeAlts}{\ensuremath{\mathit{qs}}}
\newcommand{\StarParse}[1]{\ensuremath{\mathit{starparse}(#1)}}
\newcommand{\LeftChoiceParse}[1]{\ensuremath{\mathit{l}.(#1)}}
\newcommand{\RightChoiceParse}[1]{\ensuremath{\mathit{r}.(#1)}}
\newcommand{\RangeExcInc}[2]{\ensuremath{(#1,#2]}}
\newcommand{\RangeIncInc}[2]{\ensuremath{[#1,#2]}}

\newcommand{\Lens}{\ensuremath{\mathit{\ell}}\xspace}
\newcommand{\AtomLens}{\ensuremath{\mathit{al}}}
\newcommand{\IterateAtomType}{\textit{Iterate}}
\newcommand{\ConcatedAtomsLens}{\ensuremath{\mathit{cal}}}
\newcommand{\OredClausesLens}{\ensuremath{\mathit{ocl}}}
\newcommand{\ClauseLens}{\ensuremath{\mathit{cll}}}
\newcommand{\SequenceLens}{\ensuremath{\mathit{sql}}}
\newcommand{\SequenceLensType}{\ensuremath{\mathit{SequenceLens}}}
\newcommand{\DNFLens}{\ensuremath{\mathit{dl}}}
\newcommand{\DNFLensType}{\ensuremath{\mathit{DNFLens}}}
\newcommand{\AtomLensType}{\ensuremath{\mathit{AtomLens}}}
\newcommand{\SynSim}[2]{\ensuremath{#1 \sim_{\mathit{sym}} #2}}
\newcommand{\ExdSynSim}[3]{\ensuremath{#2 \sim_{\mathit{sym},#1} #3}}

\newcommand{\PermutationSetOf}[1]{\ensuremath{S_{#1}}}
\newcommand{\Permutation}{\ensuremath{\sigma}}

\newcommand{\Star}{\ensuremath{^*}}
\newcommand{\StarOf}[1]{\ensuremath{{#1}\Star}}
\newcommand{\ConstLens}{\ensuremath{\mathit{const}}}
\newcommand{\ConstLensOf}[2]{\ensuremath{\ConstLens(#1,#2)}}
\newcommand{\ConcatLens}{\ensuremath{\KW{concat}}}
\newcommand{\ConcatLensOf}[2]{\ensuremath{\ConcatLens(#1,#2)}}
\newcommand{\ConcatLensShortOf}[2]{\ensuremath{\mathit{c}(#1,#2)}}
\newcommand{\SwapLens}{\ensuremath{\KW{swap}}\xspace}
\newcommand{\SwapLensOf}[2]{\ensuremath{\SwapLens(#1,#2)}}
\newcommand{\SwapLensShortOf}[2]{\ensuremath{\mathit{s}(#1,#2)}}
\newcommand{\OrLens}{\ensuremath{\KW{or}}\xspace}
\newcommand{\OrLensOf}[2]{\ensuremath{\OrLens(#1,#2)}}
\newcommand{\IdentityLens}{\ensuremath{\KW{id}}}
\newcommand{\IdentityLensOf}[1]{\ensuremath{\IdentityLens(#1)}}
\newcommand{\IdentityLensShortT}{\ensuremath{\mathit{id}}}
\newcommand{\IdentityLensShortOf}[1]{\ensuremath{\IdentityLensShortT_{#1}}}
\newcommand{\IterateLens}{\ensuremath{\KW{iterate}}\xspace}
\newcommand{\IterateLensOf}[1]{\ensuremath{\mathit{\IterateLens(#1)}}}
\newcommand{\Identity}{\ensuremath{\mathit{id}}}
\newcommand{\Compose}{\ensuremath{\circ}}
\newcommand{\ComposeLensOf}[2]{\ensuremath{#1\mathrel{;}#2}}
\newcommand{\Disconnect}{\ensuremath{\KW{disc}}\xspace}
\newcommand{\DisconnectOf}[4]{\ensuremath{\Disconnect(#1,#2,#3,#4)}}
\newcommand{\MergeL}{\ensuremath{\KW{merge\_left}}\xspace}
\newcommand{\MergeLOf}[2]{\ensuremath{\MergeL(#1,#2)}}
\newcommand{\MergeR}{\ensuremath{\KW{merge\_right}}\xspace}
\newcommand{\MergeROf}[2]{\ensuremath{\MergeR(#1,#2)}}
\newcommand{\Invert}{\ensuremath{\KW{invert}}}
\newcommand{\InvertOf}[1]{\ensuremath{\Invert(#1)}}

% GRAMMAR OPERATORS
\newcommand{\GBar}{\ensuremath{~|~}}
\newcommand{\GIndent}{\hspace{.5in}}
\newcommand{\GEq}{\ensuremath{::=~}}
\newcommand{\GEmp}{\ensuremath{\cdot}}
\newcommand{\Perm}{\ensuremath{\mathit{Perm}}}
\newcommand{\Nats}{\ensuremath{\mathbb{N}}}

\newcommand{\InverseOf}[1]{\ensuremath{#1^{-1}}}
\newcommand{\FloorOf}[1]{\ensuremath{\lfloor#1\rfloor}}
\newcommand{\CeilOf}[1]{\ensuremath{\lceil#1\rceil}}
\newcommand{\OfType}{\ensuremath{:}}
\newcommand{\OfRewritelessType}{\ensuremath{\,\,\tilde{\OfType}\,\,}}
\newcommand{\MapsBetweenTypeOf}[2]{\ensuremath{#1 \Leftrightarrow #2}}
\newcommand{\ArrowTypeOf}[2]{\ensuremath{#1 \rightarrow #2}}
\newcommand{\SizeOf}[1]{\ensuremath{|#1|}}

\newcommand{\ToDNFRegex}{\ensuremath{\Downarrow}}
\newcommand{\ToDNFRegexOf}[1]{\ensuremath{\ToDNFRegex\mkern-4mu #1}}
\newcommand{\ToRegex}{\ensuremath{\Uparrow}}
\newcommand{\ToRegexOf}[1]{\ensuremath{\ToRegex\mkern-4mu #1}}

\newcommand{\SuchThat}{\ensuremath{~|~}}
\newcommand{\That}{\ensuremath{~.~}}
\newcommand{\Given}{\ensuremath{~|~}}

\newcommand{\LeftQuotientOf}[2]{\ensuremath{#1\backslash#2}}
\newcommand{\RightQuotientOf}[2]{\ensuremath{#1\slash#2}}

\newcommand{\SuffixOf}[1]{\ensuremath{S_{#1}}}
\newcommand{\PrefixOf}[1]{\ensuremath{P_{#1}}}

\newcommand{\ComplementOf}[1]{\ensuremath{\bar{#1}}}

\newcommand{\Alphabet}{\ensuremath{\Sigma}}
\newcommand{\Character}{\ensuremath{c}}
\newcommand{\CharacterAlt}{\ensuremath{d}}

\newcommand{\SequenceLeft}{\ensuremath{[}}
\newcommand{\SequenceRight}{\ensuremath{]}}
\newcommand{\SequenceOf}[1]{\ensuremath{\SequenceLeft#1\SequenceRight}}
\newcommand{\SeqSep}{\ensuremath{\mkern-1mu\Concat\mkern-1mu}}
\newcommand{\DNFLeft}{\ensuremath{\langle}}
\newcommand{\DNFRight}{\ensuremath{\rangle}}
\newcommand{\DNFOf}[1]{\ensuremath{\DNFLeft#1\DNFRight}}
\newcommand{\DNFSep}{\ensuremath{\Or}}
\newcommand{\SequenceLensLeft}{\ensuremath{[}}
\newcommand{\SequenceLensRight}{\ensuremath{]}}
\newcommand{\SequenceLensOf}[1]{\ensuremath{\SequenceLensLeft#1\SequenceLensRight}}
\newcommand{\SeqLSep}{\ensuremath{\mkern-1mu\Concat\mkern-1mu}}
\newcommand{\DNFLensLeft}{\ensuremath{\langle}}
\newcommand{\DNFLensRight}{\ensuremath{\rangle}}
\newcommand{\DNFLensOf}[1]{\ensuremath{\DNFLensLeft#1\DNFLensRight}}
\newcommand{\DNFLSep}{\ensuremath{\Or}}


\newcommand{\ConstantLensRule}{\textsc{Constant Lens}}
\newcommand{\IdentityLensRule}{\textsc{Identity Lens}}
\newcommand{\IterateLensRule}{\textsc{Iterate Lens}}
\newcommand{\ConcatLensRule}{\textsc{Concat Lens}}
\newcommand{\SwapLensRule}{\textsc{Swap Lens}}
\newcommand{\OrLensRule}{\textsc{Or Lens}}
\newcommand{\ComposeLensRule}{\textsc{Compose Lens}}
\newcommand{\RewriteRegexLensRule}{\textsc{Rewrite Regex Lens}}

\newcommand{\AtomUnrollstarLeftRule}{\textsc{Atom Unrollstar\SubLeft}}
\newcommand{\AtomUnrollstarRightRule}{\textsc{Atom Unrollstar\SubRight}}
\newcommand{\ParallelAtomStructuralRewriteRule}{\textsc{Parallel Atom Structural Rewrite}}
\newcommand{\ParallelSwapAtomStructuralRewriteRule}{\textsc{Parallel Swap Atom Structural Rewrite}}
\newcommand{\AtomStructuralRewriteRule}{\textsc{Atom Structural Rewrite}}
\newcommand{\DNFStructuralRewriteRule}{\textsc{DNF Structural Rewrite}}
\newcommand{\ParallelDNFStructuralRewriteRule}{\textsc{Parallel DNF Structural Rewrite}}
\newcommand{\ParallelSwapDNFStructuralRewriteRule}{\textsc{Parallel Swap DNF Structural Rewrite}}
\newcommand{\IdentityRewriteRule}{\textsc{Identity Rewrite}}
\newcommand{\DNFReorderRule}{\textsc{DNF Reorder}}

\newcommand{\SequenceLensRule}{\textsc{Sequence Lens}}
\newcommand{\AtomLensRule}{\textsc{Atom Lens}}
\newcommand{\DNFLensRule}{\textsc{DNF Lens}}
\newcommand{\RewriteDNFRegexLensRule}{\textsc{Rewrite DNF Regex Lens}}

\newcommand{\SubLeft}{\textsubscript{L}}
\newcommand{\SubRight}{\textsubscript{R}}

\newcommand{\Set}{\ensuremath{\mathit{S}}}

\newcommand{\OrIdentityRule}{\textit{+ Ident}}
\newcommand{\EmptyProjectionRightRule}{\textit{0 Proj\SubRight{}}}
\newcommand{\EmptyProjectionLeftRule}{\textit{0 Proj\SubLeft{}}}
\newcommand{\ConcatAssocRule}{\textit{\Concat{} Assoc}}
\newcommand{\OrAssociativityRule}{\textit{\Or{} Assoc}}
\newcommand{\OrCommutativityRule}{\textit{\Or{} Comm}}
\newcommand{\DistributivityLeftRule}{\textit{Dist\SubRight{}}}
\newcommand{\DistributivityRightRule}{\textit{Dist\SubLeft{}}}
\newcommand{\ConcatIdentityLeftRule}{\textit{\Concat{} Ident\SubLeft{}}}
\newcommand{\ConcatIdentityRightRule}{\textit{\Concat{} Ident\SubRight{}}}
\newcommand{\SumstarRule}{\textit{Sumstar}}
\newcommand{\ProductstarRule}{\textit{Prodstar}}
\newcommand{\UnrollstarLeftRule}{\textit{Unrollstar\SubLeft{}}}
\newcommand{\UnrollstarRightRule}{\textit{Unrollstar\SubRight{}}}
\newcommand{\StarstarRule}{\textit{Starstar}}
\newcommand{\DicyclicityRule}{\textit{Dicyc}}
\newcommand{\Derivation}{\ensuremath{\mathcal{D}}}

\newcolumntype{q}{>{$}l<{$}}
\newcolumntype{v}{>{$}r<{$}}

\renewcommand{\subsubsection}[1]{\paragraph{{#1}}}

\newcommand{\Examples}{\ensuremath{\mathit{exs}}}

\newcommand{\ParallelReduction}{\ensuremath{\rightarrow}}
\newcommand{\ParallelRewrite}{\ensuremath{\,\mathrlap{\to}\,{\scriptstyle\parallel}\,\,\,}}
\newcommand{\ParallelRewriteAtom}{\ensuremath{\ParallelRewrite_{\Atom}}}
\newcommand{\ParallelRewriteSwap}{\ensuremath{\ParallelRewrite^{\mathit{swap}}}}
\newcommand{\ParallelRewriteSwapAtom}{\ensuremath{\ParallelRewrite^{\mathit{swap}}_{\Atom}}}

\newcommand{\Property}{\ensuremath{\mathit{p}}}
\newcommand{\Propagator}{\ensuremath{\mathit{q}}}

\newcommand{\Relation}{\ensuremath{\mathit{R}}}
\newcommand{\RelationSet}{\ensuremath{\mathit{RS}}}

\newcommand{\DiamondProperty}{\ensuremath{\mathit{confluent}}}
\newcommand{\DiamondPropertyWithPropertyOf}[1]{\ensuremath{\DiamondProperty_{#1}}}
\newcommand{\IsConfluentWithPropertyOf}[2]
    {\ensuremath{\DiamondPropertyWithPropertyOf{#2}(#1)}}
\newcommand{\BisimilarProperty}{\ensuremath{\mathit{bisimilar}}}
\newcommand{\BisimilarPropertyWithPropertyOf}[1]{\ensuremath{\BisimilarProperty_{#1}}}
\newcommand{\IsBisimilarWithPropertyOf}[2]
    {\ensuremath{\BisimilarPropertyWithPropertyOf{#2}(#1)}}

\newcommand{\Reduces}{\ensuremath{\rightarrow}}

\newcommand{\SortaEquiv}{\ensuremath{\equiv_{sorta}}}

\newcommand{\AtomEquiv}{\ensuremath{\equiv_{\Atom}}}

\newcommand{\Sep}{\ensuremath{\$}}

\newcommand{\Cross}{\ensuremath{\times}}

\newcommand{\Distance}{\ensuremath{\mathit{d}}}

\newcommand{\AbsOf}[1]{\ensuremath{|#1|}}
\newcommand{\Size}{\ensuremath{\mathit{size}}}
\newcommand{\Module}{\ensuremath{\mathit{M}}}
\newcommand{\VectorSpace}{\ensuremath{\mathit{V}}}

\newcommand{\GetDist}{\ensuremath{\mathit{dist}}}
\newcommand{\LOneNorm}{\ensuremath{\ell_1}}

\newcommand{\Sorting}{\ensuremath{\mathit{sorting}}}
\newcommand{\SortingOf}[2]{\ensuremath{\Sorting(#1,#2)}}

\newcommand{\Sort}{\ensuremath{\mathit{sort}}}
\newcommand{\SortOf}[2]{\ensuremath{\Sort(#1,#2)}}

\newcommand{\ListType}{\ensuremath{\mathit{List}}}
\newcommand{\ListTypeOf}[1]{\ensuremath{#1\,\ListType}}
\newcommand{\ListLeft}{\ensuremath{[}}
\newcommand{\ListRight}{\ensuremath{]}}
\newcommand{\ListOf}[1]{\ensuremath{\ListLeft #1 \ListRight}}
\newcommand{\DNFLeq}{\ensuremath{\leq_{DNF}}}
\newcommand{\SequenceLeq}{\ensuremath{\leq_{Seq}}}
\newcommand{\AtomLeq}{\ensuremath{\leq_{Atom}}}
\newcommand{\ILSLeq}{\ensuremath{\leq_{\mathit{intlistset}}}}
\newcommand{\ExampledDNFLeq}{\ensuremath{\leq_{DNF}^{\Examples}}}
\newcommand{\ExampledSequenceLeq}{\ensuremath{\leq_{Seq}^{\Examples}}}
\newcommand{\ExampledAtomLeq}{\ensuremath{\leq_{Atom}^{\Examples}}}
\newcommand{\DNFEq}{\ensuremath{=_{DNF}}}
\newcommand{\SequenceEq}{\ensuremath{=_{Seq}}}
\newcommand{\AtomEq}{\ensuremath{=_{Atom}}}

\newcommand{\NormalizedDNFOf}[1]{\ensuremath{\DNFOf{#1}_n}}
\newcommand{\NormalizedSequenceOf}[1]{\ensuremath{\SequenceOf{#1}_n}}
\newcommand{\NormalizedStarOf}[1]{\ensuremath{\NormalizedStarOf{#1}_n}}

\newcommand{\AtomNormalizer}{\ensuremath{\mathit{AN}}}
\newcommand{\AtomNormalizerType}{\textit{Atom Normalizer}}
\newcommand{\SequenceNormalizer}{\ensuremath{\mathit{SNN}}}
\newcommand{\SequenceNormalizerType}{\textit{Sequence Normalizer}}
\newcommand{\DNFRegexNormalizer}{\ensuremath{\mathit{DNFN}}}
\newcommand{\DNFRegexNormalizerType}{\textit{DNF Normalizer}}

\newcommand{\Normalize}{\ensuremath{\mathcal{N}}}
\newcommand{\NormalizeOf}[1]{\ensuremath{\Normalize(#1)}}

\newcommand{\DNFLensSynth}{\ensuremath{\mathit{DNFLensSynth}}}
\newcommand{\SequenceLensSynth}{\ensuremath{\mathit{SequenceLensSynth}}}
\newcommand{\AtomLensSynth}{\ensuremath{\mathit{AtomLensSynth}}}
\newcommand{\DNFLensSynthOf}[2]{\ensuremath{\DNFLensSynth(#1,#2)}}
\newcommand{\SequenceLensSynthOf}[2]{\ensuremath{\SequenceLensSynth(#1,#2)}}
\newcommand{\AtomLensSynthOf}[2]{\ensuremath{\AtomLensSynth(#1,#2)}}

\newcommand{\DNFLensHasSemanticsOf}[1]{\ensuremath{\xLeftrightarrow{#1}}}
\newcommand{\SatisfiesDNFLensHasSemanticsOf}[3]{\ensuremath{#2\DNFLensHasSemanticsOf{#1}#3}}
\newcommand{\SatisfiesIdentitySemantics}[2]
  {\ensuremath{\SatisfiesDNFLensHasSemanticsOf{\Identity}{#1}{#2}}}
\newcommand{\EquivalenceOf}[1]{\equiv_{#1}}

\newcommand{\SSREquiv}{\ensuremath{\equiv^s}}

\newcommand{\ReflexivityRule}{\textsc{Reflexivity}}
\newcommand{\BaseRule}{\textsc{Base}}
\newcommand{\SymmetryRule}{\textsc{Symmetry}}
\newcommand{\TransitivityRule}{\textsc{Transitivity}}

\newcommand{\BaseRegexType}{\textit{Base}}
\newcommand{\EmptyRegexType}{\textit{Empty}}
\newcommand{\StarRegexType}{\textit{Star}}
\newcommand{\ConcatRegexType}{\textit{Concat}}
\newcommand{\OrRegexType}{\textit{Or}}

\newcommand{\ConstLensType}{\textit{Const}}
\newcommand{\ConcatLensType}{\textit{Concat}}
\newcommand{\IterateLensType}{\textit{Iterate}}
\newcommand{\SwapLensType}{\textit{Swap}}
\newcommand{\OrLensType}{\textit{Or}}
\newcommand{\ComposeLensType}{\textit{Compose}}
\newcommand{\IdentityLensType}{\textit{Identity}}


\newcommand{\StarAtomType}{\textit{Star}}
\newcommand{\MultiConcatSequenceType}{\textit{MultiConcat}}
\newcommand{\MultiOrDNFRegexType}{\textit{MultiOr}}

\newcommand{\AtomToDNF}{\ensuremath{\mathcal{D}}}
\newcommand{\AtomToDNFOf}[1]{\ensuremath{\AtomToDNF(#1)}}
\newcommand{\AtomToDNFLens}{\ensuremath{\mathcal{D}}}
\newcommand{\AtomToDNFLensOf}[1]{\ensuremath{\AtomToDNFLens(#1)}}

\newcommand{\Queue}{\ensuremath{\mathit{Q}}}
\newcommand{\QueueElement}{\ensuremath{\mathit{qe}}}
\newcommand{\QueueElements}{\ensuremath{\QueueElement\mathit{s}}}
\newcommand{\ExpCount}{\ensuremath{\mathit{e}}}
\newcommand{\True}{\ensuremath{\mathit{true}}}
\newcommand{\False}{\ensuremath{\mathit{false}}}
\newcommand{\Null}{\ensuremath{\mathit{null}}}
\newcommand{\DNFRegexs}{\ensuremath{\DNFRegex\mathit{s}}}
\newcommand{\Types}{\ensuremath{\textit{t}}}

\newcommand{\DictionaryOrderL}{\ensuremath{[}}
\newcommand{\DictionaryOrderR}{\ensuremath{]}}
\newcommand{\DictionaryOrderOf}[1]{\ensuremath{\DictionaryOrderL #1 \DictionaryOrderR}}

\newcommand{\SetOfListOrderL}{\ensuremath{\{}}
\newcommand{\SetOfListOrderR}{\ensuremath{\}}}
\newcommand{\SetOfListOrderOf}[1]{\ensuremath{\SetOfListOrderL #1 \SetOfListOrderR}}

\newcommand{\Int}{\ensuremath{i}}
\newcommand{\UserDef}{\ensuremath{U}}
\newcommand{\UserDefAlt}{\ensuremath{V}}

\newcommand{\Optician}{Optician}
\newcommand{\SOptician}{Optician\textsubscript{S}}
\newcommand{\SynthLens}{\PCF{SynthLens}}
\newcommand{\RXSearch}{\PCF{RXSearch}\xspace}
\newcommand{\SynthDNFLens}{\PCF{SynthDNFLens}}
\newcommand{\ToLens}{\ensuremath{\Uparrow}}
\newcommand{\ToLensOf}[1]{\ensuremath{\ToLens{}\mkern-4mu #1}}
\newcommand{\ToDNFRegexText}{\PCF{ToDNFRegex}}
\newcommand{\Beautify}{\PCF{Beautify}}
\newcommand{\RigidSynth}{\PCF{RigidSynth}}
\newcommand{\GreedySynth}{\PCF{GreedySynth}\xspace}
\newcommand{\RigidSynthInternal}{\PCF{RigidSynthInternal}}
\newcommand{\RigidSynthSequence}{\PCF{RigidSynthSeq}}
\newcommand{\RigidSynthAtom}{\PCF{RigidSynthAtom}}
\newcommand{\GetDNFNormalizer}{\PCF{GetDNFNormalizer}}
\newcommand{\CreatePQueue}{\PCF{CreatePQueue}}
\newcommand{\GetTransitiveSet}{\PCF{GetTransitiveSet}}
\newcommand{\GetCurrentSet}{\PCF{GetCurrentSet}}
\newcommand{\Pop}{\PCF{Pop}}
\newcommand{\ExpandOnce}{\PCF{ExpandOnce}}
\newcommand{\ExpandRequired}{\PCF{ExpandRequired}}
\newcommand{\FixProblemElts}{\PCF{FixProblemElts}}
\newcommand{\Expand}{\PCF{Expand}}
\newcommand{\ForceExpand}{\PCF{ForceExpand}}
\newcommand{\Reveal}{\PCF{Reveal}}
\newcommand{\Map}{\PCF{Map}}
\newcommand{\EnqueueMany}{\PCF{EnqueueMany}}
\newcommand{\ReturnVal}[1]{\ensuremath{\Return\,#1}}
\newcommand{\CurrentSet}{\ensuremath{\mathit{CS}}}
\newcommand{\TransitiveSet}{\ensuremath{\mathit{TS}}}

\newcommand{\StringType}{\ensuremath{\mathit{String}}}

\newcommand{\SUBSECTION}[1]{\iffull\subsection{#1}\else\paragraph*{#1}}

\newcommand{\None}{\ensuremath{\mathit{None}}}
\newcommand{\DNFLensOption}{\ensuremath{\DNFLens\mathit{o}}}
\newcommand{\Some}{\ensuremath{\mathit{Some}}}
\newcommand{\SomeOf}[1]{\ensuremath{\Some\,#1}}
\newcommand{\Option}{\ensuremath{\mathit{Option}}}
\newcommand{\OptionOf}[1]{\ensuremath{#1 \App \Option}}

\newcommand{\Success}{\ensuremath{\boldsymbol{\color{dkgreen}\checkmark}}}
\newcommand{\Failure}{\ensuremath{\boldsymbol{\color{dkred}\times}}}

\newcommand{\Append}{\ensuremath{+\!\!\!\!+\ }}
\newcommand{\ModernTitle}{\VarCF{modern\_\discretionary{}{}{}title}}
\newcommand{\DNFModernTitle}{\VarCF{dnf\_modern\_title}}
\newcommand{\DNFLegacyTitle}{\VarCF{dnf\_legacy\_title}}
\newcommand{\LegacyTitle}{\VarCF{legacy\_\discretionary{}{}{}title}}
\newcommand{\LegacyTitleP}{\VarCF{legacy\_\discretionary{}{}{}title'}}
\newcommand{\TextChar}{\VarCF{text\_\discretionary{}{}{}char}}

\newcommand{\IntList}{\ensuremath{\mathit{il}}}
\newcommand{\IntListSet}{\ensuremath{\mathit{ils}}}
\newcommand{\StringIntListSet}{\ensuremath{\mathit{sils}}}
\newcommand{\ProjectStrings}{\ensuremath{\mathit{projectstrings}}}
\newcommand{\ProjectStringsOf}[1]{\ensuremath{\mathit{\ProjectStrings(#1)}}}
\newcommand{\ProjectILS}{\ensuremath{\mathit{projectils}}}
\newcommand{\ProjectILSOf}[1]{\ensuremath{\mathit{\ProjectILS(#1)}}}
\newcommand{\Generates}{\ensuremath{\rightsquigarrow}}

\newcommand{\ExampledDNFRegex}{\ensuremath{EDS}}
\newcommand{\ExampledDNFRegexAlt}{\ensuremath{EDT}}

\newcommand{\ExampledSequence}{\ensuremath{ESQ}}
\newcommand{\ExampledSequenceAlt}{\ensuremath{ETQ}}

\newcommand{\ExampledAtom}{\ensuremath{EA}}
\newcommand{\ExampledAtomAlt}{\ensuremath{EB}}

\newcommand{\EmbedExamples}{\ensuremath{\PCF{EmbedExamples}}}
\newcommand{\EmbedExamplesOf}[2]{\ensuremath{\EmbedExamples(#1,#2)}}

\newcommand{\Morpheus}{Morpheus}
\newcommand{\InSynth}{InSynth}

\newcommand{\FullMode}{\textbf{Full}\xspace}
\newcommand{\NoCSMode}{\textbf{NoCS}\xspace}
\newcommand{\NoFPEMode}{\textbf{NoFPE}\xspace}
\newcommand{\NoERMode}{\textbf{NoER}\xspace}
\newcommand{\NoUDMode}{\textbf{NoUD}\xspace}
\newcommand{\FlashExtractMode}{\textbf{FlashExtract}\xspace}
\newcommand{\FlashFillMode}{\textbf{Flash Fill}\xspace}
\newcommand{\NaiveMode}{\textbf{Na\"ive}\xspace}

% Asymmetric Lens Commands
\newcommand{\Put}{\ensuremath{\mathit{put}}\xspace}
\newcommand{\Get}{\ensuremath{\mathit{get}}\xspace}
\newcommand{\Create}{\ensuremath{\mathit{create}}\xspace}

% Symmetric Lens Components
\newcommand{\CreateR}{\KW{creater}\xspace}
\newcommand{\CreateL}{\KW{createl}\xspace}
\newcommand{\PutR}{\KW{putr}\xspace}
\newcommand{\PutL}{\KW{putl}\xspace}
% Symmetric Lens Commands
\newcommand{\CreateROf}[1]{\CreateR \App #1}
\newcommand{\CreateLOf}[1]{\CreateL \App #1}
\newcommand{\PutROf}[2]{\PutR \App #1 \App #2}
\newcommand{\PutLOf}[2]{\PutL \App #1 \App #2}
% Stateless Symmetric Lens Components
% Stateless Symmetric Lens Applications
\newcommand{\PutRL}{\PCF{PutRL}\xspace}
\newcommand{\PutLR}{\PCF{PutLR}\xspace}
% Symmetric Lens Laws
% Classical
% Stateless
\newcommand{\CreatePutRL}{\PCF{CreatePutRL}}
\newcommand{\CreatePutLR}{\PCF{CreatePutLR}}
% Forgetful 
\newcommand{\ForgetfulLR}{\PCF{ForgetfulLR}\xspace}
\newcommand{\ForgetfulRL}{\PCF{ForgetfulRL}\xspace}



% Symmetric DNF Lenses
\newcommand{\SDNFLens}{\ensuremath{\mathit{sdl}}}
\newcommand{\SDNFLensOf}[1]{\ensuremath{\DNFLensLeft#1\DNFLensRight}}
\newcommand{\SSQLensOf}[1]{\ensuremath{\SequenceLensLeft#1\SequenceLensRight}}
\newcommand{\SSQLens}{\ensuremath{\mathit{ssql}}}
\newcommand{\SAtomLens}{\ensuremath{\mathit{sal}}}


% Expected Information
\newcommand{\Entropy}{\ensuremath{\mathbb{H}}}
\newcommand{\EntropyOf}[1]{\ensuremath{\Entropy(#1)}}
\newcommand{\Argmin}{\ensuremath{\text{argmin}}}
\newcommand{\ArgminOver}[1]{\ensuremath{\underset{#1}{\Argmin}}}

% Regex
\newcommand{\PRegexOr}[3]{\ensuremath{#1~|_{#3}~#2}}
\newcommand{\PRegexConcat}[2]{{\ensuremath{\RegexConcat{#1}{#2}}}}
\newcommand{\PRegexStar}[2]{\ensuremath{#1^{*_{#2}}}}
\newcommand{\Probability}{\ensuremath{p}}
\newcommand{\ProbabilityAlt}{\ensuremath{q}}
\newcommand{\ProbabilityOf}[2]{P_{#1}(#2)}
\newcommand{\Undefined}{\ensuremath{\mathit{undefined}}}

\newcommand{\Fst}{\ensuremath{\mathit{fst}}}
\newcommand{\Snd}{\ensuremath{\mathit{snd}}}
\newcommand{\EditSeq}{\ensuremath{mathit{EditSeq}}}
\newcommand{\NumBenchmarks}{\ensuremath{20}\xspace}

\newcommand{\InL}{\ensuremath{\mathit{inl}}}
\newcommand{\InLOf}[1]{\ensuremath{\InL \App #1}}
\newcommand{\InR}{\ensuremath{\mathit{inr}}}
\newcommand{\InROf}[1]{\ensuremath{\InR \App #1}}

\newcommand{\Wildcard}{\ensuremath{\_}}
\newcommand{\SingleApp}{\ensuremath{\mathit{singleapp}}}
\newcommand{\Log}{\ensuremath{\mathit{log}}}

\newcommand{\Minute}{\CF{minute}\xspace}
\newcommand{\LinuxCommand}{\CF{linux\_command}\xspace}
\newcommand{\WindowsCommand}{\CF{win\_command}\xspace}

\newcommand{\Reals}{\ensuremath{\mathbb{R}}}
% \lstset{framextopmargin=50pt,frame=bottomline}

%%% Local Variables:
%%% TeX-master: "main"
%%% End:


\begin{document}
\title{Lens Synthesis}
\date{}
\maketitle

\section{Already known stuff}
$\Sigma$ alphabet.
\\

$\StarOf{\Sigma}$ all words over the alphabet $\Sigma$.
\\

The set, $\Language \subset \StarOf{\Sigma}$ is a language.
\\

Regular Expressions:

\begin{tabular}{l@{\hspace*{5mm}}l@{\ }c@{\ }l@{\hspace*{5mm}}>{\itshape\/}l}
& \Regex{},\RegexAlt{} & \GEq{} & \hspace{1.06em}s & \BaseRegexType{} \\
          & & & \GBar{} $\emptyset$ & \EmptyRegexType{} \\
          & & & \GBar{} \Regex{}* & \StarRegexType{} \\
          & & & \GBar{} $\RegexConcat{\Regex_1}{\Regex_2}$ & \ConcatRegexType{} \\
          & & & \GBar{} $\RegexOr{\Regex_1}{\Regex_2}$ & \OrRegexType{} \\
\end{tabular}
\\

Semantics of regular expressions:
$\RegexSemanticsOf{\cdot} \OfType \RegexType \rightarrow \SetType$
\begin{itemize}
\item $\RegexSemanticsOf{\String} = \SetOf{\String}$
\item $\RegexSemanticsOf{\emptyset} = \SetOf{}$
\item $\RegexSemanticsOf{\StarOf{\Regex}} = \SetOf{\String_1\ldots\String_n
    \SuchThat \String_i \in \RegexSemanticsOf{\Regex} \BooleanAnd n \in \Nats}$
\item $\RegexSemanticsOf{\RegexConcat{\Regex_1}{\Regex_2}} =
  \SetOf{\String_1\String_2 \SuchThat \String_1 \in \RegexSemanticsOf{\Regex_1}
    \BooleanAnd \String_2 \in \RegexSemanticsOf{\Regex_2}}$
\item $\RegexSemanticsOf{\RegexOr{\Regex_1}{\Regex_2}} = \SetOf{\String
    \SuchThat \String \in \LanguageOf{\Regex_1} \BooleanOr \String \in
    \LanguageOf{\Regex_2}}$.
\end{itemize}

Equivalence classes on $\Set$ are functions of type: $\Set \rightarrow \Set
\rightarrow \BooleanType$.  We require that equivalence classes are reflexive,
transitive, and symmetric.

\section{New Stuff Building Up to Regular Exps with Quotients}

\subsection{Goals for regular expressions with quotients} We would like to be able to express regular languages with quotients on them
syntactically.  What does it mean to express a regular expression with
quotients?  It means that we would like to express two things with our language:
we would like to be able to express a set, and we would also like to be able to
express equivalences on that set.

The semantics of regular expressions are merely a set.
The semantics of regular expressions with quotients are not only a set, but also
an equivalence relation on that set.  Let's give a definition for these targets,
to make it easier to think about.

\subsection{Setoids}

A \emph{Setoid} as a pair $\Setoid = (\Set,\EquivalenceRelation)$, where
$\Set : \SetType$ and $\EquivalenceRelation \OfType \Set \rightarrow \Set
\rightarrow \BooleanType$, where we also enforce that $\EquivalenceRelation$
must be an equivalence relation.

Examples:
\begin{enumerate}
\item Trivial example: $(\StarOf{\Sigma},=)$.  This is the set of all strings
  over $\Sigma$, where $=$ is the finest equivalence relation, where each string
  is only equal to itself.
\item Useful example: $(\Language,\EquivalenceRelation)$.  Let $\Language$ be a
  language.  Consider $\EquivalenceRelation$, the equivalence relation which has
  all nonempty sequences of whitespace equal to each other.  So ``a\,b'', ``a\,\,\,
  b'', and ``a\,\,\,\,\,\,\, b'' are all equivalent to each other; and ``b\,a'',
  ``b\,\,\, a'', and ``b\,\,\,\,\,\,\,\,\,\,\,\,a'' are all equivalent to each
  other.  However ``a\,b'' and ``b\,a'' are not equivalent to each other.
\end{enumerate}

Clearly, setoids should have some good language to express them, as writing them
up informally is very difficult.

Let's find a language, and semantics on that language, to express setoids.

This will be done in 2 parts: first we will define a language for writing
equivalences on sets, then we will use this language to define a language for
expressing setoids.

\subsection{Equivalences}

First a syntax for the equivalences:

\begin{tabular}{l@{\hspace*{5mm}}l@{\ }c@{\ }l@{\hspace*{5mm}}}
  & \RegularEquivalence{} & \GEq{} & \All \\
  & & & \GBar{} \LensEquivalenceOf{\Lens}
\end{tabular}

Semantics for this is slightly complex, because the semantics depend on the set
we are operating over.  This is handled through parametrizing the semantics by
sets.
\\

$\EquivalenceSemanticsOf{\Set}{\cdot} : \EquivType \rightarrow (\Set
\rightarrow \Set \rightarrow \BooleanType)$

\begin{itemize}
\item $\EquivalenceSemanticsOf{\Set}{\All} = \forall s\in\Set,\, \forall t\in\Set, \,s = t$
\item $\EquivalenceSemanticsOf{\Set}{\LensEquivalenceOf{\Lens}} =
  \forall s\in\Set,\, \forall t\in\Set,\\
  \hspace*{5em}\PutRightOf{\Lens}(s) = t\,\,\BooleanOr\\
  \hspace*{5em}\PutLeftOf{\Lens}(s) = t\,\,\BooleanOr\\
  \hspace*{5em}\PutRightOf{\Lens}(t) = s\,\,\BooleanOr\\
  \hspace*{5em}\PutLeftOf{\Lens}(t) = s$
\end{itemize}

Using this, we can now define regular expressions with equivalences, to express
setoids.

\subsection{Regular Expressions with Quotients}

In addition to expressing setoids, we would like this language to look very
similar to regular
expressions.  We add an additional quotienting combinator, but otherwise keep
the language the same.

\begin{tabular}{l@{\hspace*{5mm}}l@{\ }c@{\ }l@{\hspace*{5mm}}>{\itshape\/}l}
& \RegexWithQuot{},\RegexWithQuotAlt{} & \GEq{} & \hspace{1.06em}s & \BaseRegexWithQuotType{} \\
          & & & \GBar{} $\emptyset$ & \EmptyRegexWithQuotType{} \\
          & & & \GBar{} $\StarOf{\RegexWithQuot}$ & \StarRegexWithQuotType{} \\
          & & & \GBar{} $\RegexConcat{\RegexWithQuot_1}{\RegexWithQuot_2}$ & \ConcatRegexWithQuotType{} \\
          & & & \GBar{} $\RegexOr{\RegexWithQuot_1}{\RegexWithQuot_2}$ & \OrRegexWithQuotType{} \\
          & & & \GBar{} $\RegexQuotient{\RegexWithQuot}{\RegularEquivalence}$ & \QuotientRegexWithQuotType{} \\
\end{tabular}

This is very similar to regular expressions, but we also have a quotient
operation, which acts on regular expressions.

Now, let's define semantics.
\\

$\RXWithQuotSemanticsOf{\cdot} : \RegexWithQuotType \rightarrow \SetoidType$

\begin{itemize}
\item $\RXWithQuotSemanticsOf{\String} = (\SetOf{\String},=)$
\item $\RXWithQuotSemanticsOf{\emptyset} = (\SetOf{},=)$
\item $\RXWithQuotSemanticsOf{\StarOf{\RegexWithQuot}} =
  (\Language,\EquivalenceRelation_R)$ where, if
  $(\Language',\EquivalenceRelation') =
  \RXWithQuotSemanticsOf{\RegexWithQuot}$\\
  $\Language = \SetOf{\String_1\ldots\String_n
    \SuchThat \String_i \in \Language' \BooleanAnd n \in \Nats}$\\
  $\EquivalenceRelation_R$ is the equivalence relation
  generated by the binary relation $R$, where
  $R\,(\String_1\ldots\String_n)\,(\StringAlt_1\ldots\StringAlt_n)$ if
  $\forall i, \, \String_i \EquivalenceRelation' \StringAlt_i$.
\item $\RXWithQuotSemanticsOf{\RegexWithQuot_1\Concat\RegexWithQuot_2} =
  (\Language,\EquivalenceRelation_R)$ where, if
  $(\Language_1,\EquivalenceRelation_1) =
  \RXWithQuotSemanticsOf{\RegexWithQuot_1}$ and
  $(\Language_2,\EquivalenceRelation_2) =
  \RXWithQuotSemanticsOf{\RegexWithQuot_2}$ \\
  $\Language = \SetOf{\String_1\String_2
    \SuchThat \String_1 \in \Language_1 \BooleanAnd \String_2 \in \Language_2}$\\
  $\EquivalenceRelation_R$ is the equivalence relation
  generated by the binary relation $R$, where
  $R\,(\String_1\String_2)\,(\StringAlt_1\StringAlt_2)$ if
  $\String_1 \EquivalenceRelation_1 \StringAlt_1 \BooleanAnd
  \String_2 \EquivalenceRelation_2 \StringAlt_2$.
\item $\RXWithQuotSemanticsOf{\RegexWithQuot_1\Or\RegexWithQuot_2} =
  (\Language,\EquivalenceRelation_R)$ where, if
  $(\Language_1,\EquivalenceRelation_1) =
  \RXWithQuotSemanticsOf{\RegexWithQuot_1}$ and
  $(\Language_2,\EquivalenceRelation_2) =
  \RXWithQuotSemanticsOf{\RegexWithQuot_2}$ \\
  $\Language = \SetOf{\String
    \SuchThat \String \in \Language_1 \BooleanOr \String \in \Language_2}$\\
  $\EquivalenceRelation_R$ is the equivalence relation
  generated by the binary relation $R$, where
  $R\,\String\,\StringAlt$ if
  $\String \EquivalenceRelation_1 \StringAlt \BooleanOr
  \String \EquivalenceRelation_2 \StringAlt$.
\item $\RXWithQuotSemanticsOf{\RegexQuotient{\RegexWithQuot}{\RegularEquivalence}} =
  (\Language,\EquivalenceRelation_R)$ where, if
  $(\Language,\EquivalenceRelation_1) =
  \RXWithQuotSemanticsOf{\RegexWithQuot}$ and
  $\EquivalenceRelation_2 = \EquivalenceSemanticsOf{\Language}{\RegularEquivalence})$
  \\
  $\EquivalenceRelation_R$ is the equivalence relation
  generated by the binary relation $R$, where
  $R\,\String\,\StringAlt$ if
  $\String \EquivalenceRelation_1 \StringAlt \BooleanOr
  \String \EquivalenceRelation_2 \StringAlt$.
\end{itemize}

We add the syntactic sugar: $Reorderings(A_1 \ldots A_n)$ for
$(\ldots(((A_1 | A_2) / \LensEquivalenceOf{\Lens_1}) | A_3) /
\LensEquivalenceOf{\Lens_2}
| \ldots | A_n) / \LensEquivalenceOf{\Lens_{n-1}}$ where $\Lens_i$ is the lens
handling the maps between $A_1 \ldots A_i$ and $A_{i+1}$.

Examples:
$\RXWithQuotSemanticsOf{(``abc''|``xyz'')/All} = (\SetOf{``abc'',``xyz''},
``abc'' \cong ``xyz'')$

More involved example:

\begin{lstlisting}
let WSP = (' ' | '\n' | '\t')+ / All;;

let Name = [A-Z][a-z]*;;

let FirstLast = Name (WSP Name)+;;

let LastCommaFirst = Name ',' (WSP Name)+;;

let mapping_between_names = {lens here};;

let FullName = (FirstLast | LastCommaFirst) / mapping_between_names;;

let Month = 'Jan' | ... | 'Dec';;
let Day = '1' | ... | '31';;
let Year = (0-9){4};;

let AmericanBirthday = Month WSP Day ',' WSP YEAR;;
let BritishBirthday = Day WSP Month WSP YEAR;;

let between_birthdays = {lens here};;

let Birthday = (AmericanBirthday | BritishBirthday) / between_birthdays

let NameAndBirthday = Reorderings ((WSP FullName) (WSP Birthday));;
\end{lstlisting}

So \CF{NameAndBirthday} expresses names and birthdays, in either order, where
you can express both the name and the birthday in multiple different ways, and
where varying whitespace is treated as equivalent.

Now that we've done all these fun things on regular expressions with quotients,
we can start writing lenses with quotients!

\pagebreak

\section{Canonizers}

To build up lquot and rquot, we need to first define canonizers.  The syntax for
canonizers is:

\begin{tabular}{l@{\hspace*{5mm}}l@{\ }c@{\ }l@{\hspace*{5mm}}>{\itshape\/}l}
  & \Canonizer{} & \GEq{} & \hspace{1.06em}\ProjectCanonizerOf{\RegexWithQuot}{\String} & \ProjectCanonizerType{} \\
  & & & \GBar{} \IdentityCanonizerOf{\RegexWithQuot} & \IdentityCanonizerType{} \\
  & & & \GBar{} \SquashCanonizerOf{\Lens} & \SquashCanonizerType{} \\
          & & & \GBar{} \IterateCanonizerOf{\Canonizer} & \IterateCanonizerType{} \\
          & & & \GBar{} $\ConcatCanonizerOf{\Canonizer_1}{\Canonizer_2}$ & \ConcatCanonizerType{} \\
          & & & \GBar{} $\OrCanonizerOf{\Canonizer_1}{\Canonizer_2}$ & \OrCanonizerType{} \\
\end{tabular}

We desire a canonizer $\Canonizer$ to encode two functions,
$\CanonizeOf{\Canonizer}$ and $\ChooseOf{\Canonizer}$.  Intuitively, we want
$\CanonizeOf{\Canonizer}$ to be a projection, and $\ChooseOf{\Canonizer}$ to be
merely an inclusion, and we want both of these to be functional.  We define the
$\Canonize$ and $\Choose$ as two relations, and prove that these relations are
functional for
well typed lenses.  Furthermore, we prove that $\ChooseOf{\Canonizer}$ is the
right inverse of $\CanonizeOf{\Canonizer}$ -- in other words
$\CanonizeOf{\Canonizer}\,(\ChooseOf{\Canonizer}\,s) = s$.  We begin by defining
canonize.
\\

$\CanonizeOf{(\cdot)} \OfType \CanonizerType \rightarrow (\StringType
\rightarrow \StringType \rightarrow \BooleanType)$

\begin{itemize}
\item
  $\CanonizeOf{\ProjectCanonizerOf{\RegexWithQuot}{\String}}(\StringAlt,\String)$
  if $\StringAlt \in (\FirstOf{\RXWithQuotSemanticsOf{\RegexWithQuot}})$.
\item $\CanonizeOf{\IdentityCanonizerOf{\RegexWithQuot}}(\String,\String)$
  if $\String \in
  (\FirstOf{\RXWithQuotSemanticsOf{\RegexWithQuot}})$.
\item $\CanonizeOf{\SquashCanonizerOf{\Lens}}(\String,\StringAlt)$
  if $\String = \StringAlt$ or $\Lens.putr(\String,\StringAlt)$.
\item
  $\CanonizeOf{\IterateCanonizerOf{\Canonizer}}(\String_1\ldots\String_n,\StringAlt_1\ldots\StringAlt_n)$
  if $\forall i,\, \CanonizeOf{\Canonizer}(\String_i,\StringAlt_i)$.
\item
  $\CanonizeOf{\ConcatCanonizerOf{\Canonizer_1}{\Canonizer_2}}(\String_1\String_2,\StringAlt_1\StringAlt_2)$
  if $\CanonizeOf{\Canonizer_1}(\String_1,\StringAlt_1) \BooleanAnd
  \CanonizeOf{\Canonizer_2}(\String_2,\StringAlt_2)$.
\item
  $\CanonizeOf{\ConcatCanonizerOf{\Canonizer_1}{\Canonizer_2}}(\String,\StringAlt)$
  if $\CanonizeOf{\Canonizer_1}(\String,\StringAlt) \BooleanOr
  \CanonizeOf{\Canonizer_2}(\String,\StringAlt)$.
\end{itemize}

We can also define $\Choose$ in a similar fashion.
\\

$\ChooseOf{(\cdot)} \OfType \CanonizerType \rightarrow (\StringType
\rightarrow \StringType \rightarrow \BooleanType)$

\begin{itemize}
\item $\ChooseOf{\ProjectCanonizerOf{\Regex}{\String}}(\String,\String)$.
\item $\ChooseOf{\IdentityCanonizerOf{\RegexWithQuot}}(\String,\String)$ where
  if $\String \in (\mathit{fst}\, (\RXWithQuotSemanticsOf{\RegexWithQuot}))$.
\item $\ChooseOf{\SquashCanonizerOf{\Lens}}(\String,\String)$
  if $\exists \StringAlt, \Lens.putr(\StringAlt,\String)$.
\item
  $\ChooseOf{\IterateCanonizerOf{\Canonizer}}(\String_1\ldots\String_n,\StringAlt_1\ldots\StringAlt_n)$
  if $\forall i,\, \ChooseOf{\Canonizer}(\String_i,\StringAlt_i)$.
\item
  $\ChooseOf{\ConcatCanonizerOf{\Canonizer_1}{\Canonizer_2}}(\String_1\String_2,\StringAlt_1\StringAlt_2)$
  if $\ChooseOf{\Canonizer_1}(\String_1,\StringAlt_1) \BooleanAnd
  \ChooseOf{\Canonizer_2}(\String_2,\StringAlt_2)$.
\item
  $\ChooseOf{\ConcatCanonizerOf{\Canonizer_1}{\Canonizer_2}}(\String,\StringAlt)$
  if $\ChooseOf{\Canonizer_1}(\String,\StringAlt) \BooleanOr
  \ChooseOf{\Canonizer_2}(\String,\StringAlt)$.
\end{itemize}

We now define a typing relation of the form $\Canonizer \OfType \RegexWithQuot
\CanonizerArrow \RegexWithQuotAlt$.  If we have derived $\Canonizer \OfType
\RegexWithQuot \CanonizerArrow \RegexWithQuotAlt$, it means that $\Canonizer$'s
$\Canonize$ is a function from the language of $\RegexWithQuot$
to $\RegexWithQuotAlt$.  It also means that $\Canonizer$'s
$\Choose$ is a function from the language of $\RegexWithQuotAlt$
to $\RegexWithQuot$.  Furthermore, a well typed canonizer satisfies
round-tripping laws (choose is the right inverse of canonize up to equivalence).

\begin{mathpar}
  \inferrule[]% \ConstantLensRule{}
  {
    \String\in\RXWithQuotSemanticsOf{\RegexWithQuot}
  }
  {
    \ProjectCanonizerOf{\RegexWithQuot}{\String} \OfType
    \RegexQuotient{\RegexWithQuot}{\All} \CanonizerArrow \String
  }

  \inferrule[]%\IdentityLensRule{}
  {
    \RegexWithQuot \text{ is strongly unambiguous}
  }
  {
    \IdentityCanonizerOf{\RegexWithQuot} \OfType
    \RegexWithQuot \CanonizerArrow \RegexWithQuot
  }

  \inferrule[]%\IdentityLensRule{}
  {
    \Lens \OfType \RegexWithQuot_1 \Leftrightarrow \RegexWithQuot_2\\\\
    \UnambigOrOf{\RegexWithQuot_1}{\RegexWithQuot_2}
  }
  {
    \SquashCanonizerOf{\Lens} \OfType
    ((\RegexWithQuot_1 \Or \RegexWithQuot_2) / \LensEquivalenceOf{\Lens})\,
    \CanonizerArrow \RegexWithQuot_2
  }

  \inferrule[]
  {
    \Canonizer \OfType \RegexWithQuot \CanonizerArrow \RegexWithQuotAlt\\
    \UnambigItOf{\RegexWithQuot}\\
    \UnambigItOf{\RegexWithQuotAlt}
  }
  {
    \IterateCanonizerOf{\Lens} \OfType \StarOf{\RegexWithQuot} \CanonizerArrow \StarOf{\RegexWithQuotAlt}
  }

  \inferrule[]%\ConcatLensRule{}
  {
    \Canonizer_1 \OfType \RegexWithQuot_1 \CanonizerArrow \RegexWithQuotAlt_1\\\\
    \Canonizer_2 \OfType \RegexWithQuot_2 \CanonizerArrow \RegexWithQuotAlt_2\\\\
    \UnambigConcatOf{\RegexWithQuot_1}{\RegexWithQuot_2}\\
    \UnambigConcatOf{\RegexWithQuotAlt_1}{\RegexWithQuotAlt_2}
  }
  {
    \ConcatCanonizerOf{\Canonizer_1}{\Canonizer_2} \OfType
    \RegexWithQuot_1\RegexWithQuot_2 \CanonizerArrow \RegexWithQuotAlt_1\RegexWithQuotAlt_2
  }

  \inferrule[]%\OrLensRule{}
  {
    \Canonizer_1 \OfType \RegexWithQuot_1 \CanonizerArrow \RegexWithQuotAlt_1\\
    \Canonizer_2 \OfType \RegexWithQuot_2 \CanonizerArrow \RegexWithQuotAlt_2\\\\
    \UnambigOrOf{\RegexWithQuot_1}{\RegexWithQuot_2}\\
    \UnambigOrOf{\RegexWithQuotAlt_1}{\RegexWithQuotAlt_2}
  }
  {
    \OrCanonizerOf{\Canonizer_1}{\Canonizer_2} \OfType
    \RegexWithQuot_1 \Or \RegexWithQuot_2
    \CanonizerArrow \RegexWithQuotAlt_1 \Or \RegexWithQuotAlt_2
  }
\end{mathpar}

\section{Lenses}

Lenses with quotients are similar to lenses, but we have an additional
combinator for encorporating canonizers.

\begin{tabular}{l@{\hspace*{5mm}}l@{\ }c@{\ }l@{\hspace*{5mm}}>{\itshape\/}l}
  & \Lens{} & \GEq{} & \hspace{1.06em}\ConstLensOf{\String}{\StringAlt} & \ConstLensType{} \\
  & & & \GBar{} \IdentityLensOf{\RegexWithQuot} & \IdentityLensType{} \\
  & & & \GBar{} \IterateLensOf{\Lens} & \IterateLensType{} \\
  & & & \GBar{} $\ConcatLensOf{\Lens_1}{\Lens_2}$ & \ConcatLensType{} \\
  & & & \GBar{} $\SwapLensOf{\Lens_1}{\Lens_2}$ & \SwapLensType{} \\
  & & & \GBar{} $\OrLens{\Lens_1}{\Lens_2}$ & \OrLensType{} \\
  & & & \GBar{} $\LeftQuotientOf{\Canonizer}{\Lens}$ & \LeftQuotientType\\
  & & & \GBar{} $\RightQuotientOf{\Lens}{\Canonizer}$ & \RightQuotientType
\end{tabular}

The semantics are as before, but with the additional portion where if we perform
an lquot or an rquot, we perform the canonize or choose action on those when we
do the mapping.

The two functions for the semantics are $\Get$ and $\Put$, and are expressed as
$\GetOf{\Lens}$ and $\PutOf{\Lens}$.  These are both binary relations on strings.

Typings of lenses are of the form $\Lens \OfType \Regex \Leftrightarrow
\RegexAlt$, and means that $\Lens$ describes two functions which are inverses of
each other, with domains and codomains of the languages expressed by $\Regex$
and $\RegexAlt$.

\begin{mathpar}
  \inferrule[]% \ConstantLensRule{}
  {
    \String_1 \in \StarOf{\Sigma}\\
    \String_2 \in \StarOf{\Sigma}
  }
  {
    \ConstLensOf{\String_1}{\String_2} \OfType \String_1 \Leftrightarrow \String_2
  }

  \inferrule[]%\IdentityLensRule{}
  {
    \RegexWithQuot \text{ is strongly unambiguous}
  }
  {
    \IdentityLensOf{\RegexWithQuot} \OfType \RegexWithQuot \Leftrightarrow \RegexWithQuot
  }

  \inferrule[]
  {
    \Lens \OfType \RegexWithQuot \Leftrightarrow \RegexWithQuotAlt\\
    \UnambigItOf{\RegexWithQuot}\\
    \UnambigItOf{\RegexWithQuotAlt}
  }
  {
    \IterateLensOf{\Lens} \OfType \StarOf{\RegexWithQuot} \Leftrightarrow \StarOf{\RegexWithQuotAlt}
  }

  \inferrule[]%\ConcatLensRule{}
  {
    \Lens_1 \OfType \RegexWithQuot_1 \Leftrightarrow \RegexWithQuotAlt_1\\\\
    \Lens_2 \OfType \RegexWithQuot_2 \Leftrightarrow \RegexWithQuotAlt_2\\\\
    \UnambigConcatOf{\RegexWithQuot_1}{\RegexWithQuot_2}\\
    \UnambigConcatOf{\RegexWithQuotAlt_1}{\RegexWithQuotAlt_2}
  }
  {
    \ConcatLensOf{\Lens_1}{\Lens_2} \OfType \RegexWithQuot_1\RegexWithQuot_2 \Leftrightarrow \RegexWithQuotAlt_1\RegexWithQuotAlt_2
  }

  \inferrule[]%\SwapLensRule{}
  {
    \Lens_1 \OfType \RegexWithQuot_1 \Leftrightarrow \RegexWithQuotAlt_1\\\\
    \Lens_2 \OfType \RegexWithQuot_2 \Leftrightarrow \RegexWithQuotAlt_2\\\\
    \UnambigConcatOf{\RegexWithQuot_1}{\RegexWithQuot_2}\\
    \UnambigConcatOf{\RegexWithQuotAlt_2}{\RegexWithQuotAlt_1}
  }
  {
    \SwapLensOf{\Lens_1}{\Lens_2} \OfType \RegexWithQuot_1\RegexWithQuot_2 \Leftrightarrow \RegexWithQuotAlt_2\RegexWithQuotAlt_1
  }

  \inferrule[]%\OrLensRule{}
  {
    \Lens_1 \OfType \RegexWithQuot_1 \Leftrightarrow \RegexWithQuotAlt_1\\
    \Lens_2 \OfType \RegexWithQuot_2 \Leftrightarrow \RegexWithQuotAlt_2\\\\
    \UnambigOrOf{\RegexWithQuot_1}{\RegexWithQuot_2}\\
    \UnambigOrOf{\RegexWithQuotAlt_1}{\RegexWithQuotAlt_2}
  }
  {
    \OrLensOf{\Lens_1}{\Lens_2} \OfType
    \RegexWithQuot_1 \Or \RegexWithQuot_2
    \Leftrightarrow \RegexWithQuotAlt_1 \Or \RegexWithQuotAlt_2
  }

  \inferrule[]%\ComposeLensRule{}
  {
    \Lens_1 \OfType \RegexWithQuot_1 \Leftrightarrow \RegexWithQuot_2\\
    \Lens_2 \OfType \RegexWithQuot_2 \Leftrightarrow \RegexWithQuot_3\\
  }
  {
    \Lens_2 ; \Lens_1 \OfType \RegexWithQuot_1 \Leftrightarrow \RegexWithQuot_3
  }

  \inferrule[]%\RewriteRegexLensRule{}
  {
    \Lens \OfType \RegexWithQuot_1 \Leftrightarrow \RegexWithQuot_2\\
    \RegexWithQuot_1 \DefinitionalEquiv \RegexWithQuot_1'\\
    \RegexWithQuot_2 \DefinitionalEquiv \RegexWithQuot_2'
  }
  {
    \Lens \OfType \RegexWithQuot_1' \Leftrightarrow \RegexWithQuot_2'
  }

  \inferrule[]%\RewriteRegexLensRule{}
  {
    \Canonizer \OfType \RegexWithQuot_1' \CanonizerArrow \RegexWithQuot_1\\
    \Lens \OfType \RegexWithQuot_1 \Leftrightarrow \RegexWithQuot_2
  }
  {
    \LeftQuotientOf{\Canonizer}{\Lens} \OfType \RegexWithQuot_1' \Leftrightarrow \RegexWithQuot_2
  }

  \inferrule[]%\RewriteRegexLensRule{}
  {
    \Lens \OfType \RegexWithQuot_1 \Leftrightarrow \RegexWithQuot_2\\
    \Canonizer \OfType \RegexWithQuot_2' \CanonizerArrow \RegexWithQuot_2
  }
  {
    \RightQuotientOf{\Canonizer}{\Lens} \OfType \RegexWithQuot_1 \Leftrightarrow \RegexWithQuot_2'
  }
\end{mathpar}

\end{document}
