\newif\ifdraft\drafttrue  % set true to show comments
\newif\ifplentyoftime\plentyoftimefalse  % :-)

% % For double-blind review submission, w/o CCS and ACM Reference (max
% submission space)
\documentclass[acmsmall,review,anonymous]{acmart}\settopmatter{printfolios=true,printccs=false,printacmref=false}
% % For double-blind review submission, w/ CCS and ACM Reference
% \documentclass[acmsmall,review,anonymous]{acmart}\settopmatter{printfolios=true}
% % For single-blind review submission, w/o CCS and ACM Reference (max
% submission space)
% \documentclass[acmsmall,review]{acmart}\settopmatter{printfolios=true,printccs=false,printacmref=false}
% % For single-blind review submission, w/ CCS and ACM Reference
% \documentclass[acmsmall,review]{acmart}\settopmatter{printfolios=true} % For
% final camera-ready submission, w/ required CCS and ACM Reference
% \documentclass[acmsmall]{acmart}\settopmatter{}


% % Journal information % Supplied to authors by publisher for camera-ready
% submission; % use defaults for review submission.
\acmJournal{PACMPL}
\acmVolume{1}
\acmNumber{POPL} % CONF = POPL or ICFP or OOPSLA
\acmArticle{1}
\acmYear{2018}
\acmMonth{1}
\acmDOI{} % \acmDOI{10.1145/nnnnnnn.nnnnnnn}
\startPage{1}

% % Copyright information % Supplied to authors (based on authors' rights
% management selection; % see authors.acm.org) by publisher for camera-ready
% submission; % use 'none' for review submission.
\setcopyright{none}
% \setcopyright{acmcopyright} \setcopyright{acmlicensed}
% \setcopyright{rightsretained} \ccopyrightyear{2018}           %% If different
% from \acmYear

% % Bibliography style %\bibliographystyle{ACM-Reference-Format} % Citation
% style % Note: author/year citations are required for papers published as an %
% issue of PACMPL.
% %\citestyle{acmauthoryear}   %% For author/year citations


% %%%%%%%%%%%%%%%%%%%%%%%%%%%%%%%%%%%%%%%%%%%%%%%%%%%%%%%%%%%%%%%%%%%%% % Note:
% Authors migrating a paper from PACMPL format to traditional % SIGPLAN
% proceedings format must update the '\documentclass' and % topmatter commands
% above; see 'acmart-sigplanproc-template.tex'.
% %%%%%%%%%%%%%%%%%%%%%%%%%%%%%%%%%%%%%%%%%%%%%%%%%%%%%%%%%%%%%%%%%%%%%


\usepackage{amsmath, amssymb, amsthm, enumerate, array, extarrows, mathrsfs,
mathtools, stmaryrd, listings, xspace, bussproofs}
\theoremstyle{definition}
\usepackage[capitalize]{cleveref}
\newtheorem{definition}{Definition}
\newtheorem{example}{Example}
\newtheorem{conjecture}{Conjecture}
\newtheorem{question}{Question}
\newtheorem{fact}{Fact}
\newtheorem{claim}{Claim}
\newtheorem{remark}{Remark}

% Macros Colors
\definecolor{dkblue}{rgb}{0,0.1,0.5}
\definecolor{dkgreen}{rgb}{0,0.6,0}
\definecolor{dkred}{rgb}{0.6,0,0}
\definecolor{dkpurple}{rgb}{0.7,0,0.4}
\definecolor{olive}{rgb}{0.4, 0.4, 0.0}
\definecolor{teal}{rgb}{0.0,0.5,0.5}
\definecolor{orange}{rgb}{0.9,0.6,0.2}
\definecolor{lightyellow}{RGB}{255, 255, 179}
\definecolor{lightgreen}{RGB}{170, 255, 220}
\definecolor{teal}{RGB}{141,211,199}
\definecolor{darkbrown}{RGB}{121,37,0}


\newcommand{\FINISH}[3]{\ifdraft\textcolor{#1}{[#2: #3]}\fi}
\newcommand{\bcp}[1]{\FINISH{dkred}{B}{#1}}
\newcommand{\BCP}[1]{\FINISH{dkred}{B}{\bf #1}}
\newcommand{\afm}[1]{\FINISH{dkgreen}{A}{#1}}
\newcommand{\dpw}[1]{\FINISH{dkblue}{D}{#1}} % Toronto Maple Leafs Blue :-)
\newcommand{\saz}[1]{\FINISH{orange}{SZ}{#1}}
\newcommand{\ksf}[1]{\FINISH{teal}{K}{#1}}
\newcommand{\sam}[1]{\FINISH{dkpurple}{SM}{#1}}


\newcommand{\kw}[1]{\ensuremath{\mathsf{#1}}\xspace}
\newcommand{\get}{\ensuremath{\kw{get}}\xspace}
\newcommand{\pput}{\ensuremath{\kw{put}}\xspace}
\newcommand{\create}{\ensuremath{\kw{create}}\xspace}
\newcommand{\res}{\ensuremath{\kw{res}}\xspace}
\newcommand{\ccopy}{\ensuremath{\kw{copy}}\xspace}
\newcommand{\const}{\ensuremath{\kw{const}}\xspace}
\newcommand{\default}{\ensuremath{\kw{default}}\xspace}
\newcommand{\length}{\ensuremath{\kw{length}}\xspace}
\newcommand{\match}{\ensuremath{\kw{match}}\xspace}

\DeclareMathOperator*{\argmin}{arg\,min} % thin space, limits underneath in
\DeclareMathOperator*{\argmax}{argmax} % no space, limits underneath in


\lstset{ basicstyle=\ttfamily, escapeinside=|| }
\begin{document}

% % Title information
\title{Lenses with a Difference}         %% [Short Title] is optional;
% % when present, will be used in % header instead of Full Title.
% %\titlenote{with title note}             %% \titlenote is optional; % can be
% repeated if necessary; % contents suppressed with 'anonymous'
% %\subtitle{Subtitle}                     %% \subtitle is optional
% %\subtitlenote{with subtitle note}       %% \subtitlenote is optional; % can
% be repeated if necessary; % contents suppressed with 'anonymous'


% % Author information % Contents and number of authors suppressed with
% 'anonymous'.
% % Each author should be introduced by \author, followed by % \authornote
% (optional), \orcid (optional), \affiliation, and % \email.
% % An author may have multiple affiliations and/or emails; repeat the %
% appropriate command.
% % Many elements are not rendered, but should be provided for metadata %
% extraction tools.

% % Author with single affiliation.
% \author{First1 Last1} \authornote{with author1 note}          %% \authornote
% is optional; % can be repeated if necessary \orcid{nnnn-nnnn-nnnn-nnnn} %%
% \orcid is optional \affiliation{ \positition{Position1}
% \department{Department1} %% \department is recommended
% \institution{Institution1} %% \institution is required \streetaddress{Street1
% Address1} \city{City1} \state{State1} \posittcode{Post-Code1}
% \country{Country1}                    %% \country is recommended }
% \email{first1.last1@inst1.edu}          %% \email is recommended

\author{Solomon Maina}
\position{PhD Student}
\department{Computer Science}              %% \department is recommended
\institution{University of Pennsylvania}            %% \institution is required
\country{USA}                    %% \country is recommended }
\email{smaina@seas.upenn.edu}          %% \email is recommended

% % Abstract % Note: \begin{abstract}...\end{abstract} environment must come %
% before \maketitle command
\begin{abstract}
Many of the bidirectional (bx) programming languages that exist today focus on
restoring consistency between source and target data. Consequently, these languages often provide two guarantees: (1) that a transformation restores consistency when an
update is made to source or target data, and (2) that trivial updates to a the source result to trivial updates to the target and vice versa. Unfortunately, in situations
where there are many different ways of restoring consistency, these languages
do not constrain how a transformation chooses to do so. As a result, the transformations derived using these languages can lead to unexpected results. In this paper we present a formalization for bx transformations that requires a bx to provide guarantees on how it propagates updates. We will then describe a specific bx language which we call {\em difference lenses}; this language is based on the {\em basic lenses} defined in the bx programming language {\em Boomerang}, but each difference lens comes with explicit guarantees on update propagation. We will then express various lens programming languages as difference lenses, and hence describe non-trivial guarantees on update propagation that these languages provide as well.
\end{abstract}

%% Keywords
%% comma separated list
\keywords{bidirectional programming, lenses}  %% \keywords
% are mandatory in final camera-ready submission

%% \maketitle
%% Note: \maketitle command must come after title commands, author
%% commands, abstract environment, Computing Classification System
%% environment and commands, and keywords command.
\maketitle

\section{Introduction}
Many of the bidirectional (bx) programming languages that exist today focus on
restoring consistency between source and target data. Consequently, these languages often provide two guarantees: (1) a {\em correctness} guarantee that requires that a transformation restores consistency when an update is made to source or target data, and (2) a {\em hippocraticness} guarantee that requires that trivial updates to source data result is trivial updates to target data and vice versa. Unfortunately, in situations
where there are many different ways of restoring consistency, these languages
do not constrain how a transformation chooses to do so. As a result, the transformations derived using these languages can lead to unexpected results.

For example, assume that a source consists of a list of pairs of names, and that the target consists of a list of single names. Assume that $\ell$ is a bx which in the forward direction (i.e. the \get direction) simply discards the first of the two names:
$$\ell.\get \; [``Ada \; Lovelace", \; ``Alan \; Turing"] = [``Lovelace", \; ``Turing"]$$
If the order of the names in the target is reversed then a user might expect that in the backward direction (i.e. the \pput direction), the lens would simply restore the original names:
$$\ell.\pput \; [``Ada \; Lovelace", \; ``Alan \; Turing"] \; [``Turing", \; ``Lovelace"] = [``Ada \; Lovelace", \; ``Alan \; Turing"]$$
However, $\ell$ may not perform this alignment so that instead, the following result occurs:
$$\ell.\pput \; [``Ada \; Lovelace", \; ``Alan \; Turing"] \; [``Turing", \; ``Lovelace"] = [``Ada \; Turing", \; ``Alan \; Lovelace"]$$
Consider another example where both a source and a target consists of a list where every element on the list is a name and a year. Assume that $\ell$ is a bx which in the \get direction returns all elements of the source list that have year greater than or equal to 1900:
$$\ell.\get \; [``Ada \; Lovelace, \; 1815", \; ``Alan \; Turing, \; 1912"] = [``Alan \; Turing, \; 1912"]$$
Suppose that the entry $``Ada \; Lovelace, \; 1915"$ is then added to the target list. Should $\ell.\pput$ replace the old entry containing the name $``Ada \; Lovelace"$,
\begin{align*}
&\ell.\pput \; [``Ada \; Lovelace, \; 1815", \; ``Alan \; Turing, \; 1912"] \; [``Alan \; Turing, \; 1912", \; ``Ada \; Lovelace, \; 1915"] \\
&= [``Ada \; Lovelace, \; 1915", \; ``Alan \; Turing, \; 1912"]
\end{align*}
or should the \pput application retain both versions of $``Ada \; Lovelace"$?
\begin{align*}
&\ell.\pput \; [``Ada \; Lovelace, \; 1815", \; ``Alan \; Turing, \; 1912"] \; [``Alan \; Turing, \; 1912", \; ``Ada \; Lovelace, \; 1915"] \\
&= [``Ada \; Lovelace, \; 1815", \; ``Alan \; Turing, \; 1912", \; ``Ada \; Lovelace, \; 1915"]
\end{align*}
In both of these examples, there are multiple ways to restore consistency---that is, there are multiple ways to define $\ell.\pput \; s \; v$ in such a way that $\ell.\get \; (\ell.\pput \; s \; v) = v$---and the results from the different \pput applications may be vastly different from what a user expects. 

Various solutions have been proposed to address the issue of propagating updates in a ``minimal'' way, for example {\em matching lenses} that deal with the issue of alignment \cite{barbosa2010matching}, {\em edit lenses} \cite{hofmann2012edit} and {\em delta lenses} \cite{diskin2011asymmetric,diskin2011state,pacheco2012delta}
that explicitly propagate edits through the lens, and {\em least change
lenses} \cite{macedo2013composing} that require that the source value resulting from a \pput application is not only acceptable, but also one of closest to the original $s$ among the sources that share the same view $v$.

We take the approach that a bidirectional programming language should have explicit guarantees on how programs written in that language propagate updates; ideally these guarantees should be strictly stronger than hippocraticness. This way a programmer who derives bx transformations in this language has a clearer understanding of the behaviour of the programs that she writes. 

In this paper we will show how this can be done, by describing a formalization of bx transformations that explicitly accounts for update propagation, and presenting a series of examples showing how this principle can be applied to existing bx programming languages. More specifically, the major technical contributions that we will make in this paper are the following:
\begin{enumerate}
\item
We present a formalization for bx transformations that requires a bx to provide guarantees on how it propagates updates. 
\item
We describe a specific bx language which we call {\em difference lenses}; this language is based on the {\em basic lenses} defined in the bx programming language {\em Boomerang}\cite{foster2009bidirectional}, but each difference lens comes with explicit guarantees on update propagation. As a result of this work, we will describe non-trivial update propagation properties that Boomerang's basic lenses provide.
  \item
  We show how to represent matching lenses and  relational lenses \cite{bohannon2006relational} as difference lenses, and hence describe non-trivial guarantees that these lens languages provide as well.
\end{enumerate}
\iffalse
 we describe a compositional language of lenses called {\em difference lenses}, where each difference $\ell$ satisfies the following equation,
\begin{equation}\label{difflenslaw}
\delta_S(s, \ell.\pput \; s \; v) = \delta_V(\ell.\get \; s, v)
\end{equation}
for appropriate real-valued $\delta_S : S \times S \longrightarrow \mathbb{R}^{\geq 0}$ and $\delta_S : S \times S \longrightarrow \mathbb{R}^{\geq 0}$.

\cref{difflenslaw}, which already appears in Gibbons et. al. \cite{gibbons2017principles} as the principle of {\em Weak Least Surprise}, is simple and intuitive but presents a challenge when trying to develop a compositional lens langauge that satisfies it. For example, assume that a source consists of a name-and-date pairs while a view consists of a name. Assume that a lens $\ell$ simply discards dates in the \get direction:
\begin{lstlisting}[mathescape=true]
$\ell.\get$ "Ada Lovelace, 12-10-1815" = "Ada Lovelace"
\end{lstlisting}
Consider the iteration lens $\ell^*$, where the source is now a list of name-and-date pairs while a view consists of a list of names. Whenever the length of a view list is larger than the length of a source list, then a \pput application of $\ell^*$ must initialize a source from a view from scratch:
\begin{lstlisting}[mathescape=true]
$\ell.\pput$ [] ["Ada Lovelace"] = ["Ada Lovelace, 00-00-0000"]
\end{lstlisting}
Now if we are to define $\delta_S^*$ and $\delta_V^*$ inductively on source and view lists respectively in such a way that \cref{difflenslaw} holds of $\ell^*$, then the following equation would have to hold:
\begin{lstlisting}[mathescape=true]
$\delta_S^*$(["Ada Lovelace, 00-00-0000"], []) = $\delta_V^*$(["Ada Lovelace"], [])
\end{lstlisting}
In a case such as this when a source is {\em created} from a view as opposed to arising from a view being \pput into an existing source, the source delta will be defined so as to defer to the view delta by mapping the created source back into the view; in particular, for the lens $\ell^*$ in the example, then we have
\begin{lstlisting}[mathescape=true]
$\delta_S^*$(["Ada Lovelace, 00-00-0000"], [])
= $\delta_V^*$([$\ell.\get$ "Ada Lovelace, 00-00-0000"], [])
= $\delta_V^*$(["Ada Lovelace"], [])
\end{lstlisting}

By now there exist many programming languages that enable one to define {\em
bidirectional transformations}, programs that can be run in two
directions. Each of these languages provides guarantees on how the two
transformations interact e.g. that they are bijective, that a round-trip on
one or both directions does not cause non-trivial updates etc. Many of these
guarantees have to do with restoring or preserving consistency. To formalize
this statement, we describe a widely used formulation of bidirectional
transformations due to Stevens \cite{stevens2010bidirectional} but originally
articulated in a more general form by Meertens \cite{meertens1998designing}:
\begin{definition}
A bidirectional transformation (bx) $R: M \Leftrightarrow N$ comprises of a
consistency relation $R \subseteq M \times N$ and restorers $\overrightarrow{R}
: M \times N \rightarrow N$ and $\overleftarrow{R} : M \times N \longrightarrow
N$.
\end{definition}
\begin{definition}
A bx $R : M \Leftrightarrow N$ is {\em correct} if the consistency restorers do
restore consistency: that is, for all $m \in M, n \in N$, we have $R(m,
\overrightarrow{R}(m, n))$, and dually. $R$ is {\em hippocratic} if the
consistency restorers make no change to an already-consistent pair of models:
that is, for all $m \in M$, $n \in N$, if $R(m, n)$ then $\overrightarrow{R}(m,
n) = n$ and dually. $R$ is {\em well behaved} if it is correct and hippocratic.
\end{definition}

Many bidirectional programming languages thus enable one to define
well behaved bx's: consistency restoration is guaranteed by correctness of the
bx while consistency preservation is guaranteed by the hippocraticness of the
bx. This paper is primarily concerned with well behaved bx's where
$\overrightarrow{R}$, the \get function of the bx, ignores it's $N$-argument and
where the consistency relation $R$ is defined by $R(m, \overrightarrow{R}(m))$
for all $m \in M$--- these transformations are the well known and widely
understood {\em lenses} of Foster et al \cite{foster2007combinators}--- though
the ideas we develop could be used in other bidirectional settings as well. 

When the consistency relation of a bx $R$ is very fine, i.e. when each $m \in M$
is consistent with only a few $n \in N$ and vice versa, then a well behaved lens
gives fairly strong guarantees on $\overleftarrow{R}$ and $\overrightarrow{R}$.
However, there are many cases that arise in practice where there are many
different ways to restore consistency; when this happens, well-behavedness may
be too weak to rule out ``bad'' behaviour or to clarify the behaviour of a bx in
subtle cases. Various solutions have been proposed to address this issue, for
example {\em matching lenses} that deal with the issue of alignment
\cite{barbosa2010matching}, {\em edit lenses} \cite{hofmann2012edit} and {\em
delta lenses} \cite{diskin2011asymmetric,diskin2011state,pacheco2012delta}
that explicitly propagate edits through the lens, and {\em least change
lenses} that require that the source value resulting from a \pput application
is not only acceptable, but also one of closest to the original $s$ among the
sources that share the same view $v$ \cite{macedo2013composing}.

This paper takes a different approach to specifying update propagation by
identifying a condition called {\em $\partial D$ preservation} which we describe
below:
\begin{definition}
Let $R : M \Leftrightarrow N$ be a bidirectional transformation and $\partial D
\subseteq (M \times M) \times (N \times N)$. Then $R$ {\em preserves $\partial
D$} if for all $m \in M$, $n, n' \in N$ , if $R(m, n)$ then $\partial D((m,
\overleftarrow{R}(m, n')), (\overrightarrow{R}(m, n), n'))$. $R$ {\em
strongly preserves $\partial D$} if the dual condition also holds.
\end{definition}
\noindent The idea behind $\partial D$ preservation is that if $m$ and $n$ are
already consistent and $\overrightarrow{R}(m,n)$ is updated to produce $n'$,
then the difference between $\overrightarrow{R}(m,n)$ and $n'$ is related
to the difference between $m$ and $\overleftarrow{R}(m,n')$. $R$
strongly preserves $\partial D$ if it preserves $\partial D$ in both directions.

For lenses, $\partial D$ preservation says that difference between $\ell.\get \;
s$ and $v$ is related to the difference between $s$ and $\ell.\pput \; s \;
v$. One way to formalize this idea is to model the difference between two sources $s, s'
\in S$ and two views $v, v' \in V$ as functions $\delta_S : S \times S
\longrightarrow \partial S$ and $\delta_V : V \times V \longrightarrow \partial
V$ for some sets $\partial S$ and $\partial V$, and to capture the
relationship between $\delta_S$ and $\delta_V$ by a lens. In
particular, given a lens $\partial(S, V) : \partial S \Leftrightarrow \partial
V$ we will ask if the following equation holds of $\ell$:
\begin{equation}\label{difflenslaw}
\partial(S, V).\get \; (\delta_S(s, \ell.\pput \; s \; v)) = \delta_V(\ell.\get
\; s, v)
\end{equation}
If $\ell : S \Leftrightarrow V$ satisfies \cref{difflenslaw} we say that 
$\ell$ is a {\em difference  lens} with respect to $(\delta_S, 
\partial(S, V), \delta_V)$, which we denote with the typing relation 
$\ell : S \xLeftrightarrow{\delta_S, \partial(S, V), \delta_V} V$.

For example, consider two versions of an iteration lens $\ell_1 : S^*
\Leftrightarrow V^*$ and $\ell_2 : S^* \Leftrightarrow V^*$ defined inductively
from a lens $\ell : S \Leftrightarrow V$. Both of these lens have the same get
behaviour: given a list $s_1, \ldots, s_n$ of sources, they both produce
$\ell.\get \; s_1, \ldots, \ell.\get \; s_n$. The \pput behaviour of the two
lenses is different though: $\ell_1$ simply aligns views to sources
positionally, while $\ell_2$ computes the optimal alignment on the view side and
uses this alignment to put views back into sources.

Let \lstinline|name = [A-Z] . [a-z]*|, let the source $S$ equal 
\lstinline|name . " " . name| and the view $V$ equal \lstinline|name|.
Let $\ell : S \Leftrightarrow V$ be the lens
\lstinline|del name . del " " . copy name| which given a pair of names produces
the second name in the \get direction, and restores the first name in the \pput
direction:
$$\ell.\get \; ``Ada \; Turing" = ``Turing" \text{ and }\; \ell.\pput \; ``Ada
\; Turing" \; ``Lovelace" = ``Ada \; Lovelace"$$
Define the
function $\delta_V : V \times V \longrightarrow \mathbb{N}^{\geq 0}$ by 
$$\delta_V(n, n') = 
\begin{cases}
0 & \text{if }n=n'\\
1 & \text{otherwise}
\end{cases}$$
Define $\delta_S : S \times S \longrightarrow \mathbb{N}^{\geq 0}$ by
$$\delta_S(m \cdot `` \quad " \cdot n, m' \cdot `` \quad " \cdot n') =
\begin{cases}
0 & \text{if } m = m', n = n'\\
1 & \text{if } m = m', n \neq n'\\
1 & \text{if } m \neq m', n = n'\\
2 & \text{if } m \neq m', n \neq n'
\end{cases}$$
Let the lens $\partial(S, V) : \mathbb{N}^{\geq 0} \Leftrightarrow
\mathbb{N}^{\geq 0}$ be given by $\partial(S, V) = \ccopy(\mathbb{N}^{\geq
0})$. Then $\ell : S \xLeftrightarrow{\delta_S, \partial(S, V), \delta_V} V$.

Now let $xs, ys$ be two lists of single names, with $length(xs) \geq length(ys)$
Define $\delta_V^* : V^* \times V^* \longrightarrow \mathbb{Z}^* \times V^*$ on $xs$ and
$ys$ by aligning elements of $xs$ maximally with elements of $ys$ and then using $\delta_V$
on these pairs to produce a list $ds$ of numbers. In this process, some elements of
$ys$ will not be aligned if $length(xs) < length(ys)$, in which case we simply collect
these in another list of elements of $V$. For instance,
$$\delta_V^*([``Lovelace", ``Kleene", ``Turing"], [``Turing", ``Lovelace"]) = ([0,0], [``Kleene"])$$
since we align \lstinline|"Lovelace"| to \lstinline|"Lovelace"| and 
\lstinline|"Turing"| to \lstinline|"Turing"| for a difference of 0 in each case, which leaves
\lstinline|"Kleene"| unaligned.

On the source side, let $ss, ts$ be two lists of two names with $length(ss) \geq length(ts)$.
Define $\delta_{S, \ell}^* : S^* \times S^* \longrightarrow \mathbb{Z}^* \times S^*$ by taking the 
alignment used to compute $\delta_V^*(\ell^*.\get \; ss, \ell^*.\get \; ts)$ and then using that
alignment in the same manner that $\delta_V^*$ does. For instance
\begin{align*}
&\delta_{S, \ell}^*([``Ada \; Lovelace", ``Stephen \; Kleene", ``Alan \; Turing"], 
[``Alan \; Turing", ``Ada \; Lovelace"])\\
&= ([0,0], [``Stephen \; Kleene"])
\end{align*}
Recall that the \pput behaviour of $\ell_1$ simply aligns views to sources
positionally, while $\ell_2$ computes the optimal alignment on the view side and
uses this alignment to put views back into sources. From this we can prove that
$$\ell_2 : S^* \xLeftrightarrow{\delta_{S, \ell}^*, \partial(S, V)^* \times \ell^*, \delta_V^*} V^*$$
On the other hand, the judgement $\ell_1 : S^* \xLeftrightarrow{\delta_{S, \ell}^*, 
\partial(S, V)^* \times \ell^*, \delta_V^*} V^*$ does not holds since $\ell_1$ does not satisfy \cref{difflenslaw} :
$$(\partial(S, V)^* \times \ell^*).\get \; (\delta_{S, \ell}^*(s, \ell.\pput \; s \; v)) = \delta_V^*(\ell.\get
\; s, v)
$$
This is witnessed by the fact that if $s = [``Ada \; Lovelace", ``Stephen \; Kleene", ``Alan \; Turing"]$ and \\
$v = [``Kleene", ``Turing", ``Lovelace"]$, then 
\begin{align*}
&\ell_1.\pput \; [``Ada \; Lovelace", ``Stephen \; Kleene", ``Alan \; Turing"] \; [``Kleene", ``Turing", ``Lovelace"]\\
&= [``Ada \; Kleene", ``Stephen \; Turing", ``Alan \; Lovelace"]
\end{align*}
Also, $\delta_{S, \ell}^*(s, \ell.\pput \; s \; v) = ([1,1,1], [])$, $\delta_V^*(\ell.\get
\; s, v) = ([0,0,0], [])$, but
\begin{align*}
(\partial(S, V)^* \times \ell^*).\get \; (\delta_{S, \ell}^*(s, \ell.\pput \; s \; v)) &=
(\partial(S, V)^* \times \ell^*).\get \; ([1,1,1], [])\\
&= ([1,1,1], [])\\
&\neq ([0,0,0], [])\\
&= \delta_V^*(\ell.\get \; s, v)
\end{align*}
Here the positional lens $\ell_1$ is not a difference lens with respect to $(\delta_{S, \ell}^*, 
\partial(S, V)^* \times \ell^*, \delta_V^*)$ since it does not align sources to views in the same way that
$\delta_V^*$ does. On the other hand, $\ell_2$ does, hence is a difference lens with respect to $(\delta_{S, \ell}^*, 
\partial(S, V)^* \times \ell^*, \delta_V^*)$. 

In the rest of this paper, we give concrete examples of how to use $\partial D$ preservation to characterize 
the guarantees on update propagation provided by various lens programming languages. In particular,
the major technical contributions that we will make in this paper are the following:
\begin{enumerate}
  \item
  We formalize a condition called $\partial D$ preservation which provides a way
  of constraining the updates that can be made by a bidirectional
  transformation. We give examples of relations $\partial D$ as well as
  examples and counterexamples of lenses that are $\partial D$ preserving.
  \item
  We describe an instance of $\partial D$ preservation that is satisfied by {\em difference lenses}. We give a set of combinators for {\em difference lenses} and describe the update propagation guarantees that the lenses derived using these combinators give.
  \item
  Finally, we show how to represent matching lenses and relational lenses
  \cite{bohannon2006relational} as difference lenses, and hence
  describe non-trivial guarantees beyond correctness and hippocraticness that
  these lens languages provide as well.
\end{enumerate}
\fi
\section{Laying Down BX Laws}
\begin{definition}
A bidirectional transformation (bx) $R : M \Leftrightarrow N$ comprises of 
\begin{enumerate}
\item
a consistency relation $R \subseteq M \times N$,
\item
a relation $S \subseteq (M \times M) \times (N \times N)$ such that for all $m, m' \in M$, $n, n' \in N$,
\begin{enumerate}
\item
if $(m, m') \stackrel{S}{\sim} (n, n)$ then $m = m'$,
\item
if $(m, m) \stackrel{S}{\sim} (n, n')$ then $n = n'$, and
\end{enumerate}
\item
restorers $\overrightarrow{R} : M \times N \rightarrow N$ and $\overleftarrow{R} : M \times N \longrightarrow
N$.
\end{enumerate}
\end{definition}
Intuitively, the relation $S$ relates differences on the $M$ side to differences on the $N$ side. For instance, $S$ could be the relation $(m, m') \stackrel{S}{\sim} (n,n')$ if and only if $\delta_M(m,m') = \delta_N(n,n')$ for some metrics $\delta_M$ and $\delta_N$ on $M$ and $N$ respectively. The relation is required to send only trivial deltas to trivial deltas.
\begin{definition}
A bx $R : M \Leftrightarrow N$ is {\em $S$-preserving} for all $m \in M$, $n, n' \in N$, then if $R(m, n)$ then $(m, \overleftarrow{R}(m,n')) \stackrel{S}{\sim} (n, n')$
\end{definition}
Intuitively $S$-preservation says that if we start in a situation where $m$ is consistent with $n$, update $n$ to get $n'$, then map $n'$ back into $m$ via $\overleftarrow{R}(m,n')$, then the difference between $n$ and $n'$ is related to the difference between $m$ and $\overleftarrow{R}(m,n')$.
\begin{definition}
A bx $R : M \Leftrightarrow N$ is hippocratic if for all $m \in M$, $n \in N$, if $R(m, n)$ then $\overrightarrow{R}(m, n) = n$ and $\overleftarrow{R}(m,n) = m$.
\end{definition}
\begin{fact}
For any $R : N \Leftrightarrow N$ and any $S \subseteq (M \times M) \times (N \times N)$, if $R$ is $S$-preserving, then $R$ is hippocratic.
\end{fact}
\begin{definition}
A bx $R : M \Leftrightarrow N$ is {\em correct} if the consistency restorers do
restore consistency: that is, for all $m \in M, n \in N$, we have $R(m,
\overrightarrow{R}(m, n))$, and dually.
\end{definition}
\iffalse
A {\em lens} is a program that can be run in two directions. In the
forward direction, the \get component of a lens produces a {\em view}
from a {\em source}, while in the backward direction, the \pput component folds
an updated view back into the original source. Formally, Foster
\cite{foster2007combinators} defined a lens $\ell : S
\Leftrightarrow V$ from a source $S$ to a view $V$ to be a pair of functions
$\ell.\get : S \longrightarrow V$ and $\ell.\pput : S \longrightarrow V
\longrightarrow S$ satisfying the
following properties:
\begin{align*}
\ell.\get \; (\ell.\pput \; s \; v) &= v \tag{PUTGET}\\
\ell.\pput \; s \; (\ell.\get \; s) &= s \tag{GETPUT}
\end{align*}
The PUTGET law ensures that updates to the view are translated
``exactly''---i.e. that, given a view, the \pput function produces a source
that \get maps back to the very same view, while the GETPUT law ensures a
``stability'' property for the source---i.e. it requires that the \pput
function returns the original source unmodified whenever the update to the view
is a no-op.

Unfortunately, while these laws give some guarantees on how the \get and \pput
components of a lens interact, they do not specify the behaviour of the \pput
component when the update to the view is non-trivial. Consequently, a
non-trivial update to a view could lead to an unexpected update to the source.
Various solutions have been proposed to address this issue, for example {\em
matching lenses} to deal with the issue of alignment \cite{barbosa2010matching},
{\em edit lenses} \cite{hofmann2012edit} and {\em delta lenses}
\cite{diskin2011asymmetric,diskin2011state,pacheco2012delta} which explicitly
propagate edits through the lens, and {\em least change lenses} which
add a new LEASTPUTGET law which requires that the source value resulting from a
\pput application is not only acceptable, but also one of closest to the
original $s$ among the sources that share the same view $v$
\cite{macedo2013composing}.

This paper describes a general method for specifying what updates are allowable
for a given lens without requiring that the lens explicitly propagate updates.
Our approach is guided by the simple idea the idea that if one starts with a
source $s$, calls the \get function on $s$ to acquire the view $\ell.\get \;
s$, updates $\ell.\get \; s$ to get a new view $v$, then the ``difference''
between $v$ and $\ell.\get \; s$ and $v$ should be related to the difference
between $\ell.\pput \; s \; v$ and $v$.

To formalize this idea, we first recall a widely used definition of
bidirectional transformations
\cite{meertens1998designing,stevens2010bidirectional} definition of a
bidirectional transformations:
\begin{definition}
A bidirectional transformation (bx) $R: M \Leftrightarrow N$ comprises of a
consistency relation $R \subseteq M \times N$ and restorers $\overrightarrow{R}
: M \times N \rightarrow N$ and $\overleftarrow{R} : M \times N \longrightarrow
N$.
\end{definition}
\begin{definition}
A bx $R : M \leftrightarrow N$ is {\em correct} if the consistency restorers do
restore consistency: that is, for all $m \in M, n \in N$, we have $R(m,
\overrightarrow{R}(m, n))$, and dually. $R$ is {\em hippocratic} if the
consistency restores make no change to and already-consistent pair of models:
that is, for all $m \in M$, $n \in N$, if $R(m, n)$ then $\overrightarrow{R}(m,
n) = n$ and dually. $R$ is {\em well behaved} if it is correct and hippocratic.
\end{definition}

Observe that $\ell : S \Leftrightarrow V$ is a lens, let $R$ be the
consistency relation given by $R(s, \ell.\get \; s)$ for all $s \in
S$, $\overrightarrow{\ell}(s, v) = \ell.\get \; s$, and $\overleftarrow{\ell} =
\ell.\pput$. By the PUTGET law, $\ell.\get \; (\ell.\pput \; s \; v) = v$,
hence $\ell$ is correct, and by the GETPUT laws, $\ell.\pput \; s \; (\ell.\get
\; s) = s$ and $\ell.\get \; (\ell.\create \; v) = v$, hence $\ell$ is
hippocractic. Therefore $\ell$ is a well-behaved bx.

If $R$ is a well-behaved bx then $R$ restores consistency when a source or view
is updated (correctness) and preserves consistency when the source and view are
already consistent (hippocraticness). In order to express

We will then apply this method to define a typing judgement for lenses which
constrains how a lens propagates updates. We will describe a compositional
language of {\em difference lenses} which satisfy this new typing judgement,
and give examples of ordinary lenses that are not difference lenses. Finally,
we will show how to represent matching lenses and relational lenses
\cite{bohannon2006relational} as difference lenses, and hence describe the
guarantees that these lens languages provide.


For example suppose that the source is an XML document representing the
names, dates and nationalities of a collection of classical music
composers such as,

\begin{lstlisting}
<composers>
<composer>
<name>Jean Sibelius</name>
<dates>1865-1956</dates>
<nationality>Finnish</nationality>
</composer>
<composer>
<name>Aaron Copland</name>
<dates>1910-1990</dates>
<nationality>American</nationality>
</composer>
<composer>
<name>Benjamin Briten</name>
<dates>1913-1976</dates>
<nationality>English</nationality>
</composer>
</composers>
\end{lstlisting}

while the view is a list if ASCII text representing the names and dates of each
composer:
\begin{lstlisting}
Jean Sibelius, 1865-1956
Aaron Copland, 1910-1990
Benjamin Briten, 1913-1976
\end{lstlisting}

After computing the initial view, we might want to edit it in some way---e.g.,
correcting the error in Sibelius's death date and the misspelling in Britten's
name
\begin{lstlisting}
Jean Sibelius, 1865-|\colorbox{orange}{1957}|
Aaron Copland, 1910-1990
Benjamin |\colorbox{orange}{Britten}|, 1913-1976
\end{lstlisting}
and push the changes back into the original XML format:
\begin{lstlisting}
<composers>
<composer>
<name>Jean Sibelius</name>
<dates>1865-|\colorbox{orange}{1957}|</dates>
<nationality>Finnish</nationality>
</composer>
<composer>
<name>Aaron Copland</name>
<dates>1910-1990</dates>
<nationality>American</nationality>
</composer>
<composer>
<name>Benjamin |\colorbox{orange}{Britten}|</name>
<dates>1913-1976</dates>
<nationality>English</nationality>
</composer>
</composers>
\end{lstlisting}

Lenses are designed to perform these kinds of bidirectional transformations. If
$\ell$ is a suitably defined lens for the transformation that we
have just described, then the \get component of the lens performs the
XML to ASCII transformation that produces an ASCII view from an XML source,
while the \pput component of the lens performs the ASCII to XML transformation
that folds back an updated ASCII view back into the original XML source.


For example, suppose that a source $S$ consists of a pair consisting of a
boolean value and an integer, and that the view $V$ for $S$ consists of a
single integer. Further, suppose that we wish to define a \pput function for a
lens $\ell$ whose \get component is the function which given a pair $(b, n)$
simply produces the boolean value $b$. Let $f : \mathbb{B} \times \mathbb{Z}
\longrightarrow \mathbb{Z}$ be any function from $\mathbb{B} \times \mathbb{Z}
\longrightarrow \mathbb{Z}$. Then the lens $\ell_{f}$ defined by
\begin{align*}
\ell_{f}.\get \; (b, n) &= b\\
\ell_{f}.\pput \; (b, n) \; b' &= \begin{cases}
(b', n) & \text{if }b = b'\\
(b', f(b', n)) & \text{otherwise}\\
\end{cases}\\
\ell.\create \; b &= (b,f(b, 0))
\end{align*}
is a valid lens.

A user who applies a lens $\ell$ of type $\ell : S \Leftrightarrow V$ may expect
that $\ell.\pput \; (true, 1) \; false = (false, 0)$, reasoning that the
integer produced by the \pput function should be the number representation of
the boolean value $false$. Or perhaps she might expect that $\ell.\pput \;
(true, 1) \; false = (false, 1)$, reasoning that the two pieces of data in a
pair are independent and that the reasonable \pput behaviour in this case is to
restore the orginial integer. In either case, the user may be shocked to find
that in fact $\ell.\pput (true, 0) \; false = (false, 48)$! Given the \get
behaviour which just produces the boolean value from a pair, since $\ell_f$ is a
valid lens for any $f$, then the \pput function of a valid lens of type
$\mathbb{B} \times \mathbb{Z} \Leftrightarrow \mathbb{Z}$ can restore the
integer in a pair in any way so long as the old boolean value and the updated
boolean value are different.


In order to give a more constrained specification for the desired
lens we define the difference $\delta_S((a, b), (c, d))$ between $(a, b)$ and
$(c, d)$ by $$\delta_S((a, b), (c,
d)) = \begin{cases}
0 & \text{if }a = c, b = d\\
1 & \text{if }a=c, b \neq d\\
1 & \text{if }a \neq c, b = d\\
2 & \text{if }a \neq c, b \neq d
\end{cases}$$
On the view side, we define the difference $\delta_V(a, b)$
between $a$ and $b$ by, $$\delta_V(a, b) = \begin{cases}
0 & \text{if } a = b\\
1 & \text{if } a \neq b
\end{cases}$$

Next we define a lens $\partial(S, V) : \partial S \Leftrightarrow \partial V$
which describes how differences should be propagated. For this example, we
define $\partial(S,V)$ to simply be the \ccopy lens from $\partial S$ to
$\partial V$, keeping in mind that $\partial S = \partial V = \mathbb{Z}$.
\cref{difflenslaw} enables us to rule out $\ell_{bad}$, since
\begin{align*}
(\partial (S, V)).\get \; (\delta_S(\ell_{bad}.\pput \; (1,2) \; 3, (1,2))) &=
(\partial (S, V)).\get \; (\delta_S((0,3), (1,2)))\\
&= (\partial (S, V)).\get \; 2\\
&= 2\\
&\neq 1\\
&= \delta_V(3, 2)\\
&= \delta_V(3, \ell_{bad}.\get \; (1, 2))
\end{align*}
Consequently $\ell_{bad}$ is not a valid delta lens. On the
other hand, the lens $\ell_{good} : S \Leftrightarrow V$ defined by
\begin{align*}
\ell_{good}.\get \; (a, b) &= b\\
\ell_{good}.\pput \; (a, b) \; c &= (a, c)\\
\ell_{good}.\create \; b &= (0, b)
\end{align*}
is indeed a valid delta lens since the its \pput component always restores the
first factor.

\section{Strengthening Lens Laws}
\begin{definition}
A bidirectional transformation (bx) $R: M \Leftrightarrow N$ comprises of a
consistency relation $R \subseteq M \times N$ and restorers $\overrightarrow{R}
: M \times N \rightarrow N$ and $\overleftarrow{R} : M \times N \longrightarrow
N$.
\end{definition}
\begin{definition}
A bx is {\em correct} if the consistency restorers do restore consistency:
that is, for all $m \in M, n \in N$, we have $R(m, \overrightarrow{R}(m, n))$,
and dually.
\end{definition}
\begin{definition}
A bx is {\em hippocratic} if the consistency restores make no change
to and already-consistent pair of models: that is, for all $m \in M$, $n \in N$,
if $R(m, n)$ then $\overrightarrow{R}(m, n) = n$ and dually.
\end{definition}
\begin{definition}
A bx is {\em well behaved} if it is correct and hippocratic.
\end{definition}
For lenses, if $\ell : S \Leftrightarrow V$ is a lens, we take the
consistency relation $R$ to be given by $R(s, \ell.\get \; s)$ for all $s \in
S$, $\overrightarrow{\ell}(s, v) = \ell.\get \; s$, and $\overleftarrow{\ell} =
\ell.\pput$. By the PUTGET law, $\ell.\get \; (\ell.\pput \; s \; v) = v$,
hence $\ell$ is correct, and by the GETPUT laws, $\ell.\pput \; s \; (\ell.\get
\; s) = s$ and $\ell.\get \; (\ell.\create \; v) = v$, hence $\ell$ is
hippocractic. Therefore $\ell$ is a well-behaved bx.

Using the notation of Stevens \cite{stevens2010bidirectional},



The major shortcoming of this model of bx transformations is that it does not
take into account how changes should be propagated by transformation; a
well-behaved bx need only restore consistency when models go out of sync or
maintain consisteny if models are already in sync. This becomes an issue when
there are many different ways of restoring consistency

\section{Motivation and Related Work}
Foster et. al. \cite{foster2007combinators,foster2009bidirectional}
defined a lens $\ell : S \Leftrightarrow V$ from a source $S$ to a view $V$ to
be a triple of functions $\ell.\get : S \longrightarrow V$, $\ell.\pput : S
\longrightarrow S \longrightarrow V$ and $\ell.\create : V \longrightarrow S$
satisfying the following properties:
\begin{align*}
\ell.\get \; (\ell.\pput \; s \; v) &= v \tag{PUTGET}\\
\ell.\get \; (\ell.\create \; v) &= v \tag{CREATEGET}\\
\ell.\pput \; s \; (\ell.\get \; s) &= s \tag{GETPUT}
\end{align*}
While these two laws give some guarantees on how the \get, \pput and \create
components of a lens interact, they do not specify the behaviour of the \pput
component when the update to the view is non-trivial, because the GETPUT law
only says that the \pput component must return the original source unmodified
whenever the update to the view is a no-op. Consequently, a non-trivial update
to a view could lead to an undesired update to the source.

Therefore, several lens formalism have been proposed that explicitly have a
notion of an {\em update}. The first of these was Barbosa et. al.'s {\em
matching lenses}\cite{barbosa2010matching}. Matching lenses specifically deal
with the issue of {\em alignment}, since Foster et. al.'s original iteration
lens updates records positionally:
\begin{align*}
\ell^* .\pput \; (s_1, \ldots, s_m) \; (v_1, \ldots, v_n) &= t_1 \cdot
\ldots \cdot t_n\\
\text{ where } t_i &= \begin{cases}
\ell.\pput \; s_i \; v_i & \text{if } 1 \leq i \leq \min\{m, n\}\\
\ell.\create \; v_i & \text{if } \min\{m, n\} < i \leq n
\end{cases}\\
\end{align*}
In their work, Barbosa et. al. enrich the types of lenses with ``chunks''
identifying reorderable pieces of the source and view that should be re-aligned
after an update, and formulate behavioral laws that capture essential
constraints on the handling of chunks. They develop a core language
of matching lenses for strings, together with a set of ``alignment
combinators'' that implement a variety of alignment strategies. The obvious
shortcoming of matching lenses is that they only deal with alignment; they do
not solve the general issue of update propagation.

The next line of work is Diskin et. al.'s {\em delta
lenses}\cite{diskin2011asymmetric,diskin2011state}. Diskin et. al.'s delta
lenses work as follows: let $\ell : S \Leftrightarrow V$ be a lens from
$S$ to $V$. Assume that $s$ is a source, and that $v = \ell.\get s$. A user
performs an edits $d$ on $v$ to get $v'$: $v \xrightarrow{d} v'$. The \pput
function now takes the view edit $d$ and the original source $s$ and produces a
new source $s'$: $s \xrightarrow{\pput(d, s)} s'$. One thing that Diskin et. al.
do not do is that they do not give any concrete examples of delta lenses,
no composition, no notion of equivalence, and no combinators for constructing
delta lenses.

The next line of work is Hofmann et. al.'s {\em edit
lenses}\cite{hofmann2012edit}. Edit lenses are similar to delta lenses, but edit
lenses are more explicit. Specifically, the source and view edits are
represented as elements of a {\em monoid}, and the lens acts on sources and
views via monoid actions. That is if $v$ is a view and $d_v$ is an edit that
can be made on $v$ to produce $v'$, then $d_v \cdot v = v'$. The intended usage
of an edit lens is as follows. There are two users, one holding an element of
$S$ the other one an element of $V$ both referred to as {\em replicas}.
Initially, they hold $init_S \in S$ and $init_V \in V$ respectively. The users
then perform actions and propagate them across the lens. An action consists of
producing an edit $d_s$ (or $d_v$), applying it to one's current replica $s$
(resp. $v$), putting the edit through the lens to obtain an edit $d_v$ (resp.
$d_s$), and asking the user on the other side to apply $d_v$ (resp. $d_s$) to
their replica. One thing that Hofmann et. al. do that Diskin et. al. do not do
is that they do give concrete examples of edit lenses, define a suitable notion
of composition on them, and define combinators for constructing edit lenses.

The final line of work in this area is Pacheco et. al.'s
version of delta lenses\cite{pacheco2012delta}. Pacheco et. al. are the
first to give concrete examples of delta lenses, define a suitable notion
of composition on them, and define combinators for constructing delta
lenses. Pacheco et al. define a delta lens $\ell$ (d-lens for short), denote by
$\ell: S \trianglerighteq_{\blacktriangle} V$ to be a bidirectional
transformations that comprises four total functions:
\begin{align*}
\get &: S \longrightarrow V\\
\pput &: \forall(v, s) : V \times S. (v \triangle \get \;s) \longrightarrow
SA\\
\get_{\blacktriangle} &: \forall\{s':S, s:S\}. (s' \triangle s)
\longrightarrow (\get \; s')\triangle (\get \; s)\\
\pput_{\blacktriangle} &:\forall\{(v, s):V \times S\},d:(v \triangle \get
\;s).(\pput \; (v, s) \; d) \triangle s
\end{align*}
that satisfy the following propertes:
\begin{align*}
\get \; (\pput \; (v, s) \; d) = v \tag{PUTGET}\\
\pput \; (\get \; s, s) \; id = s \tag{GETPUT}\\
\get_{\blacktriangle} \; (\pput_{\blacktriangle} \; d) = d
\tag{$\text{PUTGET}_{\blacktriangle}$}\\
\pput_{\blacktriangle} \; id = id \tag{$\text{PUTGET}_{\blacktriangle}$}
\end{align*}
The \get function is a just like a \get function in Foster's et. al.'s original
work. The \pput function is different though. Let $s \in S$ and $v \in V$, and
let $d$ be an update from $v$ to $\ell.\get \; s$; for example, $d$ is a
sequence of insertions and deletions that transforms $v$ to $\ell.\get \; s$.
Then \pput takes in $s$ and $v$ as input, and in addition receives $d$ to
compute $\pput \; (v, s) \; d$.

The $\get_{\blacktriangle}$ and $\pput_{\blacktriangle}$ functions are functions
that transform deltas to deltas. The last two axioms for delta lenses say that
if $d$ is a delta from $v$ to $\ell.\get \; s$ and $\Delta$ is a delta from
$\pput \; (v, s) \; d$ to $s$, then $\get_{\blacktriangle} \; \Delta = d$, and
that if $d$ is the identity delta, then $\Delta$ is also the identity delta.

In addition to describing a delta lens language, Pacheco et. al. also
address the following shortcoming with edit lenses: edit lenses include
combinators for constructing lenses over {\em container types}.
In the simplest case, container types can be thought of as algebraic datatypes
such as lists and trees; each value of a container type has a certain {\em
shape}, and each shape has a set of {\em positions} associated with it. For
instance, the shape of a list of size $n$ is just $n$, while the positions of
such a list are the numbers $0, 1, \ldots, n-1$.

The shortcoming of the container edit lenses addressed by Pacheco et. al.
elements is that these lenses always preserve the number of positions; for
instance if $\ell$ is a container lens that maps lists to trees, and if $xs$ is
a list of size $n$, then the number of positions of $\ell.\get \; xs$ is also
$n$. Pacheco et. al. define a new set of combinators for container lenses that
can say map a list of size $5$ to a tree with 4 nodes.

In this article we show that it is not necessary for the \pput component of a lens
to explicitly propagate updates: instead, we can specify what the put function should
evaluate to, by imposing a more precise type judgement on lenses.

The idea of the stronger typing judgement is that we embed sources and views inside
of {\em difference structures} and then we define a lens that describes how the
differences behave as they cross in-between the source and the view. A
lens that is validly typed is then a lens that preserves this overall structure.

Here are the actual definitions:
\begin{definition}
A weighted set is a pair $(X, |\cdot|)$ where $X$ is a set and $|\cdot| : X
\longrightarrow \mathbb{R}^{\geq 0}$ is a map from $X$ to the non-negative
reals.
\end{definition}
\begin{definition}
A difference structure is a triple $(S, \partial S, \delta_S)$ where $S$ and
$\partial S$ are weighted sets, and $\delta_S:
S \times S \longrightarrow \partial S$ is a function satisfying
\begin{align*}
\delta(s, s') &= \delta(s', s)\tag{A1}\\
|\delta(s, s)| &= 0\tag{A2}\\
|\delta(s, s')| &\leq |s| + |s'|\tag{A3}
\end{align*}
\end{definition}
\begin{definition}
Let $(S, \partial S, \delta_S)$ and $(V, \partial V, \delta_V)$ be
difference structures. Let $\partial(S, V) : \partial S \Leftrightarrow \partial
V$ be a lens from $\partial S$ to $\partial V$. Then a lens $\ell : S
\Leftrightarrow V$ also has type $\ell : (S, \partial S,
\delta_S) \xLeftrightarrow{\partial(S, V)} (V, \partial V, \delta_V)$ if for
all $s \in S, v \in V$,
\begin{equation}\label{difflenslaw}
(\partial (S, V)).\get\; (\delta_S(\ell.\pput \; s \; v, s)) = \delta_V(v,
\ell.\get \; s)
\end{equation}
\end{definition}
The extra difference structure on the sources and the views, as well as the lens
$\partial(S, V)$ force the lens $\ell$ to satisfy \cref{difflenslaw}; this law is
therefore a specification that constrains what $\ell$ can do.

For example, suppose that a source $S$ consists of a pair of integers, and that the
view $V$ for $S$ consists of a single integer. Further, suppose that we wish
for the \get component of any lens from $S$ to $V$ to simply project away the
first factor:that is for any $\ell : \mathbb{Z} \times \mathbb{Z}
\Leftrightarrow \mathbb{Z}$, $\ell.\get \; (a, b) = b$ for all integers $a, b$.
Suppose further that we wish to force the \pput function of any lens $\ell$
defined on the data to restore the factor projected away by the \get function:
that is, we wish to enforce that for all integers $a, b, c$, $\pi_1(\ell.\pput
\; (a, b) \; c) = a$, where $\pi_1$ is the projection on the first factor.

Now consider the lens $\ell_{bad}$ defined by
\begin{align*}
\ell_{bad}.\get \; (a, b) &= b\\
\ell_{bad}.\pput \; (a, b) \; c &= \begin{cases}
(a, c) & \text{if }b = c\\
(0, c) & \text{otherwise}\\
\end{cases}\\
\ell.\create \; b &= (0,b)
\end{align*}
Then $\ell_{bad}$ is a valid lens even though $\ell_{bad}$ does not
always restore the first factor. For example, observe that $\ell.\pput (1, 2) \;
3 = (0,3)$. In order to give a more constrained specification for the desired
lens we define the difference $\delta_S((a, b), (c, d))$ between $(a, b)$ and
$(c, d)$ by $$\delta_S((a, b), (c,
d)) = \begin{cases}
0 & \text{if }a = c, b = d\\
1 & \text{if }a=c, b \neq d\\
1 & \text{if }a \neq c, b = d\\
2 & \text{if }a \neq c, b \neq d
\end{cases}$$
On the view side, we define the difference $\delta_V(a, b)$
between $a$ and $b$ by, $$\delta_V(a, b) = \begin{cases}
0 & \text{if } a = b\\
1 & \text{if } a \neq b
\end{cases}$$

Next we define a lens $\partial(S, V) : \partial S \Leftrightarrow \partial V$
which describes how differences should be propagated. For this example, we
define $\partial(S,V)$ to simply be the \ccopy lens from $\partial S$ to
$\partial V$, keeping in mind that $\partial S = \partial V = \mathbb{Z}$.
\cref{difflenslaw} enables us to rule out $\ell_{bad}$, since
\begin{align*}
(\partial (S, V)).\get \; (\delta_S(\ell_{bad}.\pput \; (1,2) \; 3, (1,2))) &=
(\partial (S, V)).\get \; (\delta_S((0,3), (1,2)))\\
&= (\partial (S, V)).\get \; 2\\
&= 2\\
&\neq 1\\
&= \delta_V(3, 2)\\
&= \delta_V(3, \ell_{bad}.\get \; (1, 2))
\end{align*}
Consequently $\ell_{bad}$ is not a valid delta lens. On the
other hand, the lens $\ell_{good} : S \Leftrightarrow V$ defined by
\begin{align*}
\ell_{good}.\get \; (a, b) &= b\\
\ell_{good}.\pput \; (a, b) \; c &= (a, c)\\
\ell_{good}.\create \; b &= (0, b)
\end{align*}
is indeed a valid delta lens since the its \pput component always restores the
first factor.

For a more realistic example using a difference lens combinator, let $S
= name \cdot dates$ and $V = name$. Let $\ell :
S \Leftrightarrow V = (copy \; name) \cdot (del \; dates)$, and suppose that we
wish to define a lens of type $S^* \Leftrightarrow V^*$, where the difference
function used on the view side is the min-edit distance difference function.
Assuming that we inductively use the difference function
$$ \delta(d_1, d_2) =
\begin{cases}
0 & \text{if }d_1 = d_2 \\
1 & \text{otherwise}
\end{cases}
$$
for both the name and date fields, then with the min-edit distance difference
function, the difference between the string $v$ given by
\begin{lstlisting}
John Doe
Jane Smith
\end{lstlisting}
and the string $v'$ given by
\begin{lstlisting}
Jane Smith
John Doe
\end{lstlisting}
is the empty list. Let $s$ =
\begin{lstlisting}
John Doe,1900-2000
Jane Smith,1901-2001
\end{lstlisting}Then $\ell.\get \; s = v$. Now assume that $v$ is edited to $v'$
given by
\begin{lstlisting}
Mark Moe
John Doe
Jane Smith
\end{lstlisting}
where a new record \lstinline|Mark Moe| has been added to $v$.
With the min-edit distance on the view side, $\delta(v', \ell.\get \; s)$ is
the singleton list [``Mark Moe''].

On the source side, the min-edit distance matching lens $\ell^*_{MIN-EDIT}$,
$\ell^*_{MIN-EDIT}.\pput \; s \; v'$ is the string
\begin{lstlisting}
Mark Moe,0000-0000
John Doe,1900-2000
Jane Smith,1901-2001
\end{lstlisting}
while with the positional lens $\ell^*$, then $\ell^*.\pput \; s \; v'$ is the
string
\begin{lstlisting}
Mark Moe,1900-2000
John Doe,1901-2001
Jane Smith,0000-0000
\end{lstlisting}
With the min-edit distance difference function,
$\delta(\ell^*_{MIN-EDIT}.\pput \; s \; v', s)$ is the singleton list
$[\text{``Mark Moe, 0000-0000''}]$, while with the
positional lens, $\delta(\ell^*.\pput \; s \; v', s)$ is the list
given by
\begin{lstlisting}
Mark Moe,1900-2000
(0,1)
(0,1)
\end{lstlisting}
The min-edit distance satisfies \cref{difflenslaw} since
\begin{align*}
(\partial (S, V))^*.\get\; (\delta_S(\ell^*_{MIN-EDIT}.\pput \; s \; v', s)) &=
(\partial (S, V))^*.\get\; ([\text{``Mark Moe, 1900-2000''}])\\
&= [\text{``Mark Moe''}]\\
&=
\delta_V(v', \ell.\get \; s)
\end{align*}
On the other hand, the positional lens does not satisfy \cref{difflenslaw} since
\begin{align*}
(\partial (S, V))^*.\get\; (\delta_S(\ell^*.\pput \; s \; v', s)) &=
(\partial (S, V))^*.\get\; ([\text{``Mark Moe, 1900-2000''}, (0,1), (0,1)])\\
&=[\text{``Mark Moe''}, (0,*), (0,*)]\\
& \neq \delta_V(v', \ell.\get \; s)
\end{align*}
Let $\mathcal{M}_V$ be the set of min-edit distance alignments on the view side,
and $\mathcal{M}_S$ the corresponding alignments on the source side.
Then, the typing judgement
$$\ell^*_{MIN-EDIT}:(S, (\partial S \sqcup S)^*, \match(\mathcal{M}_S, \delta_s))
\xLeftrightarrow{\partial(S, V)^*} (V, (\partial V \sqcup V)^*, \match(\mathcal{M}_V, \delta_v))$$
is valid, whereas the typing judgement
$$\ell^*:(S, (\partial S \sqcup S)^*, \match(\mathcal{M}_S, \delta_s))
\xLeftrightarrow{\partial(S, V)^*} (V, (\partial V \sqcup V)^*, \match(\mathcal{M}_V, \delta_v))$$
is not.
\iffalse
For example, if $\ell : S \Leftrightarrow V$ is a lens, Foster's initial
definition of the iteration lens $\ell^* : S^* \Leftrightarrow V^*$ was
\begin{align*}
\ell^* .\get \; (s_1, \ldots, s_n) &= (\ell.\get \; s_1) \cdot \ldots
\cdot (\ell.\get \; s_n)\\
\ell^* .\pput \; (s_1, \ldots, s_m) \; (v_1, \ldots, v_n) &= t_1 \cdot
\ldots \cdot t_n\\
\text{ where } t_i &= \begin{cases}
\ell.\pput \; s_i \; v_i & \text{if } 1 \leq i \leq \min\{m, n\}\\
\ell.\create \; v_i & \text{if } \min\{m, n\} < i \leq n
\end{cases}\\
\ell^*.\create \; (v_1, \ldots, v_n) &= (\ell.\create \; v_1) \cdot \ldots
\cdot (\ell.\create \; v_n)
\end{align*}
This lens always propagates updates positionally, which is unintended behaviour
in many cases. For example suppose that $\ell$ is a lens that transforms
source data in the form of an XML document representing the names, dates and
nationalities of a collection of classical music composers such as,

\begin{lstlisting}
<composers>
<composer>
<name>Jean Sibelius</name>
<dates>1865-1957</dates>
<nationality>Finnish</nationality>
</composer>
<composer>
<name>Aaron Copland</name>
<dates>1910-1990</dates>
<nationality>American</nationality>
</composer>
<composer>
<name>Benjamin Britten</name>
<dates>1913-1976</dates>
<nationality>English</nationality>
</composer>
</composers>
\end{lstlisting}
into view data stoored in ASCII text representing the names and dates of each
composer:
\begin{lstlisting}
Jean Sibelius, 1865-1957
Aaron Copland, 1910-1990
Benjamin Britten, 1913-1976
\end{lstlisting}
Now suppose that the elements in the view are reordered:
\begin{lstlisting}
Aaron Copland, 1910-1990
Benjamin Britten, 1913-1976
Jean Sibelius, 1865-1957
\end{lstlisting}
Then with the positional iteration lens, the updated source is
\begin{lstlisting}
<composers>
<composer>
<name>Aaron Copeland</name>
<dates>1910-1990</dates>
<nationality>Finnish</nationality>
</composer>
<composer>
<name>Benjamin Britten</name>
<dates>1913-1976</dates>
<nationality>American</nationality>
</composer>
<composer>
<name>Jean Sibelius</name>
<dates>1865-1957</dates>
<nationality>English</nationality>
</composer>
</composers>
\end{lstlisting}
so each of the composers now has the wrong nationality in the source.

\section{Difference functions}
\begin{definition}
A weighted set is a pair $(X, |\cdot|)$ where $X$ is a set and $|\cdot| : X
\longrightarrow \mathbb{R}^{\geq 0}$ is a map from $X$ to the non-negative
reals.
\end{definition}
\begin{definition}
A difference structure is a triple $(S, \partial S, \delta_S)$ where $S$ and
$\partial S$ are weighted sets, and $\delta_S:
S \times S \longrightarrow \partial S$ is a function satisfying
\begin{align*}
\delta(s, s') &= \delta(s', s)\tag{A1}\\
|\delta(s, s)| &= 0\tag{A2}\\
|\delta(s, s')| &\leq |s| + |s'|\tag{A3}
\end{align*}
We refer to the function $\delta$ as the difference function of the structure.
\end{definition}
\iffalse

\begin{example}
Let $X$ be any set. Define $x \leq y$ if and only if $x=y$, and define $|X|=1$
for all $x \in X$. Let $Q = \{0,1\}$, and define
$$\delta(a, b) = \begin{cases} 1 & \text{if } a \neq b\\
0 & \text{otherwise}
\end{cases}$$
Then $\delta$ is a difference function.
\end{example}

Various different edit distances on strings give a way of defining delta
functions on sets of strings:
\begin{example}
The {\em Levenshtein distance} between two words is the minimum number of
single-character edits (insertions, deletions or substitutions) required to
change one word into the other. For example, the Levenshtein distance between
"kitten" and "sitting" is 3, since the following three edits change one into
the other, and there is no way to do it with fewer than three edits:

\begin{enumerate}
  \item
  kitten -> sitten (substitution of "s" for "k")
  \item
  sitten -> sittin (substitution of "i" for "e")
  \item
  sittin -> sitting (insertion of "g" at the end).
\end{enumerate}
\end{example}
\begin{example}
In contrast, the Longest Common Subsequence(LCS) distance only allows for
insertions and deletions:
\begin{enumerate}
  \item
  delete k at 0
  \item
  insert s at 0
  \item
  delete e at 4
  \item
  insert i at 4
  \item
  insert g at 6
\end{enumerate}
for a total cost/distance of 5 operations.
\end{example}
Both Levenshtein and LCS distances are integer-valued distance functions since
they are metrics.
\begin{example}
If all the strings in a set are of the same length, then the {\em Hamming
distance} is a difference function since the Hamming distance is also a metric.
\end{example}
\begin{example}
Alignment algorithms also give us a way of defining integer-valued distance
functions on sets of strings by taking the difference of two strings to be the
total number of characters that are not equal according to the alignment.

When deciding which alignment to choose between two strings, we often wish to
choose an alignment that is optimal for alignment strategy. With
the positional alignment strategy, given two strings $s$ and $t$, then $s_1$ is
matched to $t_1$, $s_2$ is matched to $t_2$ and so on.
Also, $\mathcal{M}_{POS}$ is an optimal set of alignments for a positional
alignment strategy.

For example, using the positional alignment strategy on the strings "kitten" and
"sitting", then "k" is matched to "s", "i" to "i" and so on. The characters
which are included in the delta from "kitten" are "k", "e" while the
in the characters included from "sitting" are "s", "i" and "g", hence the delta
is equal to 5.

For the non-crossing strategy, we choose among alignments which are increasing
rather than positional, while for the min-edit distance strategy, we consider
all possible alignments.
\end{example}

\begin{example}\label{naturalnumbers}
Let $P$ be the set of non-negative integers with weight function $|a| = a$. Then
$\delta$ is a difference function.
Define $\delta(a, b) = |a-b|$. Then $\delta$ is a difference function.
\end{example}
\begin{example}
Let $X$ be a normed vector space and define $\delta(x, y) = |x-y|$. Then
$\delta$ is a difference function (in fact, $\delta$ is also a metric).
\end{example}
\begin{example}\label{divisibility}
Let $P$ be the set of positive integers with weight function $|\cdot|$ defined
by $|p_1^{a_1} \cdot \ldots \cdot p_n^{a_n}| = a_1 + \ldots + a_n$. Define
$\delta(a, b) = ab/(gcd(a, b))^2$.
Then $\delta$ is a difference function.
\end{example}
\begin{example}\label{boolean}
Let $X$ be a set and $P$ the set of finite subsets of $X$ with weight function
given by the size of the set. Define $\delta(A, B) = (A \setminus B) \cup (B
\setminus A)$. Then $\delta$ is a difference function.
\end{example}
\begin{example}\label{general}
Generalizing \cref{divisibility} and
\cref{boolean}, let $P$ be a graded complete semi-modular lattice satisfying
$|\bigwedge P|=0$. Define $\delta(a, b) = \bigwedge \{x \; | \; (a \wedge b)
\vee x = a\} \vee \bigwedge \{x \; | \; (a \wedge b) \vee x = b\}$. Then
$\delta$ is a difference function.
\end{example}
\begin{proof}
Let $a, b \in P$. Let $X = \{x \; | \; (a \wedge b) \vee x = a\}$ and $Y = \{y
\; | \; (a \wedge b) \vee y = b\}$
\begin{enumerate}
  \item[(A1)]
  Certainly $\delta(a, b) = \delta(b, a)$.
  \item[(A2)]
  Observe that $\bigwedge X = \bigwedge Y = \bot$, hence $\delta(a,
  a) = \bot$, and $|\delta(a, a)| = |\bot| = 0$.
  \item[(A3)]
  Since $P$ is semi-modular, then $|x \wedge y| + |x \vee y| \leq |x| + |y|$
  for all $x, y \in P$, from which it follows that $$|\delta(a, b)| =
  \left|\bigwedge X \vee \bigwedge Y\right| \leq \left|\bigwedge X \vee
  \bigwedge Y\right| + \left|\bigwedge X \wedge \bigwedge Y\right| \leq
  \left|\bigwedge X\right| + \left|\bigwedge Y\right|$$
\end{enumerate}
\end{proof}
\fi

\subsection{Constructions on difference functions}
\subsubsection{Products}
Given difference functions $\delta_i: P_i \times P_i \longrightarrow Q_i$ for
$i \in \{1,2\}$, define $\delta_1 \times \delta_2 :(P_1 \times P_2) \times
(P_1 \times P_2) \longrightarrow (Q_1 \times Q_2)$ by
$$(\delta_1 \times \delta_2)((x, y), (x', y')) = (\delta_1(x, x'), \delta_2(y,
y'))$$
\begin{claim}
$\delta_1 \times \delta_2$ is a difference function.
\end{claim}
\begin{proof}
Let $x, x' \in P_1$ and $y, y' \in P_2$.
\begin{enumerate}
  \item[(A1)]
  Observe that
  \begin{align*}
(\delta_1 \times \delta_2)((x, y), (x', y')) &= (\delta_1(x, x'),
\delta_2(y, y'))\\
&= (\delta_1(x', x), \delta_2(y', y))\\
&= (\delta_1 \times \delta_2)((x', y'), (x, y))
\end{align*}
\item[(A2)] Observe that
$$|(\delta_1 \times \delta_2)((x, y), (x, y))| = |(\delta_1(x, x),
\delta_2(y, y))| = |\delta_1(x, x)| + |\delta_2(y,y)| = 0 + 0 = 0$$
\item[(A3)]
Observe that
\begin{align*}
|(\delta_1 \times \delta_2)((x, y), (x', y'))| &= |(\delta_1(x, x'),
\delta_2(y, y'))|\\
&= |(\delta_1(x, x')| + |(\delta_2(y, y')|\\
&\leq (|x| + |x'|) + (|y| + |y'|)\\
&= (|x| + |y|) + (|x'| + |y'|)\\
&= |(x, y)| + |(x', y')|
\end{align*}
\end{enumerate}
\end{proof}
\subsubsection{Coproducts}
Given difference functions $\delta_i: P_i \times P_i \longrightarrow Q_i$ for
$i \in \{1,2\}$, define $\delta_1 \sqcup \delta_2 :P_1 \sqcup P_2
\longrightarrow Q_1 \sqcup Q_2 \sqcup (P_1 \times P_2)$ by
$$(\delta_1 \sqcup \delta_2)(x, y) =
\begin{cases}
\delta_1(x, y) & \text{if }x, y \in P_1\\
\delta_2(x, y) & \text{if }x, y \in P_2\\
(x, y) & \text{if }x \in P_1, y \in P_2\\
(y, x) & \text{if }x \in P_2, y \in P_1
\end{cases}$$
\begin{claim}
$\delta_1 \sqcup \delta_2$ is a difference function
\end{claim}
\begin{proof}
Let $x, x' \in P_1$ and $y, y' \in P_2$.
\begin{enumerate}
  \item[(A1)]
  If $x, y \in P_i$, then $(\delta_1 \sqcup \delta_2)(x, y) = \delta_i(x, y) =
  \delta_i(y, x)$. If $x \in P_1$ and $y \in P_2$, then $(\delta_1 \sqcup
  \delta_2)(x, y) = (x, y) = (\delta_1 \sqcup \delta_2)(y, x)$, and similarly
  if $x \in P_2$ and $y \in P_1$, then $(\delta_1 \sqcup \delta_2)(x, y) =
  (y,x) = (\delta_1 \sqcup \delta_2)(y, x)$
  \item[(A2)] If $x \in P_i$, then $|(\delta_1 \sqcup \delta_2)(x, x)| =
  |\delta_i(x, x)| = 0$
  \item[(A3)]
  If $x, y \in P_i$, then $|(\delta_1 \sqcup \delta_2)(x, y)| = |\delta_i(x, y)|
  \leq |x| + |y|$. If $x \in P_1$ and $y \in P_2$, then $|(\delta_1 \sqcup
  \delta_2)(x, y)| = |(x, y)| = |x| + |y|$, and if $x \in P_2$ and $y \in
  P_1$, then $|(\delta_1 \sqcup \delta_2)(x, y)| = (|(y, x)|) = |y| + |x|$.
\end{enumerate}
\end{proof}
\subsubsection{List difference functions}
Let $P$ be a weighted set. Let $x=(x_1, \ldots, x_m), y=(y_1, \ldots, y_n)$ be
in $P^*$. Given a difference function $\delta: P \times P \longrightarrow Q$, we
define the list difference function $\delta^*:P^* \longrightarrow (Q \sqcup
P)^*$ on $x$ and $y$ to be the list given by $\delta(x_i, y_i)$ for $1 \leq i
\leq m$, except that we omit $\delta(x_i, y_i)$ in position $i$ if $x_i = y_i$.
For $m < i \leq n$ we add $y_i$ at position $i$. Finally we collapse the list
to get rid of the empty spaces left after the omissions.

For instance suppose that $S$ is the set of all letters in the alphabet and
$\partial S = \{\epsilon\} \sqcup \{\{x, y\} \; | \; x, y \in S, x \neq y\}$
with $$\delta(x, y) =
\begin{cases}
\epsilon & \text{if } x = y\\
\{x,y\} & \text{if } x \neq y
\end{cases}$$
Then $\delta_S^*((a, c, b, d), (a, b, c, d, e)) = (\{b, c\}, \{b, c\},
e)$ where we match and remove the first $a$ and the fourth $d$ in. Since $b \neq
c$ we have $\{b,c\}$ in the second and third positions, and since $e$ is
unmatched, we include $e$ in the fifth position. Finally we collapse the list to
get rid of the two empty spaces created.

\begin{claim}
$\delta^*$ is a difference function
\end{claim}
\begin{proof}
Let $x = (x_1, \ldots, x_m), y = (y_1, \ldots, y_n) \in P^*$.
\begin{enumerate}
  \item[(A1)]
  By construction $\delta^*(x, y) = \delta^*(y, x)$.
  \item[(A2)] Observe that $|\delta^*(x, x)| = \sum_{i=1}^m|\delta(x_i,
  x_i)| =  \sum_{i=1}^m 0 = 0$.
  \item[(A3)]
  Assume WLOG that $m \leq n$. Then
  \begin{align*}
|\delta^*(x, y)| &=
\sum_{i=1}^m|\delta(x_i, y_i)| + \sum_{i=m+1}^n |y_i|\\
&\leq \sum_{i=1}^m|x_i| + |y_i| + \sum_{i=m+1}^n |y_i|\\
&= \sum_{i=1}^m|x_i| + \sum_{i=1}^n |y_i|\\
&= |x| + |y|
\end{align*}
\end{enumerate}
\end{proof}
\subsubsection{Matching difference functions}
Now let $\mathcal{M} = \{\varphi_{(x_1, \ldots, x_m), (y_1, \ldots, y_n)} \; |
\; x_i, y_j \in P, m \leq n\}$ be a set of injective functions $\varphi_{(x_1,
\ldots, x_m), (y_1, \ldots, y_n)} : [m] \longrightarrow [n]$. Given a difference
function $\delta:
P \times P \longrightarrow Q$, we define the difference function
$\match(\mathcal{M}, \delta):P^* \longrightarrow (Q \sqcup P)^*$ on $x$
and $y$ to be the list given by $\delta(x_i, y_{\varphi(i)})$ in position
$\varphi(i)$ for $1 \leq i \leq m$, except that we omit $\delta(x_i,
y_{\varphi(i)})$ in position $\varphi(i)$ if $x_i = y_{\varphi(i)}$. For $i \in
[n] \setminus \varphi([m])$ we add $y_i$ at position $i$. Finally we collapse
the list to get rid of the empty spaces left after the omissions.

Going back to our previous example, let $x = \{a, c, b, d\}$ and $y = \{a, b, c,
d, e\}$, and assume that $\varphi_{x, y} : [4] \longrightarrow [5]$ is given by
$$\varphi(1) = 1, \; \varphi(2) = 3, \; \varphi(3) = 2, \text{ and } \varphi(4)
= 4$$ Then $\match(\mathcal{M}, \delta)((a, c, b, d), (a, b, c, d, e)) =
(e)$.
\begin{claim}
$\match(\mathcal{M}, \delta)$ is a difference function.
\end{claim}
\begin{proof}
Let $x = \{x_1, \ldots, x_m\}, y = \{y_1, \ldots, y_n\} \in \match(P)$.
\begin{enumerate}
  \item[(A1)]
  By construction $\match(\mathcal{M}, \delta)(x, y) =
  \match(\mathcal{M}, \delta)(y, x)$.
  \item[(A2)] Observe that $|(\match(\mathcal{M}, \delta))(x, x)| =
  \sum_{i=1}^m|\delta(x_i, x_{\varphi(i)}| = |\delta(x_i, x_i)| =   \sum_{i=1}^m
  0 = 0$.
  \item[(A3)]
  Assume WLOG that $m \leq n$. Then
  \begin{align*}
|\match(\mathcal{M}, \delta)(x, y)| &=
\sum_{i=1}^m|\delta(x_i, y_{\varphi(i)})| + \sum_{i \in [n] \setminus
\varphi([m])} |y_i|\\
&\leq \sum_{i=1}^m|x_i| + |y_{\varphi(i)}| + \sum_{i \in [n] \setminus
\varphi([m])} |y_i|\\
&= \sum_{i=1}^m|x_i| + \sum_{i=1}^n |y_i|\\
&= |x| + |y|
\end{align*}
\end{enumerate}
\end{proof}
\fi
\section{Lens Combinators}
\subsection{Copy}
Let $P$ be any set. Define the $\ccopy(P) : P \Leftrightarrow P$ by
\begin{align}
\ccopy(P).\get \; s &= s\\
\ccopy(P).\pput \; s \; v &= v\\
\ccopy(P).\create \; v &= v
\end{align}
Then $\ccopy(P)$ is a lens. Now let $\partial S$ be a set, $\delta :
S \times S \longrightarrow \partial S$ a function.
\begin{claim}
$\ccopy(S) : S \xLeftrightarrow{\delta , \ccopy(\partial S), \delta} S$.
\end{claim}
\begin{proof}
We need to show that $(\ccopy(\partial S)).\get \; (\delta(\ccopy(S).\pput
\; s \; v, s)) = \delta(v, \ccopy(S).\get \; s)$ which follows from the fact
that $$ (\ccopy(\partial S)).\get \; (\delta(\ccopy(S).\pput \; s
\; v, s)) = \delta(\ccopy(S).\pput \; s \; v, s) = \delta(v, \ccopy(S).\get
\; s) $$
\end{proof}
\subsection{Constant}
Let $S$ be any set with $s'\in S$. Define the \const lens $\const(S) : S
\Leftrightarrow \{\star\}$ by
\begin{align*}
\const.\get \; s &= \star\\
\const.\pput \; s \; v &= s\\
\const.\create \; v &= s'
\end{align*}
Then $\const(S)$ is a lens. 
\begin{claim}
Let $\partial S$ be any set, and $\delta : S \times S \longrightarrow \partial
S$ any function satisfying $\delta(s, s) = \delta(s', s')$ for all $s, s' \in
S$. Define $\beta : \{\star\} \times \{\star\} \longrightarrow \partial S$ by
$\beta(\star, \star) = \delta(s, s)$ for any $s \in S$. Let $\ell : \partial S
\Leftrightarrow \partial S$ be any lens satisfying $\ell.\get \; \delta(s, s) =
\delta(s, s)$ for all $s \in S$. Then $\const(S): S \xLeftrightarrow{\delta,
\ell, \beta} \{\star\}$.
\end{claim}
\begin{proof}
We need to show that $\ell.\get \; (\delta(\const(S).\pput \; s \; \star, s)) =
\beta (\star, \const(S).\get \; s)$
which follows from the fact that
$$ \ell.\get \; (\delta(\const(S).\pput \; s \; \star, s)) = \ell.\get
\; \delta(s, s) = \delta(s, s) = \beta(\star, \star) = \beta(\star,
\const(S).\get \; s)$$
\end{proof}

\subsection{Default}
Let $\ell : S \Leftrightarrow V$ be a lens. Let $f :
V \longrightarrow S$ be any function. Define the lens $\default(\ell, f) :
S \Leftrightarrow V$ by
\begin{align*}
\default(\ell, f).\get \; s &= \ell.\get \; s\\
\default(\ell, f).\pput \; s \; v &= \ell.\pput \; s \; v\\
\default(\ell, f).\create \; v &= \ell.\pput \; (f \; v) \; v
\end{align*}
 
\begin{claim}
If $\ell : S \xLeftrightarrow{\delta_S, \partial (S, V), \partial V}
V$ and $f: S \longrightarrow V$, then $\default(\ell, f) : S
\xLeftrightarrow{\delta_S, \partial (S, V), \partial V} V$.
\end{claim}
\begin{proof}
Follows from the fact that $\ell : S \xLeftrightarrow{\delta_S, \partial (S, V),
\partial V} V$.
\end{proof}
\subsection{Product}
Let $\ell_1 : S_1 \Leftrightarrow V_1$ and $\ell_2 : S_2 \Leftrightarrow V_2$ be
lenses. Define the lens $\ell_1 \times \ell_2 : (S_1 \times S_2) \Leftrightarrow
(V_1 \times V_2)$ by
\begin{align*}
(\ell_1 \times \ell_2).\get \; (s_1, s_2) &= (\ell_1.\get \; s_1, \ell_2.\get
\; s_1)\\
(\ell_1 \times \ell_2).\pput \; (s_1, s_2) \; (v_1, v_2) &= (\ell_1.\pput \; s_1
\; v_1, \ell_2.\pput \; s_2 \; v_2)\\
(\ell_1 \times \ell_2).\create \; (v_1, v_2) &= (\ell_1.\create \; v_1,
\ell_2.\create \; v_2)
\end{align*}
\begin{claim}\label{productislens}
If $\ell_1 :S_1 \xLeftrightarrow{\delta_{S_1}, \partial (S_1, V_1),
\delta_{V_1}} V_1$ and $\ell_2 : S_2 \xLeftrightarrow{\delta_{S_2}, \partial
(S_2, V_2), \delta_{V_2}} V_2$, then \\
$\ell_1 \times \ell_2  : S_1 \times S_2 
\xLeftrightarrow{\delta_{S_1} \times \delta_{S_2}, \partial (S_1, V_1) \times
\partial (S_2, V_2), \delta_{V_1} \times \delta_{V_2}} V_1 \times V_2$.
\end{claim}
\begin{proof}
We need to show that
$$(\partial (S_1, V_1) \times \partial (S_2, V_2)).\get \; (\delta((\ell_1 \times
\ell_2).\pput \; (s_1, s_2) \; (v_1, v_2), (s_1, s_2))) = \delta((v_1, v_2),
(\ell_1 \times \ell_2).\get \; (s_1, s_2))$$ which follows from the fact that
\begin{align*}
&(\partial (S_1, V_1) \times \partial (S_2, V_2)).\get \; (\delta((\ell_1 \times
\ell_2).\pput \; (s_1, s_2) \; (v_1, v_2), (s_1, s_2)))\\
&= (\partial (S_1, V_1) \times \partial (S_2, V_2)).\get \;
(\delta((\ell_1.\pput \; s_1 \; v_1, \ell_2.\pput \; s_2 \; v_2), (s_1, s_2))\\
&= (\partial (S_1, V_1) \times \partial (S_2, V_2)).\get \;
(\delta(\ell_1.\pput \; s_1 \; v_1, s_1), \delta(\ell_2.\pput \; s_2 \; v_2,
s_2))\\
&= ((\partial (S_1, V_1)).\get \; (\delta(\ell_1.\pput \;
s_1 \; v_1, s_1)), (\partial (S_2, V_2)).\get \; (\delta(\ell_2.\pput \; s_2 \;
v_2, s_2)))\\
&= (\delta(v_1, \ell_1.\get \; s_1), \delta(v_2, \ell_2.\get \; s_2))\\
&= \delta((v_1, v_2), (\ell_1.\get \; s_1, \ell_2.\get \; s_2))\\
&= \delta((v_1, v_2), (\ell_1 \times \ell_2).\get \; (s_1, s_2))
\end{align*}
\end{proof}
\subsection{Sum}
Let $\ell_1 : S_1 \Leftrightarrow V_1$ and $\ell_2 : S_2 \Leftrightarrow V_2$ be
lenses. Define the lens
$\ell_1 \sqcup \ell_2 :
(S_1 \sqcup S_2) \Leftrightarrow (V_1 \sqcup V_2)$ by
\begin{align*}
(\ell_1 \sqcup \ell_2).\get \; s &=
\begin{cases}
\ell_1.\get \; s & \text{if } s \in S_1\\
\ell_2.\get \; s & \text{if } s \in S_2\\
\end{cases}\\
(\ell_1 \sqcup \ell_2).\pput \; s \; v&=
\begin{cases}
\ell_1.\pput \; s \; v& \text{if } s \in S_1, \; v \in V_1\\
\ell_2.\pput \; s \; v& \text{if } s \in S_2, \; v \in V_2\\
\ell_1.\create \; v & \text{if } s \in S_2, \; v \in V_1\\
\ell_2.\create \; v & \text{if } s \in S_1, \; v \in V_2\\
\end{cases}\\
(\ell_1 \sqcup \ell_2).\create \; v &=
\begin{cases}
\ell_1.\create \; v & \text{if } v \in V_1\\
\ell_2.\create \; v & \text{if } v \in V_2\\
\end{cases}
\end{align*}

\begin{claim}
If $\ell_1 : S_1  \xLeftrightarrow{\delta_{S_1}, \partial (S_1, V_1),
\delta_{V_1}} V_1$ and $\ell_2 : S_2 \xLeftrightarrow{\delta_{S_2}, \partial
(S_2, V_2), \delta_{V_2}} V_2$, then \\
$\ell_1 \sqcup \ell_2 : S_1 \sqcup S_2 \xLeftrightarrow{\delta_{S_1} \sqcup
\delta_{S_2}, \partial (S_1, V_1) \sqcup \partial (S_2, V_2) \sqcup (\ell_1
\times \ell_2), \delta_{V_1} \sqcup \delta_{V_2}} V_1 \sqcup V_2$
\end{claim}
\begin{proof}
We need to show that
$$(\partial (S_1, V_1) \sqcup \partial (S_2, V_2)).\get \;
(\delta((\ell_1 \sqcup \ell_2).\pput \; s \; v, s)) = \delta(v, (\ell_1 \sqcup
\ell_2).\get \; s)$$

The result holds inductively if $s \in S_1$ and $v \in V_1$ or $s \in S_2$ and
$v \in V_2$. Otherwise, assume without loss of generality
that $s \in S_1$ and $v \in V_2$. Then
\begin{align*}
(\partial (S_1, V_1) \sqcup \partial (S_2, V_2)).\get \;
(\delta((\ell_1 \sqcup \ell_2).\pput \; s \; v, s)) &=
(\partial (S_1, V_1) \sqcup \partial (S_2, V_2)).\get \; (\delta(\ell_2.\create
\; v, s))\\
&= (\ell_1 \times \ell_2).\get \; (s, \ell_2.\create \; v)\\
&= (\ell_1.\get \; s, \ell_2.\get \; (\ell_2.\create \; v))\\
&= (\ell_1.\get \; s, v)\\
&= \delta(v, \ell_1.\get \; s)\\
&= \delta(v, (\ell_1 \sqcup \ell_2).\get \; s)
\end{align*}
\end{proof}
\subsection{Positional Iteration}
Let $\ell : S \Leftrightarrow V$ be a lens and define the lens $\ell^*:
S^* \Leftrightarrow V^*$ by
\begin{align*}
\ell^* .\get \; (s_1, \ldots, s_n) &= (\ell.\get \; s_1) \cdot \ldots
\cdot (\ell.\get \; s_n)\\
\ell^*.\pput \; (s_1, \ldots, s_m) \; (v_1, \ldots, v_n) &= s'_1 \cdot
\ldots \cdot s'_n\\
\text{ where } s'_i &= \begin{cases}
\ell.\pput \; s_i \; v_i & \text{if } 1 \leq i \leq \min\{m, n\}\\
\ell.\create \; v_i & \text{if } \min\{m, n\} < i \leq n
\end{cases}\\
\ell^*.\create \; (v_1, \ldots, v_n) &= (\ell.\create \; v_1) \cdot \ldots
\cdot (\ell.\create \; v_n)
\end{align*}

\begin{claim}\label{positionalisvalid}
If $\ell: S \xLeftrightarrow{\delta_S, \partial (S, V), \delta_V}
V$, then $\ell^*: S^* \xLeftrightarrow{\delta_{S}^*, (\partial (S, V) \sqcup
\ell)^*, \delta_{V}^*} V^*$.
\end{claim}
\begin{proof}
We need to show that
\begin{multline}
(\partial (S, V) \sqcup \ell)^*.\get \;
(\delta_S^*(\ell^*.\pput \; (s_1, \ldots, s_n) \; (v_1, \ldots,
v_m), (s_1, \ldots, s_n))) \\
= \delta_V^*((v_1, \ldots, v_m), \ell^*.\get \; (s_1, \ldots,
s_n))
\end{multline}
Let $1 \leq i \leq \min\{m, n\}$. Observe that $v_i = \ell.\get \; s_i$ if and
only if $\ell.\pput \; s_i \; v_i = s_i$, so $\delta_V^*$ omits
$\delta_V(v_i, \ell.\get \; s_i)$ in position $i$ if and only if
$\delta_S^*$ also omits $\delta_S(\ell.\pput \; s_i \; v_i, s_i)$ in
position $i$; if no omission occurs, then $\delta(S, V).\get \;
(\delta(\ell.\pput \; s_i \; v_i, s_i)) = \delta(v_i, \ell.\get \; s_i)$ since
$(\ell, \partial(S,V))$ is a delta lens. For $\min\{m,n\} < i \leq m$, then
$\delta_V^*$ adds $v_i$ in position $i$, and $\delta_S^*$ adds
$\ell.\create \; v_i$ in position $i$, which gives the desired result since
$$(\partial (S, V) \sqcup \ell).\get \; (\ell.\create \; v_i) = \ell.\get \;
(\ell.\create \; v_i)) = v_i$$
\end{proof}
\subsection{Matching Iteration}
\iffalse
When defining a $\match(\mathcal{M}, \delta)$ difference function, we often wish for
$\mathcal{M}$ to be an optimal set of matchings for some alignment strategy.

For instance, define the set of matchings $\mathcal{M}_{POS}$ by
$$\mathcal{M}_{POS} = \{\varphi_{(a_1, \ldots, a_m), (b_1, \ldots, b_n)}:[m]
\longrightarrow [n] \; | \; m \leq n, \varphi_{(a_1, \ldots, a_m), (b_1,
\ldots, b_n)}(i) = i\}$$ Then $\mathcal{M}_{POS}$ is an optimal set of
matchings for a positional alignment strategy.

For the non-crossing strategy, we want to our matchings to be increasing rather
than positional. That is, given a valid set of matchings $\mathcal{M}$, let
$\mathcal{M}_{a, b} = \varphi_{a, b} \in \mathcal{M}$, and for $a, a' \in P$
with $a=(a_1, \ldots, a_m)$ and $a'=(a'_1, \ldots, a'_n)$, define
$\mathcal{M}^{NC}_{a, b}$ by
$$\mathcal{M}^{NC}_{a, a'} = \argmin_{\substack{\mathcal{M}_{a, a'} \\
\mathcal{M}_{a, a'} \text{ is increasing}}} |\match(\mathcal{M}_{a, a'},
\delta)(a, a')|$$
There may be multiple matchings which for which the size of the delta is
minimal, in which case we pick the lexicographically smallest matching.
Then $\mathcal{M}^{NC}$ is an optimal set of matchings for a non-crossing
alignment strategy.

Finally, for a min-edit distance strategy, we simply remove the condition that
the matchings should be increasing. That is, define
$\mathcal{M}^{MED}_{a, b}$ by
$$\mathcal{M}^{MED}_{a, a'} = \argmin_{\mathcal{M}_{a, a'}}
|\match(\mathcal{M}_{a, a'}, \delta)(a, a')|$$
Then $\mathcal{M}_{MED}$ is an optimal set of matchings for a min-edit distance
alignment strategy.
\fi
Assume that $\ell: S \Leftrightarrow V$. Let $\mathcal{M}_V = \{\varphi_{x, y}
\; | \; x, y \in V^*\}$ be a set of alignments between $x$ and $y$ for each $x,
y \in V^*$. For each $s, s' \in S^*$, let $\varphi_{s, s'} = \varphi_{\ell.\get
\; s, \ell.\get \; s'}$. Define $\mathcal{M}_S$ by $\mathcal{M}_S =
\{\varphi_{s, s'} \; | \; s, s' \in S^*\}$. Define the lens
$\match(\mathcal{M}_V, \ell) : S^* \Leftrightarrow V^*$ by
\begin{align*}
\match(\mathcal{M}_V, \ell) .\get \; (s_1, \ldots, s_n) &= (\ell.\get \; s_1)
\cdot \ldots \cdot (\ell.\get \; s_n)\\
\match(\mathcal{M}_V, \ell) .\pput \; (s_1, \ldots, s_m) \; (v_1, \ldots, v_n)
&= (s'_1 , \ldots , s'_n)\\
\text{ where } s'_i &= \begin{cases}
\ell.\pput \; s_{\psi(i)} \; v_i & \text{if } i \in Dom(\psi)\\
\ell.\create \; v_i & \text{if } i \not \in Dom(\psi)
\end{cases}\\
\text{ and } \psi &= \varphi_{(v_1, \ldots, v_n), ((\ell.\get \; s_1), \ldots,
(\ell.\get \; s_m))}\\
\match(\mathcal{M}_V, \ell).\create \; (v_1, \ldots, v_n) &= (\ell.\create \;
v_1) \cdot \ldots \cdot (\ell.\create \; v_n)
\end{align*}
Assume that 
\begin{claim}
If $\ell: S \xLeftrightarrow{\delta_{S}, \partial (S, V), \delta_{V}}
V$, then $\match(\mathcal{M}_V, \ell) : S^*
\xLeftrightarrow{\match(\mathcal{M}_S, \delta_{S}), \partial (S, V)^*,
\match(\mathcal{M}_V, \delta_V)} V^*$.
\end{claim}
\begin{proof}
Observe that the matching used to compute the delta on the source side is
the same as the matching used to compute the matching on the view side, hence we
may take the update lens to be position and proceed with the rest of the proof
as in \cref{positionalisvalid}.
\end{proof}
\subsection{Composition}
Let $\ell_1 : S \Leftrightarrow U$ and $\ell_2 : U \Leftrightarrow V$ be lenses.
Define the lens $\ell_2 \circ \ell_1 : S \Leftrightarrow V$ by
\begin{align*}
(\ell_2 \circ \ell_1).\get \; s &= \ell_2.\get \; (\ell_1.\get \; s)\\
(\ell_2 \circ \ell_1).\pput \; s \; v &= \ell_1.\pput \; s \; (\ell_2.\pput \;
(\ell_1.\get \; s) \; v) \\
(\ell_2 \circ \ell_1).\create \; v &= \ell_1.\create \; (\ell_2.\create \; v)
\end{align*}

\begin{claim}
If $\ell_1 : S_1  \xLeftrightarrow{\delta_{S_1}, \partial (S_1, V_1),
\delta_{V_1}} V_1$ and $\ell_2 : S_2 \xLeftrightarrow{\delta_{S_2}, \partial
(S_2, V_2), \delta_{V_2}} V_2$, then \\
$\ell_2 \circ \ell_1 :  S \xLeftrightarrow{\delta_S, \partial
(S_2, V_2) \circ \partial (S_1, V_1), \delta_V} V$.
\end{claim}
\begin{proof}
We need to show that
$$
(\partial (S_2, V_2) \circ \partial (S_1, V_1)).\get \; (\delta(\ell_2 \circ
\ell_1).\pput \; s \; v, s) = \delta(v, (\ell_2 \circ \ell_1).\get \; s)$$

This follows from the fact that
\begin{align*}
&(\partial (S_2, V_2) \circ \partial (S_1, V_1)).\get \; (\delta_S((\ell_2 \circ
\ell_1).\pput \; s \; v, s))\\
&=(\partial (S_2, V_2) \circ \partial (S_1, V_1)).\get \; (\delta_S(\ell_1.\pput
\; s \; (\ell_2.\pput \; (\ell_1.\get \; s) \; v), s))\\
&= (\partial (S_2, V_2)).\get \; ((\partial (S_1, V_1)).\get \;
(\delta_S(\ell_1.\pput \; s \; (\ell_2.\pput \; (\ell_1.\get \; s) \; v), s)))\\
&= (\partial (S_2, V_2)).\get \; (\delta_U(\ell_2.\pput \; (\ell_1.\get \; s)
\; v, \ell_1.\get \; s))\\
&= \delta_V(v, \ell_2.\get \; (\ell_1.\get \; s))\\
&= \delta_V(v, (\ell_2 \circ \ell_1).\get \; s)
\end{align*}
\end{proof}
\iffalse
\section{Symmetric Delta Lenses}
Assume that $(X, \partial X, \delta_X)$ and $(Y, \partial Y, \delta_Y)$
are difference structures. A symmetric delta lens is a pair $(\ell, \partial
\ell)$ where $\ell : X \Leftrightarrow Y$ is a symmetric lens, and
$\partial \ell : \partial X \Leftrightarrow \partial Y$ is a symmetric lens
such that
\begin{enumerate}
  \item
  if $\ell.putr(x, c) = (y, c)$ and $\ell.putl(y', c) = (x', c')$,
  then $\partial \ell.putl \; (\delta(y, y'), \delta(c, c')) = (\delta(x,
  x'), \delta(c, c'))$
  \item
  if $\ell.putl(y, c) = (x, c)$ and $\ell.putr(x', c) = (y', c')$, then
  $\partial \ell.putr \; (\delta(x, x'), \delta(c, c')) = (\delta(y, y'),
  \delta(c, c'))$
\end{enumerate}
\begin{fact}[Identity Lens]
Given any symmetric lens $\partial X : X \Leftrightarrow X$, $(id_X, \partial
X)$ is a symmetric delta lens
\end{fact}
\begin{fact}[Bijective Lenses]
Assume that $f : X \longrightarrow Y$ and $g : \partial X \longrightarrow
\partial Y$ are bijections. Assume that $g(\delta(x, x')) = \delta(f(x),
f(x'))$. Then $(bij_f, bij_g)$ is a symmetric delta lens.
\end{fact}
\begin{fact}[Lens Composition]
If $(\ell_1, \partial \ell_1) : (X, \partial X, \delta_X) \Leftrightarrow (Y,
\partial Y, \delta_Y)$ and $(\ell_2, \partial \ell_2): (Y, \partial Y,
\delta_Y) \Leftrightarrow (Z, \partial Z, \delta_Z)$ are symmetric delta
lenses, then $(\ell_1;\ell_2, \partial \ell_1;\partial \ell_2)$ is a symmetric
delta lens.
\end{fact}
\begin{fact}[Dual of a Lens]
If $(\ell, \partial \ell)$ is symmetric delta lens, then $(\ell^{op}, \partial
\ell^{op})$ is a symmetric delta lens.
\end{fact}
\begin{fact}[Terminal Lens]
$(term_x, term_{\delta(x, x)})$ is a symmetric delta lens
\end{fact}
\begin{corollary}[Disconnect Lens]
If $x \in X$ and $y \in Y$, then $(term_x;term_y^{op},
term_{\delta(x, x)};term_{\delta(y, y)}^{op})$ is a symmetric delta lens.
\end{corollary}
\begin{fact}[Tensor Product Lens]
If $(\ell_1, \partial \ell_1)$ and $(\ell_2, \partial \ell_2)$ are symmetric
delta lenses, then $(\ell_1 \otimes \ell_2, \partial \ell_1 \otimes \partial
\ell_2)$ is a symmetric delta lens.
\end{fact}
\begin{corollary}[Projection Lens]
Projection lenses are symmetric delta lenses.
\end{corollary}
\begin{fact}[Forgetful Tensor Sum Lens]
If $(\ell_1, \partial \ell_1)$ and $(\ell_2, \partial \ell_2)$ are symmetric
delta lenses, then $(\ell_1 \oplus^f \ell_2, \partial \ell_1 \oplus^f \partial
\ell_2 \oplus (\ell_1 \otimes \ell_2))$ is a symmetric delta lens.
\end{fact}
\begin{fact}[Forgetful List Mapping Lens]
If $(\ell, \partial \ell)$ is a symmetric delta lens, then $(map^f(\ell)
\ell_2, map^f(\partial \ell))$ is a symmetric delta lens.
\end{fact}
\begin{remark}[Retentive Lenses]
The retentive tensor sum and list mapping lenses are in general not symmetric
delta lenses with the obvious definitions of difference functions.
\end{remark}
\fi
\section{Relational Lenses}
We define a {\em relation} $R$ of type $((T_1, \ldots, T_n), P, F)$ to be a
multiset $R$ where $T_i, P$ are sets satisfying $R \subseteq P \subseteq T_1
\times \ldots \times T_n$, $F \subseteq [n] \times [n]$, and if $i, j \in F$,
then for all $r, r' \in R$, if $r_i = r'_i$ then $r_j = r'_j$. We interpret the
$T_i$ to be a set of {\em attributes}, $R$ to be a set of {\em records} with
$dom(r) = (T_1, \ldots, T_n)$ for all $r \in R$, $P$ to be a {\em predicate}
such that the records in $R$ satisfy $P$, and $F$ to be a set of {\em
functional dependencies} on the attributes of the records in $R$.

Let $G(F)$ be the graph induced by $F$ on $T_1, \ldots, T_n$ i.e. there is an
edge from $T_i$ to $T_j$ if and only if $(i, j) \in F$. We say that $F$ is
in {\em tree form} if $G(F)$ is a DAG and every vertex has indegree at most 1;
following \cite{bohannon2006relational},we will always assume that $F$ is in
tree form from here on. Observe that every record $r$ is completely determined
by nodes of indegree 0 in $G(F)$: if $Roots(G(F)) = \{T_{i_1}, \ldots,
T_{i_k}\}$ is the set of nodes of indegree 0 and outdegree greater than 0 in
$G(F)$ and $Isolated(G(F)) = \{T_{j_1}, \ldots, T_{j_\ell}\}$ is the set of
nodes of indegree 0 and outdegree 0 in $G(F)$, then for any $r = (r_1, \ldots,
r_n)\in R$, then the map $((r_{i_1}, \ldots, r_{i_k}), (r_{j_1}, \ldots,
r_{j_\ell}))$ is a bijection since the $r_{i_1}, \ldots, r_{i_k}$ determine all
nodes of indegree greater than 0, which leaves just the nodes $r_{j_1}, \ldots,
r_{j_\ell}$.

In order to define lenses in relations, we first need to describe data
structures representing the relation; lenses will operate on the data
structures rather than on relations. Let $Outputs(F) = G(F) \setminus
(Roots(G(F)) \cup Isolated(G(F)))$. We choose to represent a relation with the
following data:
\begin{enumerate}
  \item
  A partial map $data : (T_{i_1} \times \ldots \times T_{i_k})
  \longrightarrow 2^{(T_{j_1} \times \ldots \times T_{j_{\ell}})}$ such that
  there exists a record $r \in R$ with $r[T_{i_1}, \ldots, T_{i_k}] =
  (t_{i_1}, \ldots, t_{i_k})$ if and only if
  $$data(t_{i_1}, \ldots, t_{i_k}) =  \{\{r[T_{j_1}, \ldots, T_{j_{\ell}}] \in
  R \; | \; r_{i_m} = t_{i_m} \text{ for }1 \leq m \leq k\}\}$$
  \item
  A map $dependencies : outputs(F) \longrightarrow \cup_{i,j} 2^{T_i \times
  T_j}$ such that $(p, q) \in dependencies(j)$ if and only if there exists a
  record $r \in R$ such that $r[T_i, T_j] = (p, q)$, where $i$ is such that
  $T_i \longrightarrow T_j \in F$ ($i$ is unique as $F$ is assumed to be in tree
  form).
\end{enumerate}
\subsection{Select}
We say that a predicate $P$ ignores a set of attributes $X$ if for all $r, r'
\in R$, if $r[dom(R) \setminus X] = r'[dom(R) \setminus X]$ implies that $r
\in P$ if and only if $r' \in P$.

The {\em select} lens on $R$ evaluates to a relation which is the same as $R$
but which satisfies a predicate $P$. The inferrence rule for deriving the select
lens is the following:
\begin{prooftree}
\AxiomC{$R : ((T_1, \ldots, T_2), Q, F)$}
\AxiomC{$S : ((T_1, \ldots, T_2), P \cap Q, F)$}
\AxiomC{$F$ is in tree form}
\AxiomC{$P$ ignores $Outputs(F)$}
\QuaternaryInfC{$(select \; from \; R\; where \; P \; as \; S) : R
\Leftrightarrow S$}
\end{prooftree}
In the get direction, the select lens simply restricts $R$ to $P$. Now consider
$put \; s \; r$. The select lens proceeds as follows:
\begin{enumerate}
  \item
  First, it computes $(data_s, dependencies_s)$ and $(data_r, dependencies_r)$.
  \item
  Next, it updates $dependencies_r$ using $dependencies_s$ to get
  $dependencies_r'$.
  \item
  Next it filters $data_r$ to contain only records not in $P$ to get $data_r'$.
  \item
  Next, it computes $(data_r' \cup data_s, dependencies_r')$.
  \item
  Finally, it deletes the records from $data_r' \cup data_s$ that would cause
  PUT-GET to fail.
\end{enumerate}
\section{Matching Lenses}
\begin{definition}
A {\em basic lens} $\ell : S \xLeftrightarrow{C} V$ is a 4-tuple of functions
$\ell.\get : S \longrightarrow V$, $\ell.\res : S \longrightarrow C$,
$\ell.\pput : V \longrightarrow C \longrightarrow S$ and $\ell.\create : V
\longrightarrow S$ satisfying the following properties:
\begin{align*}
\ell.\get \; (\ell.\pput \; v \; c) &= v \tag{PUTGET}\\
\ell.\pput \; (\ell.\get \; s) \; (\ell.\res \; s) &= s \tag{GETPUT}
\end{align*}
\end{definition}
Idea: First remove matched parts to leave unmatched skeleton. Next remove
unmatched parts to leave matched skeleton and then flatten.
%% Acknowledgments
\begin{acks}                            %% acks environment is optional
%% contents suppressed with 'anonymous'
%% Commands \grantsponsor{<sponsorID>}{<name>}{<url>} and
%% \grantnum[<url>]{<sponsorID>}{<number>} should be used to
%% acknowledge financial support and will be used by metadata
%% extraction tools.
This material is based upon work supported by the
\grantsponsor{GS100000001}{National Science
Foundation}{http://dx.doi.org/10.13039/100000001} under Grant
No.~\grantnum{GS100000001}{nnnnnnn} and Grant
No.~\grantnum{GS100000001}{mmmmmmm}.  Any opinions, findings, and
conclusions or recommendations expressed in this material are those
of the author and do not necessarily reflect the views of the
National Science Foundation.
\end{acks}
\fi
\bibliographystyle{plain}
\bibliography{local}
%% Appendix
%%\appendix
%%\section{Appendix}

\end{document}
\newif\ifdraft\drafttrue  % set true to show comments
\newif\ifplentyoftime\plentyoftimefalse  % :-)

% % For double-blind review submission, w/o CCS and ACM Reference (max
% submission space)
\documentclass[acmsmall,review,anonymous]{acmart}\settopmatter{printfolios=true,printccs=false,printacmref=false}
% % For double-blind review submission, w/ CCS and ACM Reference
% \documentclass[acmsmall,review,anonymous]{acmart}\settopmatter{printfolios=true}
% % For single-blind review submission, w/o CCS and ACM Reference (max
% submission space)
% \documentclass[acmsmall,review]{acmart}\settopmatter{printfolios=true,printccs=false,printacmref=false}
% % For single-blind review submission, w/ CCS and ACM Reference
% \documentclass[acmsmall,review]{acmart}\settopmatter{printfolios=true} % For
% final camera-ready submission, w/ required CCS and ACM Reference
% \documentclass[acmsmall]{acmart}\settopmatter{}


% % Journal information % Supplied to authors by publisher for camera-ready
% submission; % use defaults for review submission.
\acmJournal{PACMPL}
\acmVolume{1}
\acmNumber{POPL} % CONF = POPL or ICFP or OOPSLA
\acmArticle{1}
\acmYear{2018}
\acmMonth{1}
\acmDOI{} % \acmDOI{10.1145/nnnnnnn.nnnnnnn}
\startPage{1}

% % Copyright information % Supplied to authors (based on authors' rights
% management selection; % see authors.acm.org) by publisher for camera-ready
% submission; % use 'none' for review submission.
\setcopyright{none}
% \setcopyright{acmcopyright} \setcopyright{acmlicensed}
% \setcopyright{rightsretained} \ccopyrightyear{2018}           %% If different
% from \acmYear

% % Bibliography style %\bibliographystyle{ACM-Reference-Format} % Citation
% style % Note: author/year citations are required for papers published as an %
% issue of PACMPL.
% %\citestyle{acmauthoryear}   %% For author/year citations


% %%%%%%%%%%%%%%%%%%%%%%%%%%%%%%%%%%%%%%%%%%%%%%%%%%%%%%%%%%%%%%%%%%%%% % Note:
% Authors migrating a paper from PACMPL format to traditional % SIGPLAN
% proceedings format must update the '\documentclass' and % topmatter commands
% above; see 'acmart-sigplanproc-template.tex'.
% %%%%%%%%%%%%%%%%%%%%%%%%%%%%%%%%%%%%%%%%%%%%%%%%%%%%%%%%%%%%%%%%%%%%%


\usepackage{amsmath, amssymb, amsthm, enumerate, array, extarrows, mathrsfs,
mathtools, stmaryrd, listings, xspace, bussproofs}
\theoremstyle{definition}
\usepackage[capitalize]{cleveref}
\newtheorem{definition}{Definition}
\newtheorem{example}{Example}
\newtheorem{conjecture}{Conjecture}
\newtheorem{question}{Question}
\newtheorem{fact}{Fact}
\newtheorem{claim}{Claim}
\newtheorem{remark}{Remark}

% Macros Colors
\definecolor{dkblue}{rgb}{0,0.1,0.5}
\definecolor{dkgreen}{rgb}{0,0.6,0}
\definecolor{dkred}{rgb}{0.6,0,0}
\definecolor{dkpurple}{rgb}{0.7,0,0.4}
\definecolor{olive}{rgb}{0.4, 0.4, 0.0}
\definecolor{teal}{rgb}{0.0,0.5,0.5}
\definecolor{orange}{rgb}{0.9,0.6,0.2}
\definecolor{lightyellow}{RGB}{255, 255, 179}
\definecolor{lightgreen}{RGB}{170, 255, 220}
\definecolor{teal}{RGB}{141,211,199}
\definecolor{darkbrown}{RGB}{121,37,0}


\newcommand{\FINISH}[3]{\ifdraft\textcolor{#1}{[#2: #3]}\fi}
\newcommand{\bcp}[1]{\FINISH{dkred}{B}{#1}}
\newcommand{\BCP}[1]{\FINISH{dkred}{B}{\bf #1}}
\newcommand{\afm}[1]{\FINISH{dkgreen}{A}{#1}}
\newcommand{\dpw}[1]{\FINISH{dkblue}{D}{#1}} % Toronto Maple Leafs Blue :-)
\newcommand{\saz}[1]{\FINISH{orange}{SZ}{#1}}
\newcommand{\ksf}[1]{\FINISH{teal}{K}{#1}}
\newcommand{\sam}[1]{\FINISH{dkpurple}{SM}{#1}}


\newcommand{\kw}[1]{\ensuremath{\mathsf{#1}}\xspace}
\newcommand{\get}{\ensuremath{\kw{get}}\xspace}
\newcommand{\pput}{\ensuremath{\kw{put}}\xspace}
\newcommand{\create}{\ensuremath{\kw{create}}\xspace}
\newcommand{\res}{\ensuremath{\kw{res}}\xspace}
\newcommand{\ccopy}{\ensuremath{\kw{copy}}\xspace}
\newcommand{\const}{\ensuremath{\kw{const}}\xspace}
\newcommand{\default}{\ensuremath{\kw{default}}\xspace}
\newcommand{\length}{\ensuremath{\kw{length}}\xspace}
\newcommand{\match}{\ensuremath{\kw{match}}\xspace}

\DeclareMathOperator*{\argmin}{arg\,min} % thin space, limits underneath in
\DeclareMathOperator*{\argmax}{argmax} % no space, limits underneath in


\lstset{ basicstyle=\ttfamily, escapeinside=|| }
\begin{document}

% % Title information
\title{Lenses with a Difference}         %% [Short Title] is optional;
% % when present, will be used in % header instead of Full Title.
% %\titlenote{with title note}             %% \titlenote is optional; % can be
% repeated if necessary; % contents suppressed with 'anonymous'
% %\subtitle{Subtitle}                     %% \subtitle is optional
% %\subtitlenote{with subtitle note}       %% \subtitlenote is optional; % can
% be repeated if necessary; % contents suppressed with 'anonymous'


% % Author information % Contents and number of authors suppressed with
% 'anonymous'.
% % Each author should be introduced by \author, followed by % \authornote
% (optional), \orcid (optional), \affiliation, and % \email.
% % An author may have multiple affiliations and/or emails; repeat the %
% appropriate command.
% % Many elements are not rendered, but should be provided for metadata %
% extraction tools.

% % Author with single affiliation.
% \author{First1 Last1} \authornote{with author1 note}          %% \authornote
% is optional; % can be repeated if necessary \orcid{nnnn-nnnn-nnnn-nnnn} %%
% \orcid is optional \affiliation{ \positition{Position1}
% \department{Department1} %% \department is recommended
% \institution{Institution1} %% \institution is required \streetaddress{Street1
% Address1} \city{City1} \state{State1} \posittcode{Post-Code1}
% \country{Country1}                    %% \country is recommended }
% \email{first1.last1@inst1.edu}          %% \email is recommended

\author{Solomon Maina}
\position{PhD Student}
\department{Computer Science}              %% \department is recommended
\institution{University of Pennsylvania}            %% \institution is required
\country{USA}                    %% \country is recommended }
\email{smaina@seas.upenn.edu}          %% \email is recommended

% % Abstract % Note: \begin{abstract}...\end{abstract} environment must come %
% before \maketitle command
\begin{abstract}
Many of the bidirectional (bx) programming languages that exist today focus on
restoring consistency of models. Consequently, these languages often provide two
guarantees: (1) that a transformation restores consistency when an
update is made to one of the models, and (2) that models that are already
consistent are not updated by the transformation. Unfortunately, in situations
where there are many different ways of restoring consistency, these languages
do not constrain how a lens chooses to do so. As a result, the bx
transformations derived using these languages can lead to unexpected results. In
this paper we formalize a condition called {\em $\partial D$ preservation} that
describes whether a bx transformation propagates updates correctly with respect
to the relation $\partial D$, and then use this condition to characterize the
guarantees that various lens programming languages make with regard to update
propagation.
\end{abstract}

%% Keywords
%% comma separated list
\keywords{bidirectional programming, lenses}  %% \keywords
% are mandatory in final camera-ready submission

%% \maketitle
%% Note: \maketitle command must come after title commands, author
%% commands, abstract environment, Computing Classification System
%% environment and commands, and keywords command.
\maketitle

\section{Introduction}
By now there exist many programming languages that enable one to define {\em
bidirectional transformations}, programs that can be run in two
directions. Each of these languages provides guarantees on how the two
transformations interact e.g. that they are bijective, that a round-trip on
one or both directions does not cause non-trivial updates etc. Many of these
guarantees have to do with restoring or preserving consistency. To formalize
this statement, we describe a widely used formulation of bidirectional
transformations due to Stevens \cite{stevens2010bidirectional} but originally
articulated in a more general form by Meertens \cite{meertens1998designing}:
\begin{definition}
A bidirectional transformation (bx) $R: M \Leftrightarrow N$ comprises of a
consistency relation $R \subseteq M \times N$ and restorers $\overrightarrow{R}
: M \times N \rightarrow N$ and $\overleftarrow{R} : M \times N \longrightarrow
N$.
\end{definition}
\begin{definition}
A bx $R : M \Leftrightarrow N$ is {\em correct} if the consistency restorers do
restore consistency: that is, for all $m \in M, n \in N$, we have $R(m,
\overrightarrow{R}(m, n))$, and dually. $R$ is {\em hippocratic} if the
consistency restorers make no change to an already-consistent pair of models:
that is, for all $m \in M$, $n \in N$, if $R(m, n)$ then $\overrightarrow{R}(m,
n) = n$ and dually. $R$ is {\em well behaved} if it is correct and hippocratic.
\end{definition}

Many bidirectional programming languages thus enable one to define
well behaved bx's: consistency restoration is guaranteed by correctness of the
bx while consistency preservation is guaranteed by the hippocraticness of the
bx. This paper is primarily concerned with well behaved bx's where
$\overrightarrow{R}$, the \get function of the bx, ignores it's $N$-argument and
where the consistency relation $R$ is defined by $R(m, \overrightarrow{R}(m))$
for all $m \in M$--- these transformations are the well known and widely
understood {\em lenses} of Foster et al \cite{foster2007combinators}--- though
the ideas we develop could be used in other bidirectional settings as well. 

When the consistency relation of a bx $R$ is very fine, i.e. when each $m \in M$
is consistent with only a few $n \in N$ and vice versa, then a well behaved lens
gives fairly strong guarantees on $\overleftarrow{R}$ and $\overrightarrow{R}$.
However, there are many cases that arise in practice where there are many
different ways to restore consistency; when this happens, well-behavedness may
be too weak to rule out ``bad'' behaviour or to clarify the behaviour of a bx in
subtle cases. Various solutions have been proposed to address this issue, for
example {\em matching lenses} that deal with the issue of alignment
\cite{barbosa2010matching}, {\em edit lenses} \cite{hofmann2012edit} and {\em
delta lenses} \cite{diskin2011asymmetric,diskin2011state,pacheco2012delta}
that explicitly propagate edits through the lens, and {\em least change
lenses} that require that the source value resulting from a \pput application
is not only acceptable, but also one of closest to the original $s$ among the
sources that share the same view $v$ \cite{macedo2013composing}.

This paper takes a different approach to specifying update propagation by
identifying a condition called {\em $\partial D$ preservation} which we describe
below:
\begin{definition}
Let $R : M \Leftrightarrow N$ be a bidirectional transformation and $\partial D
\subseteq (M \times M) \times (N \times N)$. Then $R$ {\em preserves $\partial
D$} if for all $m \in M$, $n, n' \in N$ , if $R(m, n)$ then $\partial D((m,
\overleftarrow{R}(m, n')), (\overrightarrow{R}(m, n), n'))$. $R$ {\em
strongly preserves $\partial D$} if the dual condition also holds.
\end{definition}
\noindent The idea behind $\partial D$ preservation is that if $m$ and $n$ are
already consistent and $\overrightarrow{R}(m,n)$ is updated to produce $n'$,
then the difference between $\overrightarrow{R}(m,n)$ and $n'$ is related
to the difference between $m$ and $\overleftarrow{R}(m,n')$. $R$
strongly preserves $\partial D$ if it preserves $\partial D$ in both directions.

For lenses, $\partial D$ preservation says that difference between $\ell.\get \;
s$ and $v$ is related to the difference between $s$ and $\ell.\pput \; s \;
v$. One way to formalize this idea is to model the difference between two sources $s, s'
\in S$ and two views $v, v' \in V$ as functions $\delta_S : S \times S
\longrightarrow \partial S$ and $\delta_V : V \times V \longrightarrow \partial
V$ for some sets $\partial S$ and $\partial V$, and to capture the
relationship between $\delta_S$ and $\delta_V$ by a lens. In
particular, given a lens $\partial(S, V) : \partial S \Leftrightarrow \partial
V$ we will ask if the following equation holds of $\ell$:
\begin{equation}\label{difflenslaw}
\partial(S, V).\get \; (\delta_S(s, \ell.\pput \; s \; v)) = \delta_V(\ell.\get
\; s, v)
\end{equation}
If $\ell : S \Leftrightarrow V$ satisfies \cref{difflenslaw} we say that 
$\ell$ is a {\em difference  lens} with respect to $(\delta_S, 
\partial(S, V), \delta_V)$, which we denote with the typing relation 
$\ell : S \xLeftrightarrow{\delta_S, \partial(S, V), \delta_V} V$.

For example, consider two versions of an iteration lens $\ell_1 : S^*
\Leftrightarrow V^*$ and $\ell_2 : S^* \Leftrightarrow V^*$ defined inductively
from a lens $\ell : S \Leftrightarrow V$. Both of these lens have the same get
behaviour: given a list $s_1, \ldots, s_n$ of sources, they both produce
$\ell.\get \; s_1, \ldots, \ell.\get \; s_n$. The \pput behaviour of the two
lenses is different though: $\ell_1$ simply aligns views to sources
positionally, while $\ell_2$ computes the optimal alignment on the view side and
uses this alignment to put views back into sources.

Let \lstinline|name = [A-Z] . [a-z]*|, let the source $S$ equal 
\lstinline|name . " " . name| and the view $V$ equal \lstinline|name|.
Let $\ell : S \Leftrightarrow V$ be the lens
\lstinline|del name . del " " . copy name| which given a pair of names produces
the second name in the \get direction, and restores the first name in the \pput
direction:
$$\ell.\get \; ``Ada \; Turing" = ``Turing" \text{ and }\; \ell.\pput \; ``Ada
\; Turing" \; ``Lovelace" = ``Ada \; Lovelace"$$
Define the
function $\delta_V : V \times V \longrightarrow \mathbb{N}^{\geq 0}$ by 
$$\delta_V(n, n') = 
\begin{cases}
0 & \text{if }n=n'\\
1 & \text{otherwise}
\end{cases}$$
Define $\delta_S : S \times S \longrightarrow \mathbb{N}^{\geq 0}$ by
$$\delta_S(m \cdot `` \quad " \cdot n, m' \cdot `` \quad " \cdot n') =
\begin{cases}
0 & \text{if } m = m', n = n'\\
1 & \text{if } m = m', n \neq n'\\
1 & \text{if } m \neq m', n = n'\\
2 & \text{if } m \neq m', n \neq n'
\end{cases}$$
Let the lens $\partial(S, V) : \mathbb{N}^{\geq 0} \Leftrightarrow
\mathbb{N}^{\geq 0}$ be given by $\partial(S, V) = \ccopy(\mathbb{N}^{\geq
0})$. Then $\ell : S \xLeftrightarrow{\delta_S, \partial(S, V), \delta_V} V$.

Now let $xs, ys$ be two lists of single names, with $length(xs) \geq length(ys)$
Define $\delta_V^* : V^* \times V^* \longrightarrow \mathbb{Z}^* \times V^*$ on $xs$ and
$ys$ by aligning elements of $xs$ maximally with elements of $ys$ and then using $\delta_V$
on these pairs to produce a list $ds$ of numbers. In this process, some elements of
$ys$ will not be aligned if $length(xs) < length(ys)$, in which case we simply collect
these in another list of elements of $V$. For instance,
$$\delta_V^*([``Lovelace", ``Kleene", ``Turing"], [``Turing", ``Lovelace"]) = ([0,0], [``Kleene"])$$
since we align \lstinline|"Lovelace"| to \lstinline|"Lovelace"| and 
\lstinline|"Turing"| to \lstinline|"Turing"| for a difference of 0 in each case, which leaves
\lstinline|"Kleene"| unaligned.

On the source side, let $ss, ts$ be two lists of two names with $length(ss) \geq length(ts)$.
Define $\delta_{S, \ell}^* : S^* \times S^* \longrightarrow \mathbb{Z}^* \times S^*$ by taking the 
alignment used to compute $\delta_V^*(\ell^*.\get \; ss, \ell^*.\get \; ts)$ and then using that
alignment in the same manner that $\delta_V^*$ does. For instance
\begin{align*}
&\delta_{S, \ell}^*([``Ada \; Lovelace", ``Stephen \; Kleene", ``Alan \; Turing"], 
[``Alan \; Turing", ``Ada \; Lovelace"])\\
&= ([0,0], [``Stephen \; Kleene"])
\end{align*}
Recall that the \pput behaviour of $\ell_1$ simply aligns views to sources
positionally, while $\ell_2$ computes the optimal alignment on the view side and
uses this alignment to put views back into sources. From this we can prove that
$$\ell_2 : S^* \xLeftrightarrow{\delta_{S, \ell}^*, \partial(S, V)^* \times \ell^*, \delta_V^*} V^*$$
On the other hand, the judgement $\ell_1 : S^* \xLeftrightarrow{\delta_{S, \ell}^*, 
\partial(S, V)^* \times \ell^*, \delta_V^*} V^*$ does not holds since $\ell_1$ does not satisfy \cref{difflenslaw} :
$$(\partial(S, V)^* \times \ell^*).\get \; (\delta_{S, \ell}^*(s, \ell.\pput \; s \; v)) = \delta_V^*(\ell.\get
\; s, v)
$$
This is witnessed by the fact that if $s = [``Ada \; Lovelace", ``Stephen \; Kleene", ``Alan \; Turing"]$ and \\
$v = [``Kleene", ``Turing", ``Lovelace"]$, then 
\begin{align*}
&\ell_1.\pput \; [``Ada \; Lovelace", ``Stephen \; Kleene", ``Alan \; Turing"] \; [``Kleene", ``Turing", ``Lovelace"]\\
&= [``Ada \; Kleene", ``Stephen \; Turing", ``Alan \; Lovelace"]
\end{align*}
Also, $\delta_{S, \ell}^*(s, \ell.\pput \; s \; v) = ([1,1,1], [])$, $\delta_V^*(\ell.\get
\; s, v) = ([0,0,0], [])$, but
\begin{align*}
(\partial(S, V)^* \times \ell^*).\get \; (\delta_{S, \ell}^*(s, \ell.\pput \; s \; v)) &=
(\partial(S, V)^* \times \ell^*).\get \; ([1,1,1], [])\\
&= ([1,1,1], [])\\
&\neq ([0,0,0], [])\\
&= \delta_V^*(\ell.\get \; s, v)
\end{align*}
Here the positional lens $\ell_1$ is not a difference lens with respect to $(\delta_{S, \ell}^*, 
\partial(S, V)^* \times \ell^*, \delta_V^*)$ since it does not align sources to views in the same way that
$\delta_V^*$ does. On the other hand, $\ell_2$ does, hence is a difference lens with respect to $(\delta_{S, \ell}^*, 
\partial(S, V)^* \times \ell^*, \delta_V^*)$. 

In the rest of this paper, we give concrete examples of how to use $\partial D$ preservation to characterize 
the guarantees on update propagation provided by various lens programming languages. In particular,
the major technical contributions that we will make in this paper are the following:
\begin{enumerate}
  \item
  We formalize a condition called $\partial D$ preservation which provides a way
  of constraining the updates that can be made by a bidirectional
  transformation. We give examples of relations $\partial D$ as well as
  examples and counterexamples of lenses that are $\partial D$ preserving.
  \item
  We describe an instance of $\partial D$ preservation that is satisfied by {\em difference lenses}. We give a set of combinators for {\em difference lenses} and describe the update propagation guarantees that the lenses derived using these combinators give.
  \item
  Finally, we show how to represent matching lenses and relational lenses
  \cite{bohannon2006relational} as difference lenses, and hence
  describe non-trivial guarantees beyond correctness and hippocraticness that
  these lens languages provide as well.
\end{enumerate}
\iffalse
A {\em lens} is a program that can be run in two directions. In the
forward direction, the \get component of a lens produces a {\em view}
from a {\em source}, while in the backward direction, the \pput component folds
an updated view back into the original source. Formally, Foster
\cite{foster2007combinators} defined a lens $\ell : S
\Leftrightarrow V$ from a source $S$ to a view $V$ to be a pair of functions
$\ell.\get : S \longrightarrow V$ and $\ell.\pput : S \longrightarrow V
\longrightarrow S$ satisfying the
following properties:
\begin{align*}
\ell.\get \; (\ell.\pput \; s \; v) &= v \tag{PUTGET}\\
\ell.\pput \; s \; (\ell.\get \; s) &= s \tag{GETPUT}
\end{align*}
The PUTGET law ensures that updates to the view are translated
``exactly''---i.e. that, given a view, the \pput function produces a source
that \get maps back to the very same view, while the GETPUT law ensures a
``stability'' property for the source---i.e. it requires that the \pput
function returns the original source unmodified whenever the update to the view
is a no-op.

Unfortunately, while these laws give some guarantees on how the \get and \pput
components of a lens interact, they do not specify the behaviour of the \pput
component when the update to the view is non-trivial. Consequently, a
non-trivial update to a view could lead to an unexpected update to the source.
Various solutions have been proposed to address this issue, for example {\em
matching lenses} to deal with the issue of alignment \cite{barbosa2010matching},
{\em edit lenses} \cite{hofmann2012edit} and {\em delta lenses}
\cite{diskin2011asymmetric,diskin2011state,pacheco2012delta} which explicitly
propagate edits through the lens, and {\em least change lenses} which
add a new LEASTPUTGET law which requires that the source value resulting from a
\pput application is not only acceptable, but also one of closest to the
original $s$ among the sources that share the same view $v$
\cite{macedo2013composing}.

This paper describes a general method for specifying what updates are allowable
for a given lens without requiring that the lens explicitly propagate updates.
Our approach is guided by the simple idea the idea that if one starts with a
source $s$, calls the \get function on $s$ to acquire the view $\ell.\get \;
s$, updates $\ell.\get \; s$ to get a new view $v$, then the ``difference''
between $v$ and $\ell.\get \; s$ and $v$ should be related to the difference
between $\ell.\pput \; s \; v$ and $v$.

To formalize this idea, we first recall a widely used definition of
bidirectional transformations
\cite{meertens1998designing,stevens2010bidirectional} definition of a
bidirectional transformations:
\begin{definition}
A bidirectional transformation (bx) $R: M \Leftrightarrow N$ comprises of a
consistency relation $R \subseteq M \times N$ and restorers $\overrightarrow{R}
: M \times N \rightarrow N$ and $\overleftarrow{R} : M \times N \longrightarrow
N$.
\end{definition}
\begin{definition}
A bx $R : M \leftrightarrow N$ is {\em correct} if the consistency restorers do
restore consistency: that is, for all $m \in M, n \in N$, we have $R(m,
\overrightarrow{R}(m, n))$, and dually. $R$ is {\em hippocratic} if the
consistency restores make no change to and already-consistent pair of models:
that is, for all $m \in M$, $n \in N$, if $R(m, n)$ then $\overrightarrow{R}(m,
n) = n$ and dually. $R$ is {\em well behaved} if it is correct and hippocratic.
\end{definition}

Observe that $\ell : S \Leftrightarrow V$ is a lens, let $R$ be the
consistency relation given by $R(s, \ell.\get \; s)$ for all $s \in
S$, $\overrightarrow{\ell}(s, v) = \ell.\get \; s$, and $\overleftarrow{\ell} =
\ell.\pput$. By the PUTGET law, $\ell.\get \; (\ell.\pput \; s \; v) = v$,
hence $\ell$ is correct, and by the GETPUT laws, $\ell.\pput \; s \; (\ell.\get
\; s) = s$ and $\ell.\get \; (\ell.\create \; v) = v$, hence $\ell$ is
hippocractic. Therefore $\ell$ is a well-behaved bx.

If $R$ is a well-behaved bx then $R$ restores consistency when a source or view
is updated (correctness) and preserves consistency when the source and view are
already consistent (hippocraticness). In order to express

We will then apply this method to define a typing judgement for lenses which
constrains how a lens propagates updates. We will describe a compositional
language of {\em difference lenses} which satisfy this new typing judgement,
and give examples of ordinary lenses that are not difference lenses. Finally,
we will show how to represent matching lenses and relational lenses
\cite{bohannon2006relational} as difference lenses, and hence describe the
guarantees that these lens languages provide.


For example suppose that the source is an XML document representing the
names, dates and nationalities of a collection of classical music
composers such as,

\begin{lstlisting}
<composers>
<composer>
<name>Jean Sibelius</name>
<dates>1865-1956</dates>
<nationality>Finnish</nationality>
</composer>
<composer>
<name>Aaron Copland</name>
<dates>1910-1990</dates>
<nationality>American</nationality>
</composer>
<composer>
<name>Benjamin Briten</name>
<dates>1913-1976</dates>
<nationality>English</nationality>
</composer>
</composers>
\end{lstlisting}

while the view is a list if ASCII text representing the names and dates of each
composer:
\begin{lstlisting}
Jean Sibelius, 1865-1956
Aaron Copland, 1910-1990
Benjamin Briten, 1913-1976
\end{lstlisting}

After computing the initial view, we might want to edit it in some way---e.g.,
correcting the error in Sibelius's death date and the misspelling in Britten's
name
\begin{lstlisting}
Jean Sibelius, 1865-|\colorbox{orange}{1957}|
Aaron Copland, 1910-1990
Benjamin |\colorbox{orange}{Britten}|, 1913-1976
\end{lstlisting}
and push the changes back into the original XML format:
\begin{lstlisting}
<composers>
<composer>
<name>Jean Sibelius</name>
<dates>1865-|\colorbox{orange}{1957}|</dates>
<nationality>Finnish</nationality>
</composer>
<composer>
<name>Aaron Copland</name>
<dates>1910-1990</dates>
<nationality>American</nationality>
</composer>
<composer>
<name>Benjamin |\colorbox{orange}{Britten}|</name>
<dates>1913-1976</dates>
<nationality>English</nationality>
</composer>
</composers>
\end{lstlisting}

Lenses are designed to perform these kinds of bidirectional transformations. If
$\ell$ is a suitably defined lens for the transformation that we
have just described, then the \get component of the lens performs the
XML to ASCII transformation that produces an ASCII view from an XML source,
while the \pput component of the lens performs the ASCII to XML transformation
that folds back an updated ASCII view back into the original XML source.


For example, suppose that a source $S$ consists of a pair consisting of a
boolean value and an integer, and that the view $V$ for $S$ consists of a
single integer. Further, suppose that we wish to define a \pput function for a
lens $\ell$ whose \get component is the function which given a pair $(b, n)$
simply produces the boolean value $b$. Let $f : \mathbb{B} \times \mathbb{Z}
\longrightarrow \mathbb{Z}$ be any function from $\mathbb{B} \times \mathbb{Z}
\longrightarrow \mathbb{Z}$. Then the lens $\ell_{f}$ defined by
\begin{align*}
\ell_{f}.\get \; (b, n) &= b\\
\ell_{f}.\pput \; (b, n) \; b' &= \begin{cases}
(b', n) & \text{if }b = b'\\
(b', f(b', n)) & \text{otherwise}\\
\end{cases}\\
\ell.\create \; b &= (b,f(b, 0))
\end{align*}
is a valid lens.

A user who applies a lens $\ell$ of type $\ell : S \Leftrightarrow V$ may expect
that $\ell.\pput \; (true, 1) \; false = (false, 0)$, reasoning that the
integer produced by the \pput function should be the number representation of
the boolean value $false$. Or perhaps she might expect that $\ell.\pput \;
(true, 1) \; false = (false, 1)$, reasoning that the two pieces of data in a
pair are independent and that the reasonable \pput behaviour in this case is to
restore the orginial integer. In either case, the user may be shocked to find
that in fact $\ell.\pput (true, 0) \; false = (false, 48)$! Given the \get
behaviour which just produces the boolean value from a pair, since $\ell_f$ is a
valid lens for any $f$, then the \pput function of a valid lens of type
$\mathbb{B} \times \mathbb{Z} \Leftrightarrow \mathbb{Z}$ can restore the
integer in a pair in any way so long as the old boolean value and the updated
boolean value are different.


In order to give a more constrained specification for the desired
lens we define the difference $\delta_S((a, b), (c, d))$ between $(a, b)$ and
$(c, d)$ by $$\delta_S((a, b), (c,
d)) = \begin{cases}
0 & \text{if }a = c, b = d\\
1 & \text{if }a=c, b \neq d\\
1 & \text{if }a \neq c, b = d\\
2 & \text{if }a \neq c, b \neq d
\end{cases}$$
On the view side, we define the difference $\delta_V(a, b)$
between $a$ and $b$ by, $$\delta_V(a, b) = \begin{cases}
0 & \text{if } a = b\\
1 & \text{if } a \neq b
\end{cases}$$

Next we define a lens $\partial(S, V) : \partial S \Leftrightarrow \partial V$
which describes how differences should be propagated. For this example, we
define $\partial(S,V)$ to simply be the \ccopy lens from $\partial S$ to
$\partial V$, keeping in mind that $\partial S = \partial V = \mathbb{Z}$.
\cref{difflenslaw} enables us to rule out $\ell_{bad}$, since
\begin{align*}
(\partial (S, V)).\get \; (\delta_S(\ell_{bad}.\pput \; (1,2) \; 3, (1,2))) &=
(\partial (S, V)).\get \; (\delta_S((0,3), (1,2)))\\
&= (\partial (S, V)).\get \; 2\\
&= 2\\
&\neq 1\\
&= \delta_V(3, 2)\\
&= \delta_V(3, \ell_{bad}.\get \; (1, 2))
\end{align*}
Consequently $\ell_{bad}$ is not a valid delta lens. On the
other hand, the lens $\ell_{good} : S \Leftrightarrow V$ defined by
\begin{align*}
\ell_{good}.\get \; (a, b) &= b\\
\ell_{good}.\pput \; (a, b) \; c &= (a, c)\\
\ell_{good}.\create \; b &= (0, b)
\end{align*}
is indeed a valid delta lens since the its \pput component always restores the
first factor.

\section{Strengthening Lens Laws}
\begin{definition}
A bidirectional transformation (bx) $R: M \Leftrightarrow N$ comprises of a
consistency relation $R \subseteq M \times N$ and restorers $\overrightarrow{R}
: M \times N \rightarrow N$ and $\overleftarrow{R} : M \times N \longrightarrow
N$.
\end{definition}
\begin{definition}
A bx is {\em correct} if the consistency restorers do restore consistency:
that is, for all $m \in M, n \in N$, we have $R(m, \overrightarrow{R}(m, n))$,
and dually.
\end{definition}
\begin{definition}
A bx is {\em hippocratic} if the consistency restores make no change
to and already-consistent pair of models: that is, for all $m \in M$, $n \in N$,
if $R(m, n)$ then $\overrightarrow{R}(m, n) = n$ and dually.
\end{definition}
\begin{definition}
A bx is {\em well behaved} if it is correct and hippocratic.
\end{definition}
For lenses, if $\ell : S \Leftrightarrow V$ is a lens, we take the
consistency relation $R$ to be given by $R(s, \ell.\get \; s)$ for all $s \in
S$, $\overrightarrow{\ell}(s, v) = \ell.\get \; s$, and $\overleftarrow{\ell} =
\ell.\pput$. By the PUTGET law, $\ell.\get \; (\ell.\pput \; s \; v) = v$,
hence $\ell$ is correct, and by the GETPUT laws, $\ell.\pput \; s \; (\ell.\get
\; s) = s$ and $\ell.\get \; (\ell.\create \; v) = v$, hence $\ell$ is
hippocractic. Therefore $\ell$ is a well-behaved bx.

Using the notation of Stevens \cite{stevens2010bidirectional},



The major shortcoming of this model of bx transformations is that it does not
take into account how changes should be propagated by transformation; a
well-behaved bx need only restore consistency when models go out of sync or
maintain consisteny if models are already in sync. This becomes an issue when
there are many different ways of restoring consistency

\section{Motivation and Related Work}
Foster et. al. \cite{foster2007combinators,foster2009bidirectional}
defined a lens $\ell : S \Leftrightarrow V$ from a source $S$ to a view $V$ to
be a triple of functions $\ell.\get : S \longrightarrow V$, $\ell.\pput : S
\longrightarrow S \longrightarrow V$ and $\ell.\create : V \longrightarrow S$
satisfying the following properties:
\begin{align*}
\ell.\get \; (\ell.\pput \; s \; v) &= v \tag{PUTGET}\\
\ell.\get \; (\ell.\create \; v) &= v \tag{CREATEGET}\\
\ell.\pput \; s \; (\ell.\get \; s) &= s \tag{GETPUT}
\end{align*}
While these two laws give some guarantees on how the \get, \pput and \create
components of a lens interact, they do not specify the behaviour of the \pput
component when the update to the view is non-trivial, because the GETPUT law
only says that the \pput component must return the original source unmodified
whenever the update to the view is a no-op. Consequently, a non-trivial update
to a view could lead to an undesired update to the source.

Therefore, several lens formalism have been proposed that explicitly have a
notion of an {\em update}. The first of these was Barbosa et. al.'s {\em
matching lenses}\cite{barbosa2010matching}. Matching lenses specifically deal
with the issue of {\em alignment}, since Foster et. al.'s original iteration
lens updates records positionally:
\begin{align*}
\ell^* .\pput \; (s_1, \ldots, s_m) \; (v_1, \ldots, v_n) &= t_1 \cdot
\ldots \cdot t_n\\
\text{ where } t_i &= \begin{cases}
\ell.\pput \; s_i \; v_i & \text{if } 1 \leq i \leq \min\{m, n\}\\
\ell.\create \; v_i & \text{if } \min\{m, n\} < i \leq n
\end{cases}\\
\end{align*}
In their work, Barbosa et. al. enrich the types of lenses with ``chunks''
identifying reorderable pieces of the source and view that should be re-aligned
after an update, and formulate behavioral laws that capture essential
constraints on the handling of chunks. They develop a core language
of matching lenses for strings, together with a set of ``alignment
combinators'' that implement a variety of alignment strategies. The obvious
shortcoming of matching lenses is that they only deal with alignment; they do
not solve the general issue of update propagation.

The next line of work is Diskin et. al.'s {\em delta
lenses}\cite{diskin2011asymmetric,diskin2011state}. Diskin et. al.'s delta
lenses work as follows: let $\ell : S \Leftrightarrow V$ be a lens from
$S$ to $V$. Assume that $s$ is a source, and that $v = \ell.\get s$. A user
performs an edits $d$ on $v$ to get $v'$: $v \xrightarrow{d} v'$. The \pput
function now takes the view edit $d$ and the original source $s$ and produces a
new source $s'$: $s \xrightarrow{\pput(d, s)} s'$. One thing that Diskin et. al.
do not do is that they do not give any concrete examples of delta lenses,
no composition, no notion of equivalence, and no combinators for constructing
delta lenses.

The next line of work is Hofmann et. al.'s {\em edit
lenses}\cite{hofmann2012edit}. Edit lenses are similar to delta lenses, but edit
lenses are more explicit. Specifically, the source and view edits are
represented as elements of a {\em monoid}, and the lens acts on sources and
views via monoid actions. That is if $v$ is a view and $d_v$ is an edit that
can be made on $v$ to produce $v'$, then $d_v \cdot v = v'$. The intended usage
of an edit lens is as follows. There are two users, one holding an element of
$S$ the other one an element of $V$ both referred to as {\em replicas}.
Initially, they hold $init_S \in S$ and $init_V \in V$ respectively. The users
then perform actions and propagate them across the lens. An action consists of
producing an edit $d_s$ (or $d_v$), applying it to one's current replica $s$
(resp. $v$), putting the edit through the lens to obtain an edit $d_v$ (resp.
$d_s$), and asking the user on the other side to apply $d_v$ (resp. $d_s$) to
their replica. One thing that Hofmann et. al. do that Diskin et. al. do not do
is that they do give concrete examples of edit lenses, define a suitable notion
of composition on them, and define combinators for constructing edit lenses.

The final line of work in this area is Pacheco et. al.'s
version of delta lenses\cite{pacheco2012delta}. Pacheco et. al. are the
first to give concrete examples of delta lenses, define a suitable notion
of composition on them, and define combinators for constructing delta
lenses. Pacheco et al. define a delta lens $\ell$ (d-lens for short), denote by
$\ell: S \trianglerighteq_{\blacktriangle} V$ to be a bidirectional
transformations that comprises four total functions:
\begin{align*}
\get &: S \longrightarrow V\\
\pput &: \forall(v, s) : V \times S. (v \triangle \get \;s) \longrightarrow
SA\\
\get_{\blacktriangle} &: \forall\{s':S, s:S\}. (s' \triangle s)
\longrightarrow (\get \; s')\triangle (\get \; s)\\
\pput_{\blacktriangle} &:\forall\{(v, s):V \times S\},d:(v \triangle \get
\;s).(\pput \; (v, s) \; d) \triangle s
\end{align*}
that satisfy the following propertes:
\begin{align*}
\get \; (\pput \; (v, s) \; d) = v \tag{PUTGET}\\
\pput \; (\get \; s, s) \; id = s \tag{GETPUT}\\
\get_{\blacktriangle} \; (\pput_{\blacktriangle} \; d) = d
\tag{$\text{PUTGET}_{\blacktriangle}$}\\
\pput_{\blacktriangle} \; id = id \tag{$\text{PUTGET}_{\blacktriangle}$}
\end{align*}
The \get function is a just like a \get function in Foster's et. al.'s original
work. The \pput function is different though. Let $s \in S$ and $v \in V$, and
let $d$ be an update from $v$ to $\ell.\get \; s$; for example, $d$ is a
sequence of insertions and deletions that transforms $v$ to $\ell.\get \; s$.
Then \pput takes in $s$ and $v$ as input, and in addition receives $d$ to
compute $\pput \; (v, s) \; d$.

The $\get_{\blacktriangle}$ and $\pput_{\blacktriangle}$ functions are functions
that transform deltas to deltas. The last two axioms for delta lenses say that
if $d$ is a delta from $v$ to $\ell.\get \; s$ and $\Delta$ is a delta from
$\pput \; (v, s) \; d$ to $s$, then $\get_{\blacktriangle} \; \Delta = d$, and
that if $d$ is the identity delta, then $\Delta$ is also the identity delta.

In addition to describing a delta lens language, Pacheco et. al. also
address the following shortcoming with edit lenses: edit lenses include
combinators for constructing lenses over {\em container types}.
In the simplest case, container types can be thought of as algebraic datatypes
such as lists and trees; each value of a container type has a certain {\em
shape}, and each shape has a set of {\em positions} associated with it. For
instance, the shape of a list of size $n$ is just $n$, while the positions of
such a list are the numbers $0, 1, \ldots, n-1$.

The shortcoming of the container edit lenses addressed by Pacheco et. al.
elements is that these lenses always preserve the number of positions; for
instance if $\ell$ is a container lens that maps lists to trees, and if $xs$ is
a list of size $n$, then the number of positions of $\ell.\get \; xs$ is also
$n$. Pacheco et. al. define a new set of combinators for container lenses that
can say map a list of size $5$ to a tree with 4 nodes.

In this article we show that it is not necessary for the \pput component of a lens
to explicitly propagate updates: instead, we can specify what the put function should
evaluate to, by imposing a more precise type judgement on lenses.

The idea of the stronger typing judgement is that we embed sources and views inside
of {\em difference structures} and then we define a lens that describes how the
differences behave as they cross in-between the source and the view. A
lens that is validly typed is then a lens that preserves this overall structure.

Here are the actual definitions:
\begin{definition}
A weighted set is a pair $(X, |\cdot|)$ where $X$ is a set and $|\cdot| : X
\longrightarrow \mathbb{R}^{\geq 0}$ is a map from $X$ to the non-negative
reals.
\end{definition}
\begin{definition}
A difference structure is a triple $(S, \partial S, \delta_S)$ where $S$ and
$\partial S$ are weighted sets, and $\delta_S:
S \times S \longrightarrow \partial S$ is a function satisfying
\begin{align*}
\delta(s, s') &= \delta(s', s)\tag{A1}\\
|\delta(s, s)| &= 0\tag{A2}\\
|\delta(s, s')| &\leq |s| + |s'|\tag{A3}
\end{align*}
\end{definition}
\begin{definition}
Let $(S, \partial S, \delta_S)$ and $(V, \partial V, \delta_V)$ be
difference structures. Let $\partial(S, V) : \partial S \Leftrightarrow \partial
V$ be a lens from $\partial S$ to $\partial V$. Then a lens $\ell : S
\Leftrightarrow V$ also has type $\ell : (S, \partial S,
\delta_S) \xLeftrightarrow{\partial(S, V)} (V, \partial V, \delta_V)$ if for
all $s \in S, v \in V$,
\begin{equation}\label{difflenslaw}
(\partial (S, V)).\get\; (\delta_S(\ell.\pput \; s \; v, s)) = \delta_V(v,
\ell.\get \; s)
\end{equation}
\end{definition}
The extra difference structure on the sources and the views, as well as the lens
$\partial(S, V)$ force the lens $\ell$ to satisfy \cref{difflenslaw}; this law is
therefore a specification that constrains what $\ell$ can do.

For example, suppose that a source $S$ consists of a pair of integers, and that the
view $V$ for $S$ consists of a single integer. Further, suppose that we wish
for the \get component of any lens from $S$ to $V$ to simply project away the
first factor:that is for any $\ell : \mathbb{Z} \times \mathbb{Z}
\Leftrightarrow \mathbb{Z}$, $\ell.\get \; (a, b) = b$ for all integers $a, b$.
Suppose further that we wish to force the \pput function of any lens $\ell$
defined on the data to restore the factor projected away by the \get function:
that is, we wish to enforce that for all integers $a, b, c$, $\pi_1(\ell.\pput
\; (a, b) \; c) = a$, where $\pi_1$ is the projection on the first factor.

Now consider the lens $\ell_{bad}$ defined by
\begin{align*}
\ell_{bad}.\get \; (a, b) &= b\\
\ell_{bad}.\pput \; (a, b) \; c &= \begin{cases}
(a, c) & \text{if }b = c\\
(0, c) & \text{otherwise}\\
\end{cases}\\
\ell.\create \; b &= (0,b)
\end{align*}
Then $\ell_{bad}$ is a valid lens even though $\ell_{bad}$ does not
always restore the first factor. For example, observe that $\ell.\pput (1, 2) \;
3 = (0,3)$. In order to give a more constrained specification for the desired
lens we define the difference $\delta_S((a, b), (c, d))$ between $(a, b)$ and
$(c, d)$ by $$\delta_S((a, b), (c,
d)) = \begin{cases}
0 & \text{if }a = c, b = d\\
1 & \text{if }a=c, b \neq d\\
1 & \text{if }a \neq c, b = d\\
2 & \text{if }a \neq c, b \neq d
\end{cases}$$
On the view side, we define the difference $\delta_V(a, b)$
between $a$ and $b$ by, $$\delta_V(a, b) = \begin{cases}
0 & \text{if } a = b\\
1 & \text{if } a \neq b
\end{cases}$$

Next we define a lens $\partial(S, V) : \partial S \Leftrightarrow \partial V$
which describes how differences should be propagated. For this example, we
define $\partial(S,V)$ to simply be the \ccopy lens from $\partial S$ to
$\partial V$, keeping in mind that $\partial S = \partial V = \mathbb{Z}$.
\cref{difflenslaw} enables us to rule out $\ell_{bad}$, since
\begin{align*}
(\partial (S, V)).\get \; (\delta_S(\ell_{bad}.\pput \; (1,2) \; 3, (1,2))) &=
(\partial (S, V)).\get \; (\delta_S((0,3), (1,2)))\\
&= (\partial (S, V)).\get \; 2\\
&= 2\\
&\neq 1\\
&= \delta_V(3, 2)\\
&= \delta_V(3, \ell_{bad}.\get \; (1, 2))
\end{align*}
Consequently $\ell_{bad}$ is not a valid delta lens. On the
other hand, the lens $\ell_{good} : S \Leftrightarrow V$ defined by
\begin{align*}
\ell_{good}.\get \; (a, b) &= b\\
\ell_{good}.\pput \; (a, b) \; c &= (a, c)\\
\ell_{good}.\create \; b &= (0, b)
\end{align*}
is indeed a valid delta lens since the its \pput component always restores the
first factor.

For a more realistic example using a difference lens combinator, let $S
= name \cdot dates$ and $V = name$. Let $\ell :
S \Leftrightarrow V = (copy \; name) \cdot (del \; dates)$, and suppose that we
wish to define a lens of type $S^* \Leftrightarrow V^*$, where the difference
function used on the view side is the min-edit distance difference function.
Assuming that we inductively use the difference function
$$ \delta(d_1, d_2) =
\begin{cases}
0 & \text{if }d_1 = d_2 \\
1 & \text{otherwise}
\end{cases}
$$
for both the name and date fields, then with the min-edit distance difference
function, the difference between the string $v$ given by
\begin{lstlisting}
John Doe
Jane Smith
\end{lstlisting}
and the string $v'$ given by
\begin{lstlisting}
Jane Smith
John Doe
\end{lstlisting}
is the empty list. Let $s$ =
\begin{lstlisting}
John Doe,1900-2000
Jane Smith,1901-2001
\end{lstlisting}Then $\ell.\get \; s = v$. Now assume that $v$ is edited to $v'$
given by
\begin{lstlisting}
Mark Moe
John Doe
Jane Smith
\end{lstlisting}
where a new record \lstinline|Mark Moe| has been added to $v$.
With the min-edit distance on the view side, $\delta(v', \ell.\get \; s)$ is
the singleton list [``Mark Moe''].

On the source side, the min-edit distance matching lens $\ell^*_{MIN-EDIT}$,
$\ell^*_{MIN-EDIT}.\pput \; s \; v'$ is the string
\begin{lstlisting}
Mark Moe,0000-0000
John Doe,1900-2000
Jane Smith,1901-2001
\end{lstlisting}
while with the positional lens $\ell^*$, then $\ell^*.\pput \; s \; v'$ is the
string
\begin{lstlisting}
Mark Moe,1900-2000
John Doe,1901-2001
Jane Smith,0000-0000
\end{lstlisting}
With the min-edit distance difference function,
$\delta(\ell^*_{MIN-EDIT}.\pput \; s \; v', s)$ is the singleton list
$[\text{``Mark Moe, 0000-0000''}]$, while with the
positional lens, $\delta(\ell^*.\pput \; s \; v', s)$ is the list
given by
\begin{lstlisting}
Mark Moe,1900-2000
(0,1)
(0,1)
\end{lstlisting}
The min-edit distance satisfies \cref{difflenslaw} since
\begin{align*}
(\partial (S, V))^*.\get\; (\delta_S(\ell^*_{MIN-EDIT}.\pput \; s \; v', s)) &=
(\partial (S, V))^*.\get\; ([\text{``Mark Moe, 1900-2000''}])\\
&= [\text{``Mark Moe''}]\\
&=
\delta_V(v', \ell.\get \; s)
\end{align*}
On the other hand, the positional lens does not satisfy \cref{difflenslaw} since
\begin{align*}
(\partial (S, V))^*.\get\; (\delta_S(\ell^*.\pput \; s \; v', s)) &=
(\partial (S, V))^*.\get\; ([\text{``Mark Moe, 1900-2000''}, (0,1), (0,1)])\\
&=[\text{``Mark Moe''}, (0,*), (0,*)]\\
& \neq \delta_V(v', \ell.\get \; s)
\end{align*}
Let $\mathcal{M}_V$ be the set of min-edit distance alignments on the view side,
and $\mathcal{M}_S$ the corresponding alignments on the source side.
Then, the typing judgement
$$\ell^*_{MIN-EDIT}:(S, (\partial S \sqcup S)^*, \match(\mathcal{M}_S, \delta_s))
\xLeftrightarrow{\partial(S, V)^*} (V, (\partial V \sqcup V)^*, \match(\mathcal{M}_V, \delta_v))$$
is valid, whereas the typing judgement
$$\ell^*:(S, (\partial S \sqcup S)^*, \match(\mathcal{M}_S, \delta_s))
\xLeftrightarrow{\partial(S, V)^*} (V, (\partial V \sqcup V)^*, \match(\mathcal{M}_V, \delta_v))$$
is not.
\iffalse
For example, if $\ell : S \Leftrightarrow V$ is a lens, Foster's initial
definition of the iteration lens $\ell^* : S^* \Leftrightarrow V^*$ was
\begin{align*}
\ell^* .\get \; (s_1, \ldots, s_n) &= (\ell.\get \; s_1) \cdot \ldots
\cdot (\ell.\get \; s_n)\\
\ell^* .\pput \; (s_1, \ldots, s_m) \; (v_1, \ldots, v_n) &= t_1 \cdot
\ldots \cdot t_n\\
\text{ where } t_i &= \begin{cases}
\ell.\pput \; s_i \; v_i & \text{if } 1 \leq i \leq \min\{m, n\}\\
\ell.\create \; v_i & \text{if } \min\{m, n\} < i \leq n
\end{cases}\\
\ell^*.\create \; (v_1, \ldots, v_n) &= (\ell.\create \; v_1) \cdot \ldots
\cdot (\ell.\create \; v_n)
\end{align*}
This lens always propagates updates positionally, which is unintended behaviour
in many cases. For example suppose that $\ell$ is a lens that transforms
source data in the form of an XML document representing the names, dates and
nationalities of a collection of classical music composers such as,

\begin{lstlisting}
<composers>
<composer>
<name>Jean Sibelius</name>
<dates>1865-1957</dates>
<nationality>Finnish</nationality>
</composer>
<composer>
<name>Aaron Copland</name>
<dates>1910-1990</dates>
<nationality>American</nationality>
</composer>
<composer>
<name>Benjamin Britten</name>
<dates>1913-1976</dates>
<nationality>English</nationality>
</composer>
</composers>
\end{lstlisting}
into view data stoored in ASCII text representing the names and dates of each
composer:
\begin{lstlisting}
Jean Sibelius, 1865-1957
Aaron Copland, 1910-1990
Benjamin Britten, 1913-1976
\end{lstlisting}
Now suppose that the elements in the view are reordered:
\begin{lstlisting}
Aaron Copland, 1910-1990
Benjamin Britten, 1913-1976
Jean Sibelius, 1865-1957
\end{lstlisting}
Then with the positional iteration lens, the updated source is
\begin{lstlisting}
<composers>
<composer>
<name>Aaron Copeland</name>
<dates>1910-1990</dates>
<nationality>Finnish</nationality>
</composer>
<composer>
<name>Benjamin Britten</name>
<dates>1913-1976</dates>
<nationality>American</nationality>
</composer>
<composer>
<name>Jean Sibelius</name>
<dates>1865-1957</dates>
<nationality>English</nationality>
</composer>
</composers>
\end{lstlisting}
so each of the composers now has the wrong nationality in the source.

\section{Difference functions}
\begin{definition}
A weighted set is a pair $(X, |\cdot|)$ where $X$ is a set and $|\cdot| : X
\longrightarrow \mathbb{R}^{\geq 0}$ is a map from $X$ to the non-negative
reals.
\end{definition}
\begin{definition}
A difference structure is a triple $(S, \partial S, \delta_S)$ where $S$ and
$\partial S$ are weighted sets, and $\delta_S:
S \times S \longrightarrow \partial S$ is a function satisfying
\begin{align*}
\delta(s, s') &= \delta(s', s)\tag{A1}\\
|\delta(s, s)| &= 0\tag{A2}\\
|\delta(s, s')| &\leq |s| + |s'|\tag{A3}
\end{align*}
We refer to the function $\delta$ as the difference function of the structure.
\end{definition}
\iffalse

\begin{example}
Let $X$ be any set. Define $x \leq y$ if and only if $x=y$, and define $|X|=1$
for all $x \in X$. Let $Q = \{0,1\}$, and define
$$\delta(a, b) = \begin{cases} 1 & \text{if } a \neq b\\
0 & \text{otherwise}
\end{cases}$$
Then $\delta$ is a difference function.
\end{example}

Various different edit distances on strings give a way of defining delta
functions on sets of strings:
\begin{example}
The {\em Levenshtein distance} between two words is the minimum number of
single-character edits (insertions, deletions or substitutions) required to
change one word into the other. For example, the Levenshtein distance between
"kitten" and "sitting" is 3, since the following three edits change one into
the other, and there is no way to do it with fewer than three edits:

\begin{enumerate}
  \item
  kitten -> sitten (substitution of "s" for "k")
  \item
  sitten -> sittin (substitution of "i" for "e")
  \item
  sittin -> sitting (insertion of "g" at the end).
\end{enumerate}
\end{example}
\begin{example}
In contrast, the Longest Common Subsequence(LCS) distance only allows for
insertions and deletions:
\begin{enumerate}
  \item
  delete k at 0
  \item
  insert s at 0
  \item
  delete e at 4
  \item
  insert i at 4
  \item
  insert g at 6
\end{enumerate}
for a total cost/distance of 5 operations.
\end{example}
Both Levenshtein and LCS distances are integer-valued distance functions since
they are metrics.
\begin{example}
If all the strings in a set are of the same length, then the {\em Hamming
distance} is a difference function since the Hamming distance is also a metric.
\end{example}
\begin{example}
Alignment algorithms also give us a way of defining integer-valued distance
functions on sets of strings by taking the difference of two strings to be the
total number of characters that are not equal according to the alignment.

When deciding which alignment to choose between two strings, we often wish to
choose an alignment that is optimal for alignment strategy. With
the positional alignment strategy, given two strings $s$ and $t$, then $s_1$ is
matched to $t_1$, $s_2$ is matched to $t_2$ and so on.
Also, $\mathcal{M}_{POS}$ is an optimal set of alignments for a positional
alignment strategy.

For example, using the positional alignment strategy on the strings "kitten" and
"sitting", then "k" is matched to "s", "i" to "i" and so on. The characters
which are included in the delta from "kitten" are "k", "e" while the
in the characters included from "sitting" are "s", "i" and "g", hence the delta
is equal to 5.

For the non-crossing strategy, we choose among alignments which are increasing
rather than positional, while for the min-edit distance strategy, we consider
all possible alignments.
\end{example}

\begin{example}\label{naturalnumbers}
Let $P$ be the set of non-negative integers with weight function $|a| = a$. Then
$\delta$ is a difference function.
Define $\delta(a, b) = |a-b|$. Then $\delta$ is a difference function.
\end{example}
\begin{example}
Let $X$ be a normed vector space and define $\delta(x, y) = |x-y|$. Then
$\delta$ is a difference function (in fact, $\delta$ is also a metric).
\end{example}
\begin{example}\label{divisibility}
Let $P$ be the set of positive integers with weight function $|\cdot|$ defined
by $|p_1^{a_1} \cdot \ldots \cdot p_n^{a_n}| = a_1 + \ldots + a_n$. Define
$\delta(a, b) = ab/(gcd(a, b))^2$.
Then $\delta$ is a difference function.
\end{example}
\begin{example}\label{boolean}
Let $X$ be a set and $P$ the set of finite subsets of $X$ with weight function
given by the size of the set. Define $\delta(A, B) = (A \setminus B) \cup (B
\setminus A)$. Then $\delta$ is a difference function.
\end{example}
\begin{example}\label{general}
Generalizing \cref{divisibility} and
\cref{boolean}, let $P$ be a graded complete semi-modular lattice satisfying
$|\bigwedge P|=0$. Define $\delta(a, b) = \bigwedge \{x \; | \; (a \wedge b)
\vee x = a\} \vee \bigwedge \{x \; | \; (a \wedge b) \vee x = b\}$. Then
$\delta$ is a difference function.
\end{example}
\begin{proof}
Let $a, b \in P$. Let $X = \{x \; | \; (a \wedge b) \vee x = a\}$ and $Y = \{y
\; | \; (a \wedge b) \vee y = b\}$
\begin{enumerate}
  \item[(A1)]
  Certainly $\delta(a, b) = \delta(b, a)$.
  \item[(A2)]
  Observe that $\bigwedge X = \bigwedge Y = \bot$, hence $\delta(a,
  a) = \bot$, and $|\delta(a, a)| = |\bot| = 0$.
  \item[(A3)]
  Since $P$ is semi-modular, then $|x \wedge y| + |x \vee y| \leq |x| + |y|$
  for all $x, y \in P$, from which it follows that $$|\delta(a, b)| =
  \left|\bigwedge X \vee \bigwedge Y\right| \leq \left|\bigwedge X \vee
  \bigwedge Y\right| + \left|\bigwedge X \wedge \bigwedge Y\right| \leq
  \left|\bigwedge X\right| + \left|\bigwedge Y\right|$$
\end{enumerate}
\end{proof}
\fi

\subsection{Constructions on difference functions}
\subsubsection{Products}
Given difference functions $\delta_i: P_i \times P_i \longrightarrow Q_i$ for
$i \in \{1,2\}$, define $\delta_1 \times \delta_2 :(P_1 \times P_2) \times
(P_1 \times P_2) \longrightarrow (Q_1 \times Q_2)$ by
$$(\delta_1 \times \delta_2)((x, y), (x', y')) = (\delta_1(x, x'), \delta_2(y,
y'))$$
\begin{claim}
$\delta_1 \times \delta_2$ is a difference function.
\end{claim}
\begin{proof}
Let $x, x' \in P_1$ and $y, y' \in P_2$.
\begin{enumerate}
  \item[(A1)]
  Observe that
  \begin{align*}
(\delta_1 \times \delta_2)((x, y), (x', y')) &= (\delta_1(x, x'),
\delta_2(y, y'))\\
&= (\delta_1(x', x), \delta_2(y', y))\\
&= (\delta_1 \times \delta_2)((x', y'), (x, y))
\end{align*}
\item[(A2)] Observe that
$$|(\delta_1 \times \delta_2)((x, y), (x, y))| = |(\delta_1(x, x),
\delta_2(y, y))| = |\delta_1(x, x)| + |\delta_2(y,y)| = 0 + 0 = 0$$
\item[(A3)]
Observe that
\begin{align*}
|(\delta_1 \times \delta_2)((x, y), (x', y'))| &= |(\delta_1(x, x'),
\delta_2(y, y'))|\\
&= |(\delta_1(x, x')| + |(\delta_2(y, y')|\\
&\leq (|x| + |x'|) + (|y| + |y'|)\\
&= (|x| + |y|) + (|x'| + |y'|)\\
&= |(x, y)| + |(x', y')|
\end{align*}
\end{enumerate}
\end{proof}
\subsubsection{Coproducts}
Given difference functions $\delta_i: P_i \times P_i \longrightarrow Q_i$ for
$i \in \{1,2\}$, define $\delta_1 \sqcup \delta_2 :P_1 \sqcup P_2
\longrightarrow Q_1 \sqcup Q_2 \sqcup (P_1 \times P_2)$ by
$$(\delta_1 \sqcup \delta_2)(x, y) =
\begin{cases}
\delta_1(x, y) & \text{if }x, y \in P_1\\
\delta_2(x, y) & \text{if }x, y \in P_2\\
(x, y) & \text{if }x \in P_1, y \in P_2\\
(y, x) & \text{if }x \in P_2, y \in P_1
\end{cases}$$
\begin{claim}
$\delta_1 \sqcup \delta_2$ is a difference function
\end{claim}
\begin{proof}
Let $x, x' \in P_1$ and $y, y' \in P_2$.
\begin{enumerate}
  \item[(A1)]
  If $x, y \in P_i$, then $(\delta_1 \sqcup \delta_2)(x, y) = \delta_i(x, y) =
  \delta_i(y, x)$. If $x \in P_1$ and $y \in P_2$, then $(\delta_1 \sqcup
  \delta_2)(x, y) = (x, y) = (\delta_1 \sqcup \delta_2)(y, x)$, and similarly
  if $x \in P_2$ and $y \in P_1$, then $(\delta_1 \sqcup \delta_2)(x, y) =
  (y,x) = (\delta_1 \sqcup \delta_2)(y, x)$
  \item[(A2)] If $x \in P_i$, then $|(\delta_1 \sqcup \delta_2)(x, x)| =
  |\delta_i(x, x)| = 0$
  \item[(A3)]
  If $x, y \in P_i$, then $|(\delta_1 \sqcup \delta_2)(x, y)| = |\delta_i(x, y)|
  \leq |x| + |y|$. If $x \in P_1$ and $y \in P_2$, then $|(\delta_1 \sqcup
  \delta_2)(x, y)| = |(x, y)| = |x| + |y|$, and if $x \in P_2$ and $y \in
  P_1$, then $|(\delta_1 \sqcup \delta_2)(x, y)| = (|(y, x)|) = |y| + |x|$.
\end{enumerate}
\end{proof}
\subsubsection{List difference functions}
Let $P$ be a weighted set. Let $x=(x_1, \ldots, x_m), y=(y_1, \ldots, y_n)$ be
in $P^*$. Given a difference function $\delta: P \times P \longrightarrow Q$, we
define the list difference function $\delta^*:P^* \longrightarrow (Q \sqcup
P)^*$ on $x$ and $y$ to be the list given by $\delta(x_i, y_i)$ for $1 \leq i
\leq m$, except that we omit $\delta(x_i, y_i)$ in position $i$ if $x_i = y_i$.
For $m < i \leq n$ we add $y_i$ at position $i$. Finally we collapse the list
to get rid of the empty spaces left after the omissions.

For instance suppose that $S$ is the set of all letters in the alphabet and
$\partial S = \{\epsilon\} \sqcup \{\{x, y\} \; | \; x, y \in S, x \neq y\}$
with $$\delta(x, y) =
\begin{cases}
\epsilon & \text{if } x = y\\
\{x,y\} & \text{if } x \neq y
\end{cases}$$
Then $\delta_S^*((a, c, b, d), (a, b, c, d, e)) = (\{b, c\}, \{b, c\},
e)$ where we match and remove the first $a$ and the fourth $d$ in. Since $b \neq
c$ we have $\{b,c\}$ in the second and third positions, and since $e$ is
unmatched, we include $e$ in the fifth position. Finally we collapse the list to
get rid of the two empty spaces created.

\begin{claim}
$\delta^*$ is a difference function
\end{claim}
\begin{proof}
Let $x = (x_1, \ldots, x_m), y = (y_1, \ldots, y_n) \in P^*$.
\begin{enumerate}
  \item[(A1)]
  By construction $\delta^*(x, y) = \delta^*(y, x)$.
  \item[(A2)] Observe that $|\delta^*(x, x)| = \sum_{i=1}^m|\delta(x_i,
  x_i)| =  \sum_{i=1}^m 0 = 0$.
  \item[(A3)]
  Assume WLOG that $m \leq n$. Then
  \begin{align*}
|\delta^*(x, y)| &=
\sum_{i=1}^m|\delta(x_i, y_i)| + \sum_{i=m+1}^n |y_i|\\
&\leq \sum_{i=1}^m|x_i| + |y_i| + \sum_{i=m+1}^n |y_i|\\
&= \sum_{i=1}^m|x_i| + \sum_{i=1}^n |y_i|\\
&= |x| + |y|
\end{align*}
\end{enumerate}
\end{proof}
\subsubsection{Matching difference functions}
Now let $\mathcal{M} = \{\varphi_{(x_1, \ldots, x_m), (y_1, \ldots, y_n)} \; |
\; x_i, y_j \in P, m \leq n\}$ be a set of injective functions $\varphi_{(x_1,
\ldots, x_m), (y_1, \ldots, y_n)} : [m] \longrightarrow [n]$. Given a difference
function $\delta:
P \times P \longrightarrow Q$, we define the difference function
$\match(\mathcal{M}, \delta):P^* \longrightarrow (Q \sqcup P)^*$ on $x$
and $y$ to be the list given by $\delta(x_i, y_{\varphi(i)})$ in position
$\varphi(i)$ for $1 \leq i \leq m$, except that we omit $\delta(x_i,
y_{\varphi(i)})$ in position $\varphi(i)$ if $x_i = y_{\varphi(i)}$. For $i \in
[n] \setminus \varphi([m])$ we add $y_i$ at position $i$. Finally we collapse
the list to get rid of the empty spaces left after the omissions.

Going back to our previous example, let $x = \{a, c, b, d\}$ and $y = \{a, b, c,
d, e\}$, and assume that $\varphi_{x, y} : [4] \longrightarrow [5]$ is given by
$$\varphi(1) = 1, \; \varphi(2) = 3, \; \varphi(3) = 2, \text{ and } \varphi(4)
= 4$$ Then $\match(\mathcal{M}, \delta)((a, c, b, d), (a, b, c, d, e)) =
(e)$.
\begin{claim}
$\match(\mathcal{M}, \delta)$ is a difference function.
\end{claim}
\begin{proof}
Let $x = \{x_1, \ldots, x_m\}, y = \{y_1, \ldots, y_n\} \in \match(P)$.
\begin{enumerate}
  \item[(A1)]
  By construction $\match(\mathcal{M}, \delta)(x, y) =
  \match(\mathcal{M}, \delta)(y, x)$.
  \item[(A2)] Observe that $|(\match(\mathcal{M}, \delta))(x, x)| =
  \sum_{i=1}^m|\delta(x_i, x_{\varphi(i)}| = |\delta(x_i, x_i)| =   \sum_{i=1}^m
  0 = 0$.
  \item[(A3)]
  Assume WLOG that $m \leq n$. Then
  \begin{align*}
|\match(\mathcal{M}, \delta)(x, y)| &=
\sum_{i=1}^m|\delta(x_i, y_{\varphi(i)})| + \sum_{i \in [n] \setminus
\varphi([m])} |y_i|\\
&\leq \sum_{i=1}^m|x_i| + |y_{\varphi(i)}| + \sum_{i \in [n] \setminus
\varphi([m])} |y_i|\\
&= \sum_{i=1}^m|x_i| + \sum_{i=1}^n |y_i|\\
&= |x| + |y|
\end{align*}
\end{enumerate}
\end{proof}
\fi
\section{Lens Combinators}
\subsection{Copy}
Let $P$ be any set. Define the $\ccopy(P) : P \Leftrightarrow P$ by
\begin{align}
\ccopy(P).\get \; s &= s\\
\ccopy(P).\pput \; s \; v &= v\\
\ccopy(P).\create \; v &= v
\end{align}
Then $\ccopy(P)$ is a lens. Now let $\partial S$ be a set, $\delta :
S \times S \longrightarrow \partial S$ a function.
\begin{claim}
$\ccopy(S) : S \xLeftrightarrow{\delta , \ccopy(\partial S), \delta} S$.
\end{claim}
\begin{proof}
We need to show that $(\ccopy(\partial S)).\get \; (\delta(\ccopy(S).\pput
\; s \; v, s)) = \delta(v, \ccopy(S).\get \; s)$ which follows from the fact
that $$ (\ccopy(\partial S)).\get \; (\delta(\ccopy(S).\pput \; s
\; v, s)) = \delta(\ccopy(S).\pput \; s \; v, s) = \delta(v, \ccopy(S).\get
\; s) $$
\end{proof}
\subsection{Constant}
Let $S$ be any set with $s'\in S$. Define the \const lens $\const(S) : S
\Leftrightarrow \{\star\}$ by
\begin{align*}
\const.\get \; s &= \star\\
\const.\pput \; s \; v &= s\\
\const.\create \; v &= s'
\end{align*}
Then $\const(S)$ is a lens. 
\begin{claim}
Let $\partial S$ be any set, and $\delta : S \times S \longrightarrow \partial
S$ any function satisfying $\delta(s, s) = \delta(s', s')$ for all $s, s' \in
S$. Define $\beta : \{\star\} \times \{\star\} \longrightarrow \partial S$ by
$\beta(\star, \star) = \delta(s, s)$ for any $s \in S$. Let $\ell : \partial S
\Leftrightarrow \partial S$ be any lens satisfying $\ell.\get \; \delta(s, s) =
\delta(s, s)$ for all $s \in S$. Then $\const(S): S \xLeftrightarrow{\delta,
\ell, \beta} \{\star\}$.
\end{claim}
\begin{proof}
We need to show that $\ell.\get \; (\delta(\const(S).\pput \; s \; \star, s)) =
\beta (\star, \const(S).\get \; s)$
which follows from the fact that
$$ \ell.\get \; (\delta(\const(S).\pput \; s \; \star, s)) = \ell.\get
\; \delta(s, s) = \delta(s, s) = \beta(\star, \star) = \beta(\star,
\const(S).\get \; s)$$
\end{proof}

\subsection{Default}
Let $\ell : S \Leftrightarrow V$ be a lens. Let $f :
V \longrightarrow S$ be any function. Define the lens $\default(\ell, f) :
S \Leftrightarrow V$ by
\begin{align*}
\default(\ell, f).\get \; s &= \ell.\get \; s\\
\default(\ell, f).\pput \; s \; v &= \ell.\pput \; s \; v\\
\default(\ell, f).\create \; v &= \ell.\pput \; (f \; v) \; v
\end{align*}
 
\begin{claim}
If $\ell : S \xLeftrightarrow{\delta_S, \partial (S, V), \partial V}
V$ and $f: S \longrightarrow V$, then $\default(\ell, f) : S
\xLeftrightarrow{\delta_S, \partial (S, V), \partial V} V$.
\end{claim}
\begin{proof}
Follows from the fact that $\ell : S \xLeftrightarrow{\delta_S, \partial (S, V),
\partial V} V$.
\end{proof}
\subsection{Product}
Let $\ell_1 : S_1 \Leftrightarrow V_1$ and $\ell_2 : S_2 \Leftrightarrow V_2$ be
lenses. Define the lens $\ell_1 \times \ell_2 : (S_1 \times S_2) \Leftrightarrow
(V_1 \times V_2)$ by
\begin{align*}
(\ell_1 \times \ell_2).\get \; (s_1, s_2) &= (\ell_1.\get \; s_1, \ell_2.\get
\; s_1)\\
(\ell_1 \times \ell_2).\pput \; (s_1, s_2) \; (v_1, v_2) &= (\ell_1.\pput \; s_1
\; v_1, \ell_2.\pput \; s_2 \; v_2)\\
(\ell_1 \times \ell_2).\create \; (v_1, v_2) &= (\ell_1.\create \; v_1,
\ell_2.\create \; v_2)
\end{align*}
\begin{claim}\label{productislens}
If $\ell_1 :S_1 \xLeftrightarrow{\delta_{S_1}, \partial (S_1, V_1),
\delta_{V_1}} V_1$ and $\ell_2 : S_2 \xLeftrightarrow{\delta_{S_2}, \partial
(S_2, V_2), \delta_{V_2}} V_2$, then \\
$\ell_1 \times \ell_2  : S_1 \times S_2 
\xLeftrightarrow{\delta_{S_1} \times \delta_{S_2}, \partial (S_1, V_1) \times
\partial (S_2, V_2), \delta_{V_1} \times \delta_{V_2}} V_1 \times V_2$.
\end{claim}
\begin{proof}
We need to show that
$$(\partial (S_1, V_1) \times \partial (S_2, V_2)).\get \; (\delta((\ell_1 \times
\ell_2).\pput \; (s_1, s_2) \; (v_1, v_2), (s_1, s_2))) = \delta((v_1, v_2),
(\ell_1 \times \ell_2).\get \; (s_1, s_2))$$ which follows from the fact that
\begin{align*}
&(\partial (S_1, V_1) \times \partial (S_2, V_2)).\get \; (\delta((\ell_1 \times
\ell_2).\pput \; (s_1, s_2) \; (v_1, v_2), (s_1, s_2)))\\
&= (\partial (S_1, V_1) \times \partial (S_2, V_2)).\get \;
(\delta((\ell_1.\pput \; s_1 \; v_1, \ell_2.\pput \; s_2 \; v_2), (s_1, s_2))\\
&= (\partial (S_1, V_1) \times \partial (S_2, V_2)).\get \;
(\delta(\ell_1.\pput \; s_1 \; v_1, s_1), \delta(\ell_2.\pput \; s_2 \; v_2,
s_2))\\
&= ((\partial (S_1, V_1)).\get \; (\delta(\ell_1.\pput \;
s_1 \; v_1, s_1)), (\partial (S_2, V_2)).\get \; (\delta(\ell_2.\pput \; s_2 \;
v_2, s_2)))\\
&= (\delta(v_1, \ell_1.\get \; s_1), \delta(v_2, \ell_2.\get \; s_2))\\
&= \delta((v_1, v_2), (\ell_1.\get \; s_1, \ell_2.\get \; s_2))\\
&= \delta((v_1, v_2), (\ell_1 \times \ell_2).\get \; (s_1, s_2))
\end{align*}
\end{proof}
\subsection{Sum}
Let $\ell_1 : S_1 \Leftrightarrow V_1$ and $\ell_2 : S_2 \Leftrightarrow V_2$ be
lenses. Define the lens
$\ell_1 \sqcup \ell_2 :
(S_1 \sqcup S_2) \Leftrightarrow (V_1 \sqcup V_2)$ by
\begin{align*}
(\ell_1 \sqcup \ell_2).\get \; s &=
\begin{cases}
\ell_1.\get \; s & \text{if } s \in S_1\\
\ell_2.\get \; s & \text{if } s \in S_2\\
\end{cases}\\
(\ell_1 \sqcup \ell_2).\pput \; s \; v&=
\begin{cases}
\ell_1.\pput \; s \; v& \text{if } s \in S_1, \; v \in V_1\\
\ell_2.\pput \; s \; v& \text{if } s \in S_2, \; v \in V_2\\
\ell_1.\create \; v & \text{if } s \in S_2, \; v \in V_1\\
\ell_2.\create \; v & \text{if } s \in S_1, \; v \in V_2\\
\end{cases}\\
(\ell_1 \sqcup \ell_2).\create \; v &=
\begin{cases}
\ell_1.\create \; v & \text{if } v \in V_1\\
\ell_2.\create \; v & \text{if } v \in V_2\\
\end{cases}
\end{align*}

\begin{claim}
If $\ell_1 : S_1  \xLeftrightarrow{\delta_{S_1}, \partial (S_1, V_1),
\delta_{V_1}} V_1$ and $\ell_2 : S_2 \xLeftrightarrow{\delta_{S_2}, \partial
(S_2, V_2), \delta_{V_2}} V_2$, then \\
$\ell_1 \sqcup \ell_2 : S_1 \sqcup S_2 \xLeftrightarrow{\delta_{S_1} \sqcup
\delta_{S_2}, \partial (S_1, V_1) \sqcup \partial (S_2, V_2) \sqcup (\ell_1
\times \ell_2), \delta_{V_1} \sqcup \delta_{V_2}} V_1 \sqcup V_2$
\end{claim}
\begin{proof}
We need to show that
$$(\partial (S_1, V_1) \sqcup \partial (S_2, V_2)).\get \;
(\delta((\ell_1 \sqcup \ell_2).\pput \; s \; v, s)) = \delta(v, (\ell_1 \sqcup
\ell_2).\get \; s)$$

The result holds inductively if $s \in S_1$ and $v \in V_1$ or $s \in S_2$ and
$v \in V_2$. Otherwise, assume without loss of generality
that $s \in S_1$ and $v \in V_2$. Then
\begin{align*}
(\partial (S_1, V_1) \sqcup \partial (S_2, V_2)).\get \;
(\delta((\ell_1 \sqcup \ell_2).\pput \; s \; v, s)) &=
(\partial (S_1, V_1) \sqcup \partial (S_2, V_2)).\get \; (\delta(\ell_2.\create
\; v, s))\\
&= (\ell_1 \times \ell_2).\get \; (s, \ell_2.\create \; v)\\
&= (\ell_1.\get \; s, \ell_2.\get \; (\ell_2.\create \; v))\\
&= (\ell_1.\get \; s, v)\\
&= \delta(v, \ell_1.\get \; s)\\
&= \delta(v, (\ell_1 \sqcup \ell_2).\get \; s)
\end{align*}
\end{proof}
\subsection{Positional Iteration}
Let $\ell : S \Leftrightarrow V$ be a lens and define the lens $\ell^*:
S^* \Leftrightarrow V^*$ by
\begin{align*}
\ell^* .\get \; (s_1, \ldots, s_n) &= (\ell.\get \; s_1) \cdot \ldots
\cdot (\ell.\get \; s_n)\\
\ell^*.\pput \; (s_1, \ldots, s_m) \; (v_1, \ldots, v_n) &= s'_1 \cdot
\ldots \cdot s'_n\\
\text{ where } s'_i &= \begin{cases}
\ell.\pput \; s_i \; v_i & \text{if } 1 \leq i \leq \min\{m, n\}\\
\ell.\create \; v_i & \text{if } \min\{m, n\} < i \leq n
\end{cases}\\
\ell^*.\create \; (v_1, \ldots, v_n) &= (\ell.\create \; v_1) \cdot \ldots
\cdot (\ell.\create \; v_n)
\end{align*}

\begin{claim}\label{positionalisvalid}
If $\ell: S \xLeftrightarrow{\delta_S, \partial (S, V), \delta_V}
V$, then $\ell^*: S^* \xLeftrightarrow{\delta_{S}^*, (\partial (S, V) \sqcup
\ell)^*, \delta_{V}^*} V^*$.
\end{claim}
\begin{proof}
We need to show that
\begin{multline}
(\partial (S, V) \sqcup \ell)^*.\get \;
(\delta_S^*(\ell^*.\pput \; (s_1, \ldots, s_n) \; (v_1, \ldots,
v_m), (s_1, \ldots, s_n))) \\
= \delta_V^*((v_1, \ldots, v_m), \ell^*.\get \; (s_1, \ldots,
s_n))
\end{multline}
Let $1 \leq i \leq \min\{m, n\}$. Observe that $v_i = \ell.\get \; s_i$ if and
only if $\ell.\pput \; s_i \; v_i = s_i$, so $\delta_V^*$ omits
$\delta_V(v_i, \ell.\get \; s_i)$ in position $i$ if and only if
$\delta_S^*$ also omits $\delta_S(\ell.\pput \; s_i \; v_i, s_i)$ in
position $i$; if no omission occurs, then $\delta(S, V).\get \;
(\delta(\ell.\pput \; s_i \; v_i, s_i)) = \delta(v_i, \ell.\get \; s_i)$ since
$(\ell, \partial(S,V))$ is a delta lens. For $\min\{m,n\} < i \leq m$, then
$\delta_V^*$ adds $v_i$ in position $i$, and $\delta_S^*$ adds
$\ell.\create \; v_i$ in position $i$, which gives the desired result since
$$(\partial (S, V) \sqcup \ell).\get \; (\ell.\create \; v_i) = \ell.\get \;
(\ell.\create \; v_i)) = v_i$$
\end{proof}
\subsection{Matching Iteration}
\iffalse
When defining a $\match(\mathcal{M}, \delta)$ difference function, we often wish for
$\mathcal{M}$ to be an optimal set of matchings for some alignment strategy.

For instance, define the set of matchings $\mathcal{M}_{POS}$ by
$$\mathcal{M}_{POS} = \{\varphi_{(a_1, \ldots, a_m), (b_1, \ldots, b_n)}:[m]
\longrightarrow [n] \; | \; m \leq n, \varphi_{(a_1, \ldots, a_m), (b_1,
\ldots, b_n)}(i) = i\}$$ Then $\mathcal{M}_{POS}$ is an optimal set of
matchings for a positional alignment strategy.

For the non-crossing strategy, we want to our matchings to be increasing rather
than positional. That is, given a valid set of matchings $\mathcal{M}$, let
$\mathcal{M}_{a, b} = \varphi_{a, b} \in \mathcal{M}$, and for $a, a' \in P$
with $a=(a_1, \ldots, a_m)$ and $a'=(a'_1, \ldots, a'_n)$, define
$\mathcal{M}^{NC}_{a, b}$ by
$$\mathcal{M}^{NC}_{a, a'} = \argmin_{\substack{\mathcal{M}_{a, a'} \\
\mathcal{M}_{a, a'} \text{ is increasing}}} |\match(\mathcal{M}_{a, a'},
\delta)(a, a')|$$
There may be multiple matchings which for which the size of the delta is
minimal, in which case we pick the lexicographically smallest matching.
Then $\mathcal{M}^{NC}$ is an optimal set of matchings for a non-crossing
alignment strategy.

Finally, for a min-edit distance strategy, we simply remove the condition that
the matchings should be increasing. That is, define
$\mathcal{M}^{MED}_{a, b}$ by
$$\mathcal{M}^{MED}_{a, a'} = \argmin_{\mathcal{M}_{a, a'}}
|\match(\mathcal{M}_{a, a'}, \delta)(a, a')|$$
Then $\mathcal{M}_{MED}$ is an optimal set of matchings for a min-edit distance
alignment strategy.
\fi
Assume that $\ell: S \Leftrightarrow V$. Let $\mathcal{M}_V = \{\varphi_{x, y}
\; | \; x, y \in V^*\}$ be a set of alignments between $x$ and $y$ for each $x,
y \in V^*$. For each $s, s' \in S^*$, let $\varphi_{s, s'} = \varphi_{\ell.\get
\; s, \ell.\get \; s'}$. Define $\mathcal{M}_S$ by $\mathcal{M}_S =
\{\varphi_{s, s'} \; | \; s, s' \in S^*\}$. Define the lens
$\match(\mathcal{M}_V, \ell) : S^* \Leftrightarrow V^*$ by
\begin{align*}
\match(\mathcal{M}_V, \ell) .\get \; (s_1, \ldots, s_n) &= (\ell.\get \; s_1)
\cdot \ldots \cdot (\ell.\get \; s_n)\\
\match(\mathcal{M}_V, \ell) .\pput \; (s_1, \ldots, s_m) \; (v_1, \ldots, v_n)
&= (s'_1 , \ldots , s'_n)\\
\text{ where } s'_i &= \begin{cases}
\ell.\pput \; s_{\psi(i)} \; v_i & \text{if } i \in Dom(\psi)\\
\ell.\create \; v_i & \text{if } i \not \in Dom(\psi)
\end{cases}\\
\text{ and } \psi &= \varphi_{(v_1, \ldots, v_n), ((\ell.\get \; s_1), \ldots,
(\ell.\get \; s_m))}\\
\match(\mathcal{M}_V, \ell).\create \; (v_1, \ldots, v_n) &= (\ell.\create \;
v_1) \cdot \ldots \cdot (\ell.\create \; v_n)
\end{align*}
Assume that 
\begin{claim}
If $\ell: S \xLeftrightarrow{\delta_{S}, \partial (S, V), \delta_{V}}
V$, then $\match(\mathcal{M}_V, \ell) : S^*
\xLeftrightarrow{\match(\mathcal{M}_S, \delta_{S}), \partial (S, V)^*,
\match(\mathcal{M}_V, \delta_V)} V^*$.
\end{claim}
\begin{proof}
Observe that the matching used to compute the delta on the source side is
the same as the matching used to compute the matching on the view side, hence we
may take the update lens to be position and proceed with the rest of the proof
as in \cref{positionalisvalid}.
\end{proof}
\subsection{Composition}
Let $\ell_1 : S \Leftrightarrow U$ and $\ell_2 : U \Leftrightarrow V$ be lenses.
Define the lens $\ell_2 \circ \ell_1 : S \Leftrightarrow V$ by
\begin{align*}
(\ell_2 \circ \ell_1).\get \; s &= \ell_2.\get \; (\ell_1.\get \; s)\\
(\ell_2 \circ \ell_1).\pput \; s \; v &= \ell_1.\pput \; s \; (\ell_2.\pput \;
(\ell_1.\get \; s) \; v) \\
(\ell_2 \circ \ell_1).\create \; v &= \ell_1.\create \; (\ell_2.\create \; v)
\end{align*}

\begin{claim}
If $\ell_1 : S_1  \xLeftrightarrow{\delta_{S_1}, \partial (S_1, V_1),
\delta_{V_1}} V_1$ and $\ell_2 : S_2 \xLeftrightarrow{\delta_{S_2}, \partial
(S_2, V_2), \delta_{V_2}} V_2$, then \\
$\ell_2 \circ \ell_1 :  S \xLeftrightarrow{\delta_S, \partial
(S_2, V_2) \circ \partial (S_1, V_1), \delta_V} V$.
\end{claim}
\begin{proof}
We need to show that
$$
(\partial (S_2, V_2) \circ \partial (S_1, V_1)).\get \; (\delta(\ell_2 \circ
\ell_1).\pput \; s \; v, s) = \delta(v, (\ell_2 \circ \ell_1).\get \; s)$$

This follows from the fact that
\begin{align*}
&(\partial (S_2, V_2) \circ \partial (S_1, V_1)).\get \; (\delta_S((\ell_2 \circ
\ell_1).\pput \; s \; v, s))\\
&=(\partial (S_2, V_2) \circ \partial (S_1, V_1)).\get \; (\delta_S(\ell_1.\pput
\; s \; (\ell_2.\pput \; (\ell_1.\get \; s) \; v), s))\\
&= (\partial (S_2, V_2)).\get \; ((\partial (S_1, V_1)).\get \;
(\delta_S(\ell_1.\pput \; s \; (\ell_2.\pput \; (\ell_1.\get \; s) \; v), s)))\\
&= (\partial (S_2, V_2)).\get \; (\delta_U(\ell_2.\pput \; (\ell_1.\get \; s)
\; v, \ell_1.\get \; s))\\
&= \delta_V(v, \ell_2.\get \; (\ell_1.\get \; s))\\
&= \delta_V(v, (\ell_2 \circ \ell_1).\get \; s)
\end{align*}
\end{proof}
\iffalse
\section{Symmetric Delta Lenses}
Assume that $(X, \partial X, \delta_X)$ and $(Y, \partial Y, \delta_Y)$
are difference structures. A symmetric delta lens is a pair $(\ell, \partial
\ell)$ where $\ell : X \Leftrightarrow Y$ is a symmetric lens, and
$\partial \ell : \partial X \Leftrightarrow \partial Y$ is a symmetric lens
such that
\begin{enumerate}
  \item
  if $\ell.putr(x, c) = (y, c)$ and $\ell.putl(y', c) = (x', c')$,
  then $\partial \ell.putl \; (\delta(y, y'), \delta(c, c')) = (\delta(x,
  x'), \delta(c, c'))$
  \item
  if $\ell.putl(y, c) = (x, c)$ and $\ell.putr(x', c) = (y', c')$, then
  $\partial \ell.putr \; (\delta(x, x'), \delta(c, c')) = (\delta(y, y'),
  \delta(c, c'))$
\end{enumerate}
\begin{fact}[Identity Lens]
Given any symmetric lens $\partial X : X \Leftrightarrow X$, $(id_X, \partial
X)$ is a symmetric delta lens
\end{fact}
\begin{fact}[Bijective Lenses]
Assume that $f : X \longrightarrow Y$ and $g : \partial X \longrightarrow
\partial Y$ are bijections. Assume that $g(\delta(x, x')) = \delta(f(x),
f(x'))$. Then $(bij_f, bij_g)$ is a symmetric delta lens.
\end{fact}
\begin{fact}[Lens Composition]
If $(\ell_1, \partial \ell_1) : (X, \partial X, \delta_X) \Leftrightarrow (Y,
\partial Y, \delta_Y)$ and $(\ell_2, \partial \ell_2): (Y, \partial Y,
\delta_Y) \Leftrightarrow (Z, \partial Z, \delta_Z)$ are symmetric delta
lenses, then $(\ell_1;\ell_2, \partial \ell_1;\partial \ell_2)$ is a symmetric
delta lens.
\end{fact}
\begin{fact}[Dual of a Lens]
If $(\ell, \partial \ell)$ is symmetric delta lens, then $(\ell^{op}, \partial
\ell^{op})$ is a symmetric delta lens.
\end{fact}
\begin{fact}[Terminal Lens]
$(term_x, term_{\delta(x, x)})$ is a symmetric delta lens
\end{fact}
\begin{corollary}[Disconnect Lens]
If $x \in X$ and $y \in Y$, then $(term_x;term_y^{op},
term_{\delta(x, x)};term_{\delta(y, y)}^{op})$ is a symmetric delta lens.
\end{corollary}
\begin{fact}[Tensor Product Lens]
If $(\ell_1, \partial \ell_1)$ and $(\ell_2, \partial \ell_2)$ are symmetric
delta lenses, then $(\ell_1 \otimes \ell_2, \partial \ell_1 \otimes \partial
\ell_2)$ is a symmetric delta lens.
\end{fact}
\begin{corollary}[Projection Lens]
Projection lenses are symmetric delta lenses.
\end{corollary}
\begin{fact}[Forgetful Tensor Sum Lens]
If $(\ell_1, \partial \ell_1)$ and $(\ell_2, \partial \ell_2)$ are symmetric
delta lenses, then $(\ell_1 \oplus^f \ell_2, \partial \ell_1 \oplus^f \partial
\ell_2 \oplus (\ell_1 \otimes \ell_2))$ is a symmetric delta lens.
\end{fact}
\begin{fact}[Forgetful List Mapping Lens]
If $(\ell, \partial \ell)$ is a symmetric delta lens, then $(map^f(\ell)
\ell_2, map^f(\partial \ell))$ is a symmetric delta lens.
\end{fact}
\begin{remark}[Retentive Lenses]
The retentive tensor sum and list mapping lenses are in general not symmetric
delta lenses with the obvious definitions of difference functions.
\end{remark}
\fi
\section{Relational Lenses}
We define a {\em relation} $R$ of type $((T_1, \ldots, T_n), P, F)$ to be a
multiset $R$ where $T_i, P$ are sets satisfying $R \subseteq P \subseteq T_1
\times \ldots \times T_n$, $F \subseteq [n] \times [n]$, and if $i, j \in F$,
then for all $r, r' \in R$, if $r_i = r'_i$ then $r_j = r'_j$. We interpret the
$T_i$ to be a set of {\em attributes}, $R$ to be a set of {\em records} with
$dom(r) = (T_1, \ldots, T_n)$ for all $r \in R$, $P$ to be a {\em predicate}
such that the records in $R$ satisfy $P$, and $F$ to be a set of {\em
functional dependencies} on the attributes of the records in $R$.

Let $G(F)$ be the graph induced by $F$ on $T_1, \ldots, T_n$ i.e. there is an
edge from $T_i$ to $T_j$ if and only if $(i, j) \in F$. We say that $F$ is
in {\em tree form} if $G(F)$ is a DAG and every vertex has indegree at most 1;
following \cite{bohannon2006relational},we will always assume that $F$ is in
tree form from here on. Observe that every record $r$ is completely determined
by nodes of indegree 0 in $G(F)$: if $Roots(G(F)) = \{T_{i_1}, \ldots,
T_{i_k}\}$ is the set of nodes of indegree 0 and outdegree greater than 0 in
$G(F)$ and $Isolated(G(F)) = \{T_{j_1}, \ldots, T_{j_\ell}\}$ is the set of
nodes of indegree 0 and outdegree 0 in $G(F)$, then for any $r = (r_1, \ldots,
r_n)\in R$, then the map $((r_{i_1}, \ldots, r_{i_k}), (r_{j_1}, \ldots,
r_{j_\ell}))$ is a bijection since the $r_{i_1}, \ldots, r_{i_k}$ determine all
nodes of indegree greater than 0, which leaves just the nodes $r_{j_1}, \ldots,
r_{j_\ell}$.

In order to define lenses in relations, we first need to describe data
structures representing the relation; lenses will operate on the data
structures rather than on relations. Let $Outputs(F) = G(F) \setminus
(Roots(G(F)) \cup Isolated(G(F)))$. We choose to represent a relation with the
following data:
\begin{enumerate}
  \item
  A partial map $data : (T_{i_1} \times \ldots \times T_{i_k})
  \longrightarrow 2^{(T_{j_1} \times \ldots \times T_{j_{\ell}})}$ such that
  there exists a record $r \in R$ with $r[T_{i_1}, \ldots, T_{i_k}] =
  (t_{i_1}, \ldots, t_{i_k})$ if and only if
  $$data(t_{i_1}, \ldots, t_{i_k}) =  \{\{r[T_{j_1}, \ldots, T_{j_{\ell}}] \in
  R \; | \; r_{i_m} = t_{i_m} \text{ for }1 \leq m \leq k\}\}$$
  \item
  A map $dependencies : outputs(F) \longrightarrow \cup_{i,j} 2^{T_i \times
  T_j}$ such that $(p, q) \in dependencies(j)$ if and only if there exists a
  record $r \in R$ such that $r[T_i, T_j] = (p, q)$, where $i$ is such that
  $T_i \longrightarrow T_j \in F$ ($i$ is unique as $F$ is assumed to be in tree
  form).
\end{enumerate}
\subsection{Select}
We say that a predicate $P$ ignores a set of attributes $X$ if for all $r, r'
\in R$, if $r[dom(R) \setminus X] = r'[dom(R) \setminus X]$ implies that $r
\in P$ if and only if $r' \in P$.

The {\em select} lens on $R$ evaluates to a relation which is the same as $R$
but which satisfies a predicate $P$. The inferrence rule for deriving the select
lens is the following:
\begin{prooftree}
\AxiomC{$R : ((T_1, \ldots, T_2), Q, F)$}
\AxiomC{$S : ((T_1, \ldots, T_2), P \cap Q, F)$}
\AxiomC{$F$ is in tree form}
\AxiomC{$P$ ignores $Outputs(F)$}
\QuaternaryInfC{$(select \; from \; R\; where \; P \; as \; S) : R
\Leftrightarrow S$}
\end{prooftree}
In the get direction, the select lens simply restricts $R$ to $P$. Now consider
$put \; s \; r$. The select lens proceeds as follows:
\begin{enumerate}
  \item
  First, it computes $(data_s, dependencies_s)$ and $(data_r, dependencies_r)$.
  \item
  Next, it updates $dependencies_r$ using $dependencies_s$ to get
  $dependencies_r'$.
  \item
  Next it filters $data_r$ to contain only records not in $P$ to get $data_r'$.
  \item
  Next, it computes $(data_r' \cup data_s, dependencies_r')$.
  \item
  Finally, it deletes the records from $data_r' \cup data_s$ that would cause
  PUT-GET to fail.
\end{enumerate}
\section{Matching Lenses}
\begin{definition}
A {\em basic lens} $\ell : S \xLeftrightarrow{C} V$ is a 4-tuple of functions
$\ell.\get : S \longrightarrow V$, $\ell.\res : S \longrightarrow C$,
$\ell.\pput : V \longrightarrow C \longrightarrow S$ and $\ell.\create : V
\longrightarrow S$ satisfying the following properties:
\begin{align*}
\ell.\get \; (\ell.\pput \; v \; c) &= v \tag{PUTGET}\\
\ell.\pput \; (\ell.\get \; s) \; (\ell.\res \; s) &= s \tag{GETPUT}
\end{align*}
\end{definition}
Idea: First remove matched parts to leave unmatched skeleton. Next remove
unmatched parts to leave matched skeleton and then flatten.
%% Acknowledgments
\begin{acks}                            %% acks environment is optional
%% contents suppressed with 'anonymous'
%% Commands \grantsponsor{<sponsorID>}{<name>}{<url>} and
%% \grantnum[<url>]{<sponsorID>}{<number>} should be used to
%% acknowledge financial support and will be used by metadata
%% extraction tools.
This material is based upon work supported by the
\grantsponsor{GS100000001}{National Science
Foundation}{http://dx.doi.org/10.13039/100000001} under Grant
No.~\grantnum{GS100000001}{nnnnnnn} and Grant
No.~\grantnum{GS100000001}{mmmmmmm}.  Any opinions, findings, and
conclusions or recommendations expressed in this material are those
of the author and do not necessarily reflect the views of the
National Science Foundation.
\end{acks}
\fi
\bibliographystyle{plain}
\bibliography{local}
%% Appendix
%%\appendix
%%\section{Appendix}

\end{document}
