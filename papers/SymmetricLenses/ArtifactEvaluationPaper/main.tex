\newif\ifdraft\drafttrue  % set true to show comments
%\newif\ifdraft\draftfalse  % set true to show comments
\newif\ifanon\anonfalse    % set true to suppress names, etc.
\newif\iffull\fullfalse   % set true for long version
\newif\ifappendices\appendicesfalse

\PassOptionsToPackage{usenames,dvipsnames,svgnames,table}{xcolor}
\documentclass[sigplan,acmsmall]{acmart}

\listfiles

\usepackage[usenames,dvipsnames,svgnames,table]{xcolor}
\usepackage{amsmath}
\usepackage{mathtools}
\usepackage{bussproofs}
\usepackage{amsthm}
\usepackage{csvsimple}
\usepackage{thmtools,thm-restate}
\usepackage{changepage}
\usepackage{booktabs}
\usepackage{amssymb}
\usepackage[inline]{enumitem}
\usepackage{multirow,bigdelim}
\usepackage{multicol}
\usepackage{siunitx}
\usepackage{listings}
\usepackage{sansmath}
\usepackage{url}
\usepackage{flushend}
\usepackage{microtype}
\usepackage[utf8]{inputenc}
\usepackage{mathpartir}
\usepackage{empheq}
\usepackage{array}
\usepackage{pgfplots}
\usepackage{stmaryrd}
\usepackage{courier}
\usepackage{qtree}
\usepackage[normalem]{ulem}
\usepackage{relsize}
\usepackage{tikz}
\usepackage{algorithm}
\usepackage[noend]{algpseudocode}
\usepackage{graphicx}
\usepackage{subcaption}
\usepackage{textcomp}
\usepackage{tabularx}
\usepackage{stackengine}
\usepackage{caption}
\usepackage{wrapfig}
\usepackage{remreset}

\settopmatter{printacmref=false}
\renewcommand\footnotetextcopyrightpermission[1]{}


\usetikzlibrary{
  er,
  matrix,
  shapes,
  arrows,
  positioning,
  fit,
  calc,
  pgfplots.groupplots,
  arrows.meta
}
\tikzset{>={Latex}}

%%%% Hyperlinks – must come late!
%\usepackage[pdftex,%
%            pdfpagelabels,%
%            linkcolor=blue,%
%            citecolor=blue,%
%            filecolor=blue,%
%            urlcolor=blue]
%           {hyperref}

\clubpenalty = 10000
\widowpenalty = 10000
\displaywidowpenalty = 10000

%\setlength{\belowcaptionskip}{-5pt}
%\setlength{\textfloatsep}{15pt}

% Creates a display mode for code in sans serif font
\lstnewenvironment{sflisting}[1][]
  {\lstset{%
    mathescape,
    basicstyle=\small\sffamily,
    aboveskip=5pt,
    belowskip=5pt,
    columns=flexible,
    frame=,
    xleftmargin=1em,#1}\sansmath}
  {}
% end

% Macros

  \acmYear{2018}

\begin{document}

\special{papersize=8.5in,11in}
\setlength{\pdfpageheight}{\paperheight}
\setlength{\pdfpagewidth}{\paperwidth}
%\toappear{}

%\conferenceinfo{POPL '16}{January 20--22, 2016, St. Petersburg, FL, USA} 
%\copyrightyear{2016} 
%\copyrightdata{978-1-nnnn-nnnn-n/yy/mm} 

% Uncomment one of the following two, if you are not going for the 
% traditional copyright transfer agreement.

%\exclusivelicense                % ACM gets exclusive license to publish, 
                                  % you retain copyright

%\permissiontopublish             % ACM gets nonexclusive license to publish
                                  % (paid open-access papers, 
                                  % short abstracts)

%\titlebanner{DRAFT---do not distribute}        % These are ignored unless
%\preprintfooter{DRAFT---do not distribute}   % 'preprint' option specified.

\title{Synthesizing Symmetric Lenses: Artifact Evaluation}

\author{Anders Miltner}
\affiliation{Princeton University, USA}
\email{amiltner@cs.princeton.edu}

\author{Kathleen Fisher}
\affiliation{Tufts University, USA}
\email{kfisher@eecs.tufts.edu}

\author{Solomon Maina}
\affiliation{University of Pennsylvania, USA}
\email{smaina@seas.upenn.edu}

\author{Benjamin Pierce}
\affiliation{University of Pennsylvania, USA}
\email{bcpierce@cis.upenn.edu}

\author{David Walker}
\affiliation{Princeton University, USA}
\email{dpw@cs.princeton.edu}

\author{Steve Zdancewic}
\affiliation{University of Pennsylvania, USA}
\email{stevez@cis.upenn.edu}
\maketitle

% \category{D.3.1}
% {Programming Languages}
% {Formal Definitions and Theory}
% [Semantics]

% begin introduction
\section{Setup Environment}
\subsection{Virtual Machine Environment (recommended)}
To make validation easier, we provide the Artifact Evaluation Committee with a
virtual machine that has all components installed.  Setup for the virtual
machine environment detailed below:
\begin{enumerate}
\item Install a Virtual Machine Manager (we tested with VirtualBox)
\item Download our virtual machine image (SymmetricLensSynthesis.ova in\\
  {\color{blue}
    \url{https://drive.google.com/drive/folders/1en7-K6Kxp7MvF3Z1GGfpCDBIKuF-84YM})}
\item Load our virtual machine from your virtual machine manager
\end{enumerate}

We suggest the committe dedicate at least 8GB RAM to this environment.

The user for the virtual machine is ae, with password 2019.

\subsection{Custom Environment}
If you want to install on a custom environment, certain programs and libraries
must be installed.  We provide the commands for installation in Ubuntu 16.04.6
in parenthesis next to the program or library name.

\begin{enumerate}
\item Install opam (sudo apt install opam; opam init; eval `opam config env`;
  opam update)
\item Install necessary system packages (opam depext conf-m4.1)
\item Install OCaml's JBuilder (opam install jbuilder)
\item Install OCaml's Core (opam install core)
\item Install OCaml's Menhir (opam install menhir)
\item Install OCaml's OUnit (opam install ounit)
\item Install OCaml's ppx\_deriving (opam install ppx\_deriving)
\item Install OCaml's ppx\_hash (opam install ppx\_hash)
  
\item Install python-pip (sudo apt install python-pip)
\item Install Python's EasyProcess (pip install EasyProcess)
\item Install Python's matplotlib (pip install matplotlib)
\item Install Python's tk (sudo apt install python-tk)
\item Install Python's numpy (sudo apt install numpy)

\item Download the codebase\\
  (git clone https://github.com/Miltnoid/SymmetricLensSynthArtifactEval.git)
\end{enumerate}


\section{Tool Validation}

\subsection{Simple Validation}

To validate: navigate to the directory the codebase is installed in
(/home/ae/SymmetricLensSynthArtifactEval in the virtual machine).
Then run the following command: ``make regenerate-data''

After this command is run, the figures will be present at the following
locations: 
\begin{itemize}
\item Figure 6 will be present at \$/generated-graphs/times.eps
\item Figure 7 will be present at \$/generated-graphs/times\_bijective.eps
\item Figure 8 will be present at \$/generated-graphs/metrics\_importance.eps
\end{itemize}

The command ``make regenerate-data'' will take a few hours to complete. It
must run to completion for the data to be saved.

The full data from the runs will be aggregated at
\$/generated-data/data.csv

\subsection{Directory Information}

If the artifact committee want to do more in-depth testing, we provide
information on the individual project directories. Boomerang can be built by
running ``make'' in the \$/boomerang directory, and run with ./boomerang.exe.

\paragraph*{\$/boomerang} Boomerang enhanced with symmetric lenses is present in
the program directory. The optician source code is provided in
\$/boomerang/optician . The directory \$/program/examples contains example
boomerang programs, and the directory \$/program/examples/synth\_examples
contains examples using synthesis.

\paragraph*{\$/generated-data} After data is created by the individual tests, it
is then combined and placed into \$/generated-data/data.csv.

\paragraph*{\$/generated-graphs} After data is created by the individual tests
and combined into \$/generated-data/data.csv, the data is converted into graphs
and placed into \$/generated-graphs.

\end{document}



%%% Local Variables:
%%% TeX-master: "main"
%%% End:
