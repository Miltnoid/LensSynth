\newif\ifdraft\drafttrue  % set true to show comments
%\newif\ifdraft\draftfalse  % set true to show comments
\newif\ifanon\anonfalse    % set true to suppress names, etc.
\newif\iffull\fullfalse   % set true for long version
\newif\ifappendices\appendicestrue

\PassOptionsToPackage{usenames,dvipsnames,svgnames,table}{xcolor}
\documentclass[acmsmall,screen,anonymous]{acmart}
\settopmatter{}

\usepackage[usenames,dvipsnames,svgnames,table]{xcolor}
\usepackage{amsmath}
\usepackage{mathtools}
\usepackage{bussproofs}
\usepackage{varwidth}
\usepackage{amsthm}
\usepackage{csvsimple}
\usepackage{thmtools,thm-restate}
\usepackage{changepage}
\usepackage{booktabs}
\usepackage{amssymb}
\usepackage[inline]{enumitem}
\usepackage{multirow,bigdelim}
\usepackage{multicol}
\usepackage{siunitx}
\usepackage{listings}
\usepackage{sansmath}
\usepackage{url}
\usepackage{flushend}
\usepackage{microtype}
\usepackage[utf8]{inputenc}
\usepackage{mathpartir}
\usepackage{empheq}
\usepackage{array}
\usepackage{pgfplots}
\usepackage{stmaryrd}
\usepackage{courier}
\usepackage{qtree}
\usepackage[normalem]{ulem}
\usepackage{relsize}
\usepackage{tikz}
\usepackage{algorithm}
\usepackage[noend]{algpseudocode}
\usepackage{graphicx}
\usepackage{subcaption}
\usepackage{textcomp}
\usepackage{tabularx}
\usepackage{stackengine}
\usepackage{caption}
\usepackage{wrapfig}
\usepackage{remreset}
\usepackage{tabulary}
\usepackage{xspace}

\usetikzlibrary{
  er,
  matrix,
  shapes,
  arrows,
  positioning,
  fit,
  calc,
  pgfplots.groupplots,
  arrows.meta
}
\tikzset{>={Latex}}

%%%% Hyperlinks – must come late!
%\usepackage[pdftex,%
%            pdfpagelabels,%
%            linkcolor=blue,%
%            citecolor=blue,%
%            filecolor=blue,%
%            urlcolor=blue]
%           {hyperref}

% Colors
\definecolor{dkblue}{rgb}{0,0.1,0.5}
\definecolor{dkgreen}{rgb}{0,0.6,0}
\definecolor{dkred}{rgb}{0.6,0,0}
\definecolor{dkpurple}{rgb}{0.4,0,0.6}
\definecolor{olive}{rgb}{0.4, 0.4, 0.0}
\definecolor{teal}{rgb}{0.0,0.5,0.5}
\definecolor{orange}{rgb}{0.9,0.6,0.2}
\definecolor{lightyellow}{RGB}{255, 255, 179}
\definecolor{lightgreen}{RGB}{170, 255, 220}
\definecolor{teal}{RGB}{141,211,199}
\definecolor{darkbrown}{RGB}{121,37,0}

% remove whitespace before and after multicols
\setlength{\multicolsep}{0pt}

% renewtheorem https://tex.stackexchange.com/questions/103013/is-there-a-renewtheorem-equivalent-of-renewcommand-using-amsthm-and-not-ntheo
\makeatletter
\def\renewtheorem#1{%
  \expandafter\let\csname#1\endcsname\relax
  \expandafter\let\csname c@#1\endcsname\relax
  \gdef\renewtheorem@envname{#1}
  \renewtheorem@secpar
}
\def\renewtheorem@secpar{\@ifnextchar[{\renewtheorem@numberedlike}{\renewtheorem@nonumberedlike}}
\def\renewtheorem@numberedlike[#1]#2{\newtheorem{\renewtheorem@envname}[#1]{#2}}
\def\renewtheorem@nonumberedlike#1{  
\def\renewtheorem@caption{#1}
\edef\renewtheorem@nowithin{\noexpand\newtheorem{\renewtheorem@envname}{\renewtheorem@caption}}
\renewtheorem@thirdpar
}
\def\renewtheorem@thirdpar{\@ifnextchar[{\renewtheorem@within}{\renewtheorem@nowithin}}
\def\renewtheorem@within[#1]{\renewtheorem@nowithin[#1]}
\makeatother

\pgfplotsset{
% override style for non-boxed plots
    % which is the case for both sub-plots
    every non boxed x axis/.style={} 
}

\newenvironment{mathprooftree}
  {\varwidth{.9\textwidth}\centering\leavevmode}
  {\DisplayProof\endvarwidth}

\newcommand{\FINISH}[3]{\ifdraft\textcolor{#1}{[#2: #3]}\fi}
\newcommand{\bcp}[1]{\FINISH{dkred}{B}{#1}}
\newcommand{\BCP}[1]{\FINISH{dkred}{B}{\bf #1}}
\newcommand{\afm}[1]{\FINISH{dkgreen}{A}{#1}}
\newcommand{\dpw}[1]{\FINISH{dkblue}{D}{#1}}
\newcommand{\saz}[1]{\FINISH{orange}{S}{#1}}
\newcommand{\ksf}[1]{\FINISH{teal}{K}{#1}}
\newcommand{\revised}[1]{\FINISH{dkred}{#1}}

\newcommand{\IE}{\emph{i.e.}}
\newcommand{\EG}{\emph{e.g.}}
\newcommand{\ETC}{\emph{etc.}}

\theoremstyle{definition}
\renewtheorem{theorem}{Theorem}
\newtheorem{mylemma}{Lemma}
\renewtheorem{corollary}{Corollary}
\renewtheorem{conjecture}{Conjecture}
\renewtheorem{definition}{Definition}
\newtheorem{property}{Property}
\theoremstyle{plain}
\theoremstyle{remark}
\newtheorem{subcase}{Subcase}
\theoremstyle{remark}
\newtheorem{case}{Case}
\makeatletter
\@addtoreset{subcase}{case}
\@addtoreset{case}{mylemma}
\@addtoreset{case}{theorem}
\@addtoreset{case}{corollary}
\@addtoreset{case}{definition}
\@addtoreset{case}{definition}
\@removefromreset{theorem}{section}
\makeatother

\algnewcommand\algorithmicswitch{\textbf{switch}}
\algnewcommand\algorithmicmatch{\textbf{match}}
\algnewcommand\algorithmiccase{\textbf{case}}
\algnewcommand\algorithmicwith{\textbf{with}}
\algnewcommand\algorithmicforeach{\textbf{foreach}}
\algnewcommand\Assert[1]{\State \algorithmicassert(#1)}%
% New "environments"
\algdef{SE}[SWITCH]{Switch}{EndSwitch}[1]{\algorithmicmatch\ #1\ \algorithmicwith}{\algorithmicend\ \algorithmicswitch}%
\algdef{SE}[CASE]{Case}{EndCase}[1]{$|~$ #1 $\rightarrow$}{\algorithmicend\ \algorithmiccase}%
\algdef{SE}[FOREACH]{ForEach}{EndForEach}[2]{\algorithmicforeach\ #1 $\in$ #2}{\algorithmicend\ \algorithmicforeach}%
\algdef{SE}[CaseTwo]{CaseTwo}{EndCaseTwo}[2]{$|~$ #1 $\rightarrow$ #2}{\algorithmicend\ \algorithmiccase}%
\algtext*{EndSwitch}%
\algtext*{EndCase}%
\algtext*{EndCaseTwo}%
\algtext*{EndSecondCase}%
\algtext*{EndForEach}%



\newcommand{\CF}[1]{\ensuremath{\mathsf{#1}}}         % Code Font
\newcommand{\SmallCF}[1]{{\small \mathsf{#1}}}
\newcommand{\PCF}[1]{\textproc{#1}}
\newcommand{\VarCF}[1]{{\color{darkbrown} \CF{#1}}}
\newcommand{\StringCF}[1]{\CF{\textcolor{blue}{#1}}}
\newcommand{\KW}[1]{\CF{\textcolor{dkpurple}{#1}}}
\newcommand{\Regex}{\ensuremath{\mathit{S}}\xspace}         % Regular Expression
\newcommand{\RegexType}{\ensuremath{\textit{Regex}}}
\newcommand{\EquivRegexType}{\ensuremath{\textit{Regex}/\sim}}
\newcommand{\BooleanAnd}{\ensuremath{~\wedge~}}
\newcommand{\BooleanOr}{\ensuremath{\vee}}
\newcommand{\BooleanImplies}{\ensuremath{\Rightarrow}}
\newcommand{\Rewrite}{\ensuremath{\rightarrow}}
\newcommand{\RewriteAtom}{\ensuremath{\Rewrite_\Atom}}
\newcommand{\RewriteDNF}{\ensuremath{\Rewrite_\DNFRegex}}
\newcommand{\ConcatDNF}{\ensuremath{\odot}}
\newcommand{\ConcatDNFOf}[2]{\ensuremath{#1\ConcatDNF#2}}
\newcommand{\BigConcatDNF}{\ensuremath{\bigodot}}
\newcommand{\ConcatSequence}{\ensuremath{\odot_{\Sequence}}}
\newcommand{\ConcatSequenceOf}[2]{\ensuremath{#1\ConcatSequence#2}}
\newcommand{\ConcatPermutation}{\ensuremath{\odot}}
\newcommand{\ConcatPermutationOf}[2]{\ensuremath{#1\ConcatPermutation#2}}
\newcommand{\SwapPermutation}{\ensuremath{\circledS}}
\newcommand{\SwapPermutationOf}[2]{\ensuremath{#1\SwapPermutation#2}}
\newcommand{\DistributePermutation}{\ensuremath{\otimes}}
\newcommand{\DistributePermutationOf}[2]{\ensuremath{#1\DistributePermutation#2}}
\newcommand{\DistributeSwapPermutation}{\ensuremath{\otimes^{\mathit{s}}}}
\newcommand{\DistributeSwapPermutationOf}[2]{\ensuremath{#1\DistributeSwapPermutation#2}}
\newcommand{\ConcatSequenceLens}{\ensuremath{\odot_{\SequenceLens}}}
\newcommand{\ConcatSequenceLensOf}[2]{\ensuremath{#1\ConcatSequenceLens#2}}
\newcommand{\ConcatDNFLens}{\ensuremath{\odot}}
\newcommand{\ConcatDNFLensOf}[2]{\ensuremath{#1\ConcatDNFLens#2}}
\newcommand{\SwapSequenceLens}{\ensuremath{\circledS_{\SequenceLens}}}
\newcommand{\SwapSequenceLensOf}[2]{\ensuremath{#1\SwapSequenceLens#2}}
\newcommand{\SwapDNFLens}{\ensuremath{\circledS}}
\newcommand{\SwapDNFLensOf}[2]{\ensuremath{#1\SwapDNFLens#2}}
\newcommand{\RepeatDNFOfTimes}[1]{\ensuremath{^{#1}}}
\newcommand{\RepeatDNFOf}[2]{\ensuremath{{#2}\RepeatDNFOfTimes{#1}}}
\newcommand{\RepeatDNFLensOfTimes}[1]{\ensuremath{^{#1}}}
\newcommand{\RepeatDNFLensOf}[2]{\ensuremath{{#2}\RepeatDNFLensOfTimes{#1}}}
\newcommand{\OrDNF}{\ensuremath{\oplus}}
\newcommand{\OrDNFOf}[3]{\ensuremath{#1\OrDNF_{#3}#2}}
\newcommand{\OrDNFLens}{\ensuremath{\oplus}}
\newcommand{\OrDNFLensOf}[2]{\ensuremath{#1\OrDNFLens#2}}
\newcommand{\PutRSym}{\ensuremath{\mathit{putr}}}
\newcommand{\PutLSym}{\ensuremath{\mathit{putl}}}
\newcommand{\PutRSymOf}[1]{\ensuremath{\PutRSym \App #1}}
\newcommand{\PutLSymOf}[1]{\ensuremath{\PutLSym \App #1}}
\newcommand{\RegexAlt}{\ensuremath{\mathit{T}}\xspace}         % Regular Expression
\newcommand{\RegexAltAlt}{\ensuremath{\mathit{U}}\xspace}         % Regular Expression
\newcommand{\Or}{\ensuremath{~|~}}
\newcommand{\RegexOr}[2]{\ensuremath{#1\Or#2}}
\newcommand{\SubN}{\textsubscript{n}}
\newcommand{\RegexConcat}[2]{\ensuremath{#1\cdot#2}}
\newcommand{\EmptyString}{\ensuremath{\epsilon}}
\newcommand{\StringConcat}[2]{\ensuremath{#1\cdot#2}}
\newcommand{\HasSemantics}{\ensuremath{\triangleright}}
\newcommand{\DerivesLens}{\ensuremath{\vdash}}
\newcommand{\DerivesDNFLens}{\ensuremath{\vdash_{\DNFLens}}}
\newcommand{\DerivesSequenceLens}{\ensuremath{\vdash_{\SequenceLens}}}
\newcommand{\DerivesAtomLens}{\ensuremath{\vdash_{\AtomLens}}}
\newcommand{\DerivesStringRegex}{\ensuremath{\vdash}}
\newcommand{\DerivesAtomRewrite}{\ensuremath{\vdash}}
\newcommand{\DerivesDNFRewrite}{\ensuremath{\vdash}}
\newcommand{\Concat}{\ensuremath{\cdot}}
\newcommand{\Union}{\ensuremath{\cup}}
\newcommand{\Intersect}{\ensuremath{\cap}}
\newcommand{\BigUnion}{\ensuremath{\bigcup}}
\newcommand{\BigIntersect}{\ensuremath{\bigcap}}
\newcommand{\denot}[1]{\ensuremath{[ \! [#1] \! ]}}
\newcommand{\SemanticsOf}[1]{\ensuremath{[ \! [#1] \! ]}}
\newcommand{\SetOf}[1]{\ensuremath{\{#1\}}}
\newcommand{\RegexVariable}{\ensuremath{\mathit{U}}}   % User Defined
\newcommand{\RegexVariableAlt}{\ensuremath{\mathit{V}}}
\newcommand{\LensVariable}{\ensuremath{\mathit{L}}}
\newcommand{\ExampledRegex}{\ensuremath{\mathit{er}}} % Exampled Regex
\newcommand{\UnambigItOf}[1]{\ensuremath{#1^{*!}}}
\newcommand{\UnambigConcat}{\ensuremath{\Concat^!}}
\newcommand{\SequenceUnambigConcatOf}[1]{\ensuremath{\UnambigConcat(#1)}}
\newcommand{\UnambigConcatOf}[2]{\ensuremath{#1 \UnambigConcat #2}}
\newcommand{\UnambigOrOf}[2]{\ensuremath{\LanguageOf{#1} \cap \LanguageOf{#2} = \emptyset}}
\newcommand{\Atom}{\ensuremath{\mathit{A}}}          % Atoms
\newcommand{\AtomAlt}{\ensuremath{\mathit{B}}}
\newcommand{\AtomType}{\ensuremath{\mathit{Atom}}}
\newcommand{\App}{\ensuremath{\,}}
\newcommand{\Sequence}{\ensuremath{\mathit{SQ}}}
\newcommand{\SequenceType}{\ensuremath{\mathit{Sequence}}}
\newcommand{\LetIn}[2]{\ensuremath{\text{let } #1 = #2\text{ in }}}
\newcommand{\LetWhereIn}[3]{\ensuremath{\text{let } #1 = #2 \text{ where } #3 \text{ in }}}
\newcommand{\Where}{\ensuremath{\text{ where }}}
\newcommand{\ClauseAlt}{\ensuremath{\mathit{bl}}}       % Clauses
\newcommand{\SequenceAlt}{\ensuremath{\mathit{TQ}}}
\newcommand{\DNFRegex}{\ensuremath{\mathit{DS}}}         % Regular Expression
\newcommand{\DNFRegexAlt}{\ensuremath{\mathit{DT}}}    %Alt Regex
\newcommand{\DNFRegexType}{\ensuremath{\mathit{DNF}}}
\newcommand{\LensContext}{\ensuremath{\Gamma}}
\newcommand{\RegexContext}{\ensuremath{\Delta}}  % Context
\newcommand{\FullContext}{\ensuremath{\Delta, \Gamma}}
\newcommand{\String}{\ensuremath{\mathit{s}}\xspace}        % String
\newcommand{\StringAlt}{\ensuremath{\mathit{t}}}        % StringAlt
\newcommand{\StringAltAlt}{\ensuremath{\mathit{u}}}        % StringAltAlt
\newcommand{\ExampleNumberList}{\ensuremath{\mathit{enl}}} %Example Number List
\newcommand{\ExampleNumberListList}{\ensuremath{\mathit{enll}}}
\newcommand{\ExampleStringList}{\ensuremath{\mathit{esl}}}
\newcommand{\StringList}{\ensuremath{\mathit{sl}}}
\newcommand{\Natural}{\ensuremath{\mathit{n}}}
\newcommand{\Interleaving}[1]{\ensuremath{\mathit{interleaving}(#1)}}
\newcommand{\Interleave}{\ensuremath{\mathit{interleave}}}
\newcommand{\BinaryInterleave}[2]{\ensuremath{\mathit{interleave}(#1,#2)}}
\newcommand{\NAryInterleave}[2]{\ensuremath{\mathit{interleave}(#1,\ldots,#2)}}
\newcommand{\Combine}{\ensuremath{\mathit{combine}}}
\newcommand{\List}{\ensuremath{\mathit{l}}}
\newcommand{\ValidCombine}[2]{\ensuremath{\mathit{validcombine}(#1,#2)}}
\newcommand{\ValidRegexContext}[2]{\ensuremath{\mathit{validregexcontext}(#1,#2)}}
\newcommand{\Parent}[1]{\ensuremath{\mathit{parent}(#1)}}
\newcommand{\Parented}[1]{\ensuremath{mathit{parented}(#1)}}
\newcommand{\CombineString}[1]{\ensuremath{\mathit{combine}_{\ExampleStringList}(#1)}}
\newcommand{\CombineList}[1]{\ensuremath{\mathit{combine}_{\ExampleNumberListList}(#1)}}
\newcommand{\Length}[1]{\ensuremath{\mathit{len}(#1)}}
\newcommand{\Language}{\ensuremath{L}}
\newcommand{\LanguageOf}[1]{\ensuremath{\mathcal{L}(#1)}}
\newcommand{\LanguageUnderContextOf}[2]{\ensuremath{\Language{}_{#1}(#2)}}
\newcommand{\ParseTree}{\ensuremath{\mathit{p}}}
\newcommand{\ParseTreeAlt}{\ensuremath{\mathit{q}}}
\newcommand{\ParseTrees}{\ensuremath{\mathit{ps}}}
\newcommand{\ParseTreeAlts}{\ensuremath{\mathit{qs}}}
\newcommand{\StarParse}[1]{\ensuremath{\mathit{starparse}(#1)}}
\newcommand{\LeftChoiceParse}[1]{\ensuremath{\mathit{l}.(#1)}}
\newcommand{\RightChoiceParse}[1]{\ensuremath{\mathit{r}.(#1)}}
\newcommand{\RangeExcInc}[2]{\ensuremath{(#1,#2]}}
\newcommand{\RangeIncInc}[2]{\ensuremath{[#1,#2]}}

\newcommand{\Lens}{\ensuremath{\mathit{\ell}}\xspace}
\newcommand{\AtomLens}{\ensuremath{\mathit{al}}}
\newcommand{\IterateAtomType}{\textit{Iterate}}
\newcommand{\ConcatedAtomsLens}{\ensuremath{\mathit{cal}}}
\newcommand{\OredClausesLens}{\ensuremath{\mathit{ocl}}}
\newcommand{\ClauseLens}{\ensuremath{\mathit{cll}}}
\newcommand{\SequenceLens}{\ensuremath{\mathit{sql}}}
\newcommand{\SequenceLensType}{\ensuremath{\mathit{SequenceLens}}}
\newcommand{\DNFLens}{\ensuremath{\mathit{dl}}}
\newcommand{\DNFLensType}{\ensuremath{\mathit{DNFLens}}}
\newcommand{\AtomLensType}{\ensuremath{\mathit{AtomLens}}}
\newcommand{\SynSim}[2]{\ensuremath{#1 \sim_{\mathit{sym}} #2}}
\newcommand{\ExdSynSim}[3]{\ensuremath{#2 \sim_{\mathit{sym},#1} #3}}

\newcommand{\PermutationSetOf}[1]{\ensuremath{S_{#1}}}
\newcommand{\Permutation}{\ensuremath{\sigma}}

\newcommand{\Star}{\ensuremath{^*}}
\newcommand{\StarOf}[1]{\ensuremath{{#1}\Star}}
\newcommand{\ConstLens}{\ensuremath{\mathit{const}}}
\newcommand{\ConstLensOf}[2]{\ensuremath{\ConstLens(#1,#2)}}
\newcommand{\ConcatLens}{\ensuremath{\KW{concat}}}
\newcommand{\ConcatLensOf}[2]{\ensuremath{\ConcatLens(#1,#2)}}
\newcommand{\ConcatLensShortOf}[2]{\ensuremath{\mathit{c}(#1,#2)}}
\newcommand{\SwapLens}{\ensuremath{\KW{swap}}\xspace}
\newcommand{\SwapLensOf}[2]{\ensuremath{\SwapLens(#1,#2)}}
\newcommand{\SwapLensShortOf}[2]{\ensuremath{\mathit{s}(#1,#2)}}
\newcommand{\OrLens}{\ensuremath{\KW{or}}\xspace}
\newcommand{\OrLensOf}[2]{\ensuremath{\OrLens(#1,#2)}}
\newcommand{\IdentityLens}{\ensuremath{\KW{id}}}
\newcommand{\IdentityLensOf}[1]{\ensuremath{\IdentityLens(#1)}}
\newcommand{\IdentityLensShortT}{\ensuremath{\mathit{id}}}
\newcommand{\IdentityLensShortOf}[1]{\ensuremath{\IdentityLensShortT_{#1}}}
\newcommand{\IterateLens}{\ensuremath{\KW{iterate}}\xspace}
\newcommand{\IterateLensOf}[1]{\ensuremath{\mathit{\IterateLens(#1)}}}
\newcommand{\Identity}{\ensuremath{\mathit{id}}}
\newcommand{\Compose}{\ensuremath{\circ}}
\newcommand{\ComposeLensOf}[2]{\ensuremath{#1\mathrel{;}#2}}
\newcommand{\Disconnect}{\ensuremath{\KW{disc}}\xspace}
\newcommand{\DisconnectOf}[4]{\ensuremath{\Disconnect(#1,#2,#3,#4)}}
\newcommand{\MergeL}{\ensuremath{\KW{merge\_left}}\xspace}
\newcommand{\MergeLOf}[2]{\ensuremath{\MergeL(#1,#2)}}
\newcommand{\MergeR}{\ensuremath{\KW{merge\_right}}\xspace}
\newcommand{\MergeROf}[2]{\ensuremath{\MergeR(#1,#2)}}
\newcommand{\Invert}{\ensuremath{\KW{invert}}}
\newcommand{\InvertOf}[1]{\ensuremath{\Invert(#1)}}

% GRAMMAR OPERATORS
\newcommand{\GBar}{\ensuremath{~|~}}
\newcommand{\GIndent}{\hspace{.5in}}
\newcommand{\GEq}{\ensuremath{::=~}}
\newcommand{\GEmp}{\ensuremath{\cdot}}
\newcommand{\Perm}{\ensuremath{\mathit{Perm}}}
\newcommand{\Nats}{\ensuremath{\mathbb{N}}}

\newcommand{\InverseOf}[1]{\ensuremath{#1^{-1}}}
\newcommand{\FloorOf}[1]{\ensuremath{\lfloor#1\rfloor}}
\newcommand{\CeilOf}[1]{\ensuremath{\lceil#1\rceil}}
\newcommand{\OfType}{\ensuremath{:}}
\newcommand{\OfRewritelessType}{\ensuremath{\,\,\tilde{\OfType}\,\,}}
\newcommand{\MapsBetweenTypeOf}[2]{\ensuremath{#1 \Leftrightarrow #2}}
\newcommand{\ArrowTypeOf}[2]{\ensuremath{#1 \rightarrow #2}}
\newcommand{\SizeOf}[1]{\ensuremath{|#1|}}

\newcommand{\ToDNFRegex}{\ensuremath{\Downarrow}}
\newcommand{\ToDNFRegexOf}[1]{\ensuremath{\ToDNFRegex\mkern-4mu #1}}
\newcommand{\ToRegex}{\ensuremath{\Uparrow}}
\newcommand{\ToRegexOf}[1]{\ensuremath{\ToRegex\mkern-4mu #1}}

\newcommand{\SuchThat}{\ensuremath{~|~}}
\newcommand{\That}{\ensuremath{~.~}}
\newcommand{\Given}{\ensuremath{~|~}}

\newcommand{\LeftQuotientOf}[2]{\ensuremath{#1\backslash#2}}
\newcommand{\RightQuotientOf}[2]{\ensuremath{#1\slash#2}}

\newcommand{\SuffixOf}[1]{\ensuremath{S_{#1}}}
\newcommand{\PrefixOf}[1]{\ensuremath{P_{#1}}}

\newcommand{\ComplementOf}[1]{\ensuremath{\bar{#1}}}

\newcommand{\Alphabet}{\ensuremath{\Sigma}}
\newcommand{\Character}{\ensuremath{c}}
\newcommand{\CharacterAlt}{\ensuremath{d}}

\newcommand{\SequenceLeft}{\ensuremath{[}}
\newcommand{\SequenceRight}{\ensuremath{]}}
\newcommand{\SequenceOf}[1]{\ensuremath{\SequenceLeft#1\SequenceRight}}
\newcommand{\SeqSep}{\ensuremath{\mkern-1mu\Concat\mkern-1mu}}
\newcommand{\DNFLeft}{\ensuremath{\langle}}
\newcommand{\DNFRight}{\ensuremath{\rangle}}
\newcommand{\DNFOf}[1]{\ensuremath{\DNFLeft#1\DNFRight}}
\newcommand{\DNFSep}{\ensuremath{\Or}}
\newcommand{\SequenceLensLeft}{\ensuremath{[}}
\newcommand{\SequenceLensRight}{\ensuremath{]}}
\newcommand{\SequenceLensOf}[1]{\ensuremath{\SequenceLensLeft#1\SequenceLensRight}}
\newcommand{\SeqLSep}{\ensuremath{\mkern-1mu\Concat\mkern-1mu}}
\newcommand{\DNFLensLeft}{\ensuremath{\langle}}
\newcommand{\DNFLensRight}{\ensuremath{\rangle}}
\newcommand{\DNFLensOf}[1]{\ensuremath{\DNFLensLeft#1\DNFLensRight}}
\newcommand{\DNFLSep}{\ensuremath{\Or}}


\newcommand{\ConstantLensRule}{\textsc{Constant Lens}}
\newcommand{\IdentityLensRule}{\textsc{Identity Lens}}
\newcommand{\IterateLensRule}{\textsc{Iterate Lens}}
\newcommand{\ConcatLensRule}{\textsc{Concat Lens}}
\newcommand{\SwapLensRule}{\textsc{Swap Lens}}
\newcommand{\OrLensRule}{\textsc{Or Lens}}
\newcommand{\ComposeLensRule}{\textsc{Compose Lens}}
\newcommand{\RewriteRegexLensRule}{\textsc{Rewrite Regex Lens}}

\newcommand{\AtomUnrollstarLeftRule}{\textsc{Atom Unrollstar\SubLeft}}
\newcommand{\AtomUnrollstarRightRule}{\textsc{Atom Unrollstar\SubRight}}
\newcommand{\ParallelAtomStructuralRewriteRule}{\textsc{Parallel Atom Structural Rewrite}}
\newcommand{\ParallelSwapAtomStructuralRewriteRule}{\textsc{Parallel Swap Atom Structural Rewrite}}
\newcommand{\AtomStructuralRewriteRule}{\textsc{Atom Structural Rewrite}}
\newcommand{\DNFStructuralRewriteRule}{\textsc{DNF Structural Rewrite}}
\newcommand{\ParallelDNFStructuralRewriteRule}{\textsc{Parallel DNF Structural Rewrite}}
\newcommand{\ParallelSwapDNFStructuralRewriteRule}{\textsc{Parallel Swap DNF Structural Rewrite}}
\newcommand{\IdentityRewriteRule}{\textsc{Identity Rewrite}}
\newcommand{\DNFReorderRule}{\textsc{DNF Reorder}}

\newcommand{\SequenceLensRule}{\textsc{Sequence Lens}}
\newcommand{\AtomLensRule}{\textsc{Atom Lens}}
\newcommand{\DNFLensRule}{\textsc{DNF Lens}}
\newcommand{\RewriteDNFRegexLensRule}{\textsc{Rewrite DNF Regex Lens}}

\newcommand{\SubLeft}{\textsubscript{L}}
\newcommand{\SubRight}{\textsubscript{R}}

\newcommand{\Set}{\ensuremath{\mathit{S}}}

\newcommand{\OrIdentityRule}{\textit{+ Ident}}
\newcommand{\EmptyProjectionRightRule}{\textit{0 Proj\SubRight{}}}
\newcommand{\EmptyProjectionLeftRule}{\textit{0 Proj\SubLeft{}}}
\newcommand{\ConcatAssocRule}{\textit{\Concat{} Assoc}}
\newcommand{\OrAssociativityRule}{\textit{\Or{} Assoc}}
\newcommand{\OrCommutativityRule}{\textit{\Or{} Comm}}
\newcommand{\DistributivityLeftRule}{\textit{Dist\SubRight{}}}
\newcommand{\DistributivityRightRule}{\textit{Dist\SubLeft{}}}
\newcommand{\ConcatIdentityLeftRule}{\textit{\Concat{} Ident\SubLeft{}}}
\newcommand{\ConcatIdentityRightRule}{\textit{\Concat{} Ident\SubRight{}}}
\newcommand{\SumstarRule}{\textit{Sumstar}}
\newcommand{\ProductstarRule}{\textit{Prodstar}}
\newcommand{\UnrollstarLeftRule}{\textit{Unrollstar\SubLeft{}}}
\newcommand{\UnrollstarRightRule}{\textit{Unrollstar\SubRight{}}}
\newcommand{\StarstarRule}{\textit{Starstar}}
\newcommand{\DicyclicityRule}{\textit{Dicyc}}
\newcommand{\Derivation}{\ensuremath{\mathcal{D}}}

\newcolumntype{q}{>{$}l<{$}}
\newcolumntype{v}{>{$}r<{$}}

\renewcommand{\subsubsection}[1]{\paragraph{{#1}}}

\newcommand{\Examples}{\ensuremath{\mathit{exs}}}

\newcommand{\ParallelReduction}{\ensuremath{\rightarrow}}
\newcommand{\ParallelRewrite}{\ensuremath{\,\mathrlap{\to}\,{\scriptstyle\parallel}\,\,\,}}
\newcommand{\ParallelRewriteAtom}{\ensuremath{\ParallelRewrite_{\Atom}}}
\newcommand{\ParallelRewriteSwap}{\ensuremath{\ParallelRewrite^{\mathit{swap}}}}
\newcommand{\ParallelRewriteSwapAtom}{\ensuremath{\ParallelRewrite^{\mathit{swap}}_{\Atom}}}

\newcommand{\Property}{\ensuremath{\mathit{p}}}
\newcommand{\Propagator}{\ensuremath{\mathit{q}}}

\newcommand{\Relation}{\ensuremath{\mathit{R}}}
\newcommand{\RelationSet}{\ensuremath{\mathit{RS}}}

\newcommand{\DiamondProperty}{\ensuremath{\mathit{confluent}}}
\newcommand{\DiamondPropertyWithPropertyOf}[1]{\ensuremath{\DiamondProperty_{#1}}}
\newcommand{\IsConfluentWithPropertyOf}[2]
    {\ensuremath{\DiamondPropertyWithPropertyOf{#2}(#1)}}
\newcommand{\BisimilarProperty}{\ensuremath{\mathit{bisimilar}}}
\newcommand{\BisimilarPropertyWithPropertyOf}[1]{\ensuremath{\BisimilarProperty_{#1}}}
\newcommand{\IsBisimilarWithPropertyOf}[2]
    {\ensuremath{\BisimilarPropertyWithPropertyOf{#2}(#1)}}

\newcommand{\Reduces}{\ensuremath{\rightarrow}}

\newcommand{\SortaEquiv}{\ensuremath{\equiv_{sorta}}}

\newcommand{\AtomEquiv}{\ensuremath{\equiv_{\Atom}}}

\newcommand{\Sep}{\ensuremath{\$}}

\newcommand{\Cross}{\ensuremath{\times}}

\newcommand{\Distance}{\ensuremath{\mathit{d}}}

\newcommand{\AbsOf}[1]{\ensuremath{|#1|}}
\newcommand{\Size}{\ensuremath{\mathit{size}}}
\newcommand{\Module}{\ensuremath{\mathit{M}}}
\newcommand{\VectorSpace}{\ensuremath{\mathit{V}}}

\newcommand{\GetDist}{\ensuremath{\mathit{dist}}}
\newcommand{\LOneNorm}{\ensuremath{\ell_1}}

\newcommand{\Sorting}{\ensuremath{\mathit{sorting}}}
\newcommand{\SortingOf}[2]{\ensuremath{\Sorting(#1,#2)}}

\newcommand{\Sort}{\ensuremath{\mathit{sort}}}
\newcommand{\SortOf}[2]{\ensuremath{\Sort(#1,#2)}}

\newcommand{\ListType}{\ensuremath{\mathit{List}}}
\newcommand{\ListTypeOf}[1]{\ensuremath{#1\,\ListType}}
\newcommand{\ListLeft}{\ensuremath{[}}
\newcommand{\ListRight}{\ensuremath{]}}
\newcommand{\ListOf}[1]{\ensuremath{\ListLeft #1 \ListRight}}
\newcommand{\DNFLeq}{\ensuremath{\leq_{DNF}}}
\newcommand{\SequenceLeq}{\ensuremath{\leq_{Seq}}}
\newcommand{\AtomLeq}{\ensuremath{\leq_{Atom}}}
\newcommand{\ILSLeq}{\ensuremath{\leq_{\mathit{intlistset}}}}
\newcommand{\ExampledDNFLeq}{\ensuremath{\leq_{DNF}^{\Examples}}}
\newcommand{\ExampledSequenceLeq}{\ensuremath{\leq_{Seq}^{\Examples}}}
\newcommand{\ExampledAtomLeq}{\ensuremath{\leq_{Atom}^{\Examples}}}
\newcommand{\DNFEq}{\ensuremath{=_{DNF}}}
\newcommand{\SequenceEq}{\ensuremath{=_{Seq}}}
\newcommand{\AtomEq}{\ensuremath{=_{Atom}}}

\newcommand{\NormalizedDNFOf}[1]{\ensuremath{\DNFOf{#1}_n}}
\newcommand{\NormalizedSequenceOf}[1]{\ensuremath{\SequenceOf{#1}_n}}
\newcommand{\NormalizedStarOf}[1]{\ensuremath{\NormalizedStarOf{#1}_n}}

\newcommand{\AtomNormalizer}{\ensuremath{\mathit{AN}}}
\newcommand{\AtomNormalizerType}{\textit{Atom Normalizer}}
\newcommand{\SequenceNormalizer}{\ensuremath{\mathit{SNN}}}
\newcommand{\SequenceNormalizerType}{\textit{Sequence Normalizer}}
\newcommand{\DNFRegexNormalizer}{\ensuremath{\mathit{DNFN}}}
\newcommand{\DNFRegexNormalizerType}{\textit{DNF Normalizer}}

\newcommand{\Normalize}{\ensuremath{\mathcal{N}}}
\newcommand{\NormalizeOf}[1]{\ensuremath{\Normalize(#1)}}

\newcommand{\DNFLensSynth}{\ensuremath{\mathit{DNFLensSynth}}}
\newcommand{\SequenceLensSynth}{\ensuremath{\mathit{SequenceLensSynth}}}
\newcommand{\AtomLensSynth}{\ensuremath{\mathit{AtomLensSynth}}}
\newcommand{\DNFLensSynthOf}[2]{\ensuremath{\DNFLensSynth(#1,#2)}}
\newcommand{\SequenceLensSynthOf}[2]{\ensuremath{\SequenceLensSynth(#1,#2)}}
\newcommand{\AtomLensSynthOf}[2]{\ensuremath{\AtomLensSynth(#1,#2)}}

\newcommand{\DNFLensHasSemanticsOf}[1]{\ensuremath{\xLeftrightarrow{#1}}}
\newcommand{\SatisfiesDNFLensHasSemanticsOf}[3]{\ensuremath{#2\DNFLensHasSemanticsOf{#1}#3}}
\newcommand{\SatisfiesIdentitySemantics}[2]
  {\ensuremath{\SatisfiesDNFLensHasSemanticsOf{\Identity}{#1}{#2}}}
\newcommand{\EquivalenceOf}[1]{\equiv_{#1}}

\newcommand{\SSREquiv}{\ensuremath{\equiv^s}}

\newcommand{\ReflexivityRule}{\textsc{Reflexivity}}
\newcommand{\BaseRule}{\textsc{Base}}
\newcommand{\SymmetryRule}{\textsc{Symmetry}}
\newcommand{\TransitivityRule}{\textsc{Transitivity}}

\newcommand{\BaseRegexType}{\textit{Base}}
\newcommand{\EmptyRegexType}{\textit{Empty}}
\newcommand{\StarRegexType}{\textit{Star}}
\newcommand{\ConcatRegexType}{\textit{Concat}}
\newcommand{\OrRegexType}{\textit{Or}}

\newcommand{\ConstLensType}{\textit{Const}}
\newcommand{\ConcatLensType}{\textit{Concat}}
\newcommand{\IterateLensType}{\textit{Iterate}}
\newcommand{\SwapLensType}{\textit{Swap}}
\newcommand{\OrLensType}{\textit{Or}}
\newcommand{\ComposeLensType}{\textit{Compose}}
\newcommand{\IdentityLensType}{\textit{Identity}}


\newcommand{\StarAtomType}{\textit{Star}}
\newcommand{\MultiConcatSequenceType}{\textit{MultiConcat}}
\newcommand{\MultiOrDNFRegexType}{\textit{MultiOr}}

\newcommand{\AtomToDNF}{\ensuremath{\mathcal{D}}}
\newcommand{\AtomToDNFOf}[1]{\ensuremath{\AtomToDNF(#1)}}
\newcommand{\AtomToDNFLens}{\ensuremath{\mathcal{D}}}
\newcommand{\AtomToDNFLensOf}[1]{\ensuremath{\AtomToDNFLens(#1)}}

\newcommand{\Queue}{\ensuremath{\mathit{Q}}}
\newcommand{\QueueElement}{\ensuremath{\mathit{qe}}}
\newcommand{\QueueElements}{\ensuremath{\QueueElement\mathit{s}}}
\newcommand{\ExpCount}{\ensuremath{\mathit{e}}}
\newcommand{\True}{\ensuremath{\mathit{true}}}
\newcommand{\False}{\ensuremath{\mathit{false}}}
\newcommand{\Null}{\ensuremath{\mathit{null}}}
\newcommand{\DNFRegexs}{\ensuremath{\DNFRegex\mathit{s}}}
\newcommand{\Types}{\ensuremath{\textit{t}}}

\newcommand{\DictionaryOrderL}{\ensuremath{[}}
\newcommand{\DictionaryOrderR}{\ensuremath{]}}
\newcommand{\DictionaryOrderOf}[1]{\ensuremath{\DictionaryOrderL #1 \DictionaryOrderR}}

\newcommand{\SetOfListOrderL}{\ensuremath{\{}}
\newcommand{\SetOfListOrderR}{\ensuremath{\}}}
\newcommand{\SetOfListOrderOf}[1]{\ensuremath{\SetOfListOrderL #1 \SetOfListOrderR}}

\newcommand{\Int}{\ensuremath{i}}
\newcommand{\UserDef}{\ensuremath{U}}
\newcommand{\UserDefAlt}{\ensuremath{V}}

\newcommand{\Optician}{Optician}
\newcommand{\SOptician}{Optician\textsubscript{S}}
\newcommand{\SynthLens}{\PCF{SynthLens}}
\newcommand{\RXSearch}{\PCF{RXSearch}\xspace}
\newcommand{\SynthDNFLens}{\PCF{SynthDNFLens}}
\newcommand{\ToLens}{\ensuremath{\Uparrow}}
\newcommand{\ToLensOf}[1]{\ensuremath{\ToLens{}\mkern-4mu #1}}
\newcommand{\ToDNFRegexText}{\PCF{ToDNFRegex}}
\newcommand{\Beautify}{\PCF{Beautify}}
\newcommand{\RigidSynth}{\PCF{RigidSynth}}
\newcommand{\GreedySynth}{\PCF{GreedySynth}\xspace}
\newcommand{\RigidSynthInternal}{\PCF{RigidSynthInternal}}
\newcommand{\RigidSynthSequence}{\PCF{RigidSynthSeq}}
\newcommand{\RigidSynthAtom}{\PCF{RigidSynthAtom}}
\newcommand{\GetDNFNormalizer}{\PCF{GetDNFNormalizer}}
\newcommand{\CreatePQueue}{\PCF{CreatePQueue}}
\newcommand{\GetTransitiveSet}{\PCF{GetTransitiveSet}}
\newcommand{\GetCurrentSet}{\PCF{GetCurrentSet}}
\newcommand{\Pop}{\PCF{Pop}}
\newcommand{\ExpandOnce}{\PCF{ExpandOnce}}
\newcommand{\ExpandRequired}{\PCF{ExpandRequired}}
\newcommand{\FixProblemElts}{\PCF{FixProblemElts}}
\newcommand{\Expand}{\PCF{Expand}}
\newcommand{\ForceExpand}{\PCF{ForceExpand}}
\newcommand{\Reveal}{\PCF{Reveal}}
\newcommand{\Map}{\PCF{Map}}
\newcommand{\EnqueueMany}{\PCF{EnqueueMany}}
\newcommand{\ReturnVal}[1]{\ensuremath{\Return\,#1}}
\newcommand{\CurrentSet}{\ensuremath{\mathit{CS}}}
\newcommand{\TransitiveSet}{\ensuremath{\mathit{TS}}}

\newcommand{\StringType}{\ensuremath{\mathit{String}}}

\newcommand{\SUBSECTION}[1]{\iffull\subsection{#1}\else\paragraph*{#1}}

\newcommand{\None}{\ensuremath{\mathit{None}}}
\newcommand{\DNFLensOption}{\ensuremath{\DNFLens\mathit{o}}}
\newcommand{\Some}{\ensuremath{\mathit{Some}}}
\newcommand{\SomeOf}[1]{\ensuremath{\Some\,#1}}
\newcommand{\Option}{\ensuremath{\mathit{Option}}}
\newcommand{\OptionOf}[1]{\ensuremath{#1 \App \Option}}

\newcommand{\Success}{\ensuremath{\boldsymbol{\color{dkgreen}\checkmark}}}
\newcommand{\Failure}{\ensuremath{\boldsymbol{\color{dkred}\times}}}

\newcommand{\Append}{\ensuremath{+\!\!\!\!+\ }}
\newcommand{\ModernTitle}{\VarCF{modern\_\discretionary{}{}{}title}}
\newcommand{\DNFModernTitle}{\VarCF{dnf\_modern\_title}}
\newcommand{\DNFLegacyTitle}{\VarCF{dnf\_legacy\_title}}
\newcommand{\LegacyTitle}{\VarCF{legacy\_\discretionary{}{}{}title}}
\newcommand{\LegacyTitleP}{\VarCF{legacy\_\discretionary{}{}{}title'}}
\newcommand{\TextChar}{\VarCF{text\_\discretionary{}{}{}char}}

\newcommand{\IntList}{\ensuremath{\mathit{il}}}
\newcommand{\IntListSet}{\ensuremath{\mathit{ils}}}
\newcommand{\StringIntListSet}{\ensuremath{\mathit{sils}}}
\newcommand{\ProjectStrings}{\ensuremath{\mathit{projectstrings}}}
\newcommand{\ProjectStringsOf}[1]{\ensuremath{\mathit{\ProjectStrings(#1)}}}
\newcommand{\ProjectILS}{\ensuremath{\mathit{projectils}}}
\newcommand{\ProjectILSOf}[1]{\ensuremath{\mathit{\ProjectILS(#1)}}}
\newcommand{\Generates}{\ensuremath{\rightsquigarrow}}

\newcommand{\ExampledDNFRegex}{\ensuremath{EDS}}
\newcommand{\ExampledDNFRegexAlt}{\ensuremath{EDT}}

\newcommand{\ExampledSequence}{\ensuremath{ESQ}}
\newcommand{\ExampledSequenceAlt}{\ensuremath{ETQ}}

\newcommand{\ExampledAtom}{\ensuremath{EA}}
\newcommand{\ExampledAtomAlt}{\ensuremath{EB}}

\newcommand{\EmbedExamples}{\ensuremath{\PCF{EmbedExamples}}}
\newcommand{\EmbedExamplesOf}[2]{\ensuremath{\EmbedExamples(#1,#2)}}

\newcommand{\Morpheus}{Morpheus}
\newcommand{\InSynth}{InSynth}

\newcommand{\FullMode}{\textbf{Full}\xspace}
\newcommand{\NoCSMode}{\textbf{NoCS}\xspace}
\newcommand{\NoFPEMode}{\textbf{NoFPE}\xspace}
\newcommand{\NoERMode}{\textbf{NoER}\xspace}
\newcommand{\NoUDMode}{\textbf{NoUD}\xspace}
\newcommand{\FlashExtractMode}{\textbf{FlashExtract}\xspace}
\newcommand{\FlashFillMode}{\textbf{Flash Fill}\xspace}
\newcommand{\NaiveMode}{\textbf{Na\"ive}\xspace}

% Asymmetric Lens Commands
\newcommand{\Put}{\ensuremath{\mathit{put}}\xspace}
\newcommand{\Get}{\ensuremath{\mathit{get}}\xspace}
\newcommand{\Create}{\ensuremath{\mathit{create}}\xspace}

% Symmetric Lens Components
\newcommand{\CreateR}{\KW{creater}\xspace}
\newcommand{\CreateL}{\KW{createl}\xspace}
\newcommand{\PutR}{\KW{putr}\xspace}
\newcommand{\PutL}{\KW{putl}\xspace}
% Symmetric Lens Commands
\newcommand{\CreateROf}[1]{\CreateR \App #1}
\newcommand{\CreateLOf}[1]{\CreateL \App #1}
\newcommand{\PutROf}[2]{\PutR \App #1 \App #2}
\newcommand{\PutLOf}[2]{\PutL \App #1 \App #2}
% Stateless Symmetric Lens Components
% Stateless Symmetric Lens Applications
\newcommand{\PutRL}{\PCF{PutRL}\xspace}
\newcommand{\PutLR}{\PCF{PutLR}\xspace}
% Symmetric Lens Laws
% Classical
% Stateless
\newcommand{\CreatePutRL}{\PCF{CreatePutRL}}
\newcommand{\CreatePutLR}{\PCF{CreatePutLR}}
% Forgetful 
\newcommand{\ForgetfulLR}{\PCF{ForgetfulLR}\xspace}
\newcommand{\ForgetfulRL}{\PCF{ForgetfulRL}\xspace}



% Symmetric DNF Lenses
\newcommand{\SDNFLens}{\ensuremath{\mathit{sdl}}}
\newcommand{\SDNFLensOf}[1]{\ensuremath{\DNFLensLeft#1\DNFLensRight}}
\newcommand{\SSQLensOf}[1]{\ensuremath{\SequenceLensLeft#1\SequenceLensRight}}
\newcommand{\SSQLens}{\ensuremath{\mathit{ssql}}}
\newcommand{\SAtomLens}{\ensuremath{\mathit{sal}}}


% Expected Information
\newcommand{\Entropy}{\ensuremath{\mathbb{H}}}
\newcommand{\EntropyOf}[1]{\ensuremath{\Entropy(#1)}}
\newcommand{\Argmin}{\ensuremath{\text{argmin}}}
\newcommand{\ArgminOver}[1]{\ensuremath{\underset{#1}{\Argmin}}}

% Regex
\newcommand{\PRegexOr}[3]{\ensuremath{#1~|_{#3}~#2}}
\newcommand{\PRegexConcat}[2]{{\ensuremath{\RegexConcat{#1}{#2}}}}
\newcommand{\PRegexStar}[2]{\ensuremath{#1^{*_{#2}}}}
\newcommand{\Probability}{\ensuremath{p}}
\newcommand{\ProbabilityAlt}{\ensuremath{q}}
\newcommand{\ProbabilityOf}[2]{P_{#1}(#2)}
\newcommand{\Undefined}{\ensuremath{\mathit{undefined}}}

\newcommand{\Fst}{\ensuremath{\mathit{fst}}}
\newcommand{\Snd}{\ensuremath{\mathit{snd}}}
\newcommand{\EditSeq}{\ensuremath{mathit{EditSeq}}}
\newcommand{\NumBenchmarks}{\ensuremath{20}\xspace}

\newcommand{\InL}{\ensuremath{\mathit{inl}}}
\newcommand{\InLOf}[1]{\ensuremath{\InL \App #1}}
\newcommand{\InR}{\ensuremath{\mathit{inr}}}
\newcommand{\InROf}[1]{\ensuremath{\InR \App #1}}

\newcommand{\Wildcard}{\ensuremath{\_}}
\newcommand{\SingleApp}{\ensuremath{\mathit{singleapp}}}
\newcommand{\Log}{\ensuremath{\mathit{log}}}

\newcommand{\Minute}{\CF{minute}\xspace}
\newcommand{\LinuxCommand}{\CF{linux\_command}\xspace}
\newcommand{\WindowsCommand}{\CF{win\_command}\xspace}

\newcommand{\Reals}{\ensuremath{\mathbb{R}}}
% \lstset{framextopmargin=50pt,frame=bottomline}

%%% Local Variables:
%%% TeX-master: "main"
%%% End:


\clubpenalty = 10000
\widowpenalty = 10000
\displaywidowpenalty = 10000

%\setlength{\belowcaptionskip}{-5pt}
%\setlength{\textfloatsep}{15pt}

% Creates a display mode for code in sans serif font
\lstnewenvironment{sflisting}[1][]
  {\lstset{%
    mathescape,
    basicstyle=\small\sffamily,
    aboveskip=5pt,
    belowskip=5pt,
    columns=flexible,
    frame=,
    xleftmargin=1em,#1}\sansmath}
  {}
% end

% Macros
  \newcommand{\NameOf}[1]{\CF{#1}}

%%% If you see 'ACMUNKNOWN' in the 'setcopyright' statement below,
%%% please first submit your publishing-rights agreement with ACM (follow link on submission page).
%%% Then please update our instructions page and copy-and-paste the NEW commands into your article.
%%% Please contact us in case of questions; allow up to 10 min for the system to propagate the information.
%%%
%%% The following is specific to POPL'18 and the paper
%%% 'Synthesizing Bijective Lenses'
%%% by Anders Miltner, Kathleen Fisher, Benjamin C. Pierce, David Walker, and Steve Zdancewic.
%%%
  
\setcopyright{rightsretained}
\acmPrice{}
\acmDOI{10.1145/3158089}
\acmYear{2018}
\copyrightyear{2018}
\acmJournal{PACMPL}
\acmVolume{2}
\acmNumber{POPL}
\acmArticle{1}
\acmMonth{1}
\startPage{1}

\bibliographystyle{ACM-Reference-Format}
\citestyle{acmauthoryear}

  \begin{document}

  %%% The following is specific to POPL'18 and the paper
%%% 'Synthesizing Bijective Lenses'
%%% by Anders Miltner, Kathleen Fisher, Benjamin C. Pierce, David Walker, and Steve Zdancewic.
%%%

%\toappear{}

%\conferenceinfo{POPL '16}{January 20--22, 2016, St. Petersburg, FL, USA} 
%\copyrightyear{2016} 
%\copyrightdata{978-1-nnnn-nnnn-n/yy/mm} 

% Uncomment one of the following two, if you are not going for the 
% traditional copyright transfer agreement.

%\exclusivelicense                % ACM gets exclusive license to publish, 
                                  % you retain copyright

%\permissiontopublish             % ACM gets nonexclusive license to publish
                                  % (paid open-access papers, 
                                  % short abstracts)

%\titlebanner{DRAFT---do not distribute}        % These are ignored unless
%\preprintfooter{DRAFT---do not distribute}   % 'preprint' option specified.

\title{Simple Symmetric Lenses - Foundations and Synthesis}

\begin{abstract}
  Lenses are bidirectional transformers that are often used for synchronizing
  data in different formats. A generalization of lenses, symmetric lenses,
  enable synchronizations between formats where each format contains information
  the other lacks. The theory of symmetric lenses relies heavily on complements
  for maintaining synchronization. Despite the increased expressivity of
  symmetric lenses, no major bidirectional languages or tools have adopted
  symmetric lenses. While the use of complements allows for a clean and general
  formulation of symmetric lenses, it also requires maintaining internal state
  -- in addition to maintaining synchronized files, the complement must also be
  maintained. We simplify the formulation of symmetric lenses to \emph{simple
    symmetric lenses}. These lenses require no complement, instead using
  familiar $create$ and $put$ functions to maintain synchronization.
  Furthermore, we demonstrate the relationship between simple symmetric lenses,
  by carving out a subset of symmetric lenses which are equivalent to simple
  symmetric lenses. We integrate symmetric lenses into a modern bidirectional
  language, Boomerang, while maintaining full backward compatibility.
  
  Additionally, to enable easy use of symmetric lens combinators, we extend the
  Optician algorithm to synthesize simple symmetric lenses from type and example
  specifications. Unlike the bijective lenses Optician originally addresses,
  there are many symmetric lenses between two data formats. Hypothesizing that
  typically users want the lens that synchronizes the most data, we use an
  information-theoretic measure based on stochastic regular expressions, to
  quantify how much data is left non-synchronized, and develop an algorithm that
  attempts to minimize this measure. Furthermore, as our synthesis algorithm
  requires rewrites to our regular expressions, we prove that these the
  probability distribution of strings within our stochastic regular expressions
  are stable with respect to these rewrites. We experimentally demonstrate that
  despite addressing a much harder problem, our extension to Optician still
  maintains synthesis times competitive to Optician on Optician's bijective
  benchmark suite, and generalizes to benchmarks synchronizing data not in
  bijective correspondence.
\end{abstract}

\begin{CCSXML}
<ccs2012>
<concept>
<concept_id>10011007.10011006.10011050.10011017</concept_id>
<concept_desc>Software and its engineering~Domain specific languages</concept_desc>
<concept_significance>500</concept_significance>
</concept>
<concept>
<concept_id>10011007.10011006.10011066.10011070</concept_id>
<concept_desc>Software and its engineering~Application specific development environments</concept_desc>
<concept_significance>300</concept_significance>
</concept>
</ccs2012>
\end{CCSXML}

\ccsdesc[500]{Software and its engineering~Domain specific languages}
\ccsdesc[300]{Software and its engineering~Application specific development environments}

\ifanon
%\authorinfo{}
%           {}
%           {}
\maketitle
% \vspace*{-6cm}
\else
\author{Anders Miltner}
\affiliation{
  \institution{Princeton University}
  \country{USA}
}
\email{amiltner@cs.princeton.edu}

\author{Solomon Maina}
\affiliation{
  \institution{Princeton University}
  \country{USA}
}
\email{smaina@cis.upenn.edu}

\author{Kathleen Fisher}
\affiliation{
  \institution{Tufts University}
  \country{USA}
}
\email{kfisher@eecs.tufts.edu}

\author{Benjamin C. Pierce}
\affiliation{
  \institution{University of Pennsylvania}
  \country{USA}
}
\email{bcpierce@cis.upenn.edu}

\author{David Walker}
\affiliation{
  \institution{Princeton University}
  \country{USA}
}
\email{dpw@cs.princeton.edu}

\author{Steve Zdancewic}
\affiliation{
  \institution{University of Pennsylvania}
  \country{USA}
}
\email{stevez@cis.upenn.edu}

\keywords{Bidirectional Programming, Program Synthesis, Type-Directed Synthesis,
Type Systems}

\maketitle
\fi

% \category{D.3.1}
% {Programming Languages}
% {Formal Definitions and Theory}
% [Semantics]
\ifanon\else
\fi

% begin introduction
\section{Introduction}
% state of the world
Lenses, groups of three functions that satisfy invertibility laws, are useful
formulations for synchronization. Given data in one format, you can view it is a
different data format using \emph{get} function. Given data in another format,
you can turn it into data in the first format using a \emph{create} function.
Given edits on the second format, you can propagate those updates to the first
format using a \emph{put} function. However, writing these three functions is
tedious and error prone. Much better would be to write a single program that
expresses all three of the functions while also guaranteeing invertibility
properties about them.  This is the approach used in many bidirectional
languages: users write code in a bidirectional DSL, and all three functions are
maintained.  These bidirectional languages are often used for data
synchronization tasks - keeping data in different forms synchronized.

While lenses are used to synchronize data formats, but they remain incapable of
synchronizing formats where each format maintains information the other lacks.
To address this, lenses have been generalized to symmetric lenses.  Symmetric
lenses address these synchronization tasks by maintaining an explicit complement
to keep track of projected data.  However, despite the benefits of symmetric
lenses, they have not been adopted into bidirectional languages, remaining instead
as a rarely used Haskell package.

We believe this is largely due to complements. While complements provide a
simple, theoretical framework to express symmetric lenses, they become tricky
when used for synchronization tasks.  Critically, instead of maintaining two
pieces of data (one for each synchronized data format), the lens must now
maintain three (one for each synchronized data format, and one for the
complement).  This can then be difficult when migrating data - I can no longer
copy my formats to a new computer and tell a bidirectional language to keep them
synchronized, I instead must copy my formats and the complement to the new
computer.

We propose an alternative method. Instead of using complements, we can instead
use familiar \emph{create} and \emph{put} methods to synchronize the data
formats. In this paper, we present \emph{simple symmetric lenses}, a
reformulation of symmetric lenses consisting of 4 functions: 2 create functions
and 2 put functions. While these are less expressive than symmetric lenses,
simple symmetric lenses are able to express all synchronization tasks tackled by
asymmetric lenses - all asymmetric lenses are simple symmetric lenses.
Furthermore, we find a predicate over symmetric lenses, where symmetric lenses
satisfy the predicate if, and only if, they can be written as a simple symmetric lens.

As part of our work, we extend Optician to handle symmetric lens combinators,
while maintaining full backwards compatibility. This means simple symmetric
lenses compose simply with both the matching lens and quotient lens combinators
Boomerang provides. Furthermore, despite providing additional capabilities, our
implementation of Boomerang uses fewer lines of code in the core lens library
than were originally present - the symmetry of our lenses enable more code
reuse than the asymmetric versions permitted.

Furthermore, we enable easy use of these combinators by developing a synthesis
algorithm for symmetric lenses within Boomerang. We do this by extending
Optician, a tool for synthesizing bijective lenses, with capabilities for
synthesizing symmetric lenses as well. However, this is not trivial. Bijective
lenses permit relatively few well-typed terms: finding a bijective lens between
two data formats is not easy. However, finding a symmetric lens between two data
formats is relatively easy, as the data synchronization function can be a
``disconnect'' synchronizer -- changing the data in one format doesn't impact
the data in the other. Therefore, we need a means to find the intended symmetric
lens, not merely a symmetric lens. While input-output examples can help, we wish
to make our tool able to synthesize many functions without requiring users to
specify their intent with large example suites.

Instead, we assume that users want to synchronize as much data as possible. To
these ends, we develop an information thereoretic metric measuring the entropy
of how much data is left non-synchronized, utilizing \emph{stochastic regular
  expressions}. We syntactically find the entropy of stochastic regular
expressions for unambiguous stochastic regular expresions. Furthermore, just as
Optician applies rewrites to regular expressions to find well-aligned ones, our
enhancement to Optician applies rewrites to stochastic regular expressions to
find well-aligned ones. We prove these rewrites maintain both the language of
the stochastic regular expressions and the probability distributions over them.

Lastly, we demonstrate the effectiveness of our algorithm by synthesizing all
the bijective transformations present in Optician's benchmark suite with the
new, symmetric algorithm.  Furthermore, we synthesize an additional
\NumBenchmarks new, non-bijective benchmarks.

In summary, our contributions are as follows:
\begin{enumerate}
\item We define simple symmetric lenses and show their relationship with
  symmetric and asymmetric lenses (\S\ref{sec:ssl} and \S\ref{sec:ssl-rels}).
\item We introduce simple symmetric lens combinators that can be used for
  synchronizing string data formats (\S\ref{sec:ssl-string}).
\item We define stochastic regular expressions, demonstrate how to find their
  entropy, and show how certain rewrites on them maintain language and
  probability distribution (\S\ref{sec:stoch-regexp}).
\item We define symmetric DNF lenses, and stochastic DNF regular expressions.
  We show the conversion of regular expression to DNF regular expression
  maintains language and probability distribution.  We show how to infer the
  entropy of a DNF regular expression (\S\ref{sec:dnf-regex-lens}).
\item We describe a synthesis algorithm that attempts to minimize unsynchronized
  information (\S\ref{sec:synthesis}).
\item We provide a fully backwards compatible implementation of symmetric lenses
  in Boomerang, and an implementation of our synthesis algorithm into Optician
  (\S\ref{sec:implementation}).
\item We evaluate our synthesis algorithm on a variety of benchmarks, and find
  that we lose very little synthesis time, while gaining a large amount of
  expressivity (\S\ref{sec:evaluation}).
\end{enumerate}

% end introduction

% begin preliminaries
\section{Preliminaries}
\label{sec:preliminaries}
A \emph{symmetric lens} between $X$ and $Y$ is an element of the type
$\exists C.\{ \mathit{missing} :
C,\,get : X \times C \to Y \times C,\,put : Y \times C \to X \times C\}$
satisfying the following laws:
\begin{equation}
  \tag{\PutRL}
  \frac{\PutR(x,c) = (y,c')}{\PutL(y,c') = (x,c')}
\end{equation}
\begin{equation}
  \tag{\PutLR}
  \frac{\PutL(y,c) = (x,c')}{\PutR(x,c') = (y,c')}
\end{equation}

Given a lens $l \in X \leftrightarrow Y$, a \emph{put object} for $l$ is a
member of $X + Y$. Consider the function $apply$, which, given a lens and an
element of that lens's complement, is a function from sequences of put objects
to sequences of put objects. If $apply(\Lens,c,es) = es'$, then that means, starting
with complement $c$, after edit $es_i$, the lens $\Lens$ synchronizes the other
data to $es'_i$.
\[
  \inferrule
  {
  }
  {
    apply(\Lens,c,[]) = []
  }
\]
\[
  \inferrule
  {
    \Lens.putr(x,c) = (y,c')\\
    apply(\Lens,c',es) = es'
  }
  {
    apply(\Lens,c,(\InLOf{x})::es) = (\InROf{y})::es'
  }
\]
\[
  \inferrule
  {
    \Lens.putl(y,c) = (x,c')\\
    apply(\Lens,c',es) = es'
  }
  {
    apply(\Lens,c,(\InROf{y})::es) = (\InLOf{x})::es'
  }
\]

Two lenses, $\Lens_1$ and $\Lens_2$, are equivalent if
$apply(\Lens_1,\Lens_1.init,es) = apply(\Lens_2,\Lens_2.init,es)$ for all edit
sequences $es$.
% end preliminaries

\section{Simple Symmetric Lenses}
\label{sec:ssl}
Classically, asymmetric lenses comprised three functions, \Put{}, \Get{}, and
\Create{}.  We extend this style of lens to symmetric lenses with
\emph{simple symmetric lenses}.  Like the classical formulation of asymmetric
lenses, simple symmetric lenses have no complement.  Simple symmetric
lenses comprise 4 functions, \CreateR, \CreateL, \PutR, and \PutL.
These functions satisfy the following round-tripping laws:
\begin{equation}
  \tag{\CreatePutRL}
  \PutLOf{(\CreateROf{x})}{x} = x
\end{equation}
\begin{equation}
  \tag{\CreatePutLR}
  \PutROf{(\CreateLOf{y})}{y} = y
\end{equation}
\begin{equation}
  \tag{\PutRL}
  \PutLOf{(\PutROf{x}{y})}{x} = x
\end{equation}
\begin{equation}
  \tag{\PutLR}
  \PutROf{(\PutLOf{y}{x})}{y} = y
\end{equation}

Two simple symmetric lenses are equivalent if the four functions they comprise
are equivalent as functions. However, to formalize how these lenses work as
synchronization tools, we define an $apply$ function on them, much like we
define an $apply$ on symmetric lenses.

The function $apply$, given a symmetric lens and an optional pair of elements of
the source and target, is a function from sequences of put objects to sequences
of put objects.  If $apply(\Lens,\None,es,es')$, then that means, starting with
no prior data, after edit $es_i$, the lens \Lens synchronizes the other data to
$es_i'$.  If $apply(\Lens,\SomeOf{(x,y)},es,es')$, then that means, starting with
data $x$ and $y$ on the left and right, respectively, after edit $es_i$, the
lens \Lens synchronizes the other data to $es_i'$.

\[
  \inferrule
  {
  }
  {
    apply(\Lens,xyo,[]) = []
  }
\]
\[
  \inferrule
  {
    \Lens.\CreateROf{x} = y\\
    apply(\Lens,\SomeOf{(x,y)},es) = es'
  }
  {
    apply(l,\None,\InLOf{x}::es) = \InROf{y}::es'
  }
\]
\[
  \inferrule
  {
    \Lens.\CreateLOf{y} = x\\
    apply(\Lens,\SomeOf{(x,y)},es) = es'
  }
  {
    apply(l,\None,\InROf{y}::es) = \InLOf{x}::es'
  }
\]
\[
  \inferrule
  {
    \Lens.\PutROf{x'}{y}  = y'\\
    apply(\Lens,\SomeOf{(x',y')},es) = es'
  }
  {
    apply(l,\SomeOf{(x,y)},\InLOf{x'}::es) = \InROf{y'}::es'
  }
\]
\[
  \inferrule
  {
    \Lens.\PutLOf{y'}{x}  = x'\\
    apply(\Lens,\SomeOf{(x',y')},es) = es'
  }
  {
    apply(l,\SomeOf{(x,y)},\InROf{y'}::es) = \InLOf{x'}::es'
  }
\]

Furthermore, equivalence of the $apply$ function corresponds exactly with
equivalence of lenses.

\begin{theorem}
  $\Lens_1 \equiv \Lens_2$ if, and only if, for all sequences $es$,
  $apply(\Lens_1,\None,es) = apply(\Lens_2,\None,es)$.
\end{theorem}

% begin relationship
\section{Relationships Between Simple Symmetric Lenses and Other Lenses}
\label{sec:ssl-rels}

We've reformulated symmetric lenses in a style similar to existing asymmetric
lens formulations. However, some symmetric lenses certainly are inexpressible in
the stateless symmetric lens formulation. We formalize which are expressible
with a restriction on symmetric lenses. We define \emph{forgetful symmetric
  lenses} to be symmetric lenses which satisfy the following additional laws:
\begin{equation}
  \tag{\ForgetfulRL}
  \begin{mathprooftree}
    \AxiomC{$\Lens.putr(x,c_1) = (\_,c_1')$}
    \def\extraVskip{.5pt}
    \noLine 
    \UnaryInfC{$\Lens.putr(x,c_2) = (\_,c_2')$}
    \AxiomC{$\Lens.putl(y,c_1') = (\_,c_1'')$}
    \def\extraVskip{.5pt}
    \noLine 
    \UnaryInfC{$\Lens.putl(y,c_2') = (\_,c_2'')$}
    \def\extraVskip{2pt}
    \singleLine
    \BinaryInfC{$c_1'' = c_2''$}
  \end{mathprooftree}
\end{equation}
\begin{equation}
  \tag{\ForgetfulLR}
  \begin{mathprooftree}
    \AxiomC{$\Lens.putl(y,c_1) = (\_,c_1')$}
    \def\extraVskip{.5pt}
    \noLine 
    \UnaryInfC{$\Lens.putl(y,c_2) = (\_,c_2')$}
    \AxiomC{$putr(x,c_1') = (\_,c_1'')$}
    \def\extraVskip{.5pt}
    \noLine 
    \UnaryInfC{$\Lens.putr(x,c_2') = (\_,c_2'')$}
    \def\extraVskip{2pt}
    \singleLine
    \BinaryInfC{$c_1'' = c_2''$}
  \end{mathprooftree}
\end{equation}

Intuitively, this says that the complements are uniquely determined by the most
recently input $x$ and $y$. This corresponds closely with simple symmetric
lenses, where all state is maintained by the current state of the $x$ and $y$
data. Forgetful symmetric lenses express exactly the same $apply$ function as
stateless symmetric lenses.

\begin{theorem}
  Let $\Lens$ be a symmetric lens. The lens $\Lens$ is equivalent to a forgetful
  lens if, and only if, there exists a simple symmetric lens $\Lens'$ where
  $apply(\Lens,\Lens.init,es) = apply(\Lens',\None,es)$, for all put sequences $es$.
\end{theorem}

Despite being less expressive than symmetric lenses, simple symmetric lenses are
strictly more expressive than classical asymmetric lenses.

\begin{theorem}
  Let $\Lens$ be an asymmetric lens. $\Lens$ is also a simple symmetric lens,
  where
  \begin{enumerate}
  \item $\Lens.creater \App x = \Lens.get \App x$
  \item $\Lens.createl \App y = \Lens.create \App y$
  \item $\Lens.putr \App x \App y = \Lens.get \App x$
  \item $\Lens.putl \App y \App x = \Lens.put \App y \App x$
  \end{enumerate}
\end{theorem}
% end relationship

% begin simple symmetric string lenses
\section{Simple Symmetric String Lenses}
\label{sec:ssl-string}
While we have a theory for simple symmetric lenses, to integrate with Boomerang,
we need combinators to express these lenses. While Boomerang has additional,
complex combinators, we focus on the core ones, which are also those we
eventually wish to synthesize.

\subsection{Identity}
The identity lens operates on a single regular expression.  It fully
synchronizes the data within that regular expression: if data gets changed in
the left, that change gets reflected in the right.

\subsection{Disconnect}
The disconnect lens performs the opposite function as the identity lens.  It
leave the two data formats provided separate.  If data gets changed on the left,
the right is left unaffected.  When the data is being newly added, default
behavior comes from one side or the other.

\subsection{Concat}
The concat lens combines two existing lenses, working on their concatenated
regular expressions.  It performs the left lens on the left portion of the
string, and the right lens on the right portion.  It requires unambiguity
constraints to break up the left and right portions.

\subsection{Swap}
The swap lens operates like the concat lens, but when going from one side to the
other, it swaps the order of the lenses.

\subsection{Compose}
The swap lens operates like the concat lens, but when going from one side to the
other, it swaps the order of the lenses.

\subsection{Iterate}
The iterate lens breaks the provided strings into chunks, and performs the
provided lens on those chunks, then concatenates them together.

\subsection{Or}
The or lens performs one lens or another, depending on the inputs to the lens.

\subsection{MergeL}
The mergel lens is used when the same piece of data on the left is synchronized
to data in potentially many different forms on the right. The first provided
lens is the one defaulted to on creates.

\subsection{MergeR}
The merger lens is symmetric to the mergel lens.

\subsection{Compose}
The compose lens applies one lens to go to an intermediary format, then applies
the other to go to the final format.
% end simple symmetric string lenses

% begin stochastic regular expressions
\section{Stochastic Regular Expressions}
\label{sec:stoch-regexp}
To define the entropy for data recovery, we must first develop a probability
model for our languages. We do this by enhancing regular expressions into
\emph{stochastic regular expressions}. With this, we can jointly
express both a language, and a probability distribution over that language.

\begin{center}
  \begin{tabular}{rcl}
    \Regex{},\RegexAlt{}
    & \GEq{} & $\String$ \\
    & \GBar{} & $\emptyset$ \\
    & \GBar{} & $\PRegexStar{\Regex}{p}$ \\
    & \GBar{} & $\RegexConcat{\Regex_1}{\Regex_2}$ \\
    & \GBar{} & $\PRegexOr{\Regex_1}{\Regex_2}{p}$
  \end{tabular}
\end{center}

The language is defined like the regular expression without probability
annotations would be.  The probabilities are defined as follows:

\begin{center}
  \begin{tabular}{rcl}
    $\ProbabilityOf{\String}{\String'}$
    & =
    & $\begin{cases*}1 & if $\String = \String'$\\ 0 & otherwise\end{cases*}$ \\
    
    $\ProbabilityOf{\emptyset}{\String}$
    & =
    & $0$ \\
    
    $\ProbabilityOf{\RegexConcat{\Regex_1}{\Regex_2}}{\Character_1 \ldots \Character_n}$

    & =
    & $\Sigma_{\String = \String_1\String_2}\ProbabilityOf{\Regex_1}{\String_1}*\ProbabilityOf{\Regex_2}{\String_2}$ \\
    
    $\ProbabilityOf{\PRegexOr{\Regex_1}{\Regex_2}{\Probability}}{\String}$
    & =
    & $\Probability * \ProbabilityOf{\Regex_1}{\String} +
      (1-\Probability) * \ProbabilityOf{\Regex_2}{\String}$\\
    
    $\ProbabilityOf{\PRegexStar{\Regex}{\Probability}}{\String}$
    & =
    & $\Sigma_{\String = \String_1 \ldots \String_n}\Probability^n*(1-\Probability)*\Pi_{i=1}^n\ProbabilityOf{\Regex}{\String_i}$\\
  \end{tabular}
\end{center}

Much like the synthesis algorithm for Optician, we need to apply rewrites to
handle equivalences on the provided stochastic regular expressions.  However, no
such equivalences have yet been defined.  We extend the \emph{star-semiring}
equivalences to stochastic regular expressions.

\begin{center}
  \begin{tabular}{@{}r@{\hspace{1em}}c@{\hspace{1em}}l@{}r@{}}
    \PRegexOr{\Regex}{\emptyset}{1} & $\SSREquiv$ & \Regex{} & \OrIdentityRule{} \\
    $\RegexConcat{\Regex}{\emptyset}$ & $\SSREquiv$ & $\emptyset$ & \EmptyProjectionRightRule{} \\
    $\RegexConcat{\emptyset}{\Regex}$ & $\SSREquiv$ & $\emptyset$ & \EmptyProjectionLeftRule{} \\
    \RegexConcat{(\RegexConcat{\Regex{}}{\Regex'})}{\Regex''} & $\SSREquiv$ & \RegexConcat{\Regex{}}{(\RegexConcat{\Regex'}{\Regex''})} & \ConcatAssocRule{}  \\
    \PRegexOr{(\PRegexOr{\Regex}{\Regex'}{\Probability_1})}{\Regex''}{\Probability_2} & $\SSREquiv$ & \PRegexOr{\Regex}{(\PRegexOr{\Regex'}{\Regex''}{(1-\Probability_1)*\Probability_2})}{\Probability_1*\Probability_2} & \OrAssociativityRule{}  \\
    \PRegexOr{\Regex{}}{\RegexAlt{}}{\Probability} & $\SSREquiv$ & \PRegexOr{\RegexAlt{}}{\Regex{}}{1-\Probability} & \OrCommutativityRule{}\\
    \RegexConcat{\Regex{}}{(\PRegexOr{\Regex{}'}{\Regex{}''}{\Probability})} & $\SSREquiv$ & \PRegexOr{(\RegexConcat{\Regex{}}{\Regex{}'})}{(\RegexConcat{\Regex{}}{\Regex{}''})}{\Probability} & \DistributivityLeftRule{} \\
    \RegexConcat{(\PRegexOr{\Regex{}'}{\Regex{}''}{\Probability})}{\Regex{}} & $\SSREquiv$ & \PRegexOr{(\RegexConcat{\Regex{}'}{\Regex{}})}{(\RegexConcat{\Regex{}''}{\Regex{}})}{\Probability} & \DistributivityRightRule{} \\
    \RegexConcat{\EmptyString{}}{\Regex{}} & $\SSREquiv$ & \Regex{} & \ConcatIdentityLeftRule{} \\
    \RegexConcat{\Regex{}}{\EmptyString{}} & $\SSREquiv$ & \Regex{} & \ConcatIdentityRightRule{} \\
    \PRegexStar{\Regex{}}{\Probability} & $\SSREquiv$ & \PRegexOr{\EmptyString{}}{(\RegexConcat{\Regex{}}{\PRegexStar{\Regex{}}{\Probability}})}{\Probability} & \UnrollstarLeftRule{} \\
    \PRegexStar{\Regex{}}{\Probability} & $\SSREquiv$ & \PRegexOr{\EmptyString{}}{(\RegexConcat{\PRegexStar{{\Regex{}}}{\Probability}}{\Regex{}})}{\Probability} & \UnrollstarRightRule{} 
  \end{tabular}
\end{center}
% \caption{Regular Expression Equivalences}
% \label{fig:regex-equivalence-rules}

\begin{theorem}
  If $\Regex \SSREquiv \RegexAlt$ then $\ProbabilityOf{\Regex}{\String} =
  \ProbabilityOf{\RegexAlt}{\String}$, for all strings $\String \in \LanguageOf{\Regex}$.
\end{theorem}


% end stochastic regular expressions

% begin DNF regular expressions and lenses
\section{Stochastic DNF Regular Expressions}
\label{sec:dnf-regex-lens}
We extend stochastic regular expressions to stochastic DNF regular expressions.
\begin{center}
  \begin{tabular}{l@{\ }c@{\ }l@{\ }>{\itshape\/}r}
    % DNF_REGEX
    \Atom{},\AtomAlt{} & \GEq{} & \PRegexStar{\DNFRegex{}}{\Probability}
% & \StarAtomType{}
\\
    \Sequence{},\SequenceAlt{} & \GEq{} &
                                                       $\SequenceOf{\String_0\SeqSep\Atom_1\SeqSep\ldots\SeqSep\Atom_n\SeqSep\String_n}$ 
%& \MultiConcatSequenceType{} 
\\
    \DNFRegex{},\DNFRegexAlt{} & \GEq{} & $\DNFOf{(\Sequence_1,\Probability_1)\DNFSep\ldots\DNFSep(\Sequence_n,\Probability_n)}$ %& \MultiOrDNFRegexType{} 
  \end{tabular}
\end{center}

As before, DNF regular expressions are essentially regular expressions with all
concatenations fully distributed over all disjunctions.  As these DNF regular
expressions are \emph{stochastic}, they are annotated with probabilities to
express a probability distribution, in addition to a language.  This probability
distribution is defined below:

\begin{center}
  \begin{tabular}{rcl}
    $\ProbabilityOf{\PRegexStar{\DNFRegex}{\Probability}}{\String}$
    & =
    & $\Sigma_{\String = \String_1 \ldots \String_n}\Probability^n*(1-\Probability)*\Pi_{i=1}^n\ProbabilityOf{\DNFRegex}{\String_i}$\\
    
    $\ProbabilityOf{\SequenceOf{\String_0 \SeqSep \Atom_1 \SeqSep \ldots \SeqSep \Atom_n \SeqSep \String_n}}{\String'}$
    & =
    & $\Sigma_{\String' = \String_0\String_1'\ldots\String_n'\String_n}\Pi_{i = 1}^n\ProbabilityOf{\Atom_i}{\String_i'}$ \\
    
    $\ProbabilityOf{\DNFOf{(\Sequence_1,\Probability_1) \DNFSep \ldots \DNFSep (\Sequence_n,\Probability_n)}}{\String}$
    & =
    & $\Sigma_{i=1}^n\Probability_i * \ProbabilityOf{\Sequence_i}{\String}$\\
  \end{tabular}
\end{center}

\begin{figure}
  \raggedright
  $\ConcatSequence{} \OfType{} \ArrowTypeOf{\SequenceType{}}{\ArrowTypeOf{\SequenceType{}}{\SequenceType{}}}$\\
  $\ConcatSequenceOf{[\String_0\SeqSep\Atom_1\SeqSep\ldots\SeqSep\Atom_n\SeqSep\String_n]}{[\StringAlt_0\SeqSep\AtomAlt_1\SeqSep\ldots\SeqSep\AtomAlt_m\SeqSep\StringAlt_m]}=
  [\String_0\SeqSep\Atom_1\SeqSep\ldots\SeqSep\Atom_n\SeqSep\String_n\Concat\StringAlt_0\SeqSep\AtomAlt_1\SeqSep\ldots\SeqSep\AtomAlt_m\SeqSep\StringAlt_m]$\\

  \medskip
  
  $\ConcatDNF{} \OfType{} \ArrowTypeOf{\DNFRegexType{}}{\ArrowTypeOf{\DNFRegexType{}}{\DNFRegexType{}}}$\\
  $\ConcatDNFOf{\DNFOf{(\Sequence_1,\Probability_1)\DNFSep\ldots\DNFSep(\Sequence_n,\Probability_n)}}{\DNFOf{(\SequenceAlt_1,\ProbabilityAlt_1)\DNFSep\ldots\DNFSep(\SequenceAlt_m,\ProbabilityAlt_m)}}=$\\
      $\DNFLeft (\ConcatSequenceOf{\Sequence_1}{\SequenceAlt_1},\Probability_1*\ProbabilityAlt_1)\DNFSep \ldots
      \DNFSep
      (\ConcatSequenceOf{\Sequence_1}{\SequenceAlt_m},\Probability_1*\ProbabilityAlt_m)\DNFSep
      \ldots$\\
      $\DNFSep
      (\ConcatSequenceOf{\Sequence_n}{\SequenceAlt_1},\Probability_n*\ProbabilityAlt_1)\DNFSep
      \ldots \DNFSep
      (\ConcatSequenceOf{\Sequence_n}{\SequenceAlt_m},\Probability_n * \ProbabilityAlt_m) \DNFRight$
  
  \medskip
  
  $\OrDNF{}_{\Probability} \OfType{}
  \ArrowTypeOf{\DNFRegexType{}}{\ArrowTypeOf{\DNFRegexType{}}{\DNFRegexType{}}
  }$ \\
  $\OrDNFOf{\DNFOf{(\Sequence_1,\Probability_1)\DNFSep\ldots\DNFSep(\Sequence_n,\Probability_n)}}{\DNFOf{(\SequenceAlt_1,\ProbabilityAlt_1)\DNFSep\ldots\DNFSep(\SequenceAlt_m,\ProbabilityAlt_m)}}{\Probability} =$\\
  $\DNFOf{(\Sequence_1,\Probability_1*\Probability)\DNFSep\ldots\DNFSep(\Sequence_n,\Probability_n*\Probability)\DNFSep(\SequenceAlt_1,\ProbabilityAlt_1*(1-\Probability))\DNFSep\ldots\DNFSep(\SequenceAlt_m,\ProbabilityAlt_m*(1-\Probability))}$
  
  \medskip
  
  \AtomToDNF{} \OfType
  \ArrowTypeOf{\AtomType{}}{\DNFRegexType{}}\\
  $\AtomToDNFOf{\Atom} = \DNFOf{(\SequenceOf{\EmptyString \SeqSep \Atom \SeqSep
      \EmptyString},1)}$
  \caption{DNF Regular Expression Functions}
  \label{fig:dnf-regex-functions}
\end{figure}

We have a means of turning stochastic regular expressions into stochastic DNF
regular expressions.
The conversion algorithm itself, written $\ToDNFRegexOf{\Regex}$, is defined below.
\[
  \begin{array}{rcl}
    \ToDNFRegexOf{\String} & = & \DNFOf{\SequenceOf{\String}}\\
    \ToDNFRegexOf{\emptyset} & = & \DNFOf{}\\
    \ToDNFRegexOf{(\PRegexStar{\Regex}{\Probability})} & = & \AtomToDNFOf{\PRegexStar{(\ToDNFRegexOf{\Regex})}{\Probability}}\\
    \ToDNFRegexOf{(\RegexConcat{\Regex_1}{\Regex_2})} & = & \ToDNFRegexOf{\Regex_1} \ConcatDNF \ToDNFRegexOf{\Regex_2}\\
    \ToDNFRegexOf{(\PRegexOr{\Regex_1}{\Regex_2}{\Probability})} & = & \ToDNFRegexOf{\Regex_1} \OrDNF_{\Probability} \ToDNFRegexOf{\Regex_2}\\
  \end{array}
\]
We have also proven an algorithm, demonstrating this conversion respects
probability distributions.

\begin{theorem}
  $\ProbabilityOf{\Regex}{\String} = \ProbabilityOf{\ToDNFRegex \Regex}{\String}$
\end{theorem}

Furthermore, in the situation where a DNF regular expression is unambiguous, we
have defined an efficient, syntactic means to find the entropy of the DNF
regular expression.

\begin{center}
  \begin{tabular}{rcl}
    $\EntropyOf{\PRegexStar{\DNFRegex}{\Probability}}$
    & =
    & $\frac{\Probability}{1-\Probability}(\EntropyOf{\DNFRegex} - \Log_2\Probability)
      - \Log_2(1-\Probability)
      $\\
    
    $\EntropyOf{\SequenceOf{\String_0 \SeqSep \Atom_1 \SeqSep \ldots \SeqSep \Atom_n \SeqSep \String_n}}{\String'}$
    & =
    & $\Sigma_{i = 1}^n\EntropyOf{\Atom_i}$ \\
    
    $\EntropyOf{\DNFOf{(\Sequence_1,\Probability_1) \DNFSep \ldots \DNFSep (\Sequence_n,\Probability_n)}}$
    & =
    & $\Sigma_{i=1}^n\Probability_i(\EntropyOf{\Sequence_i}+\Log_2\Probability_i)$\\
  \end{tabular}
\end{center}

\begin{theorem}
  $\EntropyOf{\DNFRegex}$ is the entropy of $P_{\DNFRegex}$.
\end{theorem}

% end DNF regular expressions and lenses

% begin symmetric dnf lenses
\section{Symmetric DNF Lenses}
\label{symmetric-dnf-lenses}
\subsection{Syntax}
Symmetric DNF Lenses are what we synthesize, and we then turn them into
Symmetric Lenses. The nonterminals $i$, $j$, $p$, $q$, $r$, $c$, and $d$ all
represent natural numbers. The nonterminals $\String$ and $\StringAlt$ represent
strings.

\begin{center}
  \begin{tabular}{@{}r@{\ }c@{}l@{}}
    % REGEX
    \SAtomLens{} & \GEq{} & $\IterateLensOf{\SDNFLens}$ \\
    \SSQLens{} & \GEq{} & $(\SSQLensOf{(i_1,j_1,\SAtomLens_1)\SeqLSep
                          \ldots\SeqLSep
                          (i_p,j_p,\SAtomLens_p)}
                          ,\ListOf{(\String_1,\Atom_1);\ldots;(\String_q,\Atom_q)}
                          ,\ListOf{\StringAlt_1;\ldots;\StringAlt_r})$ \\
    \SDNFLens{} & \GEq{} & $(\SDNFLensOf{(i_1,j_1,\SSQLens_1)\DNFLSep
                           \ldots\DNFLSep
                           (i_p,j_p,\SSQLens_p)}
                           ,\ListOf{c_1;\ldots;c_q}
                           ,\ListOf{d_1;\ldots;d_r})$ \\
  \end{tabular}
\end{center}

Lets go through these from bottom to top.

A DNF Lens \SDNFLens{} consists of three components. First is the mapping
component. The mapping component is a list of triples $(i,j,\SSQLens)$. In this,
$i$ and $j$ refer to sequence $i$ and sequence $j$ in the left and right DNF
regular expressions, respectively. The third component, \SSQLens{}, is a
sequence lens that maps between the two sequences specified by $i$ and $j$. The
second component is a list of integers $c$. The integer, $c_i$ at position $i$
states that when performing a $\CreateR$ on a string matching sequence
$\Sequence_i$, it will use the sequence lens that maps it to sequence
$\SequenceAlt_{c_i}$ on the right. The third component is a list of integers
$d$, and follows the same pattern as the second component, but maps from right
to left.

A Sequence lens \SSQLens{} consists of three components. First is the mapping
component. The mapping component is a list of triples $(i,j,\SAtomLens)$. In
this, $i$ and $j$ refer to atom $i$ and atom $j$ on the left and right
sequences, respectively. The third component, \SAtomLens{}, is an atom lens that
maps between the two atoms specified by $i$ and $j$. The second component is a
list of strings, $\String$. The string, $s_i$ at position $i$ states that when
performing a $\CreateL$ on a string matching atom $\Atom_i$, it will do one of
two things. If $\Atom_i$ is mapped to in the mapping component, then performing
the atom's $\CreateL$ will actually $\PutL$ into that string. If $\Atom_i$ is
not mapped to, then $\String_i$ will be created as a representative of
$\Atom_i$.

An atom lens \AtomLens{} is merely the iteration of a symmetric DNF Lens.


\subsection{Typing and Semantics}
Because these lenses only encode functions on well-typed programs, we introduce
the semantics alongside the typing rules.  The semantics are over typing
derivations, instead of over individual lenses.

\paragraph*{DNF Lenses}
The typing judgment is a 3-ary relation over a single DNF lens, and two DNF
regular expressions. If $\SDNFLens \OfType \DNFRegex \Leftrightarrow
\DNFRegexAlt$, then the $\SDNFLens.\CreateR$, $\SDNFLens.\CreateL$,
$\SDNFLens.\PutR$, and $\SDNFLens.\PutL$ functions form a symmetric lens.

The typing judgement has 3 components. The first is the sublens components,
confirming that the sequence lenses the DNF lens is comprised of are all
well-typed. The second guarantees that if each sequence on the left has a
sequence lens that can be used for \CreateR{}s and \PutR{}s. The last guarantees
the same for sequences on the right, with \CreateL{} and \PutL{}.

\[
  \inferrule*
  {
    \SSQLens_1 \OfType \Sequence_{i_1} \Leftrightarrow \SequenceAlt_{j_1}\\
    \ldots\\
    \SSQLens_p \OfType \Sequence_{i_p} \Leftrightarrow \SequenceAlt_{j_p}\\\\
    i_{c_1} = 1\\
    \ldots\\
    i_{c_q} = q\\\\
    j_{d_1} = 1\\
    \ldots\\
    j_{d_r} = r
  }
  {
    (\SDNFLensOf{(i_1,j_1,\SSQLens_1)\DNFLSep
      \ldots\DNFLSep
      (i_p,j_p,\SSQLens_p)}
    ,\ListOf{c_1;\ldots;c_q}
    ,\ListOf{d_1;\ldots;d_r})
    \OfType\\
    \DNFOf{(\Sequence_1,\Probability_1) \DNFSep \ldots \DNFSep (\Sequence_q,\Probability_q)}
    \Leftrightarrow
    \DNFOf{(\SequenceAlt_1,\ProbabilityAlt_1) \DNFSep \ldots \DNFSep (\SequenceAlt_r,\ProbabilityAlt_q)}
  }
\]

The \CreateR{} function looks for the sequence the provided string matches. If
the string matches sequence $\Sequence_x$, then the lens will look in the create
list at position $x$ to find which sequence lens to use. Then, the sequence lens
transforms the given string using that sequence lens.

The \PutR{} function finds what pairs sequences the input source and view
strings match.  If there that pair of sequences have a sequence lens between
them, then the DNF lens merely performs that sequence lens on the provided
strings.  If there isn't a pair of sequence lenses between them, then \CreateR{}
is performed on the source, with the view forgotten.

The \CreateL{} and \PutL{} functions are defined symmetrically.

\begin{tabular}{@{}r@{\ }c@{\ }l@{}}
  $\CreateR{} \App s$ & = & $\SSQLens_{c_x}.\CreateR{} \App s$ if $s \in \Sequence_x$\\
  $\CreateL{} \App v$ & = & $\SSQLens_{d_y}.\CreateL{} \App v$ if $v \in \SequenceAlt_y$\\
  $\PutR{} \App s \App v$ & = &
                               $\begin{cases*}
                                 \SSQLens_x.\PutR{} \App s \App v & if $s \in \Sequence_{i_x}$ and $v \in \SequenceAlt_{j_x}$\\
                                 \CreateR{} \App s & if $\nexists x.$ $s \in \Sequence_{i_x}$ and $v \in \SequenceAlt_{j_x}$
                               \end{cases*}$\\
  $\PutL{} \App v \App s$ & = &
                               $\begin{cases*}
                                 \SSQLens_y.\PutL{} \App v \App s & if $v \in \SequenceAlt_{j_y}$ and $s \in \Sequence_{i_y}$\\
                                 \CreateL{} \App v & if $\nexists y.$ $v \in \SequenceAlt_{j_y}$ and $s \in \Sequence_{i_y}$
                               \end{cases*}$
\end{tabular}

\paragraph*{Sequence Lenses}
The typing judgment is a 3-ary relation over a single Sequence lens, and two
sequences. If $\SSQLens \OfType \Sequence \Leftrightarrow \SequenceAlt$, then
the $\SSQLens.\CreateR$, $\SSQLens.\CreateL$, $\SSQLens.\PutR$, and
$\SSQLens.\PutL$ functions form a symmetric lens.

The typing judgement has 4 components. The first is the sublens components,
confirming that the atom lenses the sequence lens is comprised of all are
well-typed. The second guarantees that if each string that will be used for
\CreateR{}s are members of the correct atoms.  The third guarantees
the same for strings and atoms on the right, with \CreateL.  The last guarantees
that each atom is mapped by at most one atom lens.

\[
  \inferrule*
  {
    \SAtomLens_1 \OfType \Atom_{i_1} \Leftrightarrow \AtomAlt_{j_1}\\
    \ldots\\
    \SAtomLens_p \OfType \Atom_{i_p} \Leftrightarrow \AtomAlt_{j_p}\\\\
    \String_1 \in \Atom_1\\
    \ldots\\
    \String_q \in \Atom_q\\\\
    \StringAlt_1 \in \AtomAlt_1\\
    \ldots\\
    \StringAlt_r \in \AtomAlt_r\\\\
    i_x = i_y \BooleanImplies x = y\\
    j_x = j_y \BooleanImplies x = y
  }
  {
    (\SSQLensOf{(i_1,j_1,\SAtomLens_1)\SeqLSep
      \ldots\SeqLSep
      (i_p,j_p,\SAtomLens_p)}
    ,\ListOf{\String_1;\ldots;\String_q}
    ,\ListOf{\StringAlt_1;\ldots;\StringAlt_r})\\
    \OfType
    \SequenceOf{\String_0' \SeqSep \Atom_1 \SeqSep \ldots \SeqSep \Atom_q
      \SeqSep \String_q'}
    \Leftrightarrow
    \SequenceOf{\StringAlt_0' \SeqSep \AtomAlt_1 \SeqSep \ldots \SeqSep \AtomAlt_r \SeqSep \StringAlt_r'}
  }
\]

For each component of the string matching an atom, the \CreateR function looks
for the atom lens that maps on the atom. If there is such an atom lens,
\SAtomLens, then that the sequence lens puts the provided string into the
default string for the target atom. If there is no such atom lens, then the
sequence lens merely uses the default string.

For each component of the string matching an atom, the \PutR function looks
for the atom lens that maps on the atom. If there is such an atom lens,
\SAtomLens, then that the atom lens puts the provided string of the source atom into the
string of the target atom. If there is no such atom lens, then the
sequence lens merely recovers the target's string.

The \CreateL{} and \PutL{} functions are defined symmetrically.

\begin{tabular}{@{}r@{\ }c@{\ }l@{}}
  $\CreateR{} \App \String_0'\Concat \String_1'' \Concat \ldots \Concat \String_q'' \Concat \String_q'$
  & = 
  & $\StringAlt_0' \Concat \StringAlt_1'' \Concat \ldots \Concat
    \StringAlt_r'' \Concat \StringAlt_r'$\\
  & & where $\StringAlt_y'' =
    \begin{cases*}
      \SAtomLens_k.\PutRight \App \String_{i_k}'' \App \StringAlt_y & if $j_k = y$\\
      \StringAlt_y & if $\nexists k. j_k = y$\\
    \end{cases*}$\\
  $\CreateL{} \App \StringAlt_0'\Concat \StringAlt_1'' \Concat \ldots \Concat \StringAlt_q''
  \Concat \StringAlt_q'$
  & = 
  & $\String_0' \Concat \String_1'' \Concat \ldots \Concat
    \String_r'' \Concat \String_r'$\\
  & & where $\String_x'' =
    \begin{cases*}
      \SAtomLens_k.\PutRight \App \StringAlt_{j_k}'' \App \String_x & if $i_k = x$\\
      \String_x & if $\nexists k. i_k = x$\\
    \end{cases*}$\\
  $\PutR{} \App \String_0'\Concat \String_1'' \Concat \ldots \Concat \String_q'' \Concat \String_q' \App \StringAlt_0'\Concat \StringAlt_1'' \Concat \ldots \Concat \StringAlt_q'' \Concat \StringAlt_q'$
  & =
  & $\StringAlt_0'\Concat \StringAlt_1''' \Concat \ldots \Concat \StringAlt_q''' \Concat \StringAlt_q'$\\
  & & where
      $t_y''' =
      \begin{cases*}
        \SAtomLens_k.\PutRight \App \String_{i_k}'' \App \StringAlt_{y}'' & if $j_k = y$\\
        \StringAlt_y'' & if $\nexists k. j_k = y$\\
      \end{cases*}$\\
  $\PutL{} \App \StringAlt_0'\Concat \StringAlt_1'' \Concat \ldots \Concat \StringAlt_q'' \Concat \StringAlt_q' \App \String_0'\Concat \String_1'' \Concat \ldots \Concat \String_q'' \Concat \String_q'$
  & =
  & $\String_0'\Concat \String_1''' \Concat \ldots \Concat \String_r''' \Concat \String_r'$\\
  & & where
      $s_x''' =
      \begin{cases*}
        \SAtomLens_k.\PutLeft \App \StringAlt_{j_k}'' \App \String_{x}'' & if $i_k = x$\\
        \String_x'' & if $\nexists k. i_k = x$\\
      \end{cases*}$\\
\end{tabular}

\paragraph*{Atom Lenses}
The typing judgment is a 3-ary relation over a single atom lens, and two
atoms. If $\SAtomLens \OfType \Atom \Leftrightarrow \AtomAlt$, then
the $\SAtomLens.\CreateR$, $\SAtomLens.\CreateL$, $\SAtomLens.\PutR$, and
$\SAtomLens.\PutL$ functions form a symmetric lens.

The typing judgement just confirms that the DNF lens that comprises the
sequence lens is also well typed.

\[
  \inferrule*
  {
    \SDNFLens \OfType \DNFRegex \Leftrightarrow \DNFRegexAlt
  }
  {
    \IterateLensOf{\SDNFLens}
    \OfType \PRegexStar{\DNFRegex}{\Probability}
    \Leftrightarrow \PRegexStar{\DNFRegexAlt}{\ProbabilityAlt}
  }
\]

For each component of the string matching an atom, the \CreateR function looks
for the atom lens that maps on the atom. If there is such an atom lens,
\SAtomLens, then that the sequence lens puts the provided string into the
default string for the target atom. If there is no such atom lens, then the
sequence lens merely uses the default string.

For each component of the string matching an atom, the \PutR function looks
for the atom lens that maps on the atom. If there is such an atom lens,
\SAtomLens, then that the atom lens puts the provided string of the source atom into the
string of the target atom. If there is no such atom lens, then the
sequence lens merely recovers the target's string.

The \CreateL{} and \PutL{} functions are defined symmetrically.

\begin{tabular}{@{}r@{\ }c@{\ }l@{}}
  $\CreateR{} \App \String_0 \Concat \ldots \Concat \String_n$
  & = 
  & $\StringAlt_1 \Concat \ldots \Concat \StringAlt_n$
  where $\StringAlt_i = \SDNFLens.\CreateR \App \String_i$\\
  $\CreateL{} \App \StringAlt_0 \Concat \ldots \Concat \StringAlt_m$
  & = 
  & $\String_1 \Concat \ldots \Concat \String_m$
  where $\String_i = \SDNFLens.\CreateL \App \StringAlt_i$\\
  $\PutR{} \App \String_0 \Concat \ldots \Concat \String_n \App
  \StringAlt_1 \Concat \ldots \Concat \StringAlt_m$
  & = 
  & $\StringAlt_1' \Concat \ldots \Concat \StringAlt_n'$
    where $\StringAlt_i' =
    \begin{cases*}
      \SDNFLens.\PutR \App \String_i \App \StringAlt_i & if $i \leq m$\\
      \SDNFLens.\CreateR \App \String_i & otherwise
    \end{cases*}$\\
  $\PutL{} \App \StringAlt_1 \Concat \ldots \Concat \StringAlt_m \App
  \String_0 \Concat \ldots \Concat \String_n$
  & = 
  & $\String_1' \Concat \ldots \Concat \String_m'$
    where $\String_i' =
    \begin{cases*}
      \SDNFLens.\PutL \App \StringAlt_i \App \String_i & if $i \leq n$\\
      \SDNFLens.\CreateL \App \StringAlt_i & otherwise
    \end{cases*}$\\
\end{tabular}

Now that we've developed symmetric DNF lenses, it's important to develop a means
to guage how bijective the lens is.

Instead of searching for \emph{any} term that satisfies the specification, we
search for the \emph{best} term that satisfies the specification. This begs the
question, by what metric do we use to determine which term is best. Intuitively,
we aim to synthesize the lens which is \emph{closest} to being a bijection.
Somewhat more formally, we aim to synthesize a lens which minimizes the expected
number of bits which must be used to recover the source from the target, and the
target from the source. The expected number of bits that are needed to encode
informatation from a data source is a well known concept in information theory,
and is known as the \emph{entropy} of the data source.

The entropy of $\DNFRegexAlt$ given $\DNFRegex$ and $\SDNFLens$,
$\EntropyOf{\DNFRegexAlt \Given \DNFRegex,\SDNFLens}$ is defined symmetrically.
With these $\operatorname*{argmin}_\theta$ terms defined, the ``best'' DNF lens
between $\DNFRegex$ and $\DNFRegexAlt$ is $\ArgminOver{\SetOf{\SDNFLens \SuchThat
    \SDNFLens \OfType \DNFRegex \Leftrightarrow \DNFRegexAlt}}
\EntropyOf{\DNFRegex \Given \DNFRegexAlt,\SDNFLens} + \EntropyOf{\DNFRegexAlt
  \Given \DNFRegex,\SDNFLens}$.

\begin{centering}
  \[
    \begin{array}{rl}
      & \EntropyOf{\DNFOf{\Sequence_1 \DNFSep \ldots \DNFSep \Sequence_n}\\
      & \hspace*{1em}\Given
        \DNFOf{\SequenceAlt_1 \DNFSep \ldots \DNFSep \SequenceAlt_m}\\
      & \hspace*{1.21em},~(\SDNFLensOf{(i_1,j_1,\SSQLens_1)\DNFLSep
        \ldots\DNFLSep
        (i_p,j_p,\SSQLens_p)}
        ,\ListOf{c_1;\ldots;c_q}
        ,\ListOf{d_1;\ldots;d_r})}\\
      =\\
      & \frac{\Sigma_{j=1}^m(\frac{\Sigma_{k\SuchThat j_k = j}
        \EntropyOf{\Sequence_{i_k}\Given\SequenceAlt_j,\SequenceLens_k}}{\SizeOf{\SetOf{k\SuchThat
        j_k = j}}} + log_2\SizeOf{\SetOf{k\SuchThat j_k = j}})}{m}\\\\
      \\
      
      & \EntropyOf{
        \SequenceOf{\String_0,\Atom_1,\ldots,\Atom_n,\String_n}\\
      & \hspace*{1em}\Given
        \SequenceOf{\StringAlt_0,\AtomAlt_1,\ldots,\AtomAlt_m,\StringAlt_m}\\
      & \hspace*{1.21em},~
        (\SSQLensOf{(i_1,j_1,\SAtomLens_1)\SeqLSep
        \ldots\SeqLSep
        (i_p,j_p,\SAtomLens_p)}
        ,\ListOf{(k_1,\String_1);\ldots;(k_q,\String_q)}
        ,\ListOf{(l_1,\StringAlt_1);\ldots;(l_r,\StringAlt_r)})}\\
      =\\
      & \Sigma_{x=1}^p\EntropyOf{\Atom_{i_x} \Given \AtomAlt_{i_x},\AtomLens_x} +
        \Sigma_{x=1}^q\EntropyOf{\Atom_{k_x}}\\\\
      \\
      
      
      & \EntropyOf{\StarOf{\DNFRegex} \Given \StarOf{\DNFRegexAlt},\StarOf{\SDNFLens}}\\
      =\\
      & 4*\EntropyOf{\DNFRegex\Given\DNFRegexAlt,\SDNFLens}
    \end{array}
  \]
\end{centering}

% end symmetric dnf lenses


% begin synthesis
\section{Synthesis}
\label{sec:synthesis}
While examples can be used to tighten the
specification, a large number of well-constructed examples are required to
sufficiently constrain the search space.

Unfortunately, trying to find the best symmetric lens is a daunting task!
Unless we find a bijective lens (in which case the sum is zero), we can never be
sure if we will find a better lens than the one we found.
% end synthesis

% begin implementation
\section{Implementation}
\label{sec:implementation}
% end implementation

% begin evaluation
\section{Evaluation}
\label{sec:evaluation}
% end evaluation

% begin related-work
\section{Related Work}
\label{sec:related}
% end related-work

% begin conclusion
\section{Conclusion}
\label{sec:conc}
% end conclusion

% begin acknowledgements
\begin{acks}
\end{acks}
% end acknowledgements

\ifanon\else
\fi

% We recommend abbrvnat bibliography style.

% The bibliography should be embedded for final submission.

\bibliography{local,bcp}

\appendix

\ifappendices

\onecolumn

\section{Forgetful Symmetric Lenses}

\begin{property}[Starting Forgetfulness RL]
  \label{prop:forget-rl}
  Let $\Lens$ be a forgetful symmetric lens.  If $(x_1',c_1') = \Lens.\PutL \App
  (y,\Snd \App (\Lens.\PutR \App (x,c_1)))$, and
  $(x_2',c_2') = \Lens.\PutL \App
  (y,\Snd \App (\Lens.\PutR \App (x,c_2)))$, then $x_1' = x_2'$.
\end{property}
\begin{proof}
  By \ForgetfulRL, $c_1' = c_2'$. We know $(y,c_1') = \Lens.\PutR \App
  (x_1',c_1')$ and $(y,c_1') = \Lens.\PutR \App (x_2',c_1')$ by \PutLR. By
  \PutRL, we know $(x_1' = \Lens.\PutL \App (y,c_1'))$ and $(x_2' = \Lens.\PutL
  \App (y,c_1')))$. Therefore, by transitivity of equality, $x_1' = x_2'$.
\end{proof}

\begin{property}[Starting Forgetfulness LR]
  \label{prop:forget-lr}
  Let $\Lens$ be a forgetful symmetric lens.  If $(y_1',c_1') = \Lens.\PutR \App
  (x,\Snd \App (\Lens.\PutL \App (y,c_1)))$, and
  $(x_2',c_2') = \Lens.\PutL \App
  (x,\Snd \App (\Lens.\PutR \App (y,c_2)))$,
  then $x_1' = x_2'$.
\end{property}
\begin{proof}
  Symmetric to Starting Forgetfulness RL.
\end{proof}

\begin{definition}[S]
  Let $\Lens$ be a forgetful lens.

  Consider the following four functions $S(\Lens)$, that we wish to satisfy the
  simple symmetric lens laws.

  \begin{centering}
    \begin{tabular}{@{}r@{\ }c@{\ }l@{}}
      $\CreateR{} \App s$
      & =
      & $\Fst \App (\Lens.\PutR{} \App (s,\Lens.init))$\\
      
      $\CreateL{} \App v$
      & =
      & $\Fst \App (\Lens.\PutL{} \App (v,\Lens.init))$\\
      
      $\PutR{} \App s \App v$
      & =
      & $\LetIn{(\_,c)}{\Lens.\PutL \App (v,\Lens.init)}$\\
      &
      & $\LetIn{(s',\_)}{\Lens.\PutR \App (s,c)}$\\
      &
      & $s'$\\
      
      $\PutL{} \App v \App s$
      & =
      & $\LetIn{(\_,c)}{\Lens.\PutR \App (s,\Lens.init)}$\\
      &
      & $\LetIn{(y',\_)}{\Lens.\PutL \App (v,c)}$\\
      &
      & $y'$\\
    \end{tabular}
  \end{centering}
\end{definition}

\begin{mylemma}
  If $\Lens$ is a forgetful symmetric lens, then $S(\Lens)$ is a simple symmetric
  lens.
\end{mylemma}
\begin{proof}
  
  \CreatePutRL{}:
  
  \begin{centering}
    \begin{tabular}{@{}r@{\ }c@{\ }l@{\ }l}
      $S(\Lens).\PutLOf{(S(\Lens).\CreateROf{x})}{x}$
      & =
      & $S(\Lens).\PutLOf{(\Fst \App (\Lens.\PutR{(x,\Lens.init)}))}{x}$
      & By unfolding definitions
      \\
      
      & =
      & $\LetIn{(\_,c)}{\Lens.\PutR \App (x,\Lens.init)}$\\
      &
      & $\LetIn{(y',\_)}{\Lens.\PutL \App (\Fst \App (\Lens.\PutR{(x,\Lens.init)}),c)}$\\
      &
      & $y'$
      & By unfolding definitions\\
      
      & =
      & $\LetIn{(y',\_)}{\Lens.\PutL \App (\Lens.\PutR \App (x,\Lens.init))}$\\
      &
      & $y'$
      & By tuple harmony \\
      
      & =
      & $x$
      & By \PutRL \\
    \end{tabular}
  \end{centering}

  \CreatePutLR{}:  Symmetric to \CreatePutRL{}

  \PutRL{}:

  \begin{centering}
    \begin{tabular}{@{}r@{\ }c@{\ }l@{\ }l}
      $S(\Lens).\PutLOf{(S(\Lens).\PutROf{x}{y})}{x}$
      & =
      & $\LetIn{(\_,c)}{\Lens.\PutL \App (y,\Lens.init)}$\\
      & & $\LetIn{(y',c')}{\Lens.\PutR \App (x,c)}$\\
      & & $S(\Lens).\PutLOf{y'}{x}$
      & By unfolding definitions
      \\
      
      & =
      & $\LetIn{(\_,c)}{\Lens.\PutL \App (y,\Lens.init)}$\\
      & & $\LetIn{(y',c')}{\Lens.\PutR \App (x,c)}$\\
      & & $\LetIn{(\_,c'')}{\Lens.\PutR \App (x,\Lens.init)}$\\
      & & $\LetIn{(x',c''')}{\Lens.\PutL \App (y',c'')}$\\
      & & $x'$
      & By unfolding definitions
    \end{tabular}
  \end{centering}

  At this point, we know from \PutRL that $(x,c') = \Lens.\PutL \App (y',c')$.
  By Property~\ref{prop:forget-rl}, this means that $x' = x$, as desired.

  \PutLR{}:  Symmetric to \PutRL{}
\end{proof}

\begin{definition}
  Fix a symmetric lens $\Lens$ beteween $X$ and $Y$. Consider the
  function, $\SingleApp_{\Lens} \OfType (X + Y) \times \Lens.C
  \rightarrow ((X + Y) \times \Lens.C)$, defined as:

  \begin{tabular}{@{}r@{\ }c@{\ }l@{\ }l}
    $\SingleApp_{\Lens}(\InLOf{x},c)$
    & =
    & $\LetIn{(y,c')}{\Lens.\PutRight \App (x,c)}$\\
    &
    & $(\InROf{y},c')$\\
    
    $\SingleApp_{\Lens}(\InROf{y},c)$
    & =
    & $\LetIn{(x,c')}{\Lens.\PutLeft \App (y,c)}$\\
    &
    & $(\InLOf{x},\SomeOf{(x,y)})$
  \end{tabular}
\end{definition}

\begin{definition}
  Fix a simple symmetric lens $\Lens$ beteween $X$ and $Y$. Consider the
  function, $\SingleApp_{\Lens} \OfType ((X + Y) \times \OptionOf{(X \times Y)})
  \rightarrow ((X + Y) \times \OptionOf{(X \times Y)})$, defined as:

  \begin{tabular}{@{}r@{\ }c@{\ }l@{\ }l}
    $\SingleApp_{\Lens}(\InLOf{x},\None)$
    & =
    & $\LetIn{y}{\Lens.\CreateROf{x}}$\\
    &
    & $(\InROf{y},\SomeOf{(x,y)})$\\
    
    $\SingleApp_{\Lens}(\InROf{y},\None)$
    & =
    & $\LetIn{x}{\Lens.\CreateLOf{y}}$\\
    &
    & $(\InLOf{x},\SomeOf{(x,y)})$\\
    
    $\SingleApp_{\Lens}(\InLOf{x'},\SomeOf{(x,y)})$
    & =
    & $\LetIn{y'}{\Lens.\PutROf{x'}{y}}$\\
    &
    & $(\InROf{y'},\SomeOf{(x',y')})$\\
    
    $\SingleApp_{\Lens}(\InROf{y'},\SomeOf{(x,y)})$
    & =
    & $\LetIn{x'}{\Lens.\PutLOf{y'}{x}}$\\
    &
    & $(\InROf{x'},\SomeOf{(x',y')})$\\
  \end{tabular}
\end{definition}

\begin{definition}
  Fix a symmetric lens \Lens over $X$ and $Y$. We define the relation $R_\Lens$
  over $\Lens.C$ and $\OptionOf{(X \times Y)}$ as the largest relation such
  that:
  \begin{enumerate}
  \item $R_\Lens(c,None) \BooleanImplies c = \Lens.init$
  \item $R_\Lens(c,Some (x,y)) \BooleanImplies
    \Lens.\PutRight \App (x,c) = (y,c) \BooleanAnd
    \Lens.\PutLeft  \App (y,c) = (x,c)$
  \end{enumerate}
\end{definition}

\begin{mylemma}
  \label{lem:s-equiv}
  Let $\Lens$ be a symmetric lens.  Let $c \in \Lens.C$ be a complement, and
  $xyo \in \OptionOf{(X \times Y)}$.  If $R_{\Lens}(c,xyo)$, then
  $apply(\Lens,c,es) = apply(S(\Lens),xyo,es)$.
\end{mylemma}
\begin{proof}
  By induction on the derivation the application of $apply(S(\Lens),xyo,es)$
  \begin{case}[empty list]
    So by the case, $apply(S(\Lens),xyo,[]) = []$. Furthermore,
    $apply(\Lens,c,[]) = []$, as desired.
  \end{case}
  \begin{case}[first edit is a create right]
    So by the case, $apply(S(l),\None,\InLOf{x}::es) = \InROf{y}::es'$, where
    $S(\Lens).\CreateROf{x} = y$ and $apply(S(\Lens),\SomeOf{(x,y)},es) = es'$.

    As $R_{\Lens}(None,c)$, $c = \Lens.init$. So, performing $apply$ on
    $\Lens.init$, we get $apply(\Lens,\Lens.init,(\InLOf{x})::es) =
    (\InROf{y'})::es''$ where $\Lens.putr(x,c) = (y',c')$ and $apply(\Lens,c',es)
    = es'$.

    So, by definition, $S(\Lens).\CreateROf{x} = \Fst \App (\Lens.\PutRight \App
    (x,\Lens.init))$, so $y = y'$.

    Furthermore, by \PutRL, $\Lens.\PutLeft \App (y,c') = (x,c')$, and by another
    application of \PutLR, $\Lens.\PutRight \App (x,c') = (y,c')$.  This means
    that $R_{\Lens}(c',Some (x,y))$.

    So, by induction assumption, $apply(\Lens,c',es) = apply(S(\Lens),xyo,es)$, so
    $es' = es''$.  This means, $apply(S(l),\None,\InLOf{x}::es) =
    \InROf{y}::es'$ and $apply(\Lens,\Lens.init,(\InLOf{x})::es) =
    (\InROf{y})::es'$, so they are equal, as desired.
  \end{case}
  \begin{case}[first edit is a create left]
    Symmetric to previous case
  \end{case}
  \begin{case}[first edit is a put right]
    So by the case, $apply(S(\Lens),\SomeOf{(x,y)},\InLOf{x'}::es) = \InROf{y'}::es'$, where
    $S(\Lens).\PutROf{x}{y} = y'$ and $apply(S(\Lens),\SomeOf{(x',y')},es) = es'$.

    Performing $apply$ on $c$, we get $apply(\Lens,c,(\InLOf{x'})::es) =
    (\InROf{y'})::es''$ where $\Lens.\PutRight \App (x',c) = (y',c')$ and
    $apply(\Lens,c',es) = es''$.

    So, by definition, $S(\Lens).\PutROf{x}{y} = \Fst \App (\Lens.\PutRight \App
    (x,c''))$, where $c'' = \Snd \App (\Lens.\PutLeft \App (y,\Lens.init))$.

    By assumption, $R_{\Lens}(c,\SomeOf{(x,y)})$, so $\Lens.\PutLeft \App (y,c)
    = (x,c)$.  So, by Property~\ref{prop:forget-lr}, we know $y' = y''$.

    Furthermore, by \PutRL, $\Lens.\PutLeft \App (y',c') = (x',c')$, and by another
    application of \PutLR, $\Lens.\PutRight \App (x',c') = (y',c')$.  This means
    that $R_{\Lens}(c',Some (x',y'))$.

    So, by induction assumption, $apply(\Lens,c',es) = apply(S(\Lens),xyo,es)$, so
    $es' = es''$.  This means, $apply(S(l),\SomeOf{(x,y)},\InLOf{x'}::es) =
    \InROf{y'}::es'$ and $apply(\Lens,c,(\InLOf{x'})::es) =
    (\InROf{y'})::es'$, so they are equal, as desired.
  \end{case}
  \begin{case}[first edit is a put left]
    Symmetric to previous case
  \end{case}
\end{proof}

\begin{definition}[F]
  Let $\Lens$ be a simple symmetric lens between $X$ and $Y$.

  Consider the following set $C$, distinguished element of that set, $init$, and
  pair of functions $\PutRight$ and $\PutLeft$, that we wish to satisfy the
  symmetric lens laws, and that we also wish to be forgetful.

  \begin{centering}
    \begin{tabular}{@{}r@{\ }c@{\ }l@{}}
      $C$
      & =
      & $\OptionOf{(X \times Y)}$\\
      
      $init$
      & =
      & $\None$\\
      
      $\PutRight \App (x,c)$
      & =
      & $(y,\SomeOf{(x,y)})$ where $y = \begin{cases*}
        \Lens.\CreateROf{x} & if $c = \None$\\
        \Lens.\PutROf{x}{y'} & if $c = \SomeOf{(x',y')}$\\
        \end{cases*}$\\
      
      $\PutLeft \App (y,c)$
      & =
      & $(x,\SomeOf{(x,y)})$ where $x = \begin{cases*}
        \Lens.\CreateLOf{y} & if $c = \None$\\
        \Lens.\PutLOf{y}{x'} & if $c = \Some{(x',y')}$\\
        \end{cases*}$\\
    \end{tabular}
  \end{centering}
\end{definition}

\begin{mylemma}
  \label{lem:f-sym}
  If $\Lens$ is a simple symmetric lens, then $F(\Lens)$ is a symmetric lens.
\end{mylemma}
\begin{proof}
  \PutRL: There are two cases, $c = \None$, and $c = \SomeOf{(x,y)}$.

  \begin{case}[c = \None]
    Let $(y',c') = F(\Lens).\PutRight \App (x',\None)$. This means that $y' =
    \Lens.\CreateROf{x'}$, and $c' = \SomeOf{(x',y')}$.
    
    Now, consider $(x'',c'') = F(\Lens).\PutLeft \App (y',\SomeOf{(x',y')})$. By
    unfolding definitions, $x'' = \Lens.\PutLOf{y'}{x'}$. By \CreatePutRL, $x''
    = x'$, meaning $c'' = \SomeOf{(x',y')}$. This means $(x',c') =
    F(\Lens).\PutLeft \App (y',\SomeOf{(x',y')})$, as desired.
  \end{case}

  \begin{case}[c = \SomeOf{(x,y)}]
    Let $(y',c') = F(\Lens).\PutRight \App (x',\SomeOf{(x,y)})$. This means that $y' =
    \Lens.\PutROf{x'}{y}$, and $c' = \SomeOf{(x',y')}$.
    
    Now, consider $(x'',c'') = F(\Lens).\PutLeft \App (y',\SomeOf{(x',y')})$. By
    unfolding definitions, $x'' = \Lens.\PutLOf{y'}{x'}$. By \PutRL, $x''
    = x'$, meaning $c'' = \SomeOf{(x',y')}$. This means $(x',c') =
    F(\Lens).\PutLeft \App (y',\SomeOf{(x',y')})$, as desired.
  \end{case}

  The second requirement, \PutLR, is symmetric.
\end{proof}


\begin{mylemma}
  If $\Lens$ is a simple symmetric lens, then $F(\Lens)$ is a forgetful symmetric lens.
\end{mylemma}
\begin{proof}
  By Lemma~\ref{lem:f-sym}, we know $F(\Lens)$ is symmetric, so we merely need
  to show it is forgetful.  We will tackle merely \ForgetfulRL, as the proof for
  \ForgetfulLR is symmetric.

  Let $c_1$ and $c_2$ be two arbitrary complements, and $x$ and $y$ two
  arbitrary values of $X$ and $Y$, respectively.
  
  We know that $c_1' = \Snd \App (F(\Lens).\PutRight \App (x,c_1))$ and $c_2' =
  \Snd \App (F(\Lens).\PutRight \App (x,c_2))$.  Now, by inversion on
  $F(\Lens).\PutRight$, we know that both $c_1' = Some(x,y_1')$ and $c_2' =
  Some(x,y_2')$ for some values of $y_1$ and $y_2$ (though we don't actually
  care about the values of $y_1$ and $y_2$).

  Now by unfolding definitions we know, $\Snd \App (F(\Lens).\PutLeft \App
  (y,\SomeOf{(x,y_1')})) = \SomeOf(\Lens.\PutLeftOf{x}{y},y) = c_1''$.  Similarly, we
  know $\Snd \App (F(\Lens).\PutLeft \App
  (y,\SomeOf{(x,y_2')})) = \SomeOf(\Lens.\PutLeftOf{x}{y},y) = c_2''$, so $c_1''
  = c_2''$, as intended.
\end{proof}

\begin{mylemma}
\label{lem:f-equiv}
  If $\Lens$ be a simple symmetric lens, then $apply(\Lens,xyo,es) =
  apply(F(\Lens),xyo,es)$.
\end{mylemma}
\begin{proof}
  By induction on the derivation of $apply$ on $\Lens$!

  \begin{case}[empty list]
    So by the case, $apply(S(\Lens),xyo,[]) = []$. Furthermore,
    $apply(\Lens,xyo,[]) = []$, as desired.
  \end{case}

  \begin{case}[first edit is a create right]
    So by the case, $apply(\Lens,\None,\InLOf{x}::es) = \InROf{y}::es'$, where
    $\Lens.\CreateROf{x} = y$ and $apply(\Lens,\SomeOf{(x,y)},es) = es'$.

    Performing $apply$ on
    $\None$, we get $apply(F(\Lens),\None,(\InLOf{x})::es) =
    (\InROf{y'})::es''$ where $F(\Lens).putr(x,\None) = (y',c)$ and $apply(\Lens,c,es)
    = es'$.

    Unfolding definitions, $F(\Lens).putr(x,\None) = \Lens.\CreateROf{x} =
    (y,\SomeOf{(x,y)})$, so $c = \SomeOf{(x,y)}$ and $y = y'$

    So, by induction assumption, $apply(\Lens,\SomeOf{(x,y)},es) = apply(F(\Lens),\SomeOf{(x,y)},es)$, so
    $es' = es''$.  This means, $apply(\Lens,\None,\InLOf{x}::es) =
    \InROf{y}::es'$ and $apply(F(\Lens),\None,(\InLOf{x})::es) =
    (\InROf{y})::es'$, so they are equal, as desired.
  \end{case}

  \begin{case}[first edit is a create left]
    Symmetric to previous case
  \end{case}

  \begin{case}[first edit is a put right]
    So by the case, $apply(\Lens,\SomeOf{(x,y)},\InLOf{x'}::es) = \InROf{y'}::es'$, where
    $\Lens.\PutROf{x}{y} = y'$ and $apply(\Lens,\SomeOf{(x',y')},es) = es'$.

    Performing $apply$ on $F(\Lens)$, we get
    $apply(F(\Lens),\SomeOf{(x,y)},(\InLOf{x'})::es) = (\InROf{y'})::es''$ where
    $F(\Lens).\PutRight \App (x',\SomeOf{(x,y)}) = (y'',\SomeOf{(x',y'')})$ and
    $apply(\Lens,c',es) = es''$.
    
    So, by definition, $\Fst \App (F(\Lens).\PutRight \App (x',\SomeOf{(x,y)}))
    = \Lens.PutROf{x'}{y'}$, so also by definition, $c' = \SomeOf{(x',y')}$.

    So, by induction assumption, $apply(\Lens,\SomeOf{(x',y')},es) =
    apply(F(\Lens),\SomeOf{(x',y')},es)$, so $es' = es''$. This means,
    $apply(\Lens,\SomeOf{(x,y)},\InLOf{x'}::es) = \InROf{y'}::es'$ and
    $apply(F(\Lens),\SomeOf{(x,y)},(\InLOf{x'})::es) = (\InROf{y'})::es'$, so
    they are equal, as desired.
  \end{case}

  \begin{case}[first edit is a put left]
    Symmetric to previous case
  \end{case}
\end{proof}

\begin{theorem}
  Let $\Lens$ be a symmetric lens. The lens $\Lens$ is equivalent to a forgetful
  lens if, and only if, there exists a simple symmetric lens $\Lens'$ where
  $apply(\Lens,\Lens.init,es) = apply(\Lens',\None,es)$, for all put sequences
  $es$.
\end{theorem}

\begin{proof}
  \begin{case}[$\Rightarrow$]
    Let $\Lens$ be equivalent to a forgetful lens $\Lens'$.  Consider the
    simple symmetric lens, $S(\Lens')$.  By Lemma~\ref{lem:s-equiv},
    $apply(\Lens',\Lens'.init,es) = apply(S(\Lens'),\None,es)$.  As $\Lens$ is
    equivalent to $\Lens'$, $apply(\Lens,\Lens.init,es) =
    apply(\Lens',\Lens'.init,es)$.  So, by transitivity,
    $apply(\Lens,\Lens.init,es) = apply(S(\Lens),\None,es)$.
  \end{case}

  \begin{case}[$\Leftarrow$]
    Let $\Lens'$ be a simple symmetric lens where $apply(\Lens,\Lens.init,es) =
    apply(\Lens',\None,es)$, for all put sequences $es$.

    Consider $F(\Lens')$, a forgetful symmetric lens where $apply(\Lens',\None,es) =
    apply(F(\Lens'),\Lens'.init,es)$, for all put sequences $es$, as
    $\Lens'.init = \None$, by Lemma~\ref{lem:f-equiv}.  By transitivity, $apply(\Lens,\Lens.init,es) =
    apply(\Lens',\Lens'.init,es)$, so $\Lens$ and $\Lens'$ are equivalent.
  \end{case}
    
\end{proof}

\fi

\end{document}



%%% Local Variables:
%%% TeX-master: "main"
%%% End:
