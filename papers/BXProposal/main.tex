\documentclass[a4paper]{article}
\usepackage{graphicx}
\usepackage{listings}
\usepackage{onecolpceurws}
\usepackage[usenames,dvipsnames,svgnames,table]{xcolor}

\title{Expanding the Power of Lens Synthesis \\ (talk proposal)}

% I've never done a talk before, should this be everybody or just me
\author{
Anders Miltner \\ Princeton University \\ \ 
% Princeton, NJ 08540 \\ amiltner@cs.princeton.edu
\and
Solomon Maina \\ University of Pennsylvania
% Philadelphia, PA 19104 \\ smaina@seas.upenn.edu
\and
Kathleen Fisher \\ Tufts University
% Medford, MA 02155 \\ kfisher@eecs.tufts.edu
\and
Benjamin C. Pierce \\ University of Pennsylvania
% Philadelphia, PA 19104 \\ bcpierce@cis.upenn.edu
\and
David Walker \\ Princeton University
% Princeton, NJ 08540 \\ dpw@cs.princeton.edu
\and
Steve Zdancewic \\ University of Pennsylvania
% Philadelphia, PA 19104 \\ stevez@cis.upenn.edu
}

\institution{}

\begin{document}
\maketitle

\definecolor{dkblue}{rgb}{0,0.1,0.5}
\definecolor{dkgreen}{rgb}{0,0.6,0}
\definecolor{dkred}{rgb}{0.6,0,0}
\definecolor{dkpurple}{rgb}{0.4,0,0.6}
\definecolor{olive}{rgb}{0.4, 0.4, 0.0}
\definecolor{teal}{rgb}{0.0,0.5,0.5}
\definecolor{orange}{rgb}{0.9,0.6,0.2}
\definecolor{lightyellow}{RGB}{255, 255, 179}
\definecolor{vlightyellow}{RGB}{255, 255, 200}
\definecolor{lightgreen}{RGB}{170, 255, 220}
\definecolor{vlightgreen}{RGB}{190, 255, 230}
\definecolor{teal}{RGB}{141,211,199}
\definecolor{darkbrown}{RGB}{121,37,0}

\lstset{ language=Caml, basicstyle=\upshape\sffamily,
keywordstyle=\upshape\sffamily\color{dkpurple}, keepspaces=true,
framexleftmargin=1ex, framexrightmargin=1ex, showstringspaces=true,
commentstyle=\itshape\rmfamily,
emph={rep,iterate,synth,collapse,perm,squash,normalize,using,ins,del,lens,let,get,put,rquot,lquot,id,swap,concat,or,disconnect,merge_left,merge_right,const,skip},
emphstyle=\upshape\sffamily\color{dkpurple}, 
columns=fullflexible,
mathescape, 
xleftmargin=1.5em,
% BCP: I find this distracting:
stringstyle=\sffamily\color{dkblue},
}

\begin{abstract}
Recent work has shown how to synthesize a class of string bijections from
regular expression specifications. These bijective string transformations
can be useful for synchronizing simple data formats, but many interesting
formats require non-bijective lenses. In this talk, we show how to expand
the class of synthesized transformations beyond bijections to a large subset
of full symmetric lenses (including all the standard asymmetric lenses)
using lightweight user annotations that help identify a ``bijective core.''
\end{abstract}


\section{Synthesizing Bijective Transformations}

Bidirectional languages can be difficult to program in, proposing unfamiliar
programming paradigms, novel syntax, and subtle typing constraints.
Fortunately, their restricted expressive power, together with the
fine-grained type systems needed to guarantee that well-types programs
satisfy strong round-tripping properites, make them a perfect target for
program synthesis.  

The {\em Optician} tool is an extension to Boomerang~\cite{boomerang}---a
DSL for writing lenses~\cite{Focal2005-long2} on strings---that synthesizes
{\em bijective} Boomerang lenses when provided with a pair of types (regular
expressions) and a small number of examples (often zero).

\section{Building Complex Transformations With Bijective Cores}

Bijections are an important class of bidirectional transformations, but many
useful transformations are not completely bijective. Boomerang permits defining
non-bijective transformations, such as lenses and quotient lenses, though
Optician cannot synthesize them. Lenses are bidirectional transformations that
may lose information when going from one side to the other, while obeying
certain roundtripping properties. Quotient lenses~\cite{quotientlenses} obey
similar properties to lenses, but also permit treating certain pieces of
information as irrelevant -- normalizing them to a canonical representative.

While we wish to make creating these lenses easier, Optician's synthesis algorithm
only works on bijective lenses. However, lenses and quotient lenses seem to have
a ``bijective core'', a component that merely changes the shape of data.
This bijective component can then be surrounded by projection and quotient
components to form a full-fledged quotient lens.

At its base, this surrounding can be done solely with Optician and Boomerang,
but doing so requires repeated code. For example, consider the example of
converting an index of musical composers (with names, birth and death dates, and
nationalities) to a simple index of musical composers (with just names and
nationalities), but with irrelevant whitespace. In this example, ``.''
represents concatenation and ``\lstinline{*}'' represents the Kleene star.

\begin{lstlisting}
  let full_composers : regex = (name . dates . nationality)*
  let composers_nodates : regex = (nationality . " "* . name)*
\end{lstlisting}

In this example, the bijective core is the bijection that transmits nationality and
names (while swapping them), the projection component removes \lstinline{dates},
and the quotient component normalizes the whitespace.

We can build a projection component for \lstinline{full_composers} using
existing Boomerang combinators.
%
\begin{lstlisting}
  let full_projection = iterate (name . (del dates) . nationality)
\end{lstlisting}
%
Here, \lstinline{del} deletes data when going in one direction, and restores it
when going backwards.

We can also build a normalization component for \lstinline{dates} using existing
Boomerang combinators.
%
\begin{lstlisting}
  let ws_norm = normalize (del " "*)
  let nodates_norm = iterate (nationality . (rquot ws_norm) . name)
\end{lstlisting}
%
Here, \lstinline{normalize} turns a lens into a ``canonizer'' (in effect, a
normalization function). In this example, \lstinline{normalize} canonizes an
arbitrarily long list of whitespace to the empty string. Then, the canonizer is
hooked into a lens with \lstinline{rquot}, essentially normalizing strings when
the lens is being operated from right-to-left.

With these normalization and projection components defined, the bijective core
can be synthesized (using \lstinline{synth $R$ <=> $S$}), and the full lens can be
provided by composing all the subcomponents (using ``;'').
\begin{lstlisting}
  let bij_core = synth (targetType full_projection) <=> (sourceType nodates_norm)
  let composers : full_composers <=> composers_nodates = full_projections ; bij_core ; nodates_norm
\end{lstlisting}

Note how in both the projection and quotient component, the bijective aspects
are repeated with no real alteration -- they are merely required scaffolding
for expressing the projections and normalizations.

\section{Projection and Quotient Annotations}

Instead of manually writing explicit canonization and projection functions, we
consider an alternative approach: by annotating our regular expressions with
canonization and projection annotations, we can write our data formats and find
the bijective core in a single step. This approach has already been employed in
synthesizing bidirectional transformations that are bijections (up to
quotients)~\cite{maina+:quotient-synthesis}, but here we extend it with projections as well.

Consider the left-hand format of \lstinline{full_composers}. We expand our
regular expression types with the new combinator \lstinline{skip}. Annotating a
regular expression to be skipped does exactly what \lstinline{full_projection}
does in the previous code -- it sends the minified format to the core synthesis
engine, then composes the generated lens with a lens that deletes the regular
expression from the data format.
%
\begin{lstlisting}
  let full_composers' : annotated_regex = (name . (skip dates) . nationality)*
\end{lstlisting}
%

Consider the right-hand format of \lstinline{composers_nodates}. We expand our
regular expression types with a number of new combinators (to address different
types of quotients), one of which is
\lstinline{squash}~\cite{maina+:quotient-synthesis}. Annotating a regular
expression to be squashed does exactly what \lstinline{nodates_norm} does in the
previous code -- it sends the normalized format to the core synthesis engine,
then composes the generated lens with a lens that normalizes the whitespace to a
representative (in the example, the empty string).
%
\begin{lstlisting}
  let composers_nodates' : annotated_regex = (nationality . (squash " "*) . name)*
\end{lstlisting}
%

Now, with annotated regular expressions, we can synthesize the composers lens much
more compactly.
%
\begin{lstlisting}
  let composers'  = synth full_composers' <=> composers_nodates'
\end{lstlisting}
%
Synthesizing with these annotated regular expressions performs the same steps
that we previously did manually: it breaks the lens into concatenation and
normalization subcomponents, sends the bijective core to the existing synthesis
algorithm, and composes the generated subcomponents with the synthesized
bijective core.

While we have shown how these annotations can work for quotient lenses, we could
even imagine getting them to work for symmetric lenses. If we viewed symmetric
lenses as also having a bijective core (with projections on both sides instead
of just one), providing \lstinline{skip} annotations on the right could even
make synthesizing symmetric lenses easier. Of course, this would require
symmetric lenses in Boomerang for the generated lenses to be well-typed.

\bibliographystyle{alpha} 
\bibliography{local}
%inline the .bbl file directly for mailing to authors.

\end{document}


