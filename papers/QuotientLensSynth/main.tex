\newif\ifdraft\drafttrue  % set true to show comments

\documentclass{svproc}

\usepackage{amsmath, amssymb, verbatim, enumerate, 
graphicx, centernot, tikz, array, tikz-cd, extarrows, cleveref,
mathrsfs, mathtools, bussproofs, stmaryrd, enumitem, stackengine}

%%%%%
% Macros
% Colors
\definecolor{dkblue}{rgb}{0,0.1,0.5}
\definecolor{dkgreen}{rgb}{0,0.6,0}
\definecolor{dkred}{rgb}{0.6,0,0}
\definecolor{dkpurple}{rgb}{0.7,0,0.4}
\definecolor{olive}{rgb}{0.4, 0.4, 0.0}
\definecolor{teal}{rgb}{0.0,0.5,0.5}
\definecolor{orange}{rgb}{0.9,0.6,0.2}
\definecolor{lightyellow}{RGB}{255, 255, 179}
\definecolor{lightgreen}{RGB}{170, 255, 220}
\definecolor{teal}{RGB}{141,211,199}
\definecolor{darkbrown}{RGB}{121,37,0}

\newcommand{\FINISH}[3]{\ifdraft\textcolor{#1}{[#2: #3]}\fi}
\newcommand{\bcp}[1]{\FINISH{dkred}{B}{#1}}
\newcommand{\BCP}[1]{\FINISH{dkred}{B}{\bf #1}}
\newcommand{\afm}[1]{\FINISH{dkgreen}{A}{#1}}
\newcommand{\dpw}[1]{\FINISH{dkblue}{D}{#1}} % Toronto Maple Leafs Blue :-)
\newcommand{\saz}[1]{\FINISH{orange}{SZ}{#1}}
\newcommand{\ksf}[1]{\FINISH{teal}{K}{#1}}
\newcommand{\sam}[1]{\FINISH{dkpurple}{SM}{#1}}

% FOR Regular Expression names
\newcommand{\re}[1]{\ensuremath{\mathtt{#1}}}
\newcommand{\squash}[3]{\ensuremath{\mathit{squash} \; (#1, #2, #3)}}
\newcommand{\perm}[2]{\ensuremath{\mathit{perm}\; (#1)\; \mathit{with}\; #2}}
\newcommand{\normalize}[3]{\ensuremath{\mathit{normalize} \; (#1, #2, #3)}}
\newcommand{\sep}{\ensuremath{\; | \;}}
\newcommand{\canonizer}{\ensuremath{\mathit{Canonizer}}}
\newcommand{\bibtex}{\textsc{Bib}\TeX{}}
\newcommand{\get}{\ensuremath{\mathit{get}}}
\newcommand{\lput}{\ensuremath{\mathit{put}}}
\newcommand{\create}{\ensuremath{\mathit{create}}}
\newcommand{\eqrel}[1]{\ensuremath{\equiv_{#1}}}


%%%%%%%%%%%%%%%%%%%%%%%%%%%%%%%%%%%

\begin{document}
\mainmatter              % start of a contribution
%
\title{Synthesizing Quotient Lenses}
%
\titlerunning{Synthesizing Quotient Lenses}  % abbreviated title (for running
% head)                                     also used for the TOC unless
%                                     \toctitle is used
%
\author{}
%
\authorrunning{} % abbreviated author list (for running head)
%
%%%% list of authors for the TOC (use if author list has to be modified)
\tocauthor{}
%
\institute{}

\maketitle              % typeset the title of the contribution

\begin{abstract}
{\em A Quotient lens} is an expression which encapsulates a bidirectional
transformation whereby each of the two transformations that are part of the lens
have the same behaviour on data that are equivalent to each other modulo some
equivalence relation. Quotient lenses save the programmer the effort of writing
two expressions instead of one, give guarantess on how the forward and backward
transformations interact with each other, and also allow the programmer to
ignore differences in the data which are not intended to affect the
behaviour of the transformations.

However, quotient lenses are sometimes difficult to program, firstly because the
programmer has to give a specification for the data as well as give a separate
specification for the equivalence relations defined on the data. Secondly, even
after giving both of these specifications, the programmer also has to manually
write the underlying lens which maps between the respective equivalence
classes, a task which is itself challenging.

In this paper, we address both of these challenges. First, we propose the
language of {\em Quotient Regular Expressions} or QREs, which enables the
programmer to explicitly define a broad class of equivalence relations
directly on regular languages using a compact and convenient notation, and hence
negates the need to specify the data and the equivalence relation on the data
separately. We then define a class of {\em QRE lenses}, a class of quotient
lenses which are typed using QREs, and show how to {\em synthesize} QRE lenses
from their QRE types and a set of input-output examples. Finally, we
demonstrate the applicability of this approach by synthesizing QRE lenses
between several real-world data formats from the {\tt data.gov} database.
\end{abstract}

\section{Introduction}
{\em Bidirectional programming languages} such as
Boomerang~\cite{boomerang}, Augeas~\cite{augeas}, XSugar~\cite{xsugar},
biXid~\cite{bixid} and X/Inv~\cite{xinv}enable a programmer to express a
transformation that maps some input data to some output data, as well as a
transformation that maps the output data back into the input data in a single
expression. An example of a {\em bidirectional transformation} which we shall
revisit severally in this paper is a program which converts citation records
such as
\begin{verbatim}
@Book {conway,
  Author = {Conway, J. H.},
  Title = {Regular Algebra and Finite Machines},
}
\end{verbatim}

\noindent stored in the \bibtex format to their equivalent representation 
\begin{verbatim}
%0 Book
%T Regular Algebra and Finite Machines
%A Conway, J. H.
%F conway
\end{verbatim}

\noindent in the EndNote format, and vice versa.

Bidirectional programs have also been called {\em lenses}. Formally, Foster et
al~\cite{foster} defined a lens $\ell$ that maps between an ``abstract" {\em
source} $S$ and a ``concrete" {\em view} $V$ to be a triple of functions
\begin{align*}
\ell.get &: S \longrightarrow V\\
\ell.put &: V \longrightarrow S \longrightarrow S\\
\ell.create &: S \longrightarrow V
\end{align*} 

\noindent satisfying the equations
\begin{align*}
\ell.\lput \; (\ell.\get \; s) \; s &= s\\
\ell.\get \; (\ell.\lput \; v \; s) &= v\\
\ell.\lput \; (\ell.\create \; v) &= v\\
\end{align*}

The $\get$ function presents a view of the data from the source, the
$\lput$ function enables an updated view to be folded back into the source, and
the $\create$ component produces a copy of the source from a view, using
defaults to fill in information that may have been discarded by the $\get$
function in producing the view. The first law states that the $\lput$ function
must restore all the information discarded by $\get$ when its arguments are an
abstract structure and a concrete structure that generates the very same
abstract structure. The second and third laws state that $\lput$ and $\create$
must propagate all of the information contained in their abstract arguments to
the concrete structure they produce. These laws express fundamental expectations
about how the components of a lens should work together.

However, the constraints that are imposed by the lens laws are sometimes too
strict. For instance, the programmer may wish to ignore updates to the data
that only change the order of fields, or add whitespace characters that in
reality are not part of the data. Foster et al therefore proposed
\textit{quotient lenses} ~\cite{quotient} as a solution to this problem
Quotient lenses, which are implemented as a refinement of the bidirectional
string processing language Boomerang, enable a programmer to write lenses which
transform data that is quotiented out by an equivalence relation, so that
records which are equivalent to each other modulo this relation are considered
the same by the underlying lens.

\sam{Go over quotient lens laws?}

Unfortunately, quotient lenses are sometimes difficult to program in Boomerang,
firstly because the programmer has to give a specification for the data as well
as give a separate specification for the equivalence relations defined on the
data. For example, suppose that the programmer wants to program the \bibtex to
EndNote transformation in such a way that the program allows for any of two
representations of the authors name (e.g. ``Conway, J. H." and ``Conway J. H"),
ignores the order in which the author and title fields occur in each record,
and allows for arbitrary amounts of whitespace in between the author and title
fields. These rules allow for these two \bibtex citation records to be
considered the same,

\begin{verbatim}
@Book {conway,
  Author = {Conway, J. H.},
  Title = {Regular Algebra and Finite Machines},
}

@Book{conway,
    Title = {Regular Algebra and Finite Machines},
        
    Author = {Conway J. H.},
}
\end{verbatim}

with the EndNote records having a similar equivalence relation defined on them.

In Boomerang, equivalence relations are expressed using {\em canonizers}
which are functions that map each element of a regular language to a
canonical representation of that element. Therefore, in order for the programmer
to give a specification for the \bibtex records, then the programmer
has to write a regular expression which matches all possible representations of
\bibtex records, and then program a set of canonizers which does the job of
quotienting the data. This often makes the code verbose and error-prone.

\sam{TODO: PUT BOOMERANG CANONIZERS HERE!}

To address this issue, we propose the language of {\em Quotient Regular
Expressions} or QREs, which enables the programmer to explicitly define a broad
class of equivalence relations directly on regular languages using a compact
and convenient notation, and hence negates the need to specify the data and the
equivalence relation on the data separately. For example, a QRE which
specifies the equivalence relation on the \bibtex citation records is as
follows:

\begin{verbatim}
let bib_author  = 
"author = {" . squash bib_space_names -> bib_comma_names using 
(get (synth bib_space_names <=> bib_comma_names)) . "}"

let bib_ignore_wsp =  project ("," . [ \n\t\r]* ) -> ",\n"

let bib_perm : canonizer = perm (label, bib_author, bib_title) with
bib_ignore_wsp

let bibtex  = "@book{" . bib_perm . "}"
\end{verbatim}

\sam{Should we have a special construct like squash-synth(R, R') in addition to
squash(R, R', l), or maybe even just have only squash-synth(R, R')?}

The first expression specifies the author field as
having one of two representations (\verb!bib_space_names! and
\verb!bib_comma_names!), with the \verb!bib_comma_names! being the
canonical representation and the expression 

\noindent \verb!(get (synth bib_space_names <=> bib_comma_names))! giving the
function which converts names described by \verb!bib_space_names! to their canonical
form described  by \verb!bib_comma_names!.

The second expression specifies the regular language which is a comma followed
by arbitrary occurences of whitespace characters quotiented by the equivalence
relation which makes all members of this language equivalent, with the canonical
representation being a comma followed by a newline character.

The third expression specifies that the label, title and author fields
are can occur in any order, with each of these fields separated by a comma and
arbitrary amounts of whitespace. The canonical representaion is that in which
the label, author and title fields occur in that order, separated by a comma and
a newline.

However, even though QREs enable the programmer to specify the data as well as
the equivalence relation on the data in the same expression, the programmer
still has to manually program the underlying lens which maps the equivalence
classes of the \bibtex records to the equivalence classes of the EndNote
records

\sam{TODO: PUT BIJECTIVE LENS HERE!},

\noindent task which is itself challenging and erro-prone.

Therefore, to address this challenge we demonstrate how to {\em synthesize}
quotient lenses from a pair of QREs and a set of input-output example pairs. Our
key insight is to reduce the problem of synthesizing QRE lenses to the problem
of synthesizing {\em bijective lenses}, a problem which we solved in out prior
work ~\cite{popl18}.

Our approach is to define a class of {\em QRE lenses} which are the same as
bijective lenses with canonizers at the ends. That is, a QRE lens uses the
source QRE to compute a unique representative for the data modulo the
equivalence relation defined by the QRE before applying a bijective lens to
this representative in the forward direction, and operates similarly same in the
backward direction, but using the view QRE.

\sam{TODO: PUT LENS WITH CANONIZERS ON THE EDGE DIAGRAM HERE!}

One potential pitfall with QRE lenses is that the composition of QRE lenses may
not have the normal form of a bijective lens with canonizers on the ends:

\sam{TODO: PUT COMPOSITION DIAGRAM HERE!}

Our main technical contribution in this paper is a proof that QRE lenses are
closed under left quotienting, right quotienting, composition, and regular
combinators. This proof will allow justify our approach for synthesizing
quotient lenses so that the programmer can simply type

$$\mathit{synth} \; \verb!bibtex! \; \verb!endnote!$$

to derive a quotient lens that does the \bibtex to EndNote transformation.

In summary, our main contributions are:
\begin{enumerate}
\item We introduce a novel language of {\em Quotient Regular Expressions}
(QREs) that provide compact, convenient notation for a natural and useful
class of equivalence relations on regular languages (Section~\ref{QRE}).
\item We define a language of {\em QRE lenses}, a form of quotient lenses
whose types are given by QREs.  Our main technical contribution is a normal
form for QRE lenses and a proof that QRE lenses in normal form are closed
under left quotienting, right quotienting, composition, and regular
operators (Section~\ref{QRE-lenses}).
\item Using this normal form, we reduce the problem of {\em synthesizing}
QRE lenses from their QRE types and a set of examples to the previously studied
problem of synthesizing bijective lenses between regular languages 
(Section~\ref{synth}).
\item We extend the Boomerang {\em implementation} with QREs, QRE lens operators,
and QRE lens synthesis and demonstrate its applicability by using it to
synthesize QRE lenses between several real-world data formats from the
{\tt data.gov} database (Section~\ref{impl}).
\end{enumerate}
Sections~\ref{relwork} and~\ref{concl} discuss related and future work.

\section{Quotient Regular Expressions}
\label{QRE}

This section introduces Quotient Regular Expressions or QREs. QREs are regular
expressions augmented with syntax that lets them simultaneously express an
equivalence relation on the language described by the regular expression.

For example, suppose that the programmer wishes to write a regular expression
which describes \bibtex citation records such as:
\begin{verbatim}
@Book {conway,
  Author = {Conway, J. H.},
  Title = {Regular Algebra and Finite Machines},
}
\end{verbatim}

Each record may be represented in many different ways. Firstly, the author's
name can have the last name followed by a comma, and then the
first and second names separated by a space, or the last name, first name and
second name separated by spaces in between them. For example, each of these
\bibtex records is equivalent to the other:
\begin{verbatim}
@Book {conway,
  Author = {Conway, J. H.},
  Title = {Regular Algebra and Finite Machines},
}
\end{verbatim}
\begin{verbatim}
@Book {conway,
  Author = {Conway J. H.},
  Title = {Regular Algebra and Finite Machines},
}
\end{verbatim}

Secondly, the author and title fields may occur in either order:
\begin{verbatim}
@Book {conway,
  Author = {Conway, J. H.},
  Title = {Regular Algebra and Finite Machines},
}
\end{verbatim}
\begin{verbatim}
@Book {conway,
  Title = {Regular Algebra and Finite Machines},
  Author = {Conway, J. H.},
}
\end{verbatim}
Thirdly, the title and author records can be separated by arbitrary amounts of
whitespace characters. For example, if the title field occurs first, then the
author field may occur not on the line immediately below the title field, but
instead on the second line below the title field:
\begin{verbatim}
@Book {conway,
  Title = {Regular Algebra and Finite Machines},
  
  Author = {Conway, J. H.},
}
\end{verbatim}
In the example above, a \bibtex record may have different
representations for three different reasons. The first reason
is that the author's name can have the last name followed by a comma and then
the first and second names separated by a space, or the last name, first name and
second name separated by spaces in between them. The $\squash{R}{R'}{f}$ QRE
enables the programmer to choose between the representations described by $R$
and $R$', provided that they can provide a function $f$ that maps data stored in
the first format to data stored in the second format. In this example we have
the two regular expressions

\begin{align*}
\mathtt{COMMA\_NAME} &= \mathtt{NAME} \cdot ``," \cdot \mathtt{NAME} \cdot ``
\quad " \cdot \mathtt{NAME}\\
\mathtt{SPACE\_NAME} &= \mathtt{NAME} \cdot ``," \cdot \mathtt{NAME} \cdot ``
\quad " \cdot \mathtt{NAME}\\
\end{align*}

and the bijective lens 
$$ \ell = (\mathit{copy} \; \mathtt{COMMA\_NAME}) \cdot (\mathit{ins} \; ``,")
\cdot (\mathit{del} \; `` \quad") \cdot (\mathit{copy} \; \mathtt{COMMA\_NAME})
\cdot (\mathit{copy} \; \mathtt{COMMA\_NAME})$$
therefore the QRE
$$\mathtt{AUTHOR}
=
``\text{Author
= \{}" \cdot \squash{\mathtt{COMMA\_NAME}}{\mathtt{SPACE\_NAME}}{\ell.get}
\cdot ``\}"$$ enables the programmer to express the fact that the two different
representations of the author's name are equivalent, with the
$\mathtt{COMMA\_NAME}$ representation chosen as the canonical representation,
and the function $\ell.get$ giving the ``canonizing'' function which converts
data in the $\mathtt{SPACE\_NAME}$ format to its $\mathtt{COMMA\_NAME}$
representation. The $\squash{R}{R'}{f}$ has the added advantage that if the
transformation from $R$ to $R'$ can be expressed by a bijective lens,
then the programmer may attempt to \textit{synthesize} the lens from $R$ to $R'$
because of our previous work where we demonstrated how to synthesize bijective
lenses from a pair of regular expressions and a set of input-output example
pairs:
$$\mathtt{AUTHOR}
=
``\text{Author
= \{}" \cdot \squash{\mathtt{COMMA\_NAME}}{\mathtt{SPACE\_NAME}}{\mathit{synth}
\; \mathtt{COMMA\_NAME} \; \mathtt{SPACE\_NAME}} \cdot ``\}"$$

The second reason that a record may have two different representations is that
the author and title fields may occur in a different order, and the third
reason is that the author and title fields may have arbitrary occurences of
whitespace characters in between them. The $R \mapsto s$ and $\perm{q_1,
\ldots, q_n}{q}$ QREs enable the programmer to handle each of these cases:
$$\perm{\re{AUTHOR}, \re{TITLE}}{((``," \cdot \re{WHITESPACE}^*) \mapsto \string
``, \backslash n")}$$
This QRE enables the programmer to consider any permutation of the author and
title fields interspersed with a comma and arbitrary whitespace as equivalent.
The $((``," \cdot \re{WHITESPACE}^*) \mapsto \string ``, \backslash n")$
subexpression is itself a QRE that denotes that all string $s$ which consists
of a comma $``,"$ followed by any number of whitespace should be considered
equivalent, with the canonical representative being the string $\string ``,
\backslash n"$. The canonical representation of \bibtex records is that in
which the author field occurs the title field with a comma then a newline
character separating the two.

The regular combinators for QREs i.e. $\cdot$ (concatenation), $|$ (union) and
$*$ (iteration) behave like the regular combinators for regular languages. For
instance, if the QRE for matching a \bibtex \; record is named
$\mathtt{RECORD}$, then the QRE for matching two records is $\mathtt{RECORD}
\cdot \mathtt{RECORD}$, the QRE for matching one or two records is $\mathtt{RECORD}
\sep (\mathtt{RECORD} \cdot \mathtt{RECORD})$ and the QRE for matching any
number of QRE records in $\mathtt{RECORD}^*$.

The $\normalize{R}{R'}{f}$ QRE is the most general QRE in that for each of the
the other QREs $q$, there exist regular expressions $R, R'$ and a surjective,
idempotent function $f:\mathcal{L}(R) \longrightarrow \mathcal{L}(R')$ such that
$q$ $\normalize{R}{R'}{f}$ each denote the same equivalence relation on the
same regular language. $\normalize{R}{R'}{f}$ enables the programmer to
provide a canonizing function from $\mathcal{L}(R)$ to $\mathcal{L}(R')$,
with the equivalence class on $\mathcal{L}(R)$ being the equivalence relation
defined by the {\em fibres} of $f$ (i.e. $r \sim r'$ if and only if $f(r) =
f(r'))$. The canonizing function is required to be surjective and idempotent.
The $\normalize{R}{R'}{f}$ combinator therefore enables the programmer to express
equivalence relations which are difficult to express, or which cannot be
expressed using the other combinators.

In summary each QRE $q$ enables us to express
\begin{enumerate}
  \item a regular expression $W(q)$ (the ``whole'' of $q$),
  \item an equivalence relation $\eqrel{q}$ on $\mathcal{L}(W(q))$,
  \item a regular expression $K(q)$ (the ``kernel'' of $q$)
  such that $\mathcal{L}(K(q))$ forms a complete set of representatives for
  $\eqrel{q}$, and
  \item a ``canonizing'' function $\canonizer(q):\mathcal{L}(W(q))
  \longrightarrow \mathcal{L}(K(q))$ which given any $w \in \mathcal{L}(W(q))$,
  computes $\canonizer(q)(w)$ as the unique $k$ in $\mathcal{L}(K(q))$ such that
  $k$ is equivalent to $w$ mod $\eqrel{q}$
  \end{enumerate}
  Observe that since $\canonizer(q)$ computes $\canonizer(q)(w)$ as the unique
  $k$ in $\mathcal{L}(K(q))$ such that $w$ is equivalent to $k$ mod $\eqrel{q}$,
  then the equivalence classes of $\eqrel{q}$ are the same as the fibres of
  $\canonizer(q)$.
\subsection{Syntax of QREs}
The language of Quotient Regular Expressions (QREs) is given by the following
grammar:
\begin{align*}
q := \; &R \sep R \mapsto s \sep \squash{R}{R'}{f} \sep
\perm{q_1, \ldots, q_n}{q} \;  | \; \normalize{R}{R'}{f}\\
&q' \circ q \sep q \cdot q' \sep (q \sep q') \sep q^*,
\end{align*}
where $R$ ranges over regular expressions, $f$ ranges over functions between
regular languages, and $s$ ranges over character strings.
  
\subsection{Semantics of QREs}
\subsubsection{Preliminaries}
Let $q, q'$ be QREs. When applying the regular combinators to QREs, we require
that if a string $s$ matches any of the regular expressions $W(q) \cdot W(q')$,
$W(q) \sep W(q')$, $W(q)^*$, $K(q) \cdot K(q')$,
$K(q) \sep K(q')$, $K(q)^*$, then $s$ matches that regular expression in
only one way. This unambiguity condition is called {\em strong unambiguity},
and is necessary, firstly because it ensures that the canonizing function of a
QRE is well-defined. For example if $\mathtt{RECORD}$ is the QRE for a \bibtex
record in our \bibtex to and $W(\mathtt{ENDNOTE} \cdot \mathtt{ENDNOTE})$ is not
unambiguous, that is there is a string $s$ such that $s = s_1 \cdot s_2 =
{s_1}' \cdot {s_2}'$ with $s_1, s_2, {s_1}', {s_2}' \in W(\mathtt{ENDNOTE} \cdot
\mathtt{ENDNOTE})$, then the canonizer $f$ for $\mathtt{ENDNOTE} \cdot
\mathtt{ENDNOTE}$ may not be well defined since we are not guaranteed that
$f(s_1) \cdot f(s_2) = f({s_1}') \cdot f({s_2}')$. We also require that the
regular combinators applied to the kernels are unambiguous since the underlying
lenses will end up operating on the kernels, and these lenses impose the same
restrictions for similar reasons.

To this end, we say that regular expressions $R$ and $S$ are
\textit{unambiguosly concatenable}, written $R \cdot^! S$ if for all strings
$r, r' \in \mathcal{L}(R)$ and $s, s' \in \mathcal{L}(S)$, if $r \cdot s = r'
\cdot s'$, then $r = r'$ and $s = s'$. We say that a regular expression $R$ is
\textit{unambiguosly iterable}, written $R^{*!}$ if for all strings $r_1,
\ldots, r_m$ and $r'_1, \ldots, r'_n \in \mathcal{L}(R)$, if $r_1 \cdot \ldots
\cdot r_m = r'_1 \cdot \ldots \cdot r'_n$, then $m = n$ and $r_i = r'_i$.

We say that a regular expression $R$ is \textit{strongly unambiguous} if and
only if (1) $R = \varnothing$, or (2) $R = S_1 \cdot S_2$ with $S_1, S_2$
strongly unambiguous and $S_1 \cdot^! S_n$, or (3) $R = S_1 \sep S_2$ with
$S_1, S_2$ strongly unambiguous and $\mathcal{L}(S_1) \cap \mathcal{L}(S_2) =
\varnothing$, or (4) $R = S^*$ with $S$ strongly unambiguous and $S^{*!}$.

The semantics for QREs that formally define the whole language $W(q)$, the
kernel language $K(q)$, the equivalence relation $\eqrel{q}$ and the
canonizing function $\canonizer(q)$ are given in
Figures~\ref{fig:wk},~\ref{fig:relations} and~\ref{fig:canonizers}.
\begin{figure}[t]
  \centering
  \[
    \begin{array}{l@{\quad}l@{\quad}l}
   
      q & W(q) & K(q) \\ \hline
      R & R & R \\
      R \mapsto s & W(q) & s \\
      \squash{R}{R'}{f} & R \sep R' & R' \\
      \normalize{R}{R'}{f} & R & R' \\
      q_1 \circ  q_2 & W(q_2) & K(q_1) \\
      q_1 \cdot q_2 & W(q_1) \cdot W(q_2) & K(q_1) \cdot K(q_2) \\
      q_1 \sep q_2 & W(q_1) \sep W(q_2) & K(q_1) \sep K(q_2) \\
      q^* & W(q)^* & K(q)^* \\
    \end{array}
  \]
\[
\begin{array}{r@{\quad}l}
W( \perm{q_1, \ldots, q_n}{q} ) = &
\bigcup \limits_{\sigma \in S_n} W(q_{\sigma(1)}) \cdot W(q) \cdot \ldots \cdot
W(q) \cdot W(q_{\sigma(n)})
\\
K( \perm{q_1, \ldots, q_n}{q} ) = &
 K(q_1) \cdot K(q) \cdot \ldots \cdot K(q) \cdot K(q_n) 
\end{array}
\]
  \caption{Whole and Kernel Regular Expressions\bcp{left-justify columns,
      and try putting the last two clauses into the same format as the
      others (using $\mathit{qq} = ...$ and maybe reducing spacing around
      cdots, etc.)}}
  \label{fig:wk}
\end{figure}

\begin{figure}[t]
  \centering
\[
    \begin{array}{l@{\quad}l@{\quad}l} 
      w \; \equiv_R \; w' &\iff& w = w \\
      w \; \equiv_{R \mapsto s} \; w' \text{ for all }w, w'\\
      w \; \equiv_{\squash{R}{R'}{f}} \; w' &\iff& f(w) = w'
      \text{ or } w = w' \\
      w \; \equiv_{\normalize{R}{R'}{f}} \; w' &\iff&
      f(w)=f(w') \text{ or }w = w'\\
      w \; \equiv_{q_2 \circ q_1} \; w' &\iff& \exists k, k' \in
  \mathcal{L}(K(q_2)) \text{ such that } w \; \equiv_{q_1} \; k, \; w' \;
  \equiv_{q_1} \; k', \text{ and } k \; \equiv_{q_2} \; k'\\
      w \; \equiv_{q_1 \cdot q_2} \; w'  &\iff& w = r_1
      \cdot r_2, \; w' = {r'}_1 \cdot {r'}_2 \text{ with } r_1 \; \equiv_{q_1}
      \; {r'}_1, \; r_2 \; \equiv{q_n} \; {r'}_2\\
      w \; \equiv_{q' \sep q} \; w' &\iff& w \; \equiv_{q_1} \; w'
      \text{ or } \; w \; \equiv_{q_2} \; w'\\
      w \; \equiv_{q*} \; w' &\iff& w = r_1 \cdot \ldots \cdot r_n, \; w'
      = {r'}_1 \cdot \ldots \cdot {r'}_n \text{ and } r_i \equiv_{q} \; {r'}_i
      \\
      w \; \equiv_{\perm{q_1, \ldots, q_n}{q}} \; w' &\iff& w = r_{\sigma(1)}
      \cdot s_1 \cdot \ldots \cdot s_{n-1} \cdot r_{\sigma(n)}, \;
    w' = {r'}_{\theta(1)} \cdot s'_1 \cdot \ldots \cdot s'_{n-1}
    \cdot {r'}_{\theta(n)} \\
    & & \text{ for some } \sigma, \theta \in S_n, \text{ with } r_i \;
    \equiv_{q_i} \; r'_i \text{ and } s_k \; \equiv_{q} \; s'_{k}
    \end{array}
    \]
  \caption{QRE Equivalence Relations}
  \label{fig:relations}
\end{figure}
\begin{figure}[t]
  \begin{center}
\[
    \begin{array}{l@\quad l @\quad l} 
      \canonizer(R) &=& id_{\mathcal{L}(R)} \\
      \canonizer(R \mapsto s)(w) &=& s\\
      \canonizer(\squash{R}{R'}{f})(w) &=& 
\begin{cases}
f(w) & \text{if } w \in \mathcal{L}(R)\\
w & \text{otherwise}
\end{cases}\\
      \canonizer(\normalize{R}{R'}{f}) &=& f\\
      \canonizer(q' \circ q) &=& \canonizer(q') \circ \canonizer(q)\\
      \canonizer(q' \cdot q) &=& \canonizer(q') \cdot \canonizer(q)\\
      \canonizer(q_1 \sep q_2)(w) &=& 
\begin{cases}
\canonizer(q_1)(w) & \text{if } w \in \mathcal{L}(W(q_1))\\
\canonizer(q_2)(w) & \text{if } w \in \mathcal{L}(W(q_2))\\
\end{cases}\\
      \canonizer(q^*) &=& \canonizer(q)^* \\
    \end{array}
    \]
    \end{center}
    $\canonizer(\perm{q_1, \ldots, q_n}{q})(r_{\sigma(1)}
\cdot s_1 \cdot \ldots \cdot s_{n-1} \cdot r_{\sigma(n)}) \newline
= (\canonizer(q_1)(r_1)) \cdot (\canonizer(q)(s_1)) \cdot \ldots \cdot
(\canonizer(q_n)(r_n))$
  \caption{QRE canonizers}
  \label{fig:canonizers}
\end{figure}
\begin{figure}[t]
\centering
\begin{prooftree}
\AxiomC{$R$ is strongly unambiguous}
\UnaryInfC{$\mathit{id}(R)$ is well formed}
\end{prooftree}

\begin{prooftree}
\AxiomC{$s \in W(q)$}
\UnaryInfC{$R \mapsto s$ is well formed}
\end{prooftree}

\begin{prooftree}
\AxiomC{$\mathcal{L}(R) \cap \mathcal{L}(R') = \varnothing$}
\AxiomC{$f : \mathcal{L}(R) \longrightarrow \mathcal{L}(R')$}
\BinaryInfC{$squash(R, R', f)$ is well formed}
\end{prooftree}

\begin{prooftree}
\AxiomC{${\substack{\forall \sigma \neq \theta, \; W(q_{\sigma(1)}) \cdot W(q)
\cdot \ldots \cdot W(q_{\sigma(n)})\\ \cap W(q_{\theta(1)}) \cdot W(q)
\cdot \ldots \cdot W(q_{\theta(n)}) =\varnothing}}$}
\AxiomC{$q_i, q$ are well formed}
\AxiomC{$\forall \sigma, \; K(q_{\sigma(1)}) \cdot^! K(q) \cdot^! \ldots 
\cdot^! K(q_{\sigma(n)})$}
\TrinaryInfC{$\perm{q_1, \ldots, q_n}{q}$ is well formed}
\end{prooftree}

\begin{prooftree}
\AxiomC{$\mathcal{L}(R') \subseteq \mathcal{L}(R)$}
\AxiomC{$f : \mathcal{L}(R) \longrightarrow \mathcal{L}(R')$}
\AxiomC{$f$ is surjective}
\AxiomC{$f = f^2$}
\QuaternaryInfC{$normalize \; (R,R', f)$ is well formed}
\end{prooftree}

\begin{prooftree}
\AxiomC{$q$ is well formed}
\AxiomC{$q'$ is well formed}
\AxiomC{$K(q) = W(q')$}
\TrinaryInfC{$q' \circ q$ is well formed}
\end{prooftree}

\begin{prooftree}
\AxiomC{$q$ is well formed}
\AxiomC{$q'$ is well formed}
\AxiomC{$W(q) \cdot^! W(q')$}
\AxiomC{$K(q) \cdot^! K(q')$}
\QuaternaryInfC{$q \cdot q$ is well formed}
\end{prooftree}

\begin{prooftree}
\AxiomC{$q$ is well formed}
\AxiomC{$q'$ is well formed}
\AxiomC{$W(q) \cap W(q') = \varnothing$}
\TrinaryInfC{$q \cdot q$ is well formed}
\end{prooftree}

\begin{prooftree}
\AxiomC{$q$ is well formed}
\AxiomC{$W(q)^{*!}$}
\AxiomC{$K(q)^{*!}$}
\TrinaryInfC{$q^*$ is well formed}
\end{prooftree}
  \caption{QRE Inference Rules\bcp{Maybe some or all of these can be
      suppressed in the ``short version'' of the paper}\bcp{No, cancel
      that.  These are important.  And we need to comment about the decidability
    issues involving normalize.  (And a note that we don't really use
    normalize, but rather introduce more specialized instances.)}}
  \label{fig:qrerules}
\end{figure}
The theorem below confirms that the semantics of QREs are consistent with their
intended behaviour:
\begin{theorem}
If $q$ is a well formed QRE, then
\begin{enumerate}
  \item $W(q)$ and $K(q)$ are well defined regular expressions,
  \item  $\eqrel{q}$ is an equivalence relation on $\mathcal{L}(W(q))$,
  \item  $\mathcal{L}(K(q))$ forms a complete set of representatives for
  $\eqrel{q}$,
  \item $\canonizer(q):\mathcal{L}(W(q)) \longrightarrow \mathcal{L}(K(q))$ is a
  well-defined function, and
  \item  given any $w \in \mathcal{L}(W(q))$, $\canonizer(q)(w)$ is the unique
  $k$ in $\mathcal{L}(K(q))$ such that $k$ is equivalent to $w$ modulo
  $\eqrel{q}$.
  \end{enumerate}
\end{theorem}

\section{QRE Lenses}
\label{QRE-lenses} 
As we have seen, QREs enable the programmer to express a regular expression $R$
as well as an equivalence relation on $\mathcal{L}(R)$. QREs are therefore a
good \textit{specification} language for quotient lenses. In this section of the
paper, we introduce \textit{QRE Lenses}. QRE lenses are quotient lenses based on
the Boomerang quotient lens combinators, but which use QREs to describe their
source and view types.

Our approach in defining QRE lenses is as follows. Let $c, c'$ be QREs. Given a
bijection $\ell : K(c) \Leftrightarrow K(c')$, we may define a quotient lens $q
: W(c)/\eqrel{c} \Leftrightarrow W(c')/\eqrel{c'}$ by
\begin{equation}\label{normalform}
q.get = \ell \circ \canonizer(c) \text{ and } q.put = \ell^{-1} \circ
\canonizer(c'),
\end{equation}

A potential problem with this approach is that quotient lenses of this form may
not necessarily be closed under the action of the lens combinators. For
example, given two lenses $q$ and $q'$, the composition of $q' \circ q$ may not
be of this form.

Consequently, in order to define quotient lenses, we need to choose a class of
bijections that has just the right properties with respect to the quotient
lens combinators. Our choice is to use the class of \textit{bijective
lenses} which we identified and studied in our previous work~\cite{popl18}.
Bijective lenses, which we briefly discuss in the next subsection, possess
the properties needed to ensure that all QRE lenses have the normal form
suggested in \cref{normalform}.

\subsection{Bijective Lenses}
Given regular expressions $R, S$ and equivalence
relations $\sim_R, \sim_S$ defined on $\mathcal{L}(R)$ and $\mathcal{L}(S)$
respectively, a \textit{bijective quotient lens} $q :
R/{\sim_R}{\Longleftrightarrow} S/{\sim_S}$ from $R$ to $S$ is
a pair of functions $q.get:
\mathcal{L}(R) \longrightarrow \mathcal{L}(S)$ and $q.put : \mathcal{L}(S)
\longrightarrow \mathcal{L}(R)$ such that
\begin{align*}
q.put \; (q.get \; r) &\sim_R r\\
q.get \; (q.put \; s) &\sim_S s
\end{align*}
Additionally the components of $q$ must respect $\sim_R$ and $\sim_S$ i.e.
if $r \sim_R r'$ and $s \sim_S s'$ then $q.get \; r \; \sim_S q.get \; r'$
and $q.put \; s \sim_R q.put \; s' \; s'$.

We define the set of \textit{bijective lenses} to be the set of bijections
between regular languages created from Boomerang lens combinators.
The syntax for the language of bijective lenses is given by
$$\ell := \mathit{id} \; (R) \sep const(s, s') \sep  swap(\ell,
\ell) \sep \ell \cdot \ell' \; |  \; (\ell \sep \ell') \sep \ell^* \;
| \; \ell \circ \ell',$$ where $R$ ranges over regular expressions and $s$
ranges over character strings.

‘The denotation of a lens $\ell$ is $\llbracket \ell \rrbracket \subseteq
\mathit{String} \times \mathit{String}$. If $(s_1, s_2) \in \llbracket \ell
\rrbracket$, then $\ell$ maps between $s_1$ and $s_2$.

\begin{align*}
\llbracket R \rrbracket &= \{(r, r) \sep r \in \mathcal{L}(R)\}\\
\llbracket swap(\ell, \ell') \rrbracket &= \{(s \cdot t, t' \cdot s') \sep
(s, s') \in \llbracket \ell \rrbracket \text{ and } (t, t') \in \llbracket
\ell' \rrbracket\}\\
\llbracket \ell \cdot \ell' \rrbracket &= \{(s \cdot t, s' \cdot t) \sep
(s, s') \in \llbracket \ell \rrbracket \text{ and } (t, t') \in \llbracket
\ell' \rrbracket\}\\
\llbracket \ell \sep \ell' \rrbracket &= \{(s \cdot t) \sep
(s, t) \in \llbracket \ell \rrbracket \text{ or } (s, t) \in \llbracket
\ell' \rrbracket\}\\
\llbracket \ell^* \rrbracket &= \{(s_1 \cdot \ldots \cdot s_n, t_1 \cdot \ldots
\cdot t_n) \sep (s_i, t_i) \in \llbracket \ell \rrbracket \text{ for } 1
\leq i \leq n\}
\end{align*}

The $\mathit{const}(s, t)$ lens replaces the string $s$ with $t$ in the source
data in the forward direction, and $t$ with $s$ in the view data in the backward
direction. $\mathit{id}(R)$ applies the identity function to the source and view
in $\mathcal{L}(R)$ in both directions. The composition lens $\ell' \circ \ell$
applies $\ell$ followed by then $\ell'$ to the source in the forward direction,
and applies $\ell'$ followed by $\ell$ to the view in the backward direction.
The lens $\ell \cdot \ell'$ first splits the string $s$ into $s_1$ and $s_2$,
applies $\ell$ and $\ell'$ to $s_1$ and $s_2$ to get $t_1$ and $t_2$
respectively, then concatenates $t_1$ and $t_2$ and returns $t_1 \cdot t_2$ as
the final result in the forward direction. $\ell \cdot \ell'$ operates
similarly in the backward direction, but with $s, s_1$ and $s_2$ substituted
for $t, t_1$ and $t_2$. The $\mathit{swap} \; (\ell, \ell')$ lens operates
like $\ell \cdot \ell'$, except that it swaps $t_1$ and $t_2$ before
concatenating the two for a final result of $t_2 \cdot t_1$ in the forward
direction. In the backward direction, $\mathit{swap}(\ell, \ell')$ first undoes
the swap, then proceeds as expected. The $\ell \; | \; \ell'$ lens
chooses to apply $\ell$ or $\ell'$ depending on whether the source
(resp. view) data is matched by $\ell$ or $\ell'$ in the forward (resp.
backward direction. The $\ell^*$ lens splits the string $s$ into strings $s_1,
\ldots, s_n$, applies $\ell$ to each $s_i$ to get $t_i$, and then concatenates
each of the $t_i$'s for a final result of $t_1 \cdot \ldots \cdot t_n$ in the
forward direction. $\ell^*$ operates similarly in the backward direction, but
with $s, s_i$ substituted for $t, t_i$.

Each bijective lens $\ell$ has a type $\ell : R \Leftrightarrow S$ where $R$ and
$S$ are regular expressions. If $\ell : R \Leftrightarrow S$, then the source
type of $\ell$ is $\mathcal{L}(R)$ and the view type of $\ell$ is
$\mathcal{L}(S)$. Interestingly, with the bijective lens type system, a lens
$\ell : R \Leftrightarrow S$ can also be considered to be of type $\ell : R'
\Leftrightarrow S'$ provided that $R$ (resp. $S$) can be proven to be
equivalent to $R'$ (resp. $S$) from the star-semiring axioms. The typing rules
for bijective lenses are given in Figure~\ref{fig:lensrules}. 

 \begin{figure}[t]
  \begin{prooftree}
\AxiomC{$s_1 \in \Sigma^*$}
\AxiomC{$s_2 \in \Sigma^*$}
\BinaryInfC{$\mathit{const} \; (s_1, s_t): s_1 \Leftrightarrow s_2$}
\end{prooftree}

\begin{prooftree}
\AxiomC{$R$ is strongly unambiguous}
\UnaryInfC{$\mathit{id} \; (R): R \Leftrightarrow R$}
\end{prooftree}

\begin{prooftree}
\AxiomC{$\ell_1 : R_1 \Leftrightarrow S_1$}
\AxiomC{$\ell_2 : R_2 \Leftrightarrow S_2$}
\AxiomC{$R_1 \cdot^! R_2$}
\AxiomC{$S_1 \cdot^! S_2$}
\QuaternaryInfC{$\ell_1 \cdot \ell_2: R_1 \cdot R_2 \Leftrightarrow S_1 \cdot
S_2$}
\end{prooftree}

\begin{prooftree}
\AxiomC{$\ell_1 : R_1 \Leftrightarrow R_2$}
\AxiomC{$\ell_2 : R_2 \Leftrightarrow R_3$}
\BinaryInfC{$\ell_2 \circ \ell_1: R_1 \Leftrightarrow R_3$}
\end{prooftree}

\begin{prooftree}
\AxiomC{$\ell_1 : R_1 \Leftrightarrow S_1$}
\AxiomC{$\ell_2 : R_2 \Leftrightarrow S_2$}
\AxiomC{$R_1 \cdot^! R_2$}
\AxiomC{$S_2 \cdot^! S_1$}
\QuaternaryInfC{$\mathit{swap} \; (\ell_1, \ell_2): R_1 \cdot R_2
\Leftrightarrow S_2 \cdot S_1$}
\end{prooftree}

\begin{prooftree}
\AxiomC{$\ell_1 : R_1 \Leftrightarrow S_1$}
\AxiomC{$\ell_2 : R_2 \Leftrightarrow S_2$}
\AxiomC{$\mathcal{L}(R_1) \cap \mathcal{L}(R_2) = \varnothing$}
\AxiomC{$\mathcal{L}(S_1) \cap \mathcal{L}(S_2) = \varnothing$}
\QuaternaryInfC{$\ell_1 \sep \ell_2: R_1 \sep R_2 \Leftrightarrow S_1 \sep S_2$}
\end{prooftree}

\begin{prooftree}
\AxiomC{$\ell : R \Leftrightarrow S$}
\AxiomC{$R^{*!}$}
\AxiomC{$S^{*!}$}
\TrinaryInfC{$\ell^*: R \Leftrightarrow S$}
\end{prooftree}

\begin{prooftree}
\AxiomC{$\ell : R \Leftrightarrow S$}
\AxiomC{$R \equiv R'$}
\AxiomC{$S \equiv S'$}
\TrinaryInfC{$\ell : R' \Leftrightarrow S'$}
\end{prooftree}
  \caption{Bijective Lens Typing Rules}
  \label{fig:lensrules}
\end{figure}

\subsection{Syntax of QRE Lenses}
Having given a brief overview of the class of bijective lenses, we now introduce
the class of QRE lenses, whose language is given by following grammar:
$$ q := \mathit{lift}(\ell) \sep \mathit{lquot}(q, \ell) \sep
\mathit{rquot}(\ell, q) \sep q \cdot q' \sep (q \sep q') \sep q^* \sep q \circ q',$$
where $\ell$ ranges over bijective lenses.

The QRE lens combinators are inspired by the Boomerang lens combinators of the
same name. The $\mathit{lift}(\ell)$ quotient lens enables a bijective lens
$\ell$ to be considered a quotient lens where the equivalence relation applied
to the source and view types is the equality relation. The $\mathit{lquot}(c,
q)$ combinator takes a quotient lens $q$ and quotients the source data using
$c$, assuming that the source data forms a complete set of representatives for
the equivalence relation $\eqrel{c}$. The $\mathit{lquot}(q, c)$ does the same
but on the view data. The regular combinators $\cdot \; * \; |$ exhibit the
expected behaviour. \sam{Examples needed?}

The difference between QRE lenses and Boomerang quotient lenses is the typing
judgements. More concretely, each QRE lens $q$ has a type $q : c \Leftrightarrow
c'$ where $c, c'$ are QREs. In contrast, the approach used to type quotient
lenses in Boomerang is to classify equivalences according to whether they are
or are not the equality relation. This type system is based on two
observations: first, that most quotient lenses originate as lifted basic
lenses, and therefore have types whose equivalence relations are both equality;
and second, that equality is preserved by many of the quotient lens
combinators. Foster et al also discuss a second possible approach to typing
quotient lenses, where equivalence relations are represented by rational
functions that induce them. While this second approach is more refined than the
first, Boomerang favours the first approach since the second appears to be too
expensive to be useful in practice~\cite{quotientlenses}.

\subsection{Semantics of QRE Lenses}
As we mentioned earlier, every quotient lens $q$ has a type $q :
c \Longleftrightarrow c'$ where $c$ and $c'$ are QREs. The denotation
$\llbracket q \rrbracket$ of a QRE lens $q:c \Leftrightarrow c'$ is a quotient
lens $\llbracket q \rrbracket : W(c)/{\eqrel{c}} \Longleftrightarrow
W(c')/{\eqrel{c'}}$. The typing rules and denotation of QRE lenses are given in
Figure~\ref{fig:qlenssemantics}. The following theorem confirms that QRE lenses
exhibit the intended behaviour.

\begin{theorem}
If there is a derivation $q:c \Leftrightarrow c'$ then
$\llbracket q \rrbracket : W(c)/\eqrel{c} \Leftrightarrow W(c')/\eqrel{c'}$ is a
well-defined quotient lens.
\end{theorem}

 \begin{figure}[t]
  \centering
  \begin{prooftree}
\AxiomC{$\ell : R \Leftrightarrow S$}
\UnaryInfC{$\mathit{lift}(\ell): R \Leftrightarrow id(S)$}
\end{prooftree}
  \begin{align*}
  \llbracket q \rrbracket.get &=  \llbracket \ell \rrbracket, \text{ and }\\
  \llbracket q \rrbracket.put &= \llbracket \ell \rrbracket^{-1}
  \end{align*}
 
\begin{prooftree}
\AxiomC{$q' : c'  \Leftrightarrow c''$}
\AxiomC{$c$ is well formed}
\AxiomC{$K(c) = W(c')$}
\TrinaryInfC{$\mathit{lquot}(c, q'): c' \circ c \Leftrightarrow c''$}
\end{prooftree}
  \begin{align*}
  \llbracket q \rrbracket.get  &= \llbracket q'
  \rrbracket.get \circ \canonizer(c)\\
  \llbracket q \rrbracket.put &= \llbracket q' \rrbracket.put
  \end{align*}

  \begin{prooftree}
  \AxiomC{$q' : c \Leftrightarrow c'$}
  \AxiomC{$c''$ is well formed}
  
\AxiomC{$K(c'') = W(c')$}
\TrinaryInfC{$\mathit{rquot}(q', c):c \Leftrightarrow c'' \circ c'$}
\end{prooftree}
  \begin{align*}
  \llbracket q \rrbracket.get &= \llbracket q'
  \rrbracket.get\\
  \llbracket q \rrbracket.put &= \llbracket q'
  \rrbracket.put \circ \canonizer(c'')
  \end{align*}
  
  \begin{prooftree}
\AxiomC{$q_1 : c \Leftrightarrow c'$}
\AxiomC{$q_2 : c' \Leftrightarrow c''$}
\BinaryInfC{$q_2 \circ q_1: c \Leftrightarrow c''$}
\end{prooftree}
  \begin{align*}
  \llbracket q \rrbracket.get &= \llbracket q_2 \rrbracket.get\circ \llbracket
  q_1 \rrbracket.get, \text{ and }\\
  \llbracket q \rrbracket.put &= \llbracket q_1 \rrbracket.put \circ \llbracket
  q_2 \rrbracket.put
  \end{align*}

    \begin{prooftree}
\AxiomC{$q' : c \Leftrightarrow c'$}
\AxiomC{$W(c)^{*!}$ and $W(c')^{*!}$}
\AxiomC{$K(c)^{*!}$ and $K(c')^{*!}$}
\TrinaryInfC{${q'}^* : c^* \Leftrightarrow {c'}^*$}
\end{prooftree}
  \begin{align*}
  \llbracket q \rrbracket.get &= (\llbracket q' \rrbracket.get)^*, \text{
  and }\\
  \llbracket q \rrbracket.put &= (\llbracket q' \rrbracket.put)^*
  \end{align*}

    \begin{prooftree}
\AxiomC{$q_1 : c_1 \Leftrightarrow d_1 $}
\AxiomC{$q_2 : c_2 \Leftrightarrow d_2$}
\AxiomC{${\substack{W(c_1) \cdot^! W(c_2)\\ K(c_1) \cdot^! K(c_2)}}$}
\AxiomC{${\substack{W(d_1) \cdot^! W(d_2)\\ K(d_1) \cdot^! K(d_2)}}$}
\QuaternaryInfC{$q_1 \cdot q_2: c_1 \cdot c_2
\Leftrightarrow d_1 \cdot d_2$}
\end{prooftree}
  \begin{align*}
  \llbracket q \rrbracket.get &= \llbracket q_1 \rrbracket.get \cdot \llbracket
  q_2 \rrbracket.get, \text{ and }\\
  \llbracket q \rrbracket.put &= \llbracket q_1 \rrbracket.put \cdot \llbracket
  q_2 \rrbracket.put
  \end{align*}

      \begin{prooftree}
\AxiomC{$q_1 : c_1 \Leftrightarrow d_1 $}
\AxiomC{$q_2 : c_2 \Leftrightarrow d_2$}
\AxiomC{$\mathcal{L}(W(c_1)) \cap \mathcal{L}(W(c_2)) = \varnothing$}
\AxiomC{$\mathcal{L}(W(d_1)) \cap \mathcal{L}(W(d_2)) = \varnothing$}
\QuaternaryInfC{$q_1 \sep q_2: (c_1 \sep c_2)
\Leftrightarrow (d_1 \sep d_2)$}
\end{prooftree}
  $$
  \llbracket q_1 \sep q_2 \rrbracket.get(s) = 
  \begin{cases}
  \llbracket q_1 \rrbracket.get (s) & \text{if } s \in \mathcal{L}(W(c_1))\\
  \llbracket q_2 \rrbracket.get (s) & \text{if } s \in \mathcal{L}(W(c_2))\\
  \end{cases}$$
  $$\llbracket q_1 \sep q_2 \rrbracket.put(s) = 
  \begin{cases}
  \llbracket q_1 \rrbracket.put (s) & \text{if } s \in \mathcal{L}(W(d_1))\\
  \llbracket q_2 \rrbracket.put (s) & \text{if } s \in \mathcal{L}(W(d_2))\\
  \end{cases}
  $$
  \caption{Denotation and Typing Rules for QRE Lenses}
  \label{fig:qlenssemantics}
\end{figure}

\subsection{Normal Forms of QRE Lenses}
Recall that our approach in defining QRE lenses is to have each QRE lens $q: c
\Leftrightarrow c'$ be such that 
\begin{align*}
\llbracket q \rrbracket.get &= \ell \circ \canonizer(c)\\
\llbracket q \rrbracket.put &= \ell^{-1} \circ
\canonizer(c')
\end{align*}
for some bijective lens $\ell$. We now provide a proof that all QRE lenses are
of this form.
\begin{theorem}\label{normal form}
If there is a derivation $q : c \Leftrightarrow c'$, then there exists a
bijective lens $\ell : K(c) \Leftrightarrow K(c')$ such that
\begin{align*}
\llbracket q \rrbracket.get &= \llbracket \ell \rrbracket\circ \canonizer(c)\\
\llbracket q \rrbracket.put &= \llbracket \ell \rrbracket^{-1} \circ
\canonizer(c')
\end{align*}
\end{theorem}
\begin{proof}
Assume that $q : c \Leftrightarrow c'$. We proceed by induction over the
derivation $q : c \Leftrightarrow c'$.
\begin{enumerate}
  \item
  $\mathit{lift}(\ell): R/\mathit{id}(R) \Leftrightarrow S/\mathit{id}(S)$ where
  $\ell :
  R \Leftrightarrow S$. Then
  \begin{align*}
  \llbracket \mathit{lift}(\ell) \rrbracket.get &=  \llbracket \ell \rrbracket
  = \llbracket \ell \rrbracket \circ id_{\mathcal{L}(R)} =
  \llbracket \ell \rrbracket \circ \canonizer(\mathit{id}(R)), \text{ and }\\
  \llbracket \mathit{lift}(\ell) \rrbracket.put &= \llbracket \ell
  \rrbracket^{-1} = \llbracket \ell \rrbracket^{-1} \circ id_{\mathcal{L}(S)} =
  \llbracket \ell \rrbracket^{-1} \circ \canonizer(id(S))
  \end{align*}
  \item
  $\mathit{lquot}(c, q'): c' \circ c \Leftrightarrow c''$ where $q' : c' 
  \Leftrightarrow c''$, $c$ is well formed and $K(c) = W(c')$. Then
\begin{align*}
  \llbracket q \rrbracket.get  &= \llbracket q'
  \rrbracket.get \circ \canonizer(c)\\
  \llbracket q \rrbracket.put &= \llbracket q' \rrbracket.put
  \end{align*}
  By the induction hypothesis, there exists a bijective lens $\ell :
  K(c') \Leftrightarrow K(c'')$ such that 
  \begin{align*}
\llbracket q' \rrbracket.get &= \llbracket \ell \rrbracket \circ \canonizer(c')\\
\llbracket q' \rrbracket.put &= \llbracket \ell \rrbracket^{-1} \circ
Canonize(c'')
\end{align*}
Consequently
\begin{align*}
  \llbracket q \rrbracket.get  &= (\llbracket \ell \rrbracket \circ
  \canonizer(c')) \circ \canonizer(c) = \llbracket \ell \rrbracket \circ
  (\canonizer(c' \circ c))\\
  \llbracket q \rrbracket.put &= \llbracket \ell \rrbracket^{-1} \circ
  Canonize(c'')
  \end{align*}

  \item
  $\mathit{rquot}(q', c''):c \Leftrightarrow c'' \circ c'$ where $q' : c \Leftrightarrow
  c'$, $c''$ is well formed and $K(c'') = W(c')$. Proceed as in the previous
  case.
\item
$q_2 \circ q_1: c \Leftrightarrow c''$ where $q_1 : c \Leftrightarrow c'$ and
$q_2 : c' \Leftrightarrow c''$. Then
  \begin{align*}
  \llbracket q \rrbracket.get &= \llbracket q_2 \rrbracket.get\circ \llbracket
  q_1 \rrbracket.get, \text{ and }\\
  \llbracket q \rrbracket.put &= \llbracket q_1 \rrbracket.put \circ \llbracket
  q_2 \rrbracket.put
  \end{align*}
  By the induction hypothesis, there exist bijective lenses
  $\ell_1 :
  K(c) \Leftrightarrow K(c')$ and $\ell_2 : K(c') \Leftrightarrow K(c'')$ such
  that
  \begin{align*}
\llbracket q_1 \rrbracket.get &= \llbracket \ell_1 \rrbracket \circ
\canonizer(c)\\
\llbracket q_1 \rrbracket.put &= {\llbracket \ell_1 \rrbracket}^{-1} \circ
\canonizer(c')
\end{align*}
and
\begin{align*}
\llbracket q_2 \rrbracket.get &= \llbracket \ell_2 \rrbracket \circ
Canonize(c')\\
\llbracket q_2 \rrbracket.put &= {\llbracket \ell_2 \rrbracket}^{-1} \circ
Canonize(c'')
\end{align*}
Consequently,
\begin{align*}
\llbracket q_2 \rrbracket.get \circ \llbracket q_1 \rrbracket.get &=
(\llbracket \ell_2 \rrbracket \circ \canonizer(c')) \circ (\llbracket \ell_1
\rrbracket \circ \canonizer(c))\\
&= \llbracket \ell_2 \rrbracket \circ (\canonizer(c') \circ \llbracket \ell_1
\rrbracket) \circ \canonizer(c)\\
&= (\llbracket \ell_2 \rrbracket \circ \llbracket \ell_1 \rrbracket) \circ
\canonizer(c)\\
&= \llbracket \ell_2  \circ  \ell_1 \rrbracket \circ
\canonizer(c)
\end{align*} 
A similar argument shows that 
$$\llbracket q_1 \rrbracket.put \circ \llbracket q_2 \rrbracket.put =
\llbracket \ell_2  \circ  \ell_1 \rrbracket^{-1} \circ
\canonizer(c)$$
\item  
${q'}^* : c^* \Leftrightarrow {c'}^*$ where $q' : c \Leftrightarrow c'$,
$W(c)^{*!}$ and $W(c')^{*!}$ and $K(c)^{*!}$ and $K(c')^{*!}$. Then
  \begin{align*}
  \llbracket {q'}^* \rrbracket.get &= (\llbracket q' \rrbracket.get)^*, \text{
  and }\\
  \llbracket {q'}^* \rrbracket.put &= (\llbracket q' \rrbracket.put)^*
  \end{align*}
  By the induction hypothesis there exists a bijective lens $\ell : K(c)
  \Leftrightarrow K(c')$ such that 
   that
  \begin{align*}
\llbracket q' \rrbracket.get &= \llbracket \ell \rrbracket \circ
\canonizer(c)\\
\llbracket q' \rrbracket.put &= {\llbracket \ell \rrbracket}^{-1} \circ
\canonizer(c')
\end{align*}
Consequentlty
\begin{align*}
\llbracket {q'}^* \rrbracket.get &= (\llbracket \ell \rrbracket \circ
\canonizer(c))^* = \llbracket \ell \rrbracket^* \circ
\canonizer(c)^* = \llbracket \ell^* \rrbracket \circ
\canonizer(c^*)\\
\llbracket {q'}^* \rrbracket.put &= (\llbracket \ell \rrbracket^{-1} \circ
\canonizer(c'))^* = (\llbracket \ell \rrbracket^{-1})^* \circ
\canonizer(c')^* = \llbracket \ell^* \rrbracket^{-1} \circ
\canonizer(c'^*)\\
\end{align*}
\item
  $q_1 \cdot q_2: c_1 \cdot c_2 \Leftrightarrow d_1 \cdot d_2$, where $q_1 : c_1
  \Leftrightarrow d_1 $,  $q_2 : c_2 \Leftrightarrow d_2$, $W(c_1)
  \cdot^! W(c_2)$, $K(c_1) \cdot^! K(c_2)$, $W(d_1) \cdot^! W(d_2)$ and $
  K(d_1) \cdot^! K(d_2)$. Then
  \begin{align*}
  \llbracket q \rrbracket.get &= \llbracket q_1 \rrbracket.get \cdot \llbracket
  q_2 \rrbracket.get, \text{ and }\\
  \llbracket q \rrbracket.put &= \llbracket q_1 \rrbracket.put \cdot \llbracket
  q_2 \rrbracket.put
  \end{align*}
By the induction hypothesis, there exist bijective lenses $\ell_1 : K(c_1)
\Leftrightarrow K(d_1)$ and $\ell_2 : K(c_2) \Leftrightarrow K(d_2)$ such that
\begin{align*}
\llbracket q_1 \rrbracket.get &= \llbracket \ell_1 \rrbracket \circ
\canonizer(c_1)\\
\llbracket q_1 \rrbracket.put &= {\llbracket \ell_1 \rrbracket}^{-1} \circ
\canonizer(d_1)
\end{align*}
and
\begin{align*}
\llbracket q_2 \rrbracket.get &= \llbracket \ell_2 \rrbracket \circ
\canonizer(c_2)\\
\llbracket q_2 \rrbracket.put &= {\llbracket \ell_2 \rrbracket}^{-1} \circ
\canonizer(d_2)
\end{align*}
Consequently,
\begin{align*}
  \llbracket q \rrbracket.get &= (\llbracket \ell_1 \rrbracket \circ
\canonizer(c_1)) \cdot  (\llbracket \ell_2 \rrbracket \circ
\canonizer(c_2))\\
&= (\llbracket \ell_1 \rrbracket \cdot \llbracket \ell_2
\rrbracket) \circ (\canonizer(c_1) \cdot \canonizer(c_2))\\
&= \llbracket \ell_1 \cdot  \ell_2 \rrbracket \circ \canonizer(c_1 \cdot c_2)
\end{align*}
Similarly
$$
  \llbracket q \rrbracket.put = \llbracket \ell_1 \cdot  \ell_2 \rrbracket^{-1}
  \circ \canonizer(d_1 \cdot d_2) $$
  \item
  $q = q_1 \sep q_2$ where $q_1 : c_1 \Leftrightarrow d_1 $, $q_2 : c_2
  \Leftrightarrow d_2$, $\mathcal{L}(W(c_1)) \cap \mathcal{L}(W(c_2)) =
  \varnothing$ and $\mathcal{L}(W(d_1)) \cap \mathcal{L}(W(d_2)) = \varnothing$.
  Then
  $$
  \llbracket q_1 \sep q_2 \rrbracket.get(s) = 
  \begin{cases}
  \llbracket q_1 \rrbracket.get (s) & \text{if } s \in \mathcal{L}(W(c_1))\\
  \llbracket q_2 \rrbracket.get (s) & \text{if } s \in \mathcal{L}(W(c_2))\\
  \end{cases}$$
  $$\llbracket q_1 \sep q_2 \rrbracket.put(s) = 
  \begin{cases}
  \llbracket q_1 \rrbracket.put (s) & \text{if } s \in \mathcal{L}(W(d_1))\\
  \llbracket q_2 \rrbracket.put (s) & \text{if } s \in \mathcal{L}(W(d_2))\\
  \end{cases}
  $$
By the induction hypothesis, there exist bijective lenses $\ell_1 : K(c_1)
\Leftrightarrow K(d_1)$ and $\ell_2 : K(c_2) \Leftrightarrow K(d_2)$ such that
\begin{align*}
\llbracket q_1 \rrbracket.get &= \llbracket \ell_1 \rrbracket \circ
\canonizer(c_1)\\
\llbracket q_1 \rrbracket.put &= {\llbracket \ell_1 \rrbracket}^{-1} \circ
\canonizer(d_1)
\end{align*}
and
\begin{align*}
\llbracket q_2 \rrbracket.get &= \llbracket \ell_2 \rrbracket \circ
\canonizer(c_2)\\
\llbracket q_2 \rrbracket.put &= {\llbracket \ell_2 \rrbracket}^{-1} \circ
\canonizer(d_2)
\end{align*}
Consequently,
$$
  \llbracket q_1 \sep q_2 \rrbracket.get(s) = 
  \begin{cases}
  \llbracket \ell_1 \rrbracket \circ
\canonizer(c_1) (s) & \text{if } s \in \mathcal{L}(W(c_1))\\
  \llbracket \ell_2 \rrbracket \circ
\canonizer(c_2) (s) & \text{if } s \in \mathcal{L}(W(c_2)),\\
  \end{cases}$$
  so $\llbracket q_1 \sep q_2 \rrbracket.get = \llbracket \ell_1 \sep
  \ell_2 \rrbracket \circ \canonizer(c_1 \sep c_2)$. A similar argument shows
  that $\llbracket q_1 \sep q_2 \rrbracket.put = \llbracket \ell_1 \sep
  \ell_2 \rrbracket^{-1} \circ \canonizer(d_1 \sep d_2)$.\\
  This completes the proof.
\end{enumerate}
\end{proof}

\section{Synthesizing QRE Lenses}
\label{synth}
By Theorem~\ref{normal form}, if there is a derivation $q : c \Leftrightarrow
c'$ of a QRE lens, then there exists a bijective lens $\ell : K(c)
\Leftrightarrow K(c')$ such that
\begin{align*}
\llbracket q \rrbracket.get &= \llbracket \ell \rrbracket\circ \canonizer(c)\\
\llbracket q \rrbracket.put &= \llbracket \ell \rrbracket^{-1} \circ
\canonizer(c')
\end{align*}


Now suppose that we wish to synthesize a quotient lens $q$ of type $q: c
\Leftrightarrow c'$ that maps a specified set of input strings to a
specified set of output strings and vice versa.

For example, in the \textsc{Bib}\TeX\ to EndNote problem, the programmer wishes
to synthesize a QRE lens $q$ which maps the QRE
$$\perm{\re{REF}, \re{AUTHOR}, \re{TITLE}, \re{PUBLISHER},
\re{YEAR}}{(\re{WHITESPACE}^* \mapsto \string \n)}$$ 
to the QRE
$$\perm{\re{REF'}, \re{AUTHOR'}, \re{TITLE'}, \re{PUBLISHER'},
\re{YEAR'}}{(\re{WHITESPACE}^* \mapsto \string \n)}$$
where $\re{REF}, \re{AUTHOR}, \re{TITLE}, \re{PUBLISHER},$ and $\re{YEAR}$
(resp. $\re{REF'}, \re{AUTHOR'}, \re{TITLE'}, \re{PUBLISHER'},$ and
$\re{YEAR'}$) are regular expressions describing the respective
\textsc{Bib}\TeX\; (resp. EndNote\bcp{Don't like this ``resp.''
  construction---saves space, but not reader energy}) fields of the same
name, and 
$\re{WHITESPACE}$ is a regular expression describing whitespace
characters. Morever, the synthesized lens should map the equivalence
class of this \textsc{Bib}\TeX{} record
\begin{verbatim}
@Book {conway,
  Author = {Conway, J. H.},
  Title = {Regular Algebra and Finite Machines},
  Publisher = {Printed in GB by William Clowes & Sons Ltd},
  Year = {1971}
}
\end{verbatim}
\noindent to the equivalence class of this EndNote entry:
\begin{verbatim}
%0 Book
%T Regular Algebra and Finite Machines
%A Conway, J. H.
%D 1971
%I Printed in GB by William Clowes & Sons Ltd
%F conway
\end{verbatim}

In our previous work, we showed that given regular expressions $R, S$ and a set
of input-output pairs $\{(r_1, s_1), \ldots, (r_n, s_n)\}$, if there is a
derivation $\ell :, R \Leftrightarrow S$ of a bijective lens that maps $r_i$ to
$s_i$ in the forward direction and $s_i$ to $r_i$ in the backward direction,
then there is an algorithm that computes a bijective lens $\ell' : R'
\Leftrightarrow S'$ such that $R \equiv R'$, $S \equiv S'$, $\ell'$ maps $r_i$
to $s_i$ in the forward direction and $s_i$ to $r_i$ in the backward
direction, and $\llbracket \ell \rrbracket = \llbracket \ell'
\rrbracket$~\cite{popl18}.

\section{Implementation and Evaluation}
\label{impl}

\section{Related Work}
\label{relwork}

\bcp{Make sure to list/discuss:
  \begin{itemize}
  \item Seminar papers on lenses (to which we can refer readers for further
  reading about connections to view updates, etc.) and quotient lenses.
  \item Discuss what other lens-like programming languages (besides
  boomerang) do about quotienting.  E.g., Augeas, XSugar, ...
  \item How quotient lenses are used in Boomerang.
  \item Search for papers that cite the quotient lens paper and see if they
  extend the ideas there.
  \end{itemize}
}

\section{Conclusion and Future Work}
\label{concl}

\end{document}
