\newif\ifdraft\drafttrue  % set true to show comments
\newif\ifplentyoftime\plentyoftimefalse  % :-)



% % For double-blind review submission, w/o CCS and ACM Reference (max
% submission space)
\documentclass[acmsmall,review,anonymous]{acmart}\settopmatter{printfolios=true,printccs=false,printacmref=false}
% % For double-blind review submission, w/ CCS and ACM Reference
% \documentclass[acmsmall,review,anonymous]{acmart}\settopmatter{printfolios=true}
% % For single-blind review submission, w/o CCS and ACM Reference (max
% submission space)
% \documentclass[acmsmall,review]{acmart}\settopmatter{printfolios=true,printccs=false,printacmref=false}
% % For single-blind review submission, w/ CCS and ACM Reference
% \documentclass[acmsmall,review]{acmart}\settopmatter{printfolios=true} % For
% final camera-ready submission, w/ required CCS and ACM Reference
% \documentclass[acmsmall]{acmart}\settopmatter{}


% % Journal information % Supplied to authors by publisher for camera-ready
% submission; % use defaults for review submission.
\acmJournal{PACMPL}
\acmVolume{1}
\acmNumber{ICDP} % CONF = POPL or ICFP or OOPSLA
\acmArticle{1}
\acmYear{2018}
\acmMonth{1}
\acmDOI{} % \acmDOI{10.1145/nnnnnnn.nnnnnnn}
\startPage{1}

% % Copyright information % Supplied to authors (based on authors' rights
% management selection; % see authors.acm.org) by publisher for camera-ready
% submission; % use 'none' for review submission.
\setcopyright{none}
% \setcopyright{acmcopyright} \setcopyright{acmlicensed}
% \setcopyright{rightsretained} \ccopyrightyear{2018}           %% If different
% from \acmYear

% % Bibliography style %\bibliographystyle{ACM-Reference-Format} % Citation
% style % Note: author/year citations are required for papers published as an %
% issue of PACMPL.
% %\citestyle{acmauthoryear}   %% For author/year citations


% %%%%%%%%%%%%%%%%%%%%%%%%%%%%%%%%%%%%%%%%%%%%%%%%%%%%%%%%%%%%%%%%%%%%% % Note:
% Authors migrating a paper from PACMPL format to traditional % SIGPLAN
% proceedings format must update the '\documentclass' and % topmatter commands
% above; see 'acmart-sigplanproc-template.tex'.
% %%%%%%%%%%%%%%%%%%%%%%%%%%%%%%%%%%%%%%%%%%%%%%%%%%%%%%%%%%%%%%%%%%%%%


\usepackage{amsmath, amssymb, amsthm, enumerate, graphicx, centernot, tikz,
array, tikz-cd, extarrows, cleveref, mathrsfs, mathtools, stmaryrd, listings}

\newtheorem{definition}{Definition}
\newtheorem{example}{Example}
\newtheorem{conjecture}{Conjecture}
\newtheorem{question}{Question}
\newtheorem{fact}{Fact}
\newtheorem{claim}{Claim}

% Macros Colors
\definecolor{dkblue}{rgb}{0,0.1,0.5}
\definecolor{dkgreen}{rgb}{0,0.6,0}
\definecolor{dkred}{rgb}{0.6,0,0}
\definecolor{dkpurple}{rgb}{0.7,0,0.4}
\definecolor{olive}{rgb}{0.4, 0.4, 0.0}
\definecolor{teal}{rgb}{0.0,0.5,0.5}
\definecolor{orange}{rgb}{0.9,0.6,0.2}
\definecolor{lightyellow}{RGB}{255, 255, 179}
\definecolor{lightgreen}{RGB}{170, 255, 220}
\definecolor{teal}{RGB}{141,211,199}
\definecolor{darkbrown}{RGB}{121,37,0}


\newcommand{\FINISH}[3]{\ifdraft\textcolor{#1}{[#2: #3]}\fi}
\newcommand{\bcp}[1]{\FINISH{dkred}{B}{#1}}
\newcommand{\BCP}[1]{\FINISH{dkred}{B}{\bf #1}}
\newcommand{\afm}[1]{\FINISH{dkgreen}{A}{#1}}
\newcommand{\dpw}[1]{\FINISH{dkblue}{D}{#1}} % Toronto Maple Leafs Blue :-)
\newcommand{\saz}[1]{\FINISH{orange}{SZ}{#1}}
\newcommand{\ksf}[1]{\FINISH{teal}{K}{#1}}
\newcommand{\sam}[1]{\FINISH{dkpurple}{SM}{#1}}


\newcommand{\kw}[1]{\ensuremath{\mathsf{#1}}}
\newcommand{\get}{\ensuremath{\kw{get}}}
\newcommand{\pput}{\ensuremath{\kw{put}}}
\newcommand{\create}{\ensuremath{\kw{create}}}
\newcommand{\ccopy}{\ensuremath{\kw{copy}}}
\newcommand{\const}{\ensuremath{\kw{const}}}
\newcommand{\default}{\ensuremath{\kw{default}}}
\newcommand{\pos}{\ensuremath{\kw{posit}}}
\newcommand{\length}{\ensuremath{\kw{length}}}
\newcommand{\match}{\ensuremath{\kw{match}}}


\lstset{ basicstyle=\ttfamily, escapeinside=|| }
\begin{document}

% % Title information
\title{Stateful Delta Lenses}         %% [Short Title] is optional;
% % when present, will be used in % header instead of Full Title.
% %\titlenote{with title note}             %% \titlenote is optional; % can be
% repeated if necessary; % contents suppressed with 'anonymous'
% %\subtitle{Subtitle}                     %% \subtitle is optional
% %\subtitlenote{with subtitle note}       %% \subtitlenote is optional; % can
% be repeated if necessary; % contents suppressed with 'anonymous'


% % Author information % Contents and number of authors suppressed with
% 'anonymous'.
% % Each author should be introduced by \author, followed by % \authornote
% (optional), \orcid (optional), \affiliation, and % \email.
% % An author may have multiple affiliations and/or emails; repeat the %
% appropriate command.
% % Many elements are not rendered, but should be provided for metadata %
% extraction tools.

% % Author with single affiliation.
% \author{First1 Last1} \authornote{with author1 note}          %% \authornote
% is optional; % can be repeated if necessary \orcid{nnnn-nnnn-nnnn-nnnn} %%
% \orcid is optional \affiliation{ \position{Position1} \department{Department1}
%              %% \department is recommended \institution{Institution1}
%   %% \institution is required \streetaddress{Street1 Address1} \city{City1}
% \state{State1} \postcode{Post-Code1} \country{Country1}                    %%
% \country is recommended } \email{first1.last1@inst1.edu}          %% \email is
% recommended

\author{Solomon Maina}
\position{PhD Student}
\department{Computer Science}              %% \department is recommended
\institution{University of Pennsylvania}            %% \institution is required
\country{USA}                    %% \country is recommended }
\email{smaina@seas.upenn.edu}          %% \email is recommended

% % Abstract % Note: \begin{abstract}...\end{abstract} environment must come %
% before \maketitle command
\begin{abstract}
A {\em lens} is a program that can be run in two directions. In the
forward direction, the \get component of a lens produces a {\em view}
from a {\em source}, while in the backward direction, the \pput component folds
an updated view back into the original source. One of the reasons that
lenses are appealing is that they satisfy laws which give guarantees on how the
two transformations interact.

Unfortunately, the lens laws only require that the \pput component of a lens
return the original source unmodified whenever the update to the view is a
no-op; they do not specify the behaviour of the \pput component when the update
to the view is non-trivial. Consequently, a lens may be valid even though it
does not propagate non-trivial updates made to the view back into the source in
the manner intended. For example, a small update to the view can result
in a large update to the source, such as when a lens updates records
positionally even though the source and view are order agnostic. This paper
introduces {\em stateful delta lenses} or SD-lenses as a solution to this
shortcoming of traditional lenses. SD-lenses are lenses augmented with extra
information that specifies what kind of updates are valid for a given lens, and
how these updates should be propagated by the lens, and hence enable the
programmer to specify more exactly the desired behaviour of a lens.
\end{abstract}


%% Keywords
%% comma separated list
\keywords{difference structure, stateful delta lens}  %% \keywords
% are mandatory in final camera-ready submission


%% \maketitle
%% Note: \maketitle command must come after title commands, author
%% commands, abstract environment, Computing Classification System
%% environment and commands, and keywords command.
\maketitle

\section{Introduction}
A {\em lens} is a program that can be run in two directions. In the
forward direction, the \get component of a lens produces a {\em view}
from a {\em source}, while in the backward direction, the \pput component folds
an updated view back into the original source.

For example suppose that the source is an XML document representing the
names, dates and nationalities of a collection of classical music
composers such as,

\begin{lstlisting}
<composers>
    <composer>
        <name>Jean Sibelius</name>
        <dates>1865-1956</dates>
        <nationality>Finnish</nationality>
    </composer>
    <composer>
        <name>Aaron Copland</name>
        <dates>1910-1990</dates>
        <nationality>American</nationality>
    </composer>
    <composer>
        <name>Benjamin Briten</name>
        <dates>1913-1976</dates>
        <nationality>English</nationality>
    </composer>
</composers>
\end{lstlisting}

while the view is a list if ASCII text representing the names and dates of each
composer:
\begin{lstlisting}
Jean Sibelius, 1865-1956
Aaron Copland, 1910-1990
Benjamin Briten, 1913-1976
\end{lstlisting}

After computing the initial view, we might want to edit it in some way---e.g.,
correcting the error in Sibelius's death date and the misspelling in Britten's
name
\begin{lstlisting}
Jean Sibelius, 1865-|\colorbox{orange}{1957}|
Aaron Copland, 1910-1990
Benjamin |\colorbox{orange}{Britten}|, 1913-1976
\end{lstlisting}
and push the changes back into the original XML format:
\begin{lstlisting}
<composers>
    <composer>
        <name>Jean Sibelius</name>
        <dates>1865-|\colorbox{orange}{1957}|</dates>
        <nationality>Finnish</nationality>
    </composer>
        <composer>
        <name>Aaron Copland</name>
        <dates>1910-1990</dates>
        <nationality>American</nationality>
    </composer>
    <composer>
        <name>Benjamin |\colorbox{orange}{Britten}|</name>
        <dates>1913-1976</dates>
        <nationality>English</nationality>
    </composer>
</composers>
\end{lstlisting}

Lenses are designed to perfom these kinds of computations. If $\ell$ is
a suitably defined lens for the bidirectional tranfsormation that we
have just described, then the \get component of the lens performs the
XML to ASCII transformation that produces an ASCII view from an XML source,
while the \pput component of the lens performs the ASCII to XML transformation
that folds back an updated ASCII view back into the original XML source.

Formally, Foster \cite{foster2007combinators,foster2009bidirectional} defined a lens $\ell :
S \Leftrightarrow V$ from a source $S$ to a view $V$ to be a triple of
functions $\ell.\get : S \longrightarrow V$, $\ell.\pput : S \longrightarrow S
\longrightarrow V$ and $\ell.\create : V \longrightarrow S$ satisfying the
following properties:
\begin{align*}
\ell.\get \; (\ell.\pput \; s \; v) &= v \tag{PUTGET}\\
\ell.\get \; (\ell.\create \; v) &= v \tag{CREATEGET}\\
\ell.\pput \; s \; (\ell.\get \; s) &= s \tag{GETPUT}
\end{align*}
The original definition of lenses in \cite{foster2007combinators} involved only the \get
and \pput functions, but a \create function was later added in
\cite{foster2009bidirectional} to handle the special case where we need to construct a
source from a view, but we have no source to use as the original. \sam{Perhaps
we should include this in the footnotes?}

The PUTGET and CREATEGET laws ensure that updates to the view are translated
``exactly''---i.e. that, given a view, the \pput and \create functions produce
a source that \get maps back to the very same view, while the GETPUT law ensures
a ``stability'' property for the source---i.e. it requires that the \pput
function returns the original source unmodified whenever the update to the view
is a no-op.

While these two laws give some guarantees on how the \get and \pput components
of a lens interact, they are not strong enough to guarantee that the
updates made to a view are indeed the updates reflected back into the source.
This is because the GETPUT law only says that the \pput component must return
the original source unmodified whenever the update to the view is a no-op; it
does not say what the \pput component should do with a modified view.
Consequently, a non-trivial update to a view could lead to an undesired update
to the source.

For example, suppose that a source $S$ consists of a pair of integers, and
that the view $V$ for $S$ consists of a single integer. Further, suppose that we
wish for the \get of any lens from $S$ to $V$ to simply project away
the first factor:
$\ell.\get \; (a, b) = b$. Suppose further that we wish to force the \pput
function of any lens $\ell$ defined on the data to restore the factor projected
away by the \get function: that is, we wish to enforce that for
all integers $a, b, c$, $\pi_1(\ell.\pput \; (a, b) \; c) = a$, where $\pi_1$ is
the projection on the first factor.

Now consider the lens $\ell_{bad}$ defined by $\ell_{bad}.\get \; (a, b) = b$
and
\begin{align*}
\ell_{bad}.\get \; (a, b) &= b\\
\ell_{bad}.\pput \; (a, b) \; c &= \begin{cases}
(a, c) & \text{if }b = c\\
(0, c) & \text{otherwise}\\
\end{cases}\\
\ell.\create \; b &= (0,b)
\end{align*}
Then $\ell_{bad}$ is a valid lens even though $\ell_{bad}$ does not
always restore the first factor. For example, observe that $\ell.\pput (1, 2) \;
3 = (0,3)$. The lens laws do not allow us to specify how non-trivial updates are
propagated, hence the \pput component can cause any update to the source, so
long as the \get component maps the updated source back to the updated view.

In this paper, we introduce {\em stateful delta lenses} or SD-lenses, which are
lenses augmented with extra information that specifies what kind of updates are
valid for a given lens, and how these updates should be propagated.

In order to specify the intended behaviour of a lens, we will first need to
define what updates can be made to the source and view. We do this by embedding
source and views in {\em difference structures}:
\begin{definition}
A difference structure $(X, \partial X, \delta_X, i_X)$ is a 4-tuple such that
$X$ is a set, $\partial X$ is a set denoting the set of updates that can be
made in $X$, $\delta_X : X \times X \longrightarrow \partial X$, the {\em
difference function} of the structure, is a map that takes two elements of $x,
x' \in X$ and computes their difference $\delta_X$ as an element of $\partial
X$, and $i_X : X \longrightarrow \partial X$, the {\em inclusion function} of
the structure, is a map that given an $x \in X$ creates an update $i_X(x)$ in
$\partial X$.
\end{definition}

Going back to our previous example, let $\epsilon$ denote an ``empty''
symbol. On the source side, we define the ordered difference $\delta_S((a, b),
(c, d))$ between $(a, b)$ and $(c, d)$ by
$$\delta_S((a, b), (c,
d)) = \begin{cases}
(\epsilon, \epsilon) & \text{if }a = c, b = d\\
(\epsilon, b) & \text{if }a=c, b \neq d\\
(a, \epsilon) & \text{if }a \neq c, b = d\\
(a, b) & \text{if }a \neq c, b \neq d
\end{cases}$$
On the view side, we define the ordered difference $\delta_V(a, b)$
between $a$ and $b$ by, $$\delta_V(a, b) = \begin{cases}
(\epsilon, \epsilon) & \text{if } a = b\\
(\epsilon, a) & \text{if } a \neq b
\end{cases}$$
where we include an $\epsilon$ in the first factor to enforce the property that
the first factor should always be left unchanged. Thus the update space
$\partial S$ of $S$ is given by $(\{\epsilon\} \sqcup \mathbb{Z})
\times(\{\epsilon\} \sqcup \mathbb{Z})$ and the update space $\partial V$ of
$V$ is also given by $(\{\epsilon\} \sqcup \mathbb{Z}) \times(\{\epsilon\}
\sqcup \mathbb{Z})$. For inclusion functions, we define $i_S : S \longrightarrow
\partial S$ by $i_S(a, b) = (a, b)$ and $i_V : V \longrightarrow \partial V$ by
$i_V(a) = (\epsilon, a)$.

Next we define a lens $\partial(S, V) : \partial S \Leftrightarrow \partial V$
which describes how updates should be propagated. For this example, we let
define $\partial(S,V)$ to simply be the \ccopy lens from $\partial S$ to
$\partial V$, keeping in mind that $\partial S = \partial V$:
\begin{align*}
(\partial(S, V)).\get \; (a, b) &= (a, b)\\
(\partial(S, V)).\pput \; (a, b) \; (c, d) &= (c, d)\\
(\partial(S, V)).\create \; (a, b) &= (a, b)
\end{align*}
To specify the intended behaviour of a lens $\ell$ from $S$ to $V$ we
require that
\begin{equation}\label{partialpreservesdelta}
(\partial (S, V)).\get \; (\delta_S(\ell.\pput \; s \; v, s)) = \delta_V(v,
\ell.\get \; s)
\end{equation}
This condition says that if we start with a source $s$, map $s$ to the view
to get $\ell.\get \; s$ and then update $\ell.\get \; s$ to get a new view $v$,
then the \get component of $\partial(S, V)$ applied to $\delta_S(\ell.\pput \; s
\; v, s)$ is equal to $\delta_V(v, \ell.\get \; s)$.

For example, observe that $\ell_{bad}$ does not satisfy this condition since
\begin{align}
(\partial (S, V)).\get \; (\delta_S(\ell_{bad}.\pput \; (1,2) \; 3, (1,2))) &=
(\partial (S, V)).\get \; (\delta_S((0,3), (1,2)))\\
&= (\partial (S, V)).\get \; (0,3)\\
&= (0,3)\\
&\neq (\epsilon, 3)\\
&= \delta_V(3, 2)\\
&= \delta_V(3, \ell_{bad}.\get \; (1, 2))
\end{align}
Consequently $\ell_{bad}$ is not a valid SD-lens. On the
other hand, the lens $\ell_{good} : S \Leftrightarrow V$ defined by
\begin{align*}
\ell_{good}.\get \; (a, b) &= b\\
\ell_{good}.\pput \; (a, b) \; c &= (a, c)\\
\ell_{good}.\create \; b &= (0, b)
\end{align*}
is indeed a SD-lens since the its \pput component always restores the first
factor.

Up to this point we have not yet explained the role of inclusion functions. Now
assume that we have valid lenses $\ell_1$ and $\ell_2$ from source $S_i$ to
view $V_i$ for $i \in  \{1, 2\}$. One of the standard lens combinators is the
conditional combinator $\ell_1 \sqcup \ell_2$ which uses $\ell_1$ for $s \in
S_1$ and $v \in V_1$ and $\ell_2$ for $s \in S_2$ and $v \in V_2$. Now assume
that $s_1 \in S_1$ and $s_2 \in S_2$. When computing the difference between
$s_1$ and $s_2$, we wish to say that $\delta_{S_1 \sqcup S_2}(s_1, s_2) = s_1$
since $s_1$ and $s_2$ are in disjoing sets and are hence incomparable, but $s_1$
is in $S_1$ rather than in $\partial S_1$! Hence for any source $S$ we need to
embed $s$ as an update inside $\partial S$ and similarly for views, hence the
need for inclusion functions $i_S : S \longrightarrow \partial S$ and $i_V : V
\longrightarrow \partial V$. For inclusion functions $i_S$ and $i_V$, we
require a law similar to \cref{partialpreservesdelta} to hold:
$$(\partial (S, V)).\get \; (i_{S}(s)) = i_{V}(\ell.\get \; s)$$
\section{Stateful Delta Lenses}
\begin{definition}
Let $(S, \partial S, \delta_S, i_S)$ and $(V, \partial V, \delta_V, i_V)$ be
difference structures. A stateful delta-lens (SD-Lens) $\ell : (S, \partial S,
\delta_S, i_S) \Leftrightarrow (V, \partial V, \delta_V, i_V)$ is a pair
$(\ell, \partial (S, V))$ such that for all $s \in S$ and $v \in V$,
\begin{enumerate}
  \item
  $\ell : S \Leftrightarrow V$ is a lens,
  \item
  $\partial(S, V) : \partial S \Leftrightarrow \partial V$ is a lens,
  \item
  $(\partial (S, V)).\get \; (\delta_S(\ell.\pput \; s \; v, s)) = \delta_V(v,
  \ell.\get \; s)$
  \item
  $(\partial (S, V)).\get \; (i_{S}(s)) = i_{V}(\ell.\get \; s)$
\end{enumerate}

\end{definition}
\subsection{Boomerang Lens Combinators}
\subsubsection{Copy}
Let $P$ be any set. Define the $\ccopy(P) : P \Leftrightarrow P$ by
\begin{align}
\ccopy(P).\get \; s &= s\\
\ccopy(P).\pput \; s \; v &= v\\
\ccopy(P).\create \; v &= v
\end{align}
Then $\ccopy(P)$ is a lens. Now let $(S, \partial S, \delta_S, i_S)$ be a
difference structure.
\begin{claim}
$(\ccopy(S),  \ccopy(\partial S)) : (S, \partial S, \delta, i_S)
\Leftrightarrow (S, \partial S, \delta, i_S)$ is an SD-lens.
\end{claim}
\begin{proof}
First we need to show that $(\ccopy(\partial S)).\get \; (\delta(\ccopy(S).\pput
\; s \; v, s)) = \delta(v, \ccopy(S).\get \; s)$ which follows from the fact that
$$ (\ccopy(\partial S)).\get \; (\delta(\ccopy(S).\pput \; s
\; v, s)) = \delta(\ccopy(S).\pput \; s \; v, s) = \delta(v, \ccopy(S).\get
\; s) $$
Next we need to show that $(\ccopy(\partial S)).\get \; (i_{S}(s)) =
i_{S}(\ell.\get \; s)$, which follows from the fact that $(\ccopy(\partial
S)).\get \; (i_{S}(s)) = (\ccopy(\partial S)).\get \; (i_{S}(\ccopy(S).\get \;
s))$
\end{proof}
\subsubsection{Constant}
Let $P$ be any set. Define the \const lens $\const(P) : P \Leftrightarrow \{*\}$
by
\begin{align*}
\const.\get \; s &= *\\
\const.\pput \; s \; v &= s\\
\const.\create \; v &= p
\end{align*}
Now let $(S, \partial S, \delta, i_S)$ be a difference structure.
\begin{claim}
$(\const(S),  \const(\partial S)) : (S, \partial S, \delta_S, i_S)
\Leftrightarrow (\{*\}, \{*\}, \delta_{\{*\}}, i_{\{*\}})$ is an SD-Lens
\end{claim}
\begin{proof}
First we need to show that $(\const(\partial S)).\get \;
(\delta_S(\const(S).\pput \; s \; v, s)) = \delta_{\{*\}}(v, \const(S).\get \;
s)$ which follows from the fact that $ (\const(\partial S)).\get \;
(\delta_S(\const(S).\pput \; s \; v, s)) = * = \delta_{\{*\}}(v, \const(S).\get
\; s) $
Next we need to show that $(\const(\partial S)).\get \; (i_{S}(s)) =
i_{\{*\}}(\const(S).\get \; s)$ which follows from the fact that
$(\const(\partial S)).\get \; (i_{S}(s)) = * = i_{\{*\}}(s)$
\end{proof}

\subsubsection{Default}
Let $\ell : S \Leftrightarrow V$ be a lens. Let $f :
V \longrightarrow S$ be any function. Define the lens $\default(\ell, f) :
S \Leftrightarrow V$ by
\begin{align*}
\default(\ell, f).\get \; s &= \ell.\get \; s\\
\default(\ell, f).\pput \; s \; v &= \ell.\pput \; s \; v\\
\default(\ell, f).\create \; v &= \ell.\pput \; (f \; v) \; v
\end{align*}
Now let $(\ell, \partial (S, V)) : (S, \partial S, \delta_S, i_S) \Leftrightarrow
(V, \partial V, \delta_V, i_V)$ be an SD-Lens and $f: S \longrightarrow V$ be
any function.
\begin{claim}
$(\default(\ell, f), \partial (S, V))$ is an SD-lens.
\end{claim}
\begin{proof}
Follows from the fact that $(\ell, \partial (S, V))$ is an SD-lens.
\end{proof}
\subsubsection{Product}
Let $\ell_1 : S_1 \Leftrightarrow V_1$ and $\ell_2 : S_2 \Leftrightarrow V_2$ be
lenses. Define the lens $\ell_1 \times \ell_2 : (S_1 \times S_2) \Leftrightarrow
(V_1 \times V_2)$ by
\begin{align*}
(\ell_1 \times \ell_2).\get \; (s_1, s_2) &= (\ell_1.\get \; s_1, \ell_2.\get
\; s_1)\\
(\ell_1 \times \ell_2).\pput \; (s_1, s_2) \; (v_1, v_2) &= (\ell_1.\pput \; s_1
\; v_1, \ell_2.\pput \; s_2 \; v_2)\\
(\ell_1 \times \ell_2).\create \; (v_1, v_2) &= (\ell_1.\create \; v_1,
\ell_2.\create \; v_2)
\end{align*}
Given difference functions $\delta_S$ and $\delta_{S'}$ define $\delta_{S}
\times \delta_{S'}$ by $$(\delta_{S} \times \delta_{S'})((s_1, s_2), (s'_1,
s'_2)) = (\delta_S(s_1, s'_1), \delta_{S'}(s'_1, s'_2))$$
Given inclusion functions $i_S$ and $i_{S'}$ define $i_S \times i_{S'}$ by
$$(i_S \times i_{S'})(s, s') = (i_S(s), i_{S'}(s'))$$

Now let $(\ell_1, \partial (S_1, V_1)) : (S_1, \partial S_1, \delta_{S_1},
i_{S_1}) \Leftrightarrow (V_1, \partial V_1, \delta_{V_1}, i_{V_1})$ and $(\ell_2,
\partial (S_2, V_2)) : (S_2, \partial S_2, \delta_{S_2}, i_{S_2})
\Leftrightarrow (V_2, \partial V_2, \delta_{V_2}, i_{V_2})$ be SD-lenses. Define the lens
$$(\ell_1, \partial (S_1, V_1)) \times (\ell_2, \partial (S_2, V_2)) : (S_1 \times S_2,
\partial S_1 \times \partial S_2, \delta_{S_1} \times \delta_{S_2}, i_{S_1}
\times i_{S_2}) \Leftrightarrow (V_1 \times V_2, \partial V_1 \times \partial
V_2, \delta_{V_1} \times \delta_{V_2}, i_{V_1} \times i_{V_2})$$ by $(\ell_1,
\partial (S_1, V_1)) \times (\ell_2, \partial (S_2, V_2)) = (\ell_1 \times \ell_2,
\partial (S_1, V_1) \times \partial (S_2, V_2))$

\begin{claim}\label{productislens}
$(\ell_1 \times \ell_2, \partial (S_1, V_1) \times \partial (S_2, V_2))$ is an SD-lens.
\end{claim}
\begin{proof}
First need to show that
$$(\partial (S_1, V_1) \times \partial (S_2, V_2)).\get \; (\delta((\ell_1 \times
\ell_2).\pput \; (s_1, s_2) \; (v_1, v_2), (s_1, s_2))) = \delta((v_1, v_2),
(\ell_1 \times \ell_2).\get \; (s_1, s_2))$$ which follows from the fact that
\begin{align*}
&(\partial (S_1, V_1) \times \partial (S_2, V_2)).\get \; (\delta((\ell_1 \times
\ell_2).\pput \; (s_1, s_2) \; (v_1, v_2), (s_1, s_2)))\\
&= (\partial (S_1, V_1) \times \partial (S_2, V_2)).\get \; (\delta((\ell_1.\pput \;
s_1 \; v_1, \ell_2.\pput \; s_2 \; v_2), (s_1, s_2))\\
&= (\partial (S_1, V_1) \times \partial (S_2, V_2)).\get \; (\delta(\ell_1.\pput \;
s_1 \; v_1, s_1), \delta(\ell_2.\pput \; s_2 \; v_2, s_2))\\
&= ((\partial (S_1, V_1)).\get \; (\delta(\ell_1.\pput \;
s_1 \; v_1, s_1)), (\partial (S_2, V_2)).\get \; (\delta(\ell_2.\pput \; s_2 \;
v_2, s_2)))\\
&= (\delta(v_1, \ell_1.\get \; s_1), \delta(v_2, \ell_2.\get \; s_2))\\
&= \delta((v_1, v_2), (\ell_1.\get \; s_1, \ell_2.\get \; s_2))\\
&= \delta((v_1, v_2), (\ell_1 \times \ell_2).\get \; (s_1, s_2))
\end{align*}
Next, we need to show that $(\partial (S_1, V_1) \times \partial (S_2, V_2)).\get \;
(i_{S_1 \times S_2}(s_1, s_2)) = i_{V_1 \times V_2}((\ell_1 \times
\ell_2).\get \; (s_1, s_2))$ which follows from the fact that
\begin{align*}
(\partial (S_1, V_1) \times \partial (S_2, V_2)).\get \;
(i_{S_1 \times S_2}(s_1, s_2)) &= (\partial (S_1, V_1) \times \partial
\ell_2).\get \; (i_{S_1}(s_1), i_{S_2}(s_2))\\
&= ((\partial (S_1, V_1)).\get \; (i_{S_1}(s_1)), (\partial (S_2, V_2)).\get \;
(i_{S_2}(s_2)))\\
&= (i_{V_1}(\ell_1.\get \; s_1), i_{V_2}(\ell_2.\get \; s_2))\\
&= i_{V_1 \times V_2}(\ell_1.\get \; s_1, \ell_2.\get \; s_2)\\
&= i_{V_1 \times V_2}((\ell_1 \times \ell_2).\get \; (s_1, s_2))
\end{align*}
\end{proof}
\subsubsection{Sum}
Let $\ell_1 : S_1 \Leftrightarrow V_1$ and $\ell_2 : S_2 \Leftrightarrow V_2$ be
lenses. Define the lens
$\ell_1 \sqcup \ell_2 :
(S_1 \sqcup S_2) \Leftrightarrow (V_1 \sqcup V_2)$ by
\begin{align*}
(\ell_1 \sqcup \ell_2).\get \; s &=
\begin{cases}
\ell_1.\get \; s & \text{if } s \in S_1\\
\ell_2.\get \; s & \text{if } s \in S_2\\
\end{cases}\\
(\ell_1 \sqcup \ell_2).\pput \; s \; v&=
\begin{cases}
\ell_1.\pput \; s \; v& \text{if } s \in S_1, \; v \in V_1\\
\ell_2.\pput \; s \; v& \text{if } s \in S_2, \; v \in V_2\\
\ell_1.\create \; v & \text{if } s \in S_2, \; v \in V_1\\
\ell_2.\create \; v & \text{if } s \in S_1, \; v \in V_2\\
\end{cases}\\
(\ell_1 \sqcup \ell_2).\create \; v &=
\begin{cases}
\ell_1.\create \; v & \text{if } v \in V_1\\
\ell_2.\create \; v & \text{if } v \in V_2\\
\end{cases}
\end{align*}
Given difference functions $\delta_S, \delta_{S'}$ and inclusion functions $i_S,
i_{S'}$ define $\delta_{S} \sqcup \delta_{S'}$ and $i_{S} \sqcup i_{S'}$ by
$$(\delta_{S} \sqcup \delta_{S'})(x, y) =
\begin{cases}
\delta_S(x, y) & \text{if } x, y \in S\\
\delta_{S'}(x, y) & \text{if } x, y \in S'\\
i_S(x) & \text{if } x \in S, y \in S'\\
i_{S'}(x) & \text{if } x \in S', y \in S
\end{cases}$$
and
$$(i_S \sqcup i_{S'})(x) =
\begin{cases}
i_S(x) & \text{if } x \in S\\
i_{S'}(x) & \text{if } x \in S'
\end{cases}$$
Now let $(\ell_1, \partial (S_1, V_1)) : (S_1, \partial S_1, \delta_{S_1}, i_{S_1})
\Leftrightarrow (V_1, \partial V_1, \delta_{V_1}, i_{V_1})$ and $(\ell_2,
\partial (S_2, V_2)) : (S_2, \partial S_2, \delta_{S_2}, i_{S_2}) \Leftrightarrow
(V_2, \partial V_2, \delta_{V_2}, i_{V_2})$ be SD-lenses. Define the lens
$$(\ell_1, \partial (S_1, V_1)) \sqcup (\ell_2, \partial (S_2, V_2)) : (S_1 \sqcup S_2,
\partial S_1 \sqcup \partial S_2, \delta_{S_1} \sqcup \delta_{S_2}, i_{S_1}
\sqcup i_{S_2}) \Leftrightarrow (V_1 \sqcup V_2, \partial V_1 \sqcup \partial
V_2, \delta_{V_1} \sqcup \delta_{V_2}, i_{V_1} \sqcup i_{V_2})$$ by $(\ell_1,
\partial (S_1, V_1)) \sqcup (\ell_2, \partial (S_2, V_2)) = (\ell_1 \sqcup \ell_2,
\partial (S_1, V_1) \sqcup \partial (S_2, V_2))$
\begin{claim}
$\ell_1 \sqcup \ell_2$ is an SD-lens.
\end{claim}
\begin{proof}
We need to show that $(\partial (S_1, V_1) \sqcup \partial (S_2, V_2)).\get \;
(\delta((\ell_1 \sqcup \ell_2).\pput \; s \; v, s)) = \delta(v, (\ell_1 \sqcup
\ell_2).\get \; s)$. If $s \in S_1$ and $v \in V_1$ or $s \in S_2$ and $v \in
V_2$, then the result follows from the fact that $(\ell_1, \partial (S_1, V_1))$ and
$(\ell_2, \partial (S_2, V_2))$ are lenses. Otherwise, assume without loss of
generality that $s \in S_1$ and $v \in V_2$. Then
\begin{align*}
(\partial (S_1, V_1) \sqcup \partial (S_2, V_2)).\get \;
(\delta((\ell_1 \sqcup \ell_2).\pput \; s \; v, s)) &=
(\partial (S_1, V_1) \sqcup \partial (S_2, V_2)).\get \; (\delta(\ell_2.\create \; v,
s))\\
&= (\partial (S_1, V_1) \sqcup \partial (S_2, V_2)).\get \; (i_2(\ell_2.\create \; v))\\
&= (\partial (S_2, V_2)).\get \; (i_{S_2}(\ell_2.\create \; v))\\
&= i_{V_2}(\ell_2.\get \; (\ell_2.\create \; v))\\
&= i_{V_2}(v)\\
&= \delta(v, (\ell_1 \sqcup \ell_2).\get \; s)
\end{align*}
\end{proof}
\subsubsection{Positional Iteration}
Let $\ell: S \Leftrightarrow V$ be a lens. Define the lens $\pos(\ell):
S^* \Leftrightarrow V^*$ by
\begin{align*}
\pos(\ell) .\get \; (s_1, \ldots, s_n) &= (\ell.\get \; s_1) \cdot \ldots \cdot
(\ell.\get \; s_n)\\
\pos(\ell) .\pput \; (s_1, \ldots, s_m) \; (v_1, \ldots, v_n) &= s'_1 \cdot
\ldots \cdot s'_n\\
\text{ where } s'_i &= \begin{cases}
\ell.\pput \; s_i \; v_i & \text{if } 1 \leq i \leq \min\{m, n\}\\
\ell.\create \; v_i & \text{if } \min\{m, n\} < i \leq n
\end{cases}\\
\pos(\ell).\create \; (v_1, \ldots, v_n) &= (\ell.\create \; v_1) \cdot \ldots
\cdot (\ell.\create \; v_n)
\end{align*}
Now let $\delta_S$ be a difference function and $i_S$ and inclusion function.
We define the positional inclusion function $\pos(i_S)(s_1, \ldots, s_m) =
(i_S(s_1), \ldots, i_S(s_m))$. Let $(x_1, \ldots, x_m), (y_1, \ldots, y_n) \in
S^*$. We define the positional difference function $\pos(\delta)((x_1, \ldots,
x_m), (y_1, \ldots, y_n))$ to be the list given by $\delta(x_i, y_i)$ for $1
\leq i \leq \min\{m, n\}$, except that we omit $\delta(x_i, y_i)$ in position
$i$ if $x_i = y_i$. For $\min\{m, n\} < i \leq m$ we add $i_S(x_i)$ at position
$i$. Finally we collapse the list to get rid of the empty spaces left after the
omissions.

For instance suppose that $S$ is the set of all letters in the alphabet and
$\partial S = S^*$ with $$\delta(x, y) =
\begin{cases}
\epsilon & \text{if } x = y\\
x & \text{if } x \neq y
\end{cases}$$
and $i_S(x) = x$. Then $\pos(\delta_S)((a, b, c, d, e), (a, c, b, d)) = (b, c,
e)$ where we match and remove the first $a$ and the fourth $d$ in $(a, b, d, c,
e)$. Since $b \neq c$ we retain the $b$ and $c$ in the second and third
positions, and since $e$ is unmatched, we include $e$ via the inclusion
function $i_S(e) = e$. Finally we collapse the list to get rid of the two empty
spaces created.

Now let $(\ell, \partial (S, V)) : (S, \partial S, \delta_{S}, i_{S})
\Leftrightarrow (V, \partial V, \delta_{V}, i_{V})$ be an SD-lens.
\begin{claim}\label{positionalisvalid}
$(\pos(\ell), \pos(\partial (S, V))) : (S^*, \partial S^*, \pos(\delta_{S}),
\pos(i_{S})) \Leftrightarrow (V^*, \partial V^*, \pos(\delta_{V}), \pos(i_{V}))$
is an SD-lens.
\end{claim}
\begin{proof}
First, we need to show that
\begin{multline}
(\pos(\partial (S, V)).\get \;
((\pos(\delta_S))(\pos(\ell).\pput \; (s_1, \ldots, s_n) \; (v_1, \ldots,
v_m), (s_1, \ldots, s_n))) \\
= (\pos(\delta_V))((v_1, \ldots, v_m), (\pos(\ell)).\get \; (s_1, \ldots, s_n))
\end{multline}
Let $1 \leq i \leq \min\{m, n\}$. Then $v_i = \ell.\get \; s_i$ if and only if
$\ell.\pput \; s_i \; v_i = s_i$, so $\pos(\delta_V)$ omits $\delta_V(v_i,
\ell.\get \; s_i)$ in position $i$ if and only if $\pos(\delta_S)$ also omits
$\delta_S(\ell.\pput \; s_i \; v_i, s_i)$ in position $i$. For $\min\{m,n\}
< i \leq m$, then $\pos(\delta_V)$ adds $i_V(v_i)$ in position $i$, and
$\pos(\delta_S)$ adds $i_S(\ell.\create \; v_i)$ in position $i$, which gives
the desired result since
$$(\partial (S, V)).\get \; (i_S(\ell.\create \; v_i)) = i_V(\ell.\get \;
(\ell.\create \; v_i)) = i_V(v_i)$$
Next we need to show that
$$(\pos(\partial (S, V))).\get \; ({\pos(i_S)}(s_1, \ldots, s_m)) =
\pos(i_V)(\pos(\ell).\get \; (s_1, \ldots, s_m))$$
which follows from the fact that
\begin{align*}
(\pos(\partial (S, V))).\get \; ({\pos(i_S)}(s_1, \ldots, s_m)) &=
(\pos(\partial (S, V))).\get \; ({i_S}(s_1), \ldots, {i_S}(s_m))\\
&= (\ell.\get \; (i_S(s_1)), \ldots, \ell.\get \; (i_S(s_m)))\\
&= (i_V(\ell.\get \; s_1), \ldots, i_V(\ell.\get \; s_m))\\
&= \pos(i_V)(\ell.\get \; s_1, \ldots, \ell.\get \; s_m)\\
&= \pos(i_V)(\pos(\ell).\get \; (s_1, \ldots, s_m))
\end{align*}
\end{proof}
\subsubsection{Matching Lenses}
Let $X$ be a set. Let $x = (x_1, \ldots, x_m) \in X^*$. Define $\length(x) = m$
Now $(y_1, \ldots, y_n) \in X^*$. Let $A_{x, y} \subseteq [m]$. We call any
injective function $\varphi_{x, y} : A_{x, y} \longrightarrow [n]$ a {\em
matching} from $x$ to $y$.

Again let $x = (x_1, \ldots, x_m) \in X^*, (y_1, \ldots, y_n) \in X^*$ and
$\varphi$ a matching from $x$ to $y$. Let $\delta$ be a difference function on
$X$. Define $\delta^*_{\varphi}(x, y)$ to be the list given by $\delta(x_i,
y_{\varphi(i)})$ for $i \in A$, except that we omit $\delta(x_i,
y_{\varphi(i)})$ in position $i$ if $x_i = y_{\varphi(i)}$. If $i \not \in A$
we add $i_S(x_i)$ at position $i$. Finally we collapse the list to get rid of
the empty spaces. 

Going back to our previous example, let $x = (a, b, c, d, e)$ and $y = (a, c, b,
d)$, and assume that $\varphi : \{1, 2, 3, 4\} \longrightarrow [4]$ is
given by $$\varphi(1) = 1, \; \varphi(2) = 3, \; \varphi(3) = 2, \text{ and }
\varphi(4) = 4$$ Then $\delta_{\varphi}^*((a, b, c, d, e), (a, f, b, d)) = (c,
e)$ Since $x_1 = y_{\varphi(1)} = a$, $x_2 = y_{\varphi(2)} = b$ and $x_{4} =
y_{\varphi(4)} = d$, $a, b$ and $d$ will be omitted from the list at
positions 1, 2 and 4. Also since $\delta(c, f) = c$ then $c$ will be included in
the list at position 3 and since $e$ is unmatched, $i(e) = e$ will be
included in the list at position 5. Finally we collapse the list to get rid of
the three empty spaces created.

Now let $M = \{\varphi_{x, y} \; | \; x, y \in X\}$ be a set of matchings
between $x$ and $y$ for each $x, y \in X$. We define the difference function
$\delta^*_M$ on $X^*$ by $\delta^*_{M}(x, y) = \delta^*_{\varphi_{x, y}}$. Let
$(\ell, \partial (S, V)) : (S, \partial S, \delta_{S}, i_{S}) \Leftrightarrow
(V, \partial V, \delta_{V}, i_{V})$ be an SD-lens. Let $M_V = \{\varphi_{x, y}
\; | \; x, y \in V\}$ be a set of matchings between $x$ and $y$ for each $x, y
\in V$. For $s = (s_1, \ldots, s_m), s' = (s'_1, \ldots, s'_n) \in S$, let
$\varphi_{s, s'} = \varphi_{((\ell.\get \; s_1), \ldots, (\ell.\get \; s_m)),
((\ell.\get \; s'_1), \ldots, (\ell.\get \; s'_n))}$. Let $M_S = \{\varphi_{s,
s'} \; | \; s, s' \in S\}$. 

Define the lens $\match(\ell, M_V) : S^* \Leftrightarrow V^*$ by
\begin{align*}
\match(\ell, M_V) .\get \; (s_1, \ldots, s_n) &= (\ell.\get \; s_1) \cdot
\ldots \cdot (\ell.\get \; s_n)\\
\match(\ell, M_V) .\pput \; (s_1, \ldots, s_m) \; (v_1, \ldots, v_n) &= (s'_1
, \ldots , s'_n)\\
\text{ where } s'_i &= \begin{cases}
\ell.\pput \; s_{\psi(i)} \; v_i & \text{if } i \in Dom(\psi)\\
\ell.\create \; v_i & \text{if } i \not \in Dom(\psi)
\end{cases}\\
\text{ and } \psi &= \varphi_{(v_1, \ldots, v_n), ((\ell.\get \; s_1), \ldots,
(\ell.\get \; s_m))}\\
\match(\ell, M_V).\create \; (v_1, \ldots, v_n) &= (\ell.\create \; v_1) \cdot
\ldots \cdot (\ell.\create \; v_n)
\end{align*}
\begin{claim}
$(\match(\ell, M_V), \pos(\partial (S, V))) : (S^*, \partial S^*, \delta^*_{M_S},
\pos(i_{S})) \Leftrightarrow (V^*, \partial V^*, \delta^*_{M_V}, \pos(i_{V}))$
is an SD-lens.
\end{claim}
\begin{proof}
Observe that the matching used to compute the difference on the source side is
the same as the matching used to compute the matching on the view side, hence we
may take the update lens to be position and proceed with the rest of the proof
as in \cref{positionalisvalid}.
\end{proof}
The \match combinator takes an SD-lens $(\ell, \partial (S, V)) : (S, \partial
S, \delta_{S}, i_{S}) \Leftrightarrow (V, \partial V, \delta_{V}, i_{V})$ and a
set $M_V$ of matchings $V$. If $S, \partial S$ and $V, \partial V$ have an
appropriate notion of size defined on them, then we can define canonical
matchings which minimize the size of the update according to some alignment
strategy. We thus defined a {\em normed difference structure} as follows:
\begin{definition}
A normed set is a pair $(X, |\cdot|)$ where $X$ is a set and $|\cdot| : X
\longrightarrow \mathbb{R}^{\geq 0}$
\end{definition}
\begin{definition}
A difference structure $(X, \partial X, \delta_X, i_X)$ is a normed differene
structure if $X$ and $\partial X$ are normed sets satisfing
\begin{enumerate}
  \item
  $|\delta_X(x, x)| = 0$,
  \item
  $|\delta_X(x, x')| \leq |x|$, and
  \item
  $|i_X(x)| = |x|$
\end{enumerate}
\end{definition}
The first axiom says that the size of the difference between $x$ and itself is
0, the second axiom says that the size of the ordered difference between $x$
and $x'$ is at most the size of $x$, and third axiom says that the size of $x$
when it is included as an update in $\partial X$ is equal to the size of $x$.

Now given a view $V$ we may define the set of matchings $M_{POS}$ by 
$$M_{POS} = \{\varphi_{v, v'} \; | \; v = (v_1, \ldots, v_m), v' = (v'_1,
\ldots, v'_n), \varphi_{v, v'}(i) = i \text{ for }1 \leq i \leq \min\{m, n\}\}$$
Then $M_{POS}$ is an optimal set of matchings for a positional alignment
strategy.

For the non-crossing strategy, we want to our matchings to be increasing rather
than positional. Hence we define the set of matchings $M_{NC}$ by
$$M_{NC} = \{\varphi_{v, v'} \; | \; v, v' \in V, \varphi_{v, v'} \text{ is
increasing}, |\delta^*_{\varphi_{v, v}}(v, v')| \text{ is minimal among all
matchings from }v \text{ to }v'\}$$
For any $v, v'$, there may be multiple matchings which for which the size of the
difference is minimal, in which case we pick a canonical one. Then $M_{NC}$ is
an optimal set of matchings for a non-crossing alignment strategy.

Finally, for a min-edit distance strategy, we simply remove the condition that
the matchings should be increasing. Hence we define the set of matchings
$M_{MED}$ by
$$M_{MED} = \{\varphi_{v, v'} \; | \; v, v' \in V, |\delta^*_{\varphi_{v,
v'}}(v, v)| \text{ is minimal among all matchings from }v \text{ to }v'\}$$
Again, for any $v, v'$, there may be multiple matchings which for which the size
of the difference is minimal, in which case we pick a canonical one. Then
$M_{MED}$ is an optimal set of matchings for a min-edit distance alignment
strategy.
\subsubsection{Composition}
Let $\ell_1 : S \Leftrightarrow U$ and $\ell_2 : U \Leftrightarrow V$ be lenses.
Define the lens $\ell_2 \circ \ell_1 : S \Leftrightarrow V$ by
\begin{align*}
(\ell_2 \circ \ell_1).\get \; s &= \ell_2.\get \; (\ell_1.\get \; s)\\
(\ell_2 \circ \ell_1).\pput \; s \; v &= \ell_1.\pput \; s \; (\ell_2.\pput \;
(\ell_1.\get \; s) \; v) \\
(\ell_2 \circ \ell_1).\create \; v &= \ell_1.\create \; (\ell_2.\create \; v)
\end{align*}
Now let $(\ell_1, \partial (S_1, V_1)) : (S,
\partial S, \delta_S, i_{S}) \Leftrightarrow (U,
\partial U, \delta_U, i_{U})$ and $\ell_2 : (U,
\partial U, \delta_U, i_{U}) \Leftrightarrow (V,
\partial V, \delta_V, i_{V})$ be SD-lenses.
\begin{claim}
$(\ell_2 \circ \ell_1, \partial (S_2, V_2) \circ \partial (S_1, V_1))$ is an SD-lens.
\end{claim}
\begin{proof}
We need to show that
\begin{align}
\partial (S_2, V_2) \circ \partial (S_1, V_1)).\get \; (\delta(\ell_2 \circ
\ell_1).\pput \; s \; v, s)&= \delta(v, (\ell_2 \circ \ell_1).\get \; s)
\text{, and}\\
(\partial (S_2, V_2) \circ \partial (S_1, V_1)).\get \; (i_{S}(s)) &=
i_{V}((\ell_2 \circ \ell_1).\get \; s)
\end{align}

For the first equation, observe that
\begin{align*}
&(\partial (S_2, V_2) \circ \partial (S_1, V_1)).\get \; (\delta((\ell_2 \circ
\ell_1).\pput \; s \; v, s))\\
&=(\partial (S_2, V_2) \circ \partial (S_1, V_1)).\get \; (\delta(\ell_1.\pput
\; s \; (\ell_2.\pput \; (\ell_1.\get \; s) \; v), s))\\
&= (\partial (S_2, V_2)).\get \; ((\partial (S_1, V_1)).\get \; (\delta(\ell_1.\pput \;
s \; (\ell_2.\pput \; (\ell_1.\get \; s) \; v), s)))\\
&= (\partial (S_2, V_2)).\get \; (\delta(\ell_2.\pput \; (\ell_1.\get \; s) \; v,
\ell_1.\get \; s))\\
&= \delta(v, \ell_2.\get \; (\ell_1.\get \; s))\\
&= \delta(v, (\ell_2 \circ \ell_1).\get \; s)\\
\end{align*}
For the second equation, observe that
\begin{align*}
(\partial (S_2, V_2) \circ \partial (S_1, V_1)).\get \; (i_{S}(s)) &= 
(\partial (S_2, V_2)).\get \; ((\partial (S_1, V_1)).\get \; (i_{S}(s)))\\
&= (\partial (S_2, V_2)).\get \; (i_{U}(\ell_1.\get \; s))\\
&= i_{V}(\ell_2.\get \; (\ell_1.\get \; s))\\
&= i_{V}((\ell_2
\circ \ell_1).\get \; s)
\end{align*}
This completes the proof.
\end{proof}


%% Acknowledgments
\begin{acks}                            %% acks environment is optional
%% contents suppressed with 'anonymous'
%% Commands \grantsponsor{<sponsorID>}{<name>}{<url>} and
%% \grantnum[<url>]{<sponsorID>}{<number>} should be used to
%% acknowledge financial support and will be used by metadata
%% extraction tools.
This material is based upon work supported by the
\grantsponsor{GS100000001}{National Science
Foundation}{http://dx.doi.org/10.13039/100000001} under Grant
No.~\grantnum{GS100000001}{nnnnnnn} and Grant
No.~\grantnum{GS100000001}{mmmmmmm}.  Any opinions, findings, and
conclusions or recommendations expressed in this material are those
of the author and do not necessarily reflect the views of the
National Science Foundation.
\end{acks}

\section{Related Work}
In his thesis \cite{foster2009bidirectional}, Foster observed that while it would be
ideal for the lens laws to force the \pput component of a lens to always have
``minimal'' side effects, even stating this condition seems to require
building a notion of what constitutes an update into the semantics---we need to
be able to compare two updates to determine which one has a ``smaller'' effect.

This idea of lenses carrying a notion of updates and update propagation has been
explored in various ways. First, various categorical approaches have been
explored under the name of {\em delta lenses}
\cite{diskin2011state,johnson2017symmetric,pacheco2012delta,johnson2012lenses}
Delta lenses are lenses defined in an appropriate category, where sources and
views are the objects of the category, and the arrows are deltas (or updates)
between the objects of the category. In this setting, lens transformations are
functors which map objects to objects as well as deltas to deltas. Categorical
delta lenses are thus similar to the SD-lenses described in this paper in that
they map states to states as well as updates to updates. The main difference
between SD-lenses and categorical delta lenses is that SD-lenses are
set-based rather than categorical and arguably simpler to understand. Moreover,
only \cite{pacheco2012delta} defines a fairly rich set of combinators for delta
lenses over inductive types in the same vane as SD-lenses.

{\em Alignment lenses} have also been explored in tackling the issue of update
propagation in \cite{bohannon2008boomerang,barbosa2010matching}. Prior to this
work, lenses could only perform updates positionally on data lists, but
alignment lenses can match chunks using an optimal matching with
respect to a certain alignment strategy such as a non-crossing strategy or
min-edit distance strategy. While alignment lenses solved the problem of
matching chunks in a way that causes minimal change with respect to some
alignment strategy, they do not address the general problem of specifying how
updates should be propagated for an arbitrary lens. The SD-lenses introduced in
this paper do give a fairly general method for specifying and updates and update
propagation.

{\em Edit lenses} \cite{hofmann2012edit} are another approach to tackling the
issue of update propagation. Edit lenses work with descriptions of changes to
structures, rather than with the structures themselves. These lenses separate
the description of edits from the action of applying an edit to a state, with
the action of applying an edit to a state modelled by a monoid action, a design
choice which guarantees compositionality of edits. SD-lenses differ from edit
lenses in that edit lenses are not stateful---the only information passed to the
lens is the intended update, hence edit lenses pass around less information. On
the other hand, SD-lenses are again arguably simpler since they are stateful
and are also built on top of traditional lenses which are ubiquitous in the
bidirectional programming literature.
\bibliographystyle{plain}
\bibliography{local}

%% Appendix
%%\appendix
%%\section{Appendix}

\end{document}
