\newif\ifdraft\drafttrue  % set true to show comments
%\newif\ifdraft\draftfalse  % set true to show comments
\newif\ifanon\anonfalse    % set true to suppress names, etc.
\newif\iffull\fullfalse   % set true for long version
\newif\ifappendices\appendicesfalse

\PassOptionsToPackage{usenames,dvipsnames,svgnames,table}{xcolor}
\documentclass[sigplan,acmsmall]{acmart}

\listfiles

\usepackage[usenames,dvipsnames,svgnames,table]{xcolor}
\usepackage{amsmath}
\usepackage{mathtools}
\usepackage{bussproofs}
\usepackage{amsthm}
\usepackage{csvsimple}
\usepackage{thmtools,thm-restate}
\usepackage{changepage}
\usepackage{booktabs}
\usepackage{amssymb}
\usepackage[inline]{enumitem}
\usepackage{multirow,bigdelim}
\usepackage{multicol}
\usepackage{siunitx}
\usepackage{listings}
\usepackage{sansmath}
\usepackage{url}
\usepackage{flushend}
\usepackage{microtype}
\usepackage[utf8]{inputenc}
\usepackage{mathpartir}
\usepackage{empheq}
\usepackage{array}
\usepackage{pgfplots}
\usepackage{stmaryrd}
\usepackage{courier}
\usepackage{qtree}
\usepackage[normalem]{ulem}
\usepackage{relsize}
\usepackage{tikz}
\usepackage{algorithm}
\usepackage[noend]{algpseudocode}
\usepackage{graphicx}
\usepackage{subcaption}
\usepackage{textcomp}
\usepackage{tabularx}
\usepackage{stackengine}
\usepackage{caption}
\usepackage{wrapfig}
\usepackage{remreset}

\settopmatter{printacmref=false}
\renewcommand\footnotetextcopyrightpermission[1]{}


\usetikzlibrary{
  er,
  matrix,
  shapes,
  arrows,
  positioning,
  fit,
  calc,
  pgfplots.groupplots,
  arrows.meta
}
\tikzset{>={Latex}}

%%%% Hyperlinks – must come late!
%\usepackage[pdftex,%
%            pdfpagelabels,%
%            linkcolor=blue,%
%            citecolor=blue,%
%            filecolor=blue,%
%            urlcolor=blue]
%           {hyperref}

\clubpenalty = 10000
\widowpenalty = 10000
\displaywidowpenalty = 10000

%\setlength{\belowcaptionskip}{-5pt}
%\setlength{\textfloatsep}{15pt}

% Creates a display mode for code in sans serif font
\lstnewenvironment{sflisting}[1][]
  {\lstset{%
    mathescape,
    basicstyle=\small\sffamily,
    aboveskip=5pt,
    belowskip=5pt,
    columns=flexible,
    frame=,
    xleftmargin=1em,#1}\sansmath}
  {}
% end

% Macros

  \acmYear{2018}

\begin{document}

\special{papersize=8.5in,11in}
\setlength{\pdfpageheight}{\paperheight}
\setlength{\pdfpagewidth}{\paperwidth}
%\toappear{}

%\conferenceinfo{POPL '16}{January 20--22, 2016, St. Petersburg, FL, USA} 
%\copyrightyear{2016} 
%\copyrightdata{978-1-nnnn-nnnn-n/yy/mm} 

% Uncomment one of the following two, if you are not going for the 
% traditional copyright transfer agreement.

%\exclusivelicense                % ACM gets exclusive license to publish, 
                                  % you retain copyright

%\permissiontopublish             % ACM gets nonexclusive license to publish
                                  % (paid open-access papers, 
                                  % short abstracts)

%\titlebanner{DRAFT---do not distribute}        % These are ignored unless
%\preprintfooter{DRAFT---do not distribute}   % 'preprint' option specified.

\title{Synthesizing Bijective Lenses: Artifact Evaluation}

\author{Anders Miltner}
\affiliation{Princeton University, USA}
\email{amiltner@cs.princeton.edu}

\author{Kathleen Fisher}
\affiliation{Tufts University, USA}
\email{kfisher@eecs.tufts.edu}

\author{Benjamin Pierce}
\affiliation{University of Pennsylvania, USA}
\email{bcpierce@cis.upenn.edu}

\author{David Walker}
\affiliation{Princeton University, USA}
\email{dpw@cs.princeton.edu}

\author{Steve Zdancewic}
\affiliation{University of Pennsylvania, USA}
\email{stevez@cis.upenn.edu}
\maketitle

% \category{D.3.1}
% {Programming Languages}
% {Formal Definitions and Theory}
% [Semantics]

% begin introduction
\section{Setup Environment}
\subsection{Virtual Machine Environment (recommended)}
To make validation easier, we provide the Artifact Evaluation Committee with a
virtual machine that has all components installed.  Setup for the virtual
machine environment detailed below:
\begin{enumerate}
\item Install a Virtual Machine Manager (we tested with VirtualBox)
\item Download our virtual machine image (BijectiveLensSynthesis.ova in\\
  {\color{blue}
    \url{https://drive.google.com/drive/folders/0B5OYxhNvX675cTN4ZzVEMmJ0Mlk})}
\item Load our virtual machine from your virtual machine manager
\end{enumerate}

We suggest the committe dedicate over 8GB RAM to this environment.
While we get similar results with less RAM, some of the existing tools we
compare our artifact to use up to 8GB RAM.  Less RAM may cause these tools to
perform worse than presented in our data.

The user for the virtual machine is bls, with password bls.

\subsection{Custom Environment}
If you want to install on a custom environment, certain programs and libraries
must be installed.  We provide the commands for installation in Ubuntu 16.04.3
in parenthesis next to the program or library name.

\begin{enumerate}
\item Install opam (sudo apt install opam; opam init; eval `opam config env`;
  opam update)
\item Switch opam to OCaml 4.03.0 (opam switch 4.03.0; eval `opam config env`)
\item Install necessary system packages (opam depext conf-m4.1)
\item Install OCaml's Core (opam install core)
\item Install OCaml's Menhir (opam install menhir)
\item Install OCaml's OUnit (opam install ounit)
\item Install OCaml's ppx\_deriving (opam install ppx\_deriving)
  
\item Install python-pip (sudo apt install python-pip)
\item Install Python's EasyProcess (pip install EasyProcess)
\item Install Python's matplotlib (pip install matplotlib)
\item Install Python's tk (sudo apt install python-tk)
  
\item Install DotNet runtime environment, and C\# and Visual Basic libraries and
  compilers (sudo apt install mono-complete; sudo apt install mono-vbnc)

\item Install the codebase\\
  (git clone https://github.com/Miltnoid/BijectiveLensSynthArtifactEvaluation.git)
\end{enumerate}


\section{Tool Validation}

\subsection{Simple Validation}

To validate: navigate to the directory the codebase is installed in
(/home/bls/LensSynth in the virtual machine).
Then run the following commands:
\begin{enumerate}
\item make regenerate-specs
\item make regenerate-data
\end{enumerate}

After these commands are run, the figures will be present at the following
locations: 
\begin{itemize}
\item Figure 8 will be present at \$/generated-graphs/times.eps
\item Figure 9 will be present at \$/generated-graphs/uninferred.eps
\item Figure 10 will be present at \$/generated-graphs/compositional.eps
\item Figure 11 will be present at \$/generated-graphs/examples.eps
\end{itemize}

The command ''make regenerate-specs`` will take about an hour to complete.  It
must run to completion for the data to be saved.

The full data from the runs will be aggregated at
\$/generated-data/data.csv

The command, ``make regenerate-data'' can be stopped when partially completed.
After a benchmark has had all tests run on it, that data is saved, so future
runs will not rerun tests on that benchmark.  Each benchmark takes between 0 and
3 hours to complete, with information being printed to stdout about what
test and benchmark is being run.  The total time to complete all tests takes
between 2 and 3 days.

\subsection{Directory Information}

If the artifact committee want to do more in-depth testing, we provide
information on the individual project directories.  The means to run individual
programs are described in README.txt files in the directories.

\paragraph*{\$/program}  Our tool is present in the program directory.  The
source code is provided in \$/program/src .  The directory \$/program/tests
contains a test suite, consisting of microbenchmarks and the tested benchmarks.
The directory \$/program/example-inputs contains the tested benchmarks.
The directory \$/program/generated\_data is the output location for the csv created from
program runs.  The directories \$/program/generated\_extraction\_specs and
\$/program/generated\_io\_specs are the output locations for the specifications
given to Flash Fill and FlashExtract.

\paragraph*{\$/comparisons}  When created by ``make regenerate-specs'', the
directories \$/comparisons/extraction\_specs and \$/comparisons/io\_specs
contain the specifications given to Flash Fill and FlashExtract.  The directory
\$/comparisons/prose/Transformation.Text contains the generated data for its
runs under generated\_data, and the source code for our tests in
SampleProgram.cs.  The directory
\$/comparisons/prose/Extraction.Text contains the generated data for its
runs under generated\_data, and the source code for our tests in
LearningSamples.cs.

\paragraph*{\$/generated-data}  After data is created by the individual tests,
it is then combined and placed into \$/generated-data/data.csv.

\paragraph*{\$/generated-graphs}  After data is created by the individual tests
and combined into \$/generated-data/data.csv, the data is converted into graphs
and placed into \$/generated-graphs.

\end{document}



%%% Local Variables:
%%% TeX-master: "main"
%%% End:
